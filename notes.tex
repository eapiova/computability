\documentclass{amsbook}
\usepackage[utf8]{inputenc}
\usepackage[english]{babel}
\usepackage{csquotes}
\usepackage{hyperref}
\usepackage{amsmath}
\usepackage{dsfont}
\usepackage{amsfonts}
\usepackage{amssymb}
\usepackage{graphicx}
\usepackage{amsfonts}
\usepackage{parskip}
\usepackage{enumitem}
\usepackage{tabu}
\usepackage{listings}
\usepackage{xcolor}
\usepackage{mathabx}
\usepackage{quiver}

\hypersetup{
	colorlinks=false,
	linkcolor={blue!50!black}
}
\usepackage{mathtools}
\usepackage[
backend=biber,
style=alphabetic,
sorting=ynt
]{biblatex}
\addbibresource{bibliography.bib}

\DeclarePairedDelimiter\ceil{\lceil}{\rceil}
\DeclarePairedDelimiter\floor{\lfloor}{\rfloor}


% macros

% natural numbers
\newcommand{\nat}{\ensuremath{\mathbb{N}}}

% domain
\newcommand{\dom}[1]{\ensuremath{\mathit{dom}({#1})}}

% comment (in programs)
\newcommand{\comment}[1]{{\texttt{// #1}}}

% theorems
\newtheorem{theorem}{Theorem}[chapter]
\newtheorem{lemma}[theorem]{Lemma}
\newtheorem{corollary}[theorem]{Corollary}
\newtheorem{proposition}[theorem]{Proposition}

\theoremstyle{definition}
\newtheorem{definition}[theorem]{Definition}
\newtheorem{example}[theorem]{Example}
\newtheorem{exercise}[theorem]{Exercise}
\newtheorem{observation}[theorem]{Observation}

\theoremstyle{remark}
\newtheorem{remark}[theorem]{Remark}
\newtheorem{notation}[theorem]{Notation}
\newtheorem{counterexample}[theorem]{Counterexample}

\numberwithin{section}{chapter}
\numberwithin{equation}{chapter}

%    For a single index; for multiple indexes, see the manual
%    "Instructions for preparation of papers and monographs:
%    AMS-LaTeX" (instr-l.pdf in the AMS-LaTeX distribution).
\makeindex

\begin{document}

\frontmatter

\title{Computability\\
Some unofficial notes}

\author{Master's degree in Computer Science\\
  A.Y. 2022/2023\\
  Prof. Paolo Baldan}

\address{}
\curraddr{}
\email{}
\thanks{}

\keywords{}

\date{}

\begin{abstract}
\end{abstract}

\maketitle

%    Dedication.  If the dedication is longer than a line or two,
%    remove the centering instructions and the line break.
\cleardoublepage
\thispagestyle{empty}
\vspace*{13.5pc}
%\begin{center}
These LaTeX notes originates from some handwritten material
distributed for the course in Computability at the Master Degree in
Computer Science at the University of Padua.  I am grateful to Riccardo Borsetto, Mert Anil Hasret, Luca Zaninotto for translating the original notes and producing the LaTeX version.

They are unofficial sketchy notes, which are intended as a support for
students for understanding what is discussed in the course. They are
not intended in any sense as a replacement for the official book
(Nigel Cutland ``Computability. An Introduction to Recursive Function
Theory''), on which they are heavily based.

Paolo Baldan

% \end{center}
%\cleardoublepage

%    Change page number to 6 if a dedication is present.
\setcounter{page}{4}

\tableofcontents


\mainmatter

\chapter{Introduction}\label{chap:intro}

In this chapter, we informally discuss the notion of \textbf{effective procedure} and \textbf{function computable} by the means of an effective
procedure. This will lead us to single out the main features of an
algorithm/computational model.  Despite being informal, these
considerations will allow us to derive the existence of non-computable
functions for any chosen effective computational model.
%
Later these notions and considerations are going to be formalised by fixing a
specific computational model, a sort of idealized computer.

\section{Algorithm or effective procedure (Informal)}

Even though we do not always refer to them by their technical terms when we apply them, \textbf{effective procedure}s and \textbf{algorithms} are a part of our everyday life.

For example; at the primary school we are not only taught that given two numbers their sum exists, but we are also provided a procedure to compute the sum of two numbers!

In general terms, an \emph{algorithm} can be defined as the
description of a sequence of \emph{elementary steps} (where
``elementary'' means that they can be performed mechanically, without
any intelligence) which allows one reach some objective.  Typically,
the aim is transforming some input into a corresponding output,
suitably related to the input.
%
This could be transforming ingredients into a cake, although normally
we are interested in computational problems.

\begin{example}
  Some examples are:
  \begin{enumerate}

  \item given $n \in \nat$ establish whether $n$ is prime;
  \item find the $n^{th}$ prime number;
  \item derive a polynomial;
  \item perform the square root $\sqrt{n}$;
  \item least common multiple \emph{lcm} and largest common divisor \emph{LCD}.
  \end{enumerate}
\end{example}

Therefore we can think of an algorithm as a black box.
\begin{center}
  in $\rightarrow$ \boxed{black box} $\rightarrow$ out
\end{center}
where the transformation is performed by executing a sequence of
``mechanical'' instructions.

If each step is \emph{deterministic} (i.e., in each state of the
system the instruction to execute and the new state it produces are
uniquely determined), then each possible input will uniquely determine
the corresponding output (the procedure might not terminate, in which case we will have no output).

In mathematical terms the algorithm induces a \emph{(partial) function}
\begin{center}
  $f : \mathit{input} \rightarrow \mathit{output}$.
\end{center}
We say that $f$ is the \emph{function computed} by the algorithm and
that $f$ is effectively computable. Thus, we can give the following
first definition of an algorithm (still informal since it refers to a
generic notion of algorithm).

\begin{definition}[computable function]
  A function $f$ is computable if \emph{there exists} an algorithm
  that computes $f$.
\end{definition}

We stress that for $f$ to be computable, it is not important to know the algorithm that computes $f$, but rather we need to know that some algorithm that computes $f$ exists.

\begin{example}
  According to the above definition, we expect the the following
  functions to be computable.

  \begin{itemize}

  \item GCD (greatest common divisor), e.g., exploiting Euclid's
    algorithm.

  \item the function $f : \nat \to \nat$ defined as

    \begin{equation*}
      f(n)=
      \begin{cases}
        1 & n \mbox{ is prime} \\
        0 &   \mbox{otherwise}
      \end{cases}
    \end{equation*}

  \item
    $g(n)= n$-th prime number\\
    (this is computable, maybe inefficiently by generating numbers
    and testing for primality until the $n$-th prime is found)


  \item
    $h(n) = n$-th digit of the decimal representation of $\pi$.

    In fact, from analysis we know that
    \begin{itemize}
    \item There are series that converge to $\pi$
    \item There are techniques to estimate (by excess) the error caused:
      \begin{itemize}
      \item by truncating a series
      \item by rounding in the calculation of the value of the truncated series
      \end{itemize}
    \end{itemize}
  \end{itemize}
\end{example}

What about the function below?

\begin{equation*}
  g(n) = \begin{cases}
    1 & $ there is a sequence of exactly  \textit{n} consecutive $5$'s in $ \pi \\
    0 & $ otherwise $
	\end{cases}
\end{equation*}

For example $g(3) = 1$ if and only if  $\pi = 3.14 \dots k 555 h \dots $, with $k, h \neq 5$.

Computability is unclear. A naive algorithm could be:
\begin{itemize}
\item generate the digits of $\pi$
\item until a sequence of $5$'s of the desired length $n$ is found.
\end{itemize}
Clearly, if such a sequence exists, it will be eventually found and the answer $1$ will be given. However, at no point the computation, not having found the desired sequence of $n$ 5's, we can certainly say that it will never appear later on! Hence, we currently have no way of answering $0$.

\begin{remark}
  Clearly, something of the kind:
   \begin{itemize}
   \item generate all digits in the decimal representation of $\pi$;
   \item if they include a sequence of $n$ consecutive 5's
    $\rightarrow g(n) = 1$
  \item otherwise $\rightarrow g(n) = 0$
\end{itemize}
is \emph{not} an effective procedure.
\end{remark}

Note that this doesn't mean that $g$ is not computable, i.e., that an
effective procedure couldn't be found (e.g. on the basis of properties
of $\pi$), but at the moment this procedure is not known! (at least by
me!)

We don't really know if it's computable, but there might be a property
of $\pi$ that allows us to conclude. In particular, there is a conjecture that all
finite sequences of digits appear in $\pi$, which would imply that $g$
is the constant $1$, whence computable.

\medskip

Consider now a slightly different function $h: \nat \to \nat$, defined by
\begin{equation*}
  g(n) = \begin{cases}
    1 & $ there is a sequence of at least  \textit{n} consecutive $5$'s in $ \pi \\
    0 & $ otherwise $
  \end{cases}
\end{equation*}

The function seems very similar to the one considered before. However, note that if $\pi = 3.14 \dots k 555 h \dots $, then we deduce, not only that $h(3)=5$, but also $h(2)=h(1)=h(0)=1$. More generally, whenever $h(n) =1$ then $h(m)=1$ for all $m < n$. This suggests that $h$ could be quite ``simple''.

More precisely, consider $K = sup\{ n \mid \pi\ \text{contains}\ n\ \text{consecutive digits}\ 5 \}$. Then we have 2 possibilities:
\begin{enumerate}
\item $K$ is finite, and thus $h(n) = 1$ if $ n\leq k$, $0$ otherwise
\item $K$ is infinite, and thus $ h(n) = 1$ for all $n \in \nat$
\end{enumerate}

This implies that $h$ is computable because it is either a step
function or a constant function, that are computed by simple
programs. One could object that we don't know which shape the function
has and thus we do not know the program that calculates the
function. Fine. This doesn't mean that it's not computable.

Trying to repeat the same argument for function $g$ fails. In fact, one could think of defining $A = \{n \mid \mbox{$\pi$ contains exactly $n$ consecutive 5's}\}$. Then

\begin{equation*}
  g(x) =
    \begin{cases}
      0 & x \in A     \\
      1 & x \not\in A
    \end{cases}
\end{equation*}

This does not suggest that $g$ is computable. Set $A$ is possibly infinite and we do not see a way of providing a finite representation of $A$ which can be included in a program.

Bringing the argument to the extreme, one could consider the function
$G : \nat \to \nat$ defined by
\begin{center}
  $G(x) = \begin{cases}
    1 & $ if God exists $ \\
    0 & $ otherwise $
  \end{cases}.
  $
\end{center}
This is either the constant $0$ or the constant $1$. Independently of which of the two cases applies, the function is computable.


TO BE PROCESSED


\chapter{Algorithms and existence of non-computable functions}

\section{Characteristics of an algorithm}
\label{se:alg-char}

We present a list of features that an algorithm should satisfy in
order to capture the intuitive idea of \textbf{effective procedure}. Roughly,
what we will ask is that it is ``implementable'' on some sort of
idealised machine, the computational model. Hence, in turn, we will
list some requirements that the computational model should meet to be
considered effective.

An \textbf{algorithm} is a sequence of instructions with the following characteristics:
\begin{enumerate}[label=\alph*)]
\item
  \label{as:prog_fin}
  it is of \textbf{finite length};
\item there exists a \textbf{computing agent} able to execute its instructions;
\item the agent as a \textbf{memory} available (to store input, intermediate results to be used in subsequent steps and output)
\item the computation consists of discrete steps (it does not rely on analog devices)*
\item the computation is neither non-deterministic nor probabilistic (
  we model a  digital computer)

\item there must be no limit to the size of the input data\\
  (we want to be able to define algorithms that work on any possible
  input, e.g. sum \dots operating on summands of any size);

\item there is no limit to the memory that can be used\\
  This requirement could appear less natural, but having an unlimited
  memory is essential to avoid the notion of computability depends
  on the specific resources which are available. In fact, for many
  functions the space required for the intermediate results depends on
  the size of the input.

  e.g. $f(n) = n^2$ $(1000)^2 = 1000000 \leftarrow$ I must add a
  number of zeroes that depends on $n \rightarrow n$ must be stored
  (the states are finite).

\item
  \label{as:istr_fin}
  there must exist a finite limit to the number/complexity of the
  instructions

  This is intended to capture the intrinsic finiteness of the
  calculation device (justified by Turing with the limits of the human
  mind/memory);

  E.g. for a computer, the memory that can be accessed with a single
  instruction must be finite (even if by (g), the memory is unlimited);

\item computations might
  \begin{enumerate}

  \item  end and return a result after a finite, but unlimited number of steps
    (e.g. the square function requires a number of steps proportional to the argument);

  \item continue forever, and not return a result.
  \end{enumerate}
\end{enumerate}

\section{Existence of non-computable functions}

Later on, we will focus on a concrete computational model that will allow
us to give a completely formal definition of computable function. Simply on the basis of the assumptions above, we observe that we can infer the existence of non-computable functions for every ``effective'' computational model.

We start by recalling some basic notions and introducing useful notation.

\begin{itemize}
\item We will consider the set of \emph{natural numbers}
  $\nat = \{ 0, 1, 2, \dots \}$;

\item Given the sets $A, B$ their \emph{Cartesian product} is
  $A \times B = \{ (a,b) \mid a \in A\ \land\ b \in B\}$. We will
  write $A^n$ for $A \times A \times A \times \ldots \times A$ $n$
  times. More formally $A^1 = A$ and $ A^{n+1} = A \times A^n$.

\item A (binary) \emph{relation} or \emph{predicate} is
  $r \subseteq A \times B$.

\item A \emph{(partial) function} $f : A \to B$ is a special relation $f \subseteq A\times B$ such that if $(a, b_1), (a, b_2) \in f$ then  $b_1 = b_2$.  Following the standard convention, we will write $f(a) = b$ instead
  of $(a, b)\in f$
  \begin{itemize}
  \item the \emph{domain} of $f$ is
    $\dom{f} = \{a \mid \exists b \in B.\ f(a) = b \}$;

  \item we write $f(a) \downarrow$ for $a \in dom (f)$ and
    $f(a) \uparrow$ for $a \not\in dom (f)$;
  \end{itemize}

\item Given a set $A$ we indicate with $|A|$ its \emph{cardinality}
  (intuitively, the number of elements of $A$, but the notion extends
  to infinite sets). Given the sets $A$ and $B$ we have
  \begin{itemize}
  \item $|A| = |B|$ if there exists a bijective function $f : A \to B$;
  \item $|A| \leq |B|$ if there exists an injective function
    $f: A \to B$ injective or equivalently\footnote{Stritly speaking,
      the equivalence requires the axiom of choice.} a surjective
    function $g : B \to A$.
  \end{itemize}
  Observe that if $A \subseteq B$ then $|A| \leq |B|$ as witnessed by
  the inclusion, which is an injective function
  \begin{quote}
    $\begin{array}{cc}
       i: & A \to B  \\
          & a \mapsto a
     \end{array}$
   \end{quote}

 \item We say that $A$ is \emph{countable} or \emph{denumerable} when
   $|A| \leq |\nat|$, i.e., there is a surjective function
   $f: B \to A$. Note that, when this is the case, we can
   list (enumerate, whence the name) the elements of $A$ as
   \begin{center}
     $\begin{array}{cccc}
        f(0) & f(1) & f(2) & \dots\\
        a_0  & a_1  & a_2 & \dots
      \end{array}
      $
    \end{center}

  \item When $A, B$ are countable then $A\times B$ is countable.

    Idea of the proof:
    \begin{itemize}
    \item Since $A$ and $B$ are countable, we can consider the
      corresponding enumerations

      \begin{quote}
        $
        \begin{array}{cccc}
          A & a_0 & a_1 & a_2 \\
          B & b_0 & b_1 & b_2
        \end{array}
        $
      \end{quote}
      and place the elements of $A \times B$  in a sort of matrix
      \begin{center}
        $
        \begin{tabu}{c|ccc}
          & b_0       & b_1       & b_2       \\
          \hline
          a_0 & (a_0,b_0) & (a_0,b_1) & (a_0,b_2) \\
          a_1 & (a_1,b_0) & (a_1,b_1) & (a_1,b_2) \\
          a_2 & (a_2,b_0) & (a_2,b_1) & (a_2,b_2)
        \end{tabu}
        $
      \end{center}
      in a way that they can be enumerated following along the diagonals
      as follows:
      $(a_0,b_0), (a_0,b_1), (a_1,b_0), (a_0,b_2), (a_1,b_1), (a_2,b_0),
      \dots$ (this is referred to as ``dove tail'' enumeration)
    \end{itemize}


  \item A countable union of countable sets is countable: if
    $\{A_i\}_{i\in\nat}$ is a collection of countable sets then
    $\bigcup \limits_{i \in \nat} A_i$ is countable.
  \end{itemize}

  \section{Existence of non-computable functions in each computational model}

  Let us consider some fixed computational model satisfying the
  assumptions in \S\ref{se:alg-char}. We want to show that there are
  functions which are not computable in such a model.

  We focus on unary functions over the natural numbers. Let
  $\mathcal{F} = \{f \mid f:\nat\rightarrow\nat\}$ be the set of all the
  (partial) unary functions on $\nat$.

  Let $\mathcal{A}$ be the set of all algorithms in our fixed
  computational model.
  %
  Every algorithm $A \in \mathcal{A}$ computes a function
  $f_A: \nat \to \nat$ and a function is said to be computable in our model if
  there exists an algorithm that computes it. Hence the set
  $F_\mathcal{A}$ set of computable functions in the given computational
  model is
  \begin{center}
    $\mathcal{F}_{\mathcal{A}} = \{ f_A \mid A \in \mathcal{A} \}$.
  \end{center}

  Certainly $\mathcal{F}_A \subseteq \mathcal{F}$. But, is the inclusion
  strict (i.e., is there a non-computable function)?

  The answer is yes. Essentially, algorithms are considered too few to compute all of the functions for combinatory reasons.


  In fact, an algorithm $A \in \mathcal{A}$ will be a finite, by
  assumption (\ref{as:prog_fin}), sequence of instructions taken from
  some instruction set $I$. Moreover, by assumption (\ref{as:istr_fin}),
  $I$ must be finite. Hence:
  \begin{center}
    $\mathcal{A} \subseteq \bigcup_{i \in \nat} I^n$
  \end{center}
  Since  a countable union of finite (hence countable) sets is countable, we have:
  \begin{center}
    $|\mathcal{A}| \leq |\bigcup_{n\in\nat} I^n| \leq |\nat|$
  \end{center}
  and since the function
  \begin{quote}
    $\mathcal{A} \to F_\mathcal{A}$\\
    $A \mapsto f_A$
  \end{quote}
  is surjective by definition, we have that
  \begin{center}
    $|F_\mathcal{A}| \leq |\mathcal{A}| \leq |\nat|$
  \end{center}

  On the other hand the set of all functions, $\mathcal{F}$, is not countable. Let $\mathcal{T}$ the subset of $\mathcal{F}$ consisting of the total functions $\mathcal{T} = \{ f \mid f \in \mathcal{F}\ \land\ \dom{f} = \nat\}$. We show that
  \begin{center}
    $|\mathcal{F}| \geq |\mathcal{T}| > |\nat|$.
  \end{center}

  We prove that $|\mathcal{T}| > |\nat|$ by contradiction. Let us suppose that $\mathcal{T}$ is countable. Then we can consider an enumeration $f_0, f_1, f_2, \ldots$ of $\mathcal{F}$ and thus a matrix like the following:
  \begin{center}
    \begin{tabular}{c|ccc}
      & $f_0$    & $f_1$    & $f_2$\\
      \hline
      0 & $f_0(0)$ & $f_1(0)$ & $f_2(0)$ \\
      1 & $f_1(0)$ & $f_1(1)$ & $f_1(2)$ \\
      2 & $f_2(0)$ & $f_2(1)$ & $f_2(2)$
    \end{tabular}
  \end{center}
  and build a function, that consists of the values that are on the diagonal, systematically changed:
  \begin{quote}
    $d: \nat \to \nat$\\
    $d(n) = f_n(n)+1$
  \end{quote}

  We can observe that
  \begin{itemize}
  \item $d$ total, by definition;
  \item $d \neq f_n$ for all $n \in \nat$ (since $d(n) = f(n)+1 \neq f(n)$.
  \end{itemize}
  This is absurd since $f_0, f_1, f_2, \ldots$ is an enumeration of all the total functions.

  \medskip

  Summing up:
  \begin{center}
    $\mathcal{F}_A \subsetneq F$ and
    $|F_A| \leq |\nat| < |\mathcal{T}| = |\mathcal{F}|$
  \end{center}
  therefore $F_A \subsetneq F$, as desired.

  Note that non-computable functions are not countable:
  \begin{center}
    $|\mathcal{F} \setminus \mathcal{F}_{\mathcal{A}}| > |\nat|$.
  \end{center}
  In fact, $\mathcal{F} = \mathcal{F}_{\mathcal{A}} \cup (\mathcal{F} \setminus \mathcal{F}_{\mathcal{A}})$. Thus, if it were to be $|\mathcal{F} \setminus \mathcal{F}_{\mathcal{A}}| \leq |\nat|$, we would have $\|\mathcal{F}| \leq |\nat$ simply because of the fact that union of countable sets is countable.

  We conclude that
  \begin{enumerate}
  \item no computational model can compute all functions;
  \item the non-computable functions are the majority.
  \end{enumerate}

\chapter{URM computability}

\section {Which model?}

To give a formal notion of computability we must choose a concrete model of computation that induces a class of algorithms and therefore of computable functions. 
Despite the fact that we focus on an abstract ideal model, there are still a lot of possibilities. Many models have been considered in the literature:

\begin{enumerate}
\item Turing machine (Turing, 1936)
\item $\lambda$-calculus (Church, 1930)
\item Partial recursive functions (Godel-Kleene 1930)
\item Canonical deductive systems (Post, 1943)
\item Markov systems (Markov, 1951)
\item Unlimited register machine (URM) (Shepherdson - Sturgis, 1963)
\end{enumerate}

In principle, each computational model determines a class of computable functions.
We may be concerned thinking that the developed theory is valid only for a specific
model chosen. In fact, it can be verified that the class of computable functions for all
models cited (for all the models ``sufficiently expressive'' considered
in literature) is always the same. This leads to the so-called Church-Turing
thesis:

\textbf{Church-Turing thesis}: A function is computable by an
effective procedure (i.e., in a finitary computational model, obeying the conditions (a)-(e) from the chapter before) if and only if it is computable
by a Turing machine.


This means that the notion of ``computable function'' is robust (i.e. independent of the specific computational model), and we can choose our favorite one for developing our theory.

\begin{remark}
  The \emph{Church-Turing thesis} is called a thesis and not a theorem due
  of its informal nature. 
  It cannot be proved w.r.t effective procedures, but is supported only by evidence: 
  several computational models have been considered and all respect the thesis 
  (e.g. Yuri Gurevich, argues that it should be proved on the basis of a formal
  axiomatization of conditions (a) - (e)).
\end{remark}

Sometimes we resort to the Church-Turing thesis to shorten the proof that a certain entity is computable, however it can only be used when it is not strictly necessary, i.e. when it could be replaced by a formal proof
(and providing all the details could hide the intuitive idea under a bunch of technicalities).

\section{URM (Unlimited register machine)}

We will formalise the notion of \textbf{computable function} by using an \textbf{abstract machine} 
called \textbf{URM-machine} (Unlimited Register Machine), 
which is an abstraction of a computer based on the Von Neumann's model. It is characterized by

\begin{itemize}
\item \textbf{unbounded memory} that consists of a infinite sequence of \textbf{registers}, each of which can store a  natural number


  $\begin{tabu}{|c|c|c|c|c|}
    \hline
    R_1 & R_2 & \dots & R_n & \dots \\
    \hline
    r_1 & r_2 & \dots & r_n & \dots \\
    \hline
  \end{tabu}$

  the $n$-th register is indicated with $R_n$, its content with $r_n$

  the sequence $(r_1, r_2,\dots, r_n,\dots) \in \nat^\omega$ is called
  \textbf{configuration} of the URM;

\item a \textbf{computing agent} capable of executing an URM program;

\item  a \textbf{URM program}, i.e. a finite sequence of instructions
  $I_1, I_2, \dots, I_s$ that can ``locally'' alter the configuration
  of the URM.
\end{itemize}


Program instructions can be the following

\begin{itemize}

\item \textbf{zero} $Z(n)$ sets the content of the register $R_n$ to zero: $r_n \leftarrow 0$;

\item \textbf{successor} $S(n)$ increments the content of the $R_n$ register by 1: $r_n \leftarrow r_n+1$;

\item \textbf{transfer} $T(m,n)$ transfers the content of the register $R_m$ in the register $R_n$, $R_m$ stays untouched: $r_n\leftarrow r_m$.
\end{itemize}
The above are often referred to as \emph{arithmetic instructions}. They are characterised by the fact that the instruction to be executed in the next step
 is the one following the current instruction in the program.

Then last instruction  is 
\begin{itemize}
\item \textbf{conditional jump} $J(m,n,t)$ compares the content of the registers $R_m$ and $R_n$
  \begin{itemize}
  \item if $r_m = r_n$ it jumps to the $t$-th instruction;
  \item otherwise, it continues with the next instruction.
  \end{itemize}
\end{itemize}


\begin{example}
  An example of program is the following:
  \begin{quote}
    \begin{tabular}{llr}
      $I_1$: & J(2,3,5) &                       \\
      $I_2$: & S(1)     &                       \\
      $I_3$: & S(3)     &                       \\
      $I_4$: & J(1,1,1) &  \comment{unconditional jump}
    \end{tabular}
  \end{quote}

  Disregard what this program computes for the moment. The computation starting from the configuration below is:

  \begin{center}
    $\begin{tabu}{|c|c|c|c|}
      \hline
      R_1 & R_2 & R_3 & \dots \\
      \hline
      1   & 2   & 0   & \dots \\
      \hline
    \end{tabu}
    %
    \xrightarrow{I_1, I_2}
    %
    \begin{tabu}{|c|c|c|c|}
      \hline
      R_1 & R_2 & R_3 & \dots \\
      \hline
      2   & 2   & 0   & \dots \\
      \hline
    \end{tabu}
    %
    \xrightarrow{I_3}
    %
    \begin{tabu}{|c|c|c|c|}
      \hline
      R_1 & R_2 & R_3 & \dots \\
      \hline
      2   & 2   & 1   & \dots \\
      \hline
    \end{tabu}
    %
    \xrightarrow{I_4, I_1, I_2}
    %
    \begin{tabu}{|c|c|c|c|}
      \hline
      R_1 & R_2 & R_3 & \dots \\
      \hline
      3   & 2   & 1   & \dots \\
      \hline
    \end{tabu}
    %
    \xrightarrow{I_3}
    %
    \begin{tabu}{|c|c|c|c|}
      \hline
      R_1 & R_2 & R_3 & \dots \\
      \hline
      3   & 2   & 2   & \dots \\
      \hline
    \end{tabu}
    \xrightarrow{I_4, I_1, I_5}
    $
  \end{center}
\end{example}


The \textbf{state} of the URM machine in which it executes a program $P = I_1 \dots I_s$ 
is given by a pair $\langle c, t \rangle$ that consists of a

\begin{itemize}
\item \emph{register configuration} $c$\\
  a total function $c : \nat \to \nat$ such that $c(n)$ is the content
  of register $R_n$;

\item \emph{program counter} $t$, i.e., index of the current instruction.
\end{itemize}

An \emph{operational semantics} can easily be defined via a set of deduction rules 
axiomatising the state transitions  $\langle c, t \rangle \rightarrow \langle c', t' \rangle$. 
However we do not need this level of formality, and we will rely on an informal description of the program execution.


\begin{remark}
  A computation might \textbf{not terminate}! Consider for instance the program

  \begin{quote}
    \begin{tabular}{ll}
      $I_1$: & S(1)     \\
      $I_2$: & J(1,1,1)
    \end{tabular}
  \end{quote}

  Then the computation will not terminate. For instance
  \begin{center}
    $\begin{tabu}{|c|c|c|c|}
      \hline
      R_1 & R_2 & R_3 & \dots \\
      \hline
      0  & 0   & 0   & \dots \\
      \hline
    \end{tabu}
    %
    \xrightarrow{I_1, I_2}
    %
    \begin{tabu}{|c|c|c|c|}
      \hline
      R_1 & R_2 & R_3 & \dots \\
      \hline
      1   & 0   & 0   & \dots \\
      \hline
    \end{tabu}
    %
    \xrightarrow{I_1, I_2}
    %
    \begin{tabu}{|c|c|c|c|}
      \hline
      R_1 & R_2 & R_3 & \dots \\
      \hline
      2   & 0  & 0   & \dots \\
      \hline
    \end{tabu}
    %
    \xrightarrow{\ldots}
    %
    $
  \end{center}
\end{remark}


\begin{notation}
  Let $P$ be an URM program, and $(a_1,a_2,a_3,\dots) \in \nat^\omega$ a sequence
  of natural numbers. We indicate the computation of $P$ starting from the
  initial configuration by $P(a_1,a_2,\dots)$:

  \begin{center}
    $\begin{tabu}{|c|c|c|c|}
      \hline
      R_1 & R_2 & R_3 & \dots \\
      \hline
      a_1 & a_2 & a_3 & \dots \\
      \hline
    \end{tabu}$
  \end{center}

  and

  \begin{itemize}
  \item $P(a_1,a_2,\dots) \downarrow$ if the computation \textbf{halts}.
  \item $P(a_1,a_2,\dots) \uparrow$ if the computation \textbf{never
      halts} (i.e., it \textbf{diverges}).
  \end{itemize}


  We will work on computations that start from an initial configuration
  where only a \textbf{finite number of registers contain a non-zero value} for
  the majority of the time (almost always for obvious reasons of input
  finiteness). Hence; given $a_1,a_2,\dots,a_k \in \nat$ we will write

  \begin{center}
    $P(a_1,\dots,a_k)$ for 
    $P(a_1,\dots,a_k,0,\dots,0)$
  \end{center}

  The notation extends to $P(a_1,\dots,a_k)\downarrow$ or
  $P(a_1,\dots,a_k)\uparrow$.
\end{notation}

\section{URM-computable functions}

Let $f : \nat^k \rightarrow \nat$ be a partial function. What does it mean for  $f$ to be computable by an URM machine?

Intuitively, it means that there exists a program $P$ such that for each $(a_1,\dots,a_k) \in \nat^k$, 
$P(a_1,\dots,a_k)$ computes the value of $f$, i.e. when $(a_1,\dots,a_k) \in \dom{f}$, 
$P$ terminates and outputs $f(a_1, \ldots, a_k)$. 
However, $P$ does not terminate if $(a_1,\dots,a_k) \not\in \dom{f}$.

A doubt could be about where the output is stored. 
We conventionally decide that the output will be in the first register $R_1 $(at the end of the computation, any register other than the first register contains irrelevant data). 
For this reason we introduce the following notation

\begin{notation}
  Let $P$ be a program and $(a_1,\dots,a_k) \in \nat^k$, we write
  $P(a_1,\dots,a_k)\downarrow a$ if $P(a_1,\dots,a_k) \downarrow$ and
  the final configuration contains $a$ in $R_1$
\end{notation}

\begin{definition}[URM-computable function]
  A function $f:\nat^k\rightarrow\nat$ is said to be
  \textbf{URM-computable} if there exists a URM program $P$ such that for all
  $(a_1,\dots,a_k) \in \nat^k$ and $a\in\nat$,
  $P(a_1,\dots,a_k)\downarrow$ if and only if $(a_1,\dots,a_k)\in dom(f)$ and
  $f(a_1,\dots,a_k) = a$. 
  
  In this case we say that $P$ computes $f$.

  We denote by $\mathcal{C}$ the class of all URM-computable
  functions and by $\mathcal{C}^{(k)}$ the class of the k-ary
  URM-computable functions.
  Therefore we have
  $\mathcal{C} = \bigcup_{k\geq 1} \mathcal{C}^{(k)}$
\end{definition}

\section{Examples of URM-computable functions}

We next list some URM-computable functions, providing the corresponding programs.

\begin{enumerate}
\item $f:\nat^2 \rightarrow \nat$\\
  $f(x,y) = x+y$

  \begin{quote}
    \begin{tabular}{lll}
      $I_1$: & J(2,3,5) &                    \\
      $I_2$: & S(1)     &                    \\
      $I_3$: & S(3)     &                    \\
      $I_4$: & J(1,1,1) &  \comment{unconditional jump}
    \end{tabular}
  \end{quote}

  \begin{center}
    $\begin{tabu}{|c|c|c|c|}
      \hline
      R_1 & R_2 & R_3 & \dots \\
      \hline
      x   & y   & 0   & \dots \\
      \hline
    \end{tabu}$
  \end{center}

  \emph{Idea}: Increment $R_1$ and $R_3$ until $R_2$ and $R_3$ contain
  the same value. This results in adding to $R_1$ the content of
  $R_2$.

\item $f:\nat \rightarrow \nat$\\
  $f(x) = x\dot{-}1 = \begin{cases} 0 & x=0 \\ x-1 & x>0 \end{cases}$

  \begin{center}
    $\begin{tabu}{|c|c|c|c|}
      \hline
      R_1 & R_2 & R_3 & \dots \\
      \hline
      x   & 0   & 0   & \dots \\
      \hline
    \end{tabu}$
  \end{center}

  \emph{Idea}: if $x=0$ it trivially terminates; if $x>0$ keep a value $k-1$ in
  $R_2$ and $k$ in $R_5$, with $k>1$ ascending until $R_3=x$, at that
  point $R_2 = x-1$.

  Here's the program

  \begin{quote}
    \begin{tabular}{lll}
      $I_1$: & J(1,3,8) \\
      $I_2$: & S(3)     \\
      $I_3$: & J(1,3,7) \\
      $I_4$: & S(2)     \\
      $I_5$: & S(3)     \\
      $I_6$: & J(1,1,3) \\
      $I_7$: & T(2,1)   \\
    \end{tabular}
  \end{quote}


\item $f:\nat \rightarrow \nat$\\
  $f(x) = \begin{cases}
    \dfrac{x}{2} & \mbox{if $x$ even}\\
    \uparrow      & \mbox{otherwise}
  \end{cases}$

  \emph{Idea:} Store and increasing even number in $R_2$ and store its' half in
  $R_3$.
  \begin{center}
    $\begin{tabu}{|c|c|c|c|}
      \hline
      R_1 & R_2 & R_3 & \dots \\
      \hline
      x   &  2k  & k   & \dots \\
      \hline
    \end{tabu}$
  \end{center}

  \begin{quote}
    \begin{tabular}{lll}
      $I_1$: & J(1,2,6) \\
      $I_2$: & S(2)     \\
      $I_3$: & S(2)     \\
      $I_4$: & S(3)     \\
      $I_5$: & J(1,1,1) \\
      $I_6$: & T(3,1)   \\
    \end{tabular}
  \end{quote}

\end{enumerate}

\section {Function computed by a program}
Given a program $P$, for some fixed number $k \geq 1$ of parameters, there exists a unique \textbf{function computed by $P$} that we denote by $f_P^{(k)} : \nat^k \to \nat$ defined by:

\begin{equation*}
  f_P^{(k)}(a_1, \dots, a_k) = \begin{cases}
    a        & $ if $ P(a_1, \dots, a_k) \downarrow a  \quad \\
    \uparrow & $ if $ P(a_1, \dots, a_k) \uparrow
  \end{cases}
\end{equation*}

\begin{remark}
  The same function can be computed by different programs, for the following two reasons

  \begin{itemize}
  \item we can add useless instructions to a program (dead code, $T(n,n)$, ...)

  \item the same function can be computed via different algorithms
    (e.g., for sorting we have quicksort, mergesort, heapsort, etc.)
  \end{itemize}

  A function can be computed either by no program or by infinitely many programs.
\end{remark}

\section{Exercises}

\begin{exercise}[Reduced URM]
Consider the URM machine without transfer instruction. We indicate the class
of functions that can be computed with the reduced machine $\mathcal{C}^- $ and
we compare it with $\mathcal{C} $. Obviously $\mathcal{C}^- \subseteq \mathcal{C}
$. Let us see if $\mathcal{C} \subseteq \mathcal{C}^-$. 
\begin{proof}
Informally an instruction $T(m,n)$ at the $t$ step
can be replaced with the following subroutine
\begin{quote}
  \begin{tabular}{lll}
    $I_{t'}$:   & $J(1,1,t'+5)$  \\
    $I_{t'+1}$: & $Z(n)$        \\
    $I_{t'+2}$: & $J(m,n,t+1)$  \\
    $I_{t'+3}$: & $S(n)$        \\
    $I_{t'+4}$: & $J(1,1,t'+2)$ \\
  \end{tabular}
\end{quote}
where $t'>l(P)$.

We prove it formally. 
Given $f \in \mathcal{C} $, $f: \nat^k \rightarrow \nat $, there is an URM program $P$ such that $f_P^{(K)}  = f$. The program $P$ can be transformed into $P ^R $ of the reduced URM machine such that $f_{P^R}^{(K)}  = f_{P}^{(K)}$.

There is a proof by induction. We show that $P$ can be transformed into $P' $ s.t. $ f_{P'}^{(K)}  = f_{P}^{(K)} $ by induction on $h$, number of transfer instructions $T$ in $P$.

\begin{enumerate}
  \item $h = 0$ trivial.
  \item $h \rightarrow  h + 1$: $P$ contains $h + 1$ transfer instructions.
  Transform $P$ into $P''$ where all instructions from $1$ to $l(P)$ are as same as before, while instead of $T$ we put a jump $J(1,1, SUB)$ where the subroutine is written before. 
  We assume that if $P$ ends, it does so at instruction $l(P) + 1$, then at position $l(P) + 1$ we insert $J (1,1, END)$. 
  After these replacements, we have $h$ instructions $T$ and therefore we can say that we have a reduced URM program such that the computed function is the same by inductive assumption.
\end{enumerate}
\end{proof}
\end{exercise}

\begin{exercise}[URM with swap instruction]
  Let $URM^S $ be the model obtained by removing the transfer
  instruction and inserting a swap instruction $ T_S(m,n) $,
  exchanging the contents of registers $m$ and $n$, and let $\mathcal{C}^S$ be the corresponding class of computable functions.. How do the classes $\mathcal{C}$ and $\mathcal{C}^S$ relate?
\begin{proof}
\begin{itemize}
  \item[($\mathcal{C} \subseteq \mathcal{C}^S$)]
  We already know that $\mathcal{C} \subseteq \mathcal{C}^R $ by the previous exercise and therefore, since $\mathcal{C}^R \subseteq \mathcal{C}^S$, the desired inclusion follows by transitivity.
  
  \item[($\mathcal{C}^S \subseteq \mathcal{C}$)]
  We first prove that the swap instruction $T_S (m, n)$ is equivalent to the program below, where $i$ is a ``new'' register:
  \begin{quote}
    \begin{tabular}{l}
    $T(n,i)$ \\
    $T(m,n)$ \\
    $T(i,m)$
    \end{tabular}
    \end{quote}

More formally, let $ f \in \mathcal{C}^S$, $f:\nat\rightarrow \nat $. There exists $P$
$URM^S $ s.t. $ f_P^{(K)} = f $. Let us proceed by induction on the
number of swap instructions $h$.
\begin{enumerate}
  \item[($h = 0$)] the program is already a URM program. therefore $P' = P$.
  \item[($h \rightarrow h+1$)] Assume that $P$ includes $h+1$ swap instructions. 

  Let $i$ be a register not used by $P$ (observe that it can be found by just inspecting the program). Let $t$ be the index of a swap instructions.
  We replace such instruction by a jump
  \begin{quote}
    $I_t : J (1,1, SUB)$
  \end{quote}
  to a subroutine encoding the swap. At the end of the subroutine we return to the starting point with $J (1,1, t + 1)$. Let $P''$ be the program obtained in this way. Since it has only $h$ swap instructions, by inductive hypothesis there is $P'$ URM such that $ f_{P'}^{(K)} = f_{P''}^{(K)} = f_{P}^{(K)}$.
  
  \textbf{But all of this is wrong!}
  
  Why is it wrong? Because the program $P''$ obtained from $P$ replacing a swap instruction will indeed have $n-1$ swap instructions but also some transfer instructions, hence it is not a $URM^S$ program.
  
  We can overcome the problem by proving the following stronger
  statement: given a program $P$ that uses both URM instructions and
  $URM^S$ instructions, there is a program $P'$ that uses $URM$
  instructions only such that $ f_{P}^{(K)} = f_{P'}^{(K)} $.
  
  The proof procedure is the same but, since we are proving a stronger
  statement, the inductive case now works smoothly. This proves that
  $ \mathcal{C}^S \subseteq \mathcal{C} $.
\end{enumerate}
\end{itemize}



Therefore we deduce $\mathcal{C}^S = \mathcal{C}$, as desired.
\end{proof}

\end{exercise}

\begin{exercise}[URM without jump instructions]
Consider an URM machine without jump instructions  $J(m,n,t)$ and call it URM$ ^{nj}$. Let $\mathcal{C}^{nj}$ be the corresponding class of computable functions. How does this class relate to $\mathcal{C}$?


\begin{proof}

  Clearly $\mathcal{C}^{nj} \subseteq \mathcal{C}$ and the inclusion is strict since,  $f: \nat \rightarrow \nat$ with $f(x)\uparrow \forall x$ is computable in URM, but it is not computable in URM $ ^{nj} $.  In fact, all functions in $\mathcal{C}^{nj}$  are total since programs without jump instructions always terminate.

  We can characterise precisely the (unary) functions in $\mathcal{C}^{nj}$. They are of the shape:
\begin{itemize}
  \item $f(x) = c$
  \item $ f(x) = x + c$
\end{itemize}
where $c$ is a constant in $\nat$.

This can be proved as follows. Denote by $r_1(h,x) $ the content of register $R_1$ after $h$ steps starting from an initial configuration where $R_1$ is $x$ and the other registers contain $0$.

We show by induction on $h$ that after $h$ execution steps $r_1(h,x) $ is equal to $x + c$ or to $c$.

\begin{itemize}
  \item Case $h = 0: r_1(0,x) = x $

  \item Case $h \rightarrow h+1$: We know $ r_1(h,x) = x+c $ or $ c $ by inductive hypothesis. The next instruction can be of three shapes:
  \begin{itemize}
  \item The instruction is $Z (n)$. If $n = 1$, $r_1(h+1,x) = 0 $, otherwise $r_1(h+1,x) = r_1(h,x)$, and we conclude by induction hypothesis.
    
  \item The instruction is $S (n)$. If $n = 1$ we have that $r_1(h+1,x) = r_1(h,x)+1 $ which, by induction hypothesis, is fine. Otherwise,  $r_1(h+1,x) = r_1(h,x)$ and we conclude by induction hypothesis.
    
  \item The instruction is $T (m, n)$. When $n>1 $ or $n=m=1$ then $r_1(h+1,x) = r_1(h,x)$ and we conclude by inductive hypothesis. Otherwise, if $n = 1,\ m > 1$ we do know nothing about the content of $r_1(h+1,x) $. We are stuck ...knows?
  \end{itemize}

  \medskip The problem can be solved by observing that register $1$
  has nothing special and the same result can be proved for all
  registers. More precisely, denote by Denote by $r_n(h,x)$ the
  content of register $R_n$ after $h$ steps starting from an initial
  configuration where $R_1$ is $x$ and the other registers contain
  $0$. Then one can show that $r_n(h,x)$ contains either $c$ or $x+c$
  for a suitable constant $c$. In this case the proof goes smoothly.
\end{itemize}
\end{proof}
\end{exercise}

\chapter{Decidable Predicates}

In mathematics we often want to establish \textbf{properties}, for example \emph{$m$ is a divisor of $n$}. 
We can define it using a relation such as
\begin{align*}
    div & \subseteq \nat \times \nat\\
    div & = \{(m,k \cdot m) \mid m \in \nat, k \in \nat \}
\end{align*}

We can also view $div$ as a function
\begin{align*}
    div & : \nat \times \nat \rightarrow \{true, false\}\\
    div & = \begin{cases}
                true & \mbox{if $m$ is a divisor of $n$}\\
                false & \mbox{otherwise}
            \end{cases}
\end{align*}

In the field of computability theory one normally use the term
\textbf{predicates}.

Thus a  \textbf{k-ary predicate} on $\nat $ indicated with
$Q(x_1,\dots,x_k)$ is a property that can be true or false, formally
we can see it as

\begin{itemize}
\item a function $Q: \nat^k\rightarrow \{true,false\}$;
\item a set $Q \subseteq \nat^k$.
\end{itemize}

We write $Q(x_1,\dots,x_k)$ to denote $(x_1,\dots,x_k) \in Q$ or $Q(x_1,\dots,x_k) = true$

When is $Q$ computable? When there exists a URM such that given a k-tuple $(x_1,\dots,x_k)$ in input, it returns $true$ if $Q(x_1,\dots,x_k)$ and $false$ otherwise. 

To represent $\mathit{true}$ and $\mathit{false}$ we conventionally use values $1$ and $0$.

\begin{definition}[decidable predicate]
    A predicate $Q \subseteq \nat^k$ is said to be \textbf{decidable} if its \textbf{characteristic function}
\begin{equation*}
\mathcal{X}_Q(x_1,\dots,x_k) = \begin{cases}
1 & $ if $ Q(x_1,\dots,x_k) \\
0 & $ otherwise $
\end{cases}
\end{equation*}

is (URM) computable.
\end{definition}


\begin{remark}
    $\mathcal{X}_Q$ is a \textbf{total} function (dealing with decidability of predicates, involves only total functions).
\end{remark}


\section {Examples of decidable predicates}
\begin{enumerate}
    \item Equality\\
    $Q \subseteq \nat^2$, $Q(x,y) \equiv x = y$

    The characteristic function
    \begin{equation*}
      \mathcal{X}_Q(x,y) =
      \begin{cases}
        1 & $ if $ x = y  \\
        0 & $ otherwise $
      \end{cases}
    \end{equation*}
    is computed, for instance, by the program
    \begin{quote}
    \begin{tabular}{lll}
    $I_1$ & J(1,2,3)  \\
    $I_2$ & J(1,1,4)       \\
    $I_3$ & S(3) \\
    $I_4$ & T(3,1)
    \end{tabular}
    \end{quote}

    \item $Q(x) \equiv x \text{ is even}$

    \begin{quote}
    \begin{tabular}{lll}
    $I_1$ & J(1,2,6)   \\
    $I_2$ & S(2)        \\
    $I_3$ & J(1,2,7)   \\
    $I_4$ & S(2)        \\
    $I_5$ & J(1,1,1) \\
    $I_6$ & S(3)        \\
    $I_7$ & T(3,1)
    \end{tabular}
    \end{quote}
    
    \begin{tabu}{|c|c|c|}
    \hline
    x & k & r \\
    \hline
    \end{tabu} in memory where k is a growing index and r is the result.
    
    \item $Q(x,y) \equiv x \leq y$
    
    We can either increment both $x$ and $y$ until $x+k=y$, so that $x\leq y$, or until $y+k=x$, so that $x>y$ (not equal for the order of comparisons).
    
    \begin{quote}
    \begin{tabular}{lll}
    & T(1,3)      &        \\
    & T(2,4)      &        \\
    LOOP: & J(2,3,SI)   & \comment{x+k=y?} \\
    & J(1,4,NO)   & \comment{y+k=x?} \\
    & S(3)        &        \\
    & S(4)        &        \\
    & J(1,1,LOOP) &        \\
    SI:   & S(5)        &        \\
    NO:   & T(5,1)      &
    \end{tabular}
    \end{quote}
    
    Memory: $\begin{tabu}{|c|c|c|c|c|}
    \hline
    x & y & x+k & y+k & r \\
    \hline
    \end{tabu}$ where $r$ is the result.
    
    Another approach is to increment a register starting from 0. If we reach $x$ first then $x \leq y$, otherwise $x > y$.
    
    \begin{quote}
    \begin{tabular}{lll}            
    LOOP: & J(1,3,SI)   & \\
    & J(2,3,NO)   & \\
    & S(3)        & \\
    & J(1,1,LOOP) & \\
    SI:   & S(4)        & \\
    NO:   & T(4,1)      &
    \end{tabular}
    \end{quote}
    
    $\begin{tabu}{|c|c|c|c|}
    \hline
    x+k & y & k & r \\
    \hline
    \end{tabu}$ where $r$ is the result.
    
    \item $div(x,y)$ with $x \not= 0$
    
    \begin{quote}
    \begin{tabular}{lll}            
    LOOP: & J(2,3,SI)   &                                   \\
    & Z(4)        & \comment{sum $x$ to $R_2$}         \\
    ADDX: & J(1,4,LOOP) &                                   \\
    & J(2,3,NO)   & \comment{$kx+h=y$?} \\
    & S(3)        &                                   \\
    & S(4)        &                                   \\
    & J(1,1,ADDX) &                                   \\
    SI:   & S(5)        &                                   \\
    NO:   & T(5,1)      &
    \end{tabular}
    \end{quote}
    
    $\begin{tabu}{|c|c|c|c|c|}
    \hline
    x & y & kx+h & h & r \\
    \hline
    \end{tabu}$ where $r$ is the result.
\end{enumerate}

\chapter{Computability on other domains}
Since the URM is only able to manipulate natural numbers, our definition of computability concerns only functions and predicates on $\nat$.

The concept of computability can be extended to other domains referencing a notion of effective encoding.

Suppose that we are interested in computability on an object domain $D$. 
Does our concept of computability extend to this domain? 
One of the necessary conditions is the possibility of encoding the elements of $D$ as natural numbers. 
Suppose there exists $ \alpha: D \rightarrow \nat $, which is biunivocal and that $ \alpha, \alpha^{-1} $ are ``effective''. 
We can't have a formal notion of effectiveness; it means that everyone would agree on its calculability (if there's any justice in this world).

The domain $D$ must be countable. For example, take the strings of a certain alphabet $ \Sigma $, $ D = \Sigma^* $. 
The set of rational numbers $ \mathbb{Q} $ is also countable, and so is the set of integers $\mathbb{Z}$, while $D$ can't be  $ \mathbb{R} $ or $A^\omega$ (streams).

At this point we must ask ourselves when a function on a generic domain $D$ is URM-computable. 

\begin{definition}[Computable function on generic domain]
  Given $ f: D \rightarrow D $, we say it is \textbf{computable} if $ f^* = \alpha \circ f \circ \alpha^{-1}$
  \[\begin{tikzcd}
    D & D \\
    \nat & \nat
    \arrow["{\alpha^{-1}}", from=2-1, to=1-1]
    \arrow["f", from=1-1, to=1-2]
    \arrow["\alpha", from=1-2, to=2-2]
    \arrow["{f^*}"', from=2-1, to=2-2]
  \end{tikzcd}\]
  is URM-computable.
\end{definition}

We will see that if $\alpha$ is effective, its' inverse is also effective.

\begin{example}
  To work on integers we need a function $ \alpha: \mathbb{Z} \rightarrow \nat $; one way to define it is
  \begin{equation*}
    \alpha(z) = \begin{cases}
      2z    & z \geq 0 \\
      -2z-1 & z < 0
    \end{cases} 
  \end{equation*}
which is an effective function with inverse
\begin{equation*}
  \alpha^{-1}(n) = \begin{cases}
    \dfrac{n}{2}    & n $ is even $ \\
    -\dfrac{(n+1)}{2} & n $ is odd $
  \end{cases}
\end{equation*}

An example of a computable function is $f (z) =  |z| $. 
It is computable if $ f^*=\alpha\circ f\circ \alpha^{-1} $ is URM-computable. We have
\begin{align*}
  f^*(n) &= (\alpha\circ f\circ \alpha^{-1})(n)\\ 
         &=  
\begin{cases}
  (\alpha\circ f)\left(\dfrac{n}{2}\right) & n $ even $ \\
  (\alpha\circ f)\left(-\dfrac{n+1}{2}\right) & $ otherwise $
\end{cases}\\
&=
\begin{cases}
  \alpha\left(\dfrac{n}{2}\right) & n $ even $\\
  \alpha\left(\dfrac{n+1}{2}\right) & $ otherwise $
\end{cases}\\
&= \begin{cases}
  n   & n $ even $ \\
  n+1 & $ otherwise $
\end{cases} 
\end{align*}
that is URM-computable, so $f$ is computable.
\end{example}


\chapter {Generation of computable functions}

We can prove that certain functions are computable if they are simpler combinations of modules that are computable. We have the set $ \mathcal{C} $ of the computable functions and if we take $ f_1, f_2 \in \mathcal{C} $ and compose them with a suitable operation $ op(f_1, f_2) $ we are still in the set $ \mathcal{C} $.

More precisely we will prove that the $\mathcal{C}$ class is closed with respect to the following operations:
\begin{itemize}
\item composition (generalized)
\item primitive recursion
\item minimization (unlimited)
\end{itemize}

So to prove that the function $f:\nat^k\rightarrow \nat$ is computable we could write the URM program P that computes $f$ ($f_P^{(k)} = f$), or we could use the theorems of the closure of $\mathcal{C}$.

Actually the three operations we consider are not chosed randomly; the long term objective is to show that $\mathcal{C}$ coincides with the class of generable fucntions through composition, primitive recursion and minimization, starting from a restricted core of basic functions (\textbf{partial recursive functions} of Godel-Kleene).

\begin{notation}
  Given $f,g:\nat^k\rightarrow\nat$ and
  $\vec{x}\in\nat^k$ we write $f(\vec{x}) \simeq g(\vec{x}) $ for
  $f(\vec{x})\downarrow$ if and only if $g(\vec{x})\downarrow$ and if
  $f(\vec{x})\downarrow \Rightarrow f(\vec{x}) = g(\vec{x})$.
\end{notation}

\section {The basic functions are computable}
Basic functions:
\begin{enumerate}
\item $ \underline{0}: \nat^k \rightarrow \nat $, $\lambda x_1\dots x_k . 0$
\item $ s: \nat \rightarrow \nat $, $\lambda x . x+1$
\item projection $ U_i^k: \nat_i^k \rightarrow \nat $,  $\lambda x_1\dots x_k . x_i$
\end{enumerate}

Identity is a sub-case of projection.

The functions that calculate these basic functions are the arithmetic instructions:
\begin{enumerate}
\item $\underline{0}$ computed by z(1)
\item $s$ computed by s(1)
\item $ U_i^k$ computed by T(i, 1)
\end{enumerate}

To prove the properties of closure we will need to ``combine'' programs\dots we need a bit of notation.

Given $P$ URM program we indicate with $ \rho(P) $ \textbf{the maximum register index} used by $P$. $P$ is in \textbf{standard form} if, being $S$ its length, for each $J(m,n,t)$ instruction $t\leq s+1$ (if it ends, it does so at the instruction $s+1$).

Obviously considering only standard form programs is not limitative, meaning the following holds:

\textbf{Lemma:} For each URM program $P$ there exists a program $P'$ in satndard form which is equivalent, meaning s.t. $\forall k f_p^{(k)} = f_{P'}^{(k)}$

\textbf{proof}: It's enough to replace every instruciton $J(m,n,t)$ in $P$ s.t. $t>s+1$ with $J(m,n,s+1)$

Often we will have to \textbf{concatenate} programs, let $P, Q$ be programs, the concatenation consists in executing $P$ and when it terminates executing $Q$, that is, every jump instruction in $Q$ is replaced by adding the length of $P$, $S$.

\textbf{Note:} if $P,Q$ in standard form $\Rightarrow PQ$ is in standard form; moreover $(PQ)R = P(QR)$. We will assume every program is in standard form and we will use concatenation ``freely''.

It will be useful to take the input and give the output in arbitrary registers. Given $P$ I want $ P[i_1,\dots,i_k \rightarrow h] $ ie that it has input in the registers of index $ R_{i1},\dots,R_{ik} $, puts the output in $ R_h $ and does not assume that the rest of the registers is 0. This is easily obtainable with transfer and reset operations to move the contents of registers from $ i_1,\dots,i_k $ to $ 1,\dots,k $ and the output from $h$ to 1.

\section {Generalized composition}
given a function
$ f: \nat^k \rightarrow \nat $ and functions
$ g_1,\dots,g_k: \nat^m \rightarrow \nat $
the composition $ h: \nat^m \rightarrow \nat $
is $ \forall \vec{x} \in \nat^m $ then
$ h(\vec{x}) = f(g_1(\vec{x}), \dots g_k(\vec{x}))
$ means that $ h(\vec{x}) \downarrow $ if $ n_1 = g_1(\vec{x}) \downarrow $
\dots $ n_k = g_k(\vec{x}) \downarrow $ and $ f(n_1,\dots,n_k) \downarrow $

\textbf{Proposition:} if $f:\nat^k\rightarrow\nat$ and $ g_1,\dots,g_k $ computable \textbf{then} $h(\vec{x}) = f(g_1(\vec{x}), \dots g_k(\vec{x}))$ is computable.

\textbf{proof}: Let's take URM programs in standard form $ F, G_1, \dots, G_k $ for the computable functions mentioned above. Let us consider a register number not used by anyone, $m = max\{\rho(F),\rho(G_1), \dots \rho(G_k),k,n\}$, the program for the composition can be:

$\begin{tabu}{|c|c|c|c|c|c|c|c|c|}
  \hline
  1                      & \dots                        & m                                 & m+1   & \dots & m+n   & m+n+1 & \dots & m+n+k \\
  \hline
  \dots                  & \dots                        & x1                                & \dots & n_n   & \dots & \dots & \dots &       \\
  \hline
  \multicolumn{3}{|c|}{} & \multicolumn{3}{c|}{\vec{x}} & \multicolumn{3}{c|}{g_i(\vec{x})}                                                 \\
  \hline
\end{tabu}$

$\begin{tabu}{l}
  T([1,n], [m+1,m+n])                   \\
  G_1 [m+1,\dots,m+n \rightarrow m+n+1] \\
  \dots                                 \\
  G_k [m+1,\dots,m+n\rightarrow b+n+k]  \\
  P[m+n+1,\dots,m+n+k \rightarrow 1]
\end{tabu}$

\textbf{Corollary:} Let $f:\nat^k\rightarrow \nat$ be computable. Then $g:\nat^n\rightarrow \nat$, where $g(x_1,\dots,x_n) = f(x_{i1},\dots,x_{ik})$ is computable, where $(x_{i1},\dots,x_{ik})$ is a sequence of variables in $x_1,\dots,x_n$ with repetitions and missing variables.

\textbf{proof}: if $\vec{x} = (x_1,\dots, x_n)$,

\begin{equation*}
  g(\vec{x}) = f(\cup_{i1}^n(\vec{x}),\dots,\cup_{ik}^n\vec{x})
\end{equation*}

\textbf{Examples}:
if $f:\nat^2 \rightarrow \nat $ is computable, then the following are also computable:
\begin{itemize}
\item $f_1(x,y) = f(y,x)$
\item $f_2(x) ? f(x,x)$
\item $f_3(x,y,z) = f(x,y)$
\end{itemize}

\textbf{Note}: On the basis of this result we can utilize generalized composition when the $g_i$ are not functions of all the variables or are functions with repetitions.

\textbf{Example}: Knowing that: $ sum: \nat^2 \rightarrow \nat $ where $ sum(x_1,x_2) = x_1 + x_2 $ is computable, we can deduce that $ f: \nat^3 \rightarrow \nat $ where $ f(x_1,x_2,x_3) = x_1 + x_2 + x_3 $ is also computable.

In fact $f(x_1,x_2,x_3) = f(f(x_1,x_2),x_3) $. We can think about them as functions $\nat^3\rightarrow\nat$, so we get to $sum(sum(U_1^3(\vec{x}),U_2^3(\vec{x})), U_3^3(\vec{x}))$

The following functions are computable:
\begin{itemize}
\item \textbf{constant m} $\lambda \vec{x}.m$, as $m(\vec{x}) = S(S(\dots S(\underline{0}(\vec{x}))))$
\item \textbf{sum of the arguments} $g(x_1,\dots,x_k) = x1 + \dots + x_k$ (see above)
\item \textbf{product by a contant} $mx = g(x,\dots,x) \quad m$ times, where $g$ is the function at the previous step
\item if $f(x,y)$ is computable, then also $f'(x) = f(x,m)$ is

  in fact $f'(x) = f(x,m) = f(U_1^1(x)), m(x))$
\item if $f:\nat\rightarrow\nat$ is total computable, the predicate $Q(x,y)\equiv f(x) = y$ is decidable.

  in fact, we know that $\mathcal{X}_{Eq}(x,y) = \begin{cases}
    1 & x=y         \\
    0 & $otherwise$
  \end{cases}$ is computable

  therefore $\mathcal{X}_Q(x,y) = \mathcal{X}_{Eq}(g(x),y) = \mathcal{X}_{Eq}(g((U_1^2(x,y)), (U_2^2(x,y))$
\end{itemize}

\section {Primitive recursion}

\textbf{recursion} is a familiar concept; it allows to define a function specifying the values in terms of other values of the function itself (and possibly using other already defined functions).

\textbf{factorial}: $0! = 1$ and $(n+1)! = n!(n+1)$

\textbf{fibonacci}: $f(0) = 1$, $f(1) = 1$ and $f(n+2) = f(n) + f(n+1)$

There are many types, here we use a ``controlled'' version of recursion.

So given $f:\nat^k\rightarrow\nat$ and $g:\nat^{k+2}\rightarrow\nat$ functions

we define $ h(\vec{x},0) = f(\vec{x}) $ and $ h(\vec{x}, y+1) = g(\vec{x},y,h(\vec{x},y)) $

\textbf{Note}: The function $h$ is defined in an equational manner, with $h$ that appears on both sides: implicit definition, not obvious that such $h$ exists or that it is unique. Actually it does exist and it is unique, but a general theory that supports this observation is not trivial.

The argument proceeds as follows:

\begin{itemize}
\item we indicate with $[\nat^n\rightarrow\nat]$ the set of functions with $n$ arguments
\item we define an operator:\\
  $T: [\nat^{k+1}\rightarrow\nat] \rightarrow [\nat^{k+1}\rightarrow\nat]$

  $T(h)(\vec{x},0) = f(\vec{x})$

  $T(h)(\vec{x},y+1) = g(\vec{x},y,h(\vec{x},y))$, no circularity
\item the searched functions are the fixed points of $T$, $h$ s.t. $T(h) = h$
\item existence of the fixed point:
  \begin{itemize}
  \item $[\nat^{k+1}\rightarrow\nat]$ cpo
  \item $T$ continuous
  \item Scott $\rightarrow$ it has a least fixed point
  \end{itemize}
\item uniqueness: inductively if $h,h'$ fixed points $\Rightarrow h=h'$
\end{itemize}

For example sum and successor: $ h(x,y) = x+y $ if $ h(x,0) = x = f(x) $ and $ h(x,y+1) = h(x,y) + 1 $ therefore $f$ is the identity and $g$ is the successor, both computable therefore the sum is computable by primitive recursion.

\textbf{Observation}
Functions obtained from total functions by:
\begin{enumerate}
\item generalized composition
\item primitive recursion
\end{enumerate}
are still total.

\textbf{proof}
1 is obvious by definition.

Let $f\nat^k\rightarrow\nat, g:\nat^{k+2}\rightarrow\nat$ be total functions and let us define:

$h: \nat^{k+1} \rightarrow \nat$

$h(\vec{x},y) = f(\vec{x})$

$h(\vec{x},y+1) = g(\vec{x},y,h(\vec{x},y))$

It can be proved by induction on $y$ that $\forall \vec{x} \quad (\vec{x},y) \in dom(h)$

$(y=0): h(\vec{x},0) \simeq f(\vec{x})\downarrow$

$(y\rightarrow y+1): h(\vec{x},y+1) \simeq g(\vec{x},y,h(\vec{x},y))\downarrow$ by inductive hypothesis

\textbf{Examples}:

\begin{itemize}
\item \textbf{sum} $x+y$\\
  $x+0 = x\\
  x+(y+1) = x+y+1\\\\
  h(x,0) = x\\
  h(x,y+1) = h(x,y)+1\\\\
  f(x) = x\\
  g(x,y,z) = z+1$
\item \textbf{product}
  $x\cdot y\\
  x\cdot 0 = 0\\
  x\cdot (y+1) = xy+x\\
  \\
  h(x,0) = 0\\
  h(x,y+1) = h(x,y)+x\\
  \\
  f(x) = 0\\
  g(x,y,z) = z+y$
\item \textbf{factorial}
  $y!\\
  0! = 1\\
  (y+1)! = y!(y+1)\\
  \\
  h(0) = 1\\
  h(y+1) = h(y)(y+1)\\
  \\
  f(0) = 1$ constant $\\
  g(y,z) = z(y+1)$
\end{itemize}

\textbf{Proposition} (Closure with 	respect to the primitive recursion of $\mathcal{C}$)

Let $f:\nat^k\rightarrow\nat$ and $g:\nat^{k+2}\rightarrow\nat$ computable.

Then $h:\nat^{k+1}\rightarrow\nat$ defined through primitive recursion:\\
$ h(\vec{x},0) = f(\vec{x}) $\\
$ h(\vec{x}, y+1) = g(\vec{x},y,h(\vec{x},y)) $\\
is computable.

\textbf{proof}
Let $F,G$ programs in standard form for $f,g$, let us show how a program for $h$ can be constructed.
We proceed as suggested by the definition.

We start from $\begin{tabu}{|c|c|c|c|c|c|}
  \hline
  x_1 & \dots & x_k & y & 0 & \dots \\
  \hline
\end{tabu}$

we save the parameters and we start to calculate $h(\vec{x},0)$ using $F$.

If $y=0$ we are done, otherwise we save $h(\vec{x},0)$ and i calculate $h(\vec{x},1) = g(\vec{x},0,h(\vec{x},0))$ with $G$. Do the same for $h(\vec{x},i)$ until we arrive to $i=y$

As usual we need registers not used by $F$ and $G$, $m = max\{\rho(F),\rho(G),k+2\}$ and we build the program for $h$ as follows:

$\begin{tabu}{|c|c|c|c|c|c|c|c|c|}
  \hline
  1                     & \dots                                  & m+1                    & \dots   & m+k   & m+k+1 & \dots        & m+k+3 &   \\
  \hline
  \dots                 & \dots                                  & \dots                  & \vec{x} & \dots & i     & h(\vec{x},2) & y     & 0 \\
  \hline
  \multicolumn{2}{|c}{} & \multicolumn{5}{|c}{$arguments of g$ } & \multicolumn{2}{|c|}{}                                                      \\
  \hline
\end{tabu}$

$\begin{tabu}{lll}
  & T([1,k],[m+1,m+k])                   & $copy $\vec{x}                             \\
  & T(k+1,m+k+3)                         & $copy $ y                                  \\
  & F[m+1,\dots,m+k\rightarrow m+k+2]    & h(\vec{x},0)                               \\
  LOOP: & J(m+k+1,m+k+3,END)                   & i=y?                                       \\
  & G[m+1,\dots,m+k+2 \rightarrow m+k+2] & h(\vec{x},i+1) = g(\vec{x},i,h(\vec{x},i)) \\
  & S(m+k+1)                             & i = i+1                                    \\
  & J(1,1,LOOP)                          &                                            \\
  END:  & T(m+k+2,1)
\end{tabu}$

\textbf{Note:} We do nothing more than implementing recursion through iteration!

\textbf{Theorem:} the following functions are computable.

\begin{enumerate}
\item \textbf{Sum:} $x+y$, see above;
\item \textbf{Product:} $x \cdot y$ see above;
\item \textbf{Exponential:} $x^y$\\
  $x^0 = 1, h(x,0) = 1, f(x) = 1$\\
  $x^{y+1} = x^y\cdot x, h(x,y+1) = h(x,y)\cdot x, g(x,y,z) = z\cdot x$;
\item \textbf{Predecessor:} $x \dot - 1$\\
  $0 \dot -1 = 0, h(0) = 0, f \equiv \underline{0}$\\
  $(x+1)\dotdiv  1 = x, h(x+1) = x, g(y,z) = y$;
\item \textbf{Subtraction:} $x\dotdiv  y = \begin{cases}
    x-y & x \geq y    \\
    0   & $otherwise$
  \end{cases}$\\
  $x\dotdiv  0 = x, f(x) = x\\
  x\dotdiv (x+1) = (x\dotdiv  y)\dotdiv  1, g(x,y,z) = z\dotdiv  1$;
\item \textbf{Sign} $sg(x) = \begin{cases}
    0 & x=0   \\
    1 & x > 0
  \end{cases}\\
  sg(0) = 0, f \equiv \underline{0}\\
  sg(x+1) = 1, g(y,z) = 1$;
\item \textbf{Complement sign:} $\bar{sg}(x) = \begin{cases}
    0 & x=0 \\
    1 & x>0
  \end{cases}\\
  \bar{sg}(x) = 1 \dotdiv  sg(x), $ composition and (6);
\item $ |x - y| = \begin{cases}
    x-y & x\geq y \\
    y-x & x < y
  \end{cases}$\\
  $ |x - y| = (x\dotdiv y)+(y\dotdiv x)$ from (1), (6) and composition;
\item \textbf{Factorial:} $y!\\
  0! = 1, f \equiv 1
  (y+1)! = y!(y+1), g(y,z) = (y+1)z $;
\item \textbf{Minimum:} $min(x,y) = x\dotdiv  (x\dotdiv  y)$;
\item \textbf{Maximum:} $ max(x,y) = (x \dotdiv  y) + y $;
\item \textbf{Remainder:} $rm(x,y) = \begin{cases}
    y mod y & x \not= 0 \\
    y       & x=0
  \end{cases}$ \\ rest of the integer dicision of $y$ by $x$ (convention! reasonable if the rest $rm(x,y)$ must be such that $\exists k \mid kx + rm(x,y) = y$)\\
  $rm(x,0) = 0\\
  rm(x,y+1) = \begin{cases}
    rm(x,y)+1 & rm(x,y)+1 \not= x \\
    0         & $otherwise$
  \end{cases}\\
  = (rm(x,y)+1) sg((x\dotdiv  1)\dotdiv  rm(x,y))\\
  f(x) = 0, g(x,y,z) = z * sg(x\dotdiv 1\dotdiv z)$ OK!

\item \textbf{Quotient} = $qt(x,y) = y$ div $x$, by convention $qt(0,y) = y$\\
  we define:\\
  $qt(x,0) = 0\\
  qt(x,y+1) = \begin{cases}
    qt(x,y)+1 & rm(x,y)+1=x  \\
    qt(x,y)   & $ otherwise$
  \end{cases}\\
  = qt(x,y) + sg((x\dotdiv 1)\dotdiv rm(x,y))$

\item $div(x,y) = \begin{cases}
    1 & x|y                                   \\
    0 & $otherwise, $0|0 $ and $ 0\not|y, y>0
  \end{cases}\\
  div(x,y) = \bar{sg}(rm(x,y))$
\end{enumerate}

\subsection{Definition by cases}
\textbf{Corollary:} given $ f_1,\dots,f_n: \nat^k \rightarrow \nat $ total computable and $ Q_1,\dots,Q_n \subseteq \nat^k $ decidable predicate and mutually exclusive (for each $\vec{x} \in \nat^k$ \textbf{exactly one} of $ Q_1,\dots,Q_n$ holds) then $ f:\nat^k \rightarrow \nat $ is total computable where

\begin{equation*}
  f(\vec{x}) = \begin{cases}
    f_1(\vec{x}) & Q_1(\vec{x}) \\
    f_2(\vec{x}) & Q_2(\vec{x}) \\
    \dots        &              \\
    f_n(\vec{x}) & Q_n(\vec{x})
  \end{cases}
\end{equation*}

\textbf{proof}:
$f(\vec{x}) = f_1(\vec{x})\mathcal{X}_{Q1}(\vec{x}) + \dots + f_n(\vec{x})\mathcal{X}_{Qn}(\vec{x})$

We conclude, for composition using the calculability of sum and product.

It also applies to partial functions but we will prove it later.

\section{Algebra of decidability}
Having $ Q, Q' $   decidable predicates, then also $ \neg Q, Q \wedge Q', Q \vee Q' $ all decidable.

\textbf{proof}:
It's enough to observe that:
\begin{enumerate}
\item $ \mathcal{X}_{\lnot Q}(\vec{x}) =  \overline{sg}(\mathcal{X}_Q(\vec{x})) $
\item $\mathcal{X}_{Q \vee Q'}(\vec{x}) = \mathcal{X}_{Q}(\vec{x}) \cdot \mathcal{X}_{Q'}(\vec{x})$
\item observe that $Q \wedge Q' = \lnot (\lnot Q \vee \lnot Q')$
\end{enumerate}

We remind that $\{\neg, \wedge, \vee \}$ ($\{\neg, \vee \}$ is enough) is a set of connectives functionally complete (it allows to express any function $\{0,1\}^n \rightarrow \{0,1\}$). We deduce that:

\textbf{Corollary}: Let $Q_1, \dots, Q_n \subseteq \nat^k$ decidable predicates and let $f:\{0,1\}^n \rightarrow \{0,1\}$ a function, let us consider:\\
$\mathcal{X}: \nat^k\rightarrow\{0,1\}\\
\mathcal{X}(\vec{x}) = f(\mathcal{X}_{Q1}(\vec{x})), \dots, \mathcal{X}_{Qn}(\vec{x}) )$\\
Then the predicate $Q$ which corresponds to $\mathcal{X}$ is decidable, and therefore $\mathcal{X}$ is computable.

\section{Sum, product, limited quantification}

\textbf{Definition}: (limited sum and product), let $f:\nat^{k+1}\rightarrow\nat$ total.

$\sum_{z<y}f(\vec{x},z)$ is defined by

$\sum_{z<0}f(\vec{x},z) = 0$

$\sum_{z<y+1}f(\vec{x},z) = \sum_{z<y}f(\vec{x},z) + f(\vec{x},y)$

$\prod_{z<y}f(\vec{x},z)$ is defined by:

$\prod_{z<1}f(\vec{x},z) = 1$

$\prod_{z<y+1}f(\vec{x},z) = \prod_{z<y}f(\vec{x},z) \cdot f(\vec{x},y)$

\textbf{Observation}: if $f:\nat^{k+1}\rightarrow\nat$ is total computable then
\begin{enumerate}
\item $g(\vec{x},y) = \sum_{z<y}f(\vec{x},y)$
\item $h(\vec{x},y) = \prod_{z<y}f(\vec{x},y)$
\end{enumerate}
are total computable.

\textbf{proof}: they are defined by primitive recursion!

$g(\vec{x},0) = 0\\
g(\vec{x},y+1) = g(\vec{x},y) + f(\vec{x},y)$

and $+,f$ are computable.

Same for 2.

Obviously, for composition, the bound can be a total computable function.

Another immediate consequence concerns the decidibility of the limited quantification on the predicates.

\textbf{Lemma}: Let $Q\subseteq \nat^{k+1}$ be a decidable predicate, then:

\begin{enumerate}
\item $Q_1(\vec{x},y) \equiv \forall z<y. Q(\vec{x},z)$
\item $Q_2(\vec{x},y) \equiv \exists z<y. Q(\vec{x},z)$
\end{enumerate}

are decidable.

\textbf{proof}:
\begin{enumerate}
\item observe that $\mathcal{X}_{Q1}(\vec{x},y) = \prod_{z<y}\mathcal{X}_Q(\vec{x},z)$
\item observe that $\mathcal{X}_{Q2}(\vec{x},y) = sg(\sum_{z<y}\mathcal{X}_Q(\vec{x},z))$
\end{enumerate}

\section{Limited minimization}
Given a function $ f: \nat^{k+1} \rightarrow \nat $, we define a function $ h: \nat^{k+1} \rightarrow \nat $ as follows:

\begin{equation*}
  h(\vec{x},y) = \mu z<y . f(\vec{x},z) = \begin{cases}
    $min. $z<y$ s.t. $ g(\vec{x},z) = 0 & $ if it exists$ \\
    y                                   & $ otherwise $
  \end{cases}
\end{equation*}

\textbf{Lemma}: Let $ f: \nat^{k+1} \rightarrow \nat $ total computable. Then also $ h: \nat^{k} \rightarrow \nat $ defined by $ h(\vec{x},y) = \mu z<y. f(\vec{x},z) $ is (total) computable.

\textbf{proof}: we observe that $h$ can be defined as:

$h(\vec{x},y) = \sum_{z<y}\prod_{w\leq z} sg(f(\vec{x},w))$

The product value is 1 on the intervals $[0,z]$ in which $f\not= 0$, that if $z_0$ is the min $z<y$ where $f$ is null, they're exactly $z_0$, therefore the external sum counts them.

Alternatively $h$ can be defined directly through primitive recursion:

$
h(\vec{x},0) = 0\\
\\
h(\vec{x},y+1) = \begin{cases}
  h(\vec{x},y)               & h(\vec{x},y)\not= y \\
  \begin{cases}
    y   & f(\vec{x},y) = 0 \\
    y+1 & $otherwise$
  \end{cases} & $ otherwise $
\end{cases}\\
\\
sg(y-h(\vec{x},y)) \cdot h(\vec{x},y) + \bar{sg}(y-h(\vec{x},y))(y+sg(f(\vec{x},y)))
$

\textbf{Note}: Here too the bound can be a total computable function.

\textbf{Lemma}: The following functions are computable:
\begin{enumerate}[label=\alph*)]
\item $D(x) = $ number of divisors of $x$
\item $Pr(x) = \begin{cases}
    1 & $ x is prime $ \\
    0 & $ otherwise $
  \end{cases}$ (x prime is decidable)
\item $p_x$ = $x$-th prime number (convention: $p_0=0, p_1=2,p_2=3\dots$)
\item $(x)_y = \begin{cases}
    $exponent of $p_y$ in the factorization of $x & x,y > 0      \\
    0                                             & x=0 \vee y=0
  \end{cases}$\\
  e.g. $72 = 2^3\cdot 3^2, (72)_1 = 3, (72)_2 = 2, (72)_3 = 0$
\end{enumerate}

\textbf{proof}:
\begin{enumerate}[label=\alph*)]
\item $D(x) = \sum_{y\leq x}div(y,x)$
\item $Pr(x) = \begin{cases}
    1 & D(x) = 2      \\
    0 & $ otherwise $
  \end{cases}$ 1 if $x>1$ and is divided only by 1 and itself.\\\\
  $= \bar{sg}(|D(x)-2|)$
\item $P_x$ can be defined by primitive recursion

  $P_0=0$

  $P_{x+1} = \mu z \leq (P_x!+1) . \bar{sg}(P_z(z)\cdot \mathcal{X}_{z>Px}(z))$

  \textbf{Note}: certainly $P_{x+1} \leq P_x!+1$, and this allows us to fix the bound.

  in fact call $p$ a prime in the decomposition of $p_x!+1 (\geq 2)$, therefore $p|p_x!+1$, certainly $p>p_x$, otherwise $p|p_x!$ and therefore $p|1$. Therefore $p_x < p_{x+1} \leq p$

\item note that $(x)_y = max \quad z$ s.t. $p_y^z|x = \\
  min  \quad  z$ s.t. $p_y^{z+2}\not|x = \mu z\leq x . \lnot div((p_y)^{z+1},x)$
\end{enumerate}

\subsection{Exercizes}
Prove that the following functions are computable:

\begin{enumerate}[label=\alph*)]
\item $\floor{\sqrt{x}}$
  
  $\floor{\sqrt{x}} = max\, y\leq x \quad y^2 \leq x\\
  = min \, y \leq x \quad (y+1)^2 > x\\
  \mu y\leq x. ((x+1)-(y+1)^2)$
\item $lcm(x,y)\\
  mxm(x,y) = \mu < \leq x\cdot y . (x|z $ and $ y|z)\\
  = \mu z \leq x\cdot y. \bar{sg}(dic(x,z)\cdot div(y,z))$
\item $GCD(x,y)$
  
  Certainly $GCD(x,y)\leq min\{x,y\}$ and it can be characterized using the minimum number that can be subtracted to $min\{x,y\}$ to obtain the divisor of $x,y$

  $GCD(x,y)\leq min(x,y)-\mu z\leq min(x,y).(1\dotdiv div(min(x,y)-z,x)\cdot div(min(x,y)-z, y))$
\item number of prime divisors of $x$

  $\sum_{z\leq x} pr(z)\cdot div(z,x)$
\end{enumerate}

\section{Encoding of couples (and n-tuples)}

Let us see an encoding in $\nat$ of couples (and n-tuples) of natural numbers that will later be proved useful for some considerations on recursion (and for the furute\dots)

Let us define as a \textbf{couple encoding}:

$\pi: \nat^2\rightarrow\nat\\
\\
\pi(x,y) = 2^x(2y+1)-1$

Notice that $\pi$ is bijective and effective (computable).

The inverse can be characterized in terms of two computable functions that give the first and second component of a natural $x$ seen as couple:

$\pi^{-1}:\nat\rightarrow\nat^2\\
\\
\pi^{-1}(x) = (\pi_1(x),\pi_2(x))$

where $\pi_1(x) = (x+1)_1\\
\pi_2(x) = (\frac{x+1}{2\pi_1(x)}-1)/2$

(the division is $qt(\_,\_)$)

It can be generalized to an encoding of $n$-tuples:

$\pi^n: \nat^n\rightarrow\nat \quad n\geq2$

defining inductively

$\pi^2 = \pi\\
\\
\pi^{n+1}(\vec{x},y) = \pi(\pi^n(\vec{x},y)) \quad \vec{x} \in \nat^n, y \in \nat$

and correspondingly we can define the projections $pi_j^n:\nat\rightarrow\nat^n$

\subsection{Considerations on recursion}

The Fibonacci function is defined by:

$ fib(0) = fib(1) = 1\\
\\
fib(n+2) = fib(n) + fib(n+1) $

This isn't exactly a definition by primitive recursion, given that $f(y+2)$ is defined in terms of $f(y)$ as well as $f(y+1)$, it does not respect the schema\dots

We can show that $f$ is computable resorting to the encoding and defining:

$g:\nat\rightarrow\nat\\
g(y) = \pi(f(y),f(y+1))$

therefore $g$ can be defined by primitive recursion:

$g(0) = \pi(f(0),f(1)) = \pi(1,1)\\
g(y+1) = \pi(f(y+1),f(y+2)) = \pi(f(y+1),f(y)+f(y+1))\\
= \pi(\pi_2(g(y)), \pi_1(g(y)) + \pi_2(g(y)))$

so $g$ is computable.

$f(y) = \pi_1(g(y))$ computable.

In general we could have a function $f$ defined using $k$ previous values

$\begin{cases}
  f(0) = c_0   \\
  f(k-1) = c_k \\
  f(y+k) = h(f(y),\dots,f(y+k-1))
\end{cases}$

with $h:\nat^k\rightarrow\nat$ computable.

One can proceed like before and define

$g:\nat\rightarrow\nat\\
\\
g(y) = \pi^k(f(y),\dots,f(y+k-1))$

the $g$ function can be defined by primitive recursion

$g(0) = \pi^k(c_0,\dots,c_{k-1})\\
\\
g(y+1) = \pi^k(f(y+1),\dots,f(y+k-1),f(y+k))\\
\\
f(y+1) = \pi_2^k(g(y))\\
\\
f(y+k-1) = \pi_k^k(g(y))\\
\\
f(y+k) = h(f(y),\dots,f(y+k-1)) = h(\pi_1^k(g(y)),\dots,\pi_k^k(g(y)))\\
\\
= \pi^k(\pi_2^k(g(y)),\dots,\pi_k^k(g(y)),h(\pi_1^k(g(y)),\dots,\pi_k^k(g(y))))$

g is computable, so $f(y) = \pi_1(g(y))$ is computable.

\section{Minimalization}
The operators to manupulate functions seen until now, generalized composition and primitive recursion, starting from \textbf{total} functions return total functions. Another essential operator, in a way (that will be explained) which allows to build partial functions is the \textbf{unlimited minimalization} operator.

Similar to the limited minimalization, but \dots, given $f(\vec{x},y)$ not necessarily total, it defines more or less the following function:

$\mu y . f(\vec{x},y) $ = minimum $y$ s.t. $f(\vec{x},y) = 0$.

But there are two problems:
\begin{enumerate}
\item if there is no $y$ s.t. $f(\vec{x},y) = 0 \uparrow$
\item if before finding a $y$ s.t. $f(\vec{x},y) = 0$ happens that $f(\vec{x},z)\uparrow$ the minimalization is $\uparrow$
\end{enumerate}

intuitive if we think about the obvious algorithm to calculate it: start from 0, $f(\vec{x},0) = 0$? if yes then $out(0)$, otherwise $f(\vec{x},1) = 0$? until $f(\vec{x},y) = 0$.

\textbf{Definition}: Let $f\nat^{k+1}\rightarrow\nat$ be a function. Then the function $h:\nat^k\rightarrow\nat$ defined through \textbf{unlimited minimalization} is:

\begin{equation*}
  h(\vec{x}) = \mu y. f(\vec{x},y) = \begin{cases}
    $least $ z$ s.t. $ & \begin{cases}
      f(\vec{x},z) = 0 \\
      f(\vec{x},z)\downarrow \quad f(\vec{x},z') \not= 0 \quad $ for $ z<z'
    \end{cases} \\
    \uparrow           & $ otherwise $
  \end{cases}
\end{equation*}

\section{Closure of $\mathcal{C}$ for minimization}

\textbf{Theorem}: Let $f:\nat^{k+1}\rightarrow\nat$ a computable function (not necessarily total). Then $h:\nat^k\rightarrow\nat$ defined by $h(\vec{x}) = \mu y. f(\vec{x},y)$ is computable.

\textbf{proof}: Let $F$ be a program in standard form for $f$.

\textbf{idea:} for $z=0,1,2,\dots$ we calculate $f(\vec{x},z)$ until we find a zero\dots

We need to save the argument $\vec{x}$ in a zone that's not used by the program $F$.

$m = max\{\rho(F),k+1\}$

So the program for $h$ is obtained as follows:

$\begin{tabu}{cccccccc}
  1                            & \dots                  & k                             & \dots                  & m+1 & \dots & m+k & m+k+1 \\
  \hline
  \multicolumn{3}{|c}{\vec{x}} & \multicolumn{1}{|c|}{} & \multicolumn{3}{|c|}{\vec{x}} & \multicolumn{1}{c|}{z}                             \\
  \hline
\end{tabu}$

$\begin{tabu}{lll}
  & T([1,k],[m+1,m+k])              & saves \vec{x}                                       \\
  LOOP: & F[m+1,\dots,m+k+1\rightarrow 1] & f(\vec{x},z) \rightarrow R_1                        \\
  & J(1,m+k+2,END)                  & m+k+z $ contains $ 0' \Rightarrow f(\vec{x},z) = 0? \\
  & S(m+k+1)                        & z=z+1                                               \\
  & J(1,1,LOOP)                     &                                                     \\
  END:  & T(m+k+1,1)
\end{tabu}$

\textbf{Note} $F$ may not terminate\dots it is correct! The entire program doesn't terminate and $\mu$ is undefined!

\textbf{Note:} \textbf{While} loop implemented with \textbf{goto}.

\textbf{Observation}: The $\mu$ operator allows us to obtain \textbf{non total} functions starting from total functions.

\textbf{Example}: Given $f(x,y) = |x-y^2|\\
\\
\mu u. f(x,y) = \begin{cases}
  \sqrt{x} & x $ is a perfect square $ \\
  \uparrow & $ otherwise $
\end{cases}$

\textbf{Exercise}: Let $f:\nat\rightarrow\nat$ be computable, total and injective. The the \textbf{inverse} $f^{-1} = \begin{cases}
  y        & f(y) = x                \\
  \uparrow & \not\exists y. f(y) = x
\end{cases}$

is computable. In fact, in our hypothesis' $f^{-1}(x) = \mu y. |f(y)-x|$

\textbf{Note} This proof uses in an essential way the fact that $f$ is total, but the result is completely general \dots

Intuitively, when $f$ is not total, to find $f^{-1}(x)$ we consider a program $P$ for $f$ and I execute it as follows:
\begin{itemize}
\item 0 steps of the program on argument 0
\item 1 step on 0
\item 0 steps on 1
\item 2 steps on 0\\
  \dots
\end{itemize}

in a dove tail execution pattern.

Every time for a certain number of steps $k$ on argument $y$, the program \textbf{terminates} we check the output $f(y)$, if $f(y) = x$ we stop, otherwise we continue.

Informal \dots we will see how to formalize it.

\textbf{Exercize}: Prove that the following function is computable.

$f(x,y) = \begin{cases}
  \frac{x}{y} & y\not= 0 \land y|z \\
  \uparrow    & $ otherwise $
\end{cases}$

First attempt:

$f(x,y) = \mu z. |yz - x|$

It's almost ok\dots except for the fact that $f(0,0) = 0$ instead of $\uparrow$

The problem can be solved with a (classic) trick:

$f(x,y) = \mu z. (|yz-x| + \mathcal{X}_{x=0\land y=0}(x,y))$

\textbf{Exercise}: All the functions with finite domain are computable, meaning let $\theta: \nat\rightarrow\nat \quad dom(\theta)$ finite $ \Rightarrow \theta$ computable.

Formulated for unary functions for the sake of notation, but easily generalizable.

\textbf{proof}: Let $\theta:\nat\rightarrow\nat$ a finite domain:

$\theta=\{(x_1,y_1),\dots,(x_n,y_n)\}$

meaning:

$\theta(x) = \begin{cases}
  y_1      & x=x_1         \\
  \dots                    \\
  y_n      & x=x_n         \\
  \uparrow & $ otherwise $
\end{cases}$

therefore

$\theta(x) = \sum_{i=1}^{n}y_i \cdot \bar{sg}(|x-x_i) + \mu z. (\prod_{i=1}^{n}|x-x_i|)$

The minimalization is needed only to make the function $\uparrow$ when $x\not= x_1,\dots,x_n$, it is 0 otherwise.

\chapter{Other approaches to computability}
We already observed that the URM machine is just one of the many possible computational models that allow us to formalize the notion of computable functions.

We could have used:
\begin{itemize}
\item Turing machine
\item Canonical deduction systems of Post
\item $\lambda$-calculus of Church
\item Partial recursive functions of Gödel-Kleene
\end{itemize}

All of these approaches define the \textbf{same class of computable functions}, leading to the

\textbf{Church-Turing thesis}: a function is computable through an effective procedure 
if and only if it is URM-computable

Now, we introduce another formalism for the definition of computable functions, the set $\mathcal{R}$ of \textbf{partial recursive functions} of Gödel-Kleene and prove that it is equivalent to the URM, meaning it defines the same class of functions: $\mathcal{R} = \mathcal{C}$.

\section{Partially recursive functions}

\begin{definition}
  The class $ \mathcal{R} $ of \textbf{partially recursive functions} is the least class of partial functions on the natural numbers which contains
  \begin{enumerate}[label=(\alph*)]
    \item zero function;
    \item successor;
    \item projections
    \end{enumerate}
    
    and \textbf{closed} under
    \begin{enumerate}
    \item composition;
    \item primitive recursion;
    \item minimalisation.
    \end{enumerate}
\end{definition}

It is a well given definition.

\begin{definition}[Rich class]
    A class of functions $\mathcal{A}$ is said to be \textbf{rich} if it includes (a),(b) and (c) and it's closed under (1), (2) and (3).
\end{definition}

$\mathcal{R}$ is rich and for all $\mathcal{A}$, we have $\mathcal{R}\subseteq\mathcal{A}$

We observe that the property of being ``rich class'' is \textbf{closed for intersection}:

Let $\{\mathcal{A}_i\}i\in I$ a family of rich classes, then $\bigcap_{i\in I}\mathcal{A}_i$ rich.

And finally we define:

\textbf{Definition} The set of the partially recursive functions is:

$\mathcal{R} = \bigcap_{\mathcal{A} \subseteq \bigcup_k \nat^k\rightarrow\nat
  \land
  \mathcal{A}rich} \mathcal{A}$

\textbf{Note} $\mathcal{R}$ admits an inductive characteristic\\
$\mathcal{R}_0 = \{a,b,c\}$\\
$\mathcal{R}_{i+1} = \mathcal{R}_i \cup \{1,2,3 $ applied to $ R_i\}\\
R = \bigcup_i \mathcal{R}_i$

Another interesting class, on which we will come back:

\textbf{Definition} (Primitive recursive functions).

The class of the primitive recursive functions is the least class of functions $\mathcal{PR}\subseteq \bigcup_k\nat^k\rightarrow\nat$ that contains (a),(b) and (c) and is closed for (1) and (2).

\begin{theorem}[$\mathcal{R} = \mathcal{C}$]\label{reqc} The partially recursive
  functions coincide with those URM-computable
  
  \begin{proof}
    $ \mathcal{R} \subseteq \mathcal{C} $ because $\mathcal{R}$ is the
    least rich class, the other is a rich class
    $\Rightarrow \mathcal{R}\subseteq\mathcal{C}$.

    Let us now consider $ \mathcal{C} \subseteq \mathcal{R} $. Let
    $ f:\nat^k\rightarrow\nat \in \mathcal{C} $ be a function and show
    that $ f \in \mathcal{R} $. We know that $ \exists P \in URM $
    programs $ f_P^{(k)} = f$.

    Considering the following functions (dependant on $P$)

    With $ C_p^1(\vec{x}, t) $ we indicate contents of register 1 after $t$ steps of $ P(\vec{x}) $. A computation step is the execution of an instruction.

    It is understood that if $P(\vec{x})$ terminates in less than $t$ steps, $ C_p^1(\vec{x}, t) $ gives the content of $R_1$ in the final configuration.

    With $ J_P(\vec{x},t) $ we indicate instruction to be executed at the $t+1$-th step (after $t$ steps) of $P(\vec{x})$ (program counter) If the program has already ended, it is worth 0.

    Clearly $C_p^1$ and $J_p$ are total functions (we will need this later\dots)

    If $ f(\vec{x})\downarrow $ then $ P(\vec{x})\downarrow $ after $ t_0 $ pass, $ t_0 = \mu t. J_P(\vec{x},t) \Rightarrow f(\vec{x}) = C_p^1(\vec{x},t_0) = C_P^1(\vec{x}, \mu t.J_P(\vec{x},t)) $.
    Otherwise, if $ f(\vec{x})\uparrow $ then $P(\vec{x})\uparrow$ too and $ \mu t.J_P(\vec{x},t)\uparrow $

    Therefore:
    $f(\vec{x}) = c_p^1(\vec{x}),\mu t.J_p(\vec{x},t)$

    % =======================================================

    If you knew that $ C_P^1, J_P \in \mathcal{R} $ then here it is shown that $ f \in \mathcal{R} $

    We prove that they are even in $ \mathcal{PR} $

    The idea of the proof is the following:

    \begin{itemize}
    \item we can work on sequences encodings taht rapresents the registers and program counter configuraiton
    \item we then manipulate such sequences  with the funcitons (\( p_x, q_t, \text{div}, \dots \) ) that we built by:
      \begin{itemize}
      \item composition
      \item primitive recursion
      \end{itemize}
    \item this way we obtain $C_p^1, J_p$ through primitive recursion
    \end{itemize}
    More precisely, the computation state is a tuple
    $\langle \vec{R}, t \rangle$ where $\vec{R}$ are the registers of the
    machine and $t$ is the current instruction.

    To a register configuration in which a finite number of registers
    contains a value other than 0 can be encoded with

    \begin{center}
      $\text{cod}(\vec{R}) = \prod\limits_{i \geq 1}p_i^{r_i}$
    \end{center}

    where just a finite number of factors is $\neq 1$. At this point,
    given $x \geq 1$ interpreted as $\text{cod}(\vec{R})$

    \begin{center}
      $r_i = (x)_i$
    \end{center}

    Using this encoding, we can consider the function $C_p(\vec{x},t)$
    (the registers' configuration after t steps of $P(\vec{x})$) as a
    function $C_p : \nat^{k+1} \rightarrow \nat$. This way we can define:

    \begin{center}
      $\sigma_p(\vec{x},t) = \langle C_p(\vec{x},t), J_p(\vec{x},t) \rangle$
    \end{center}

    the state of the computation of $P(\vec{x})$ after $t$ steps. And
    using the encoding function for the $\pi$ couples we can view
    $\sigma_p$ as:

    $\sigma_p : \nat^{k+1} \rightarrow \nat$

    $\sigma_p(\vec{x}, t) = \pi(C_p(\vec{x},t), J_p(\vec{x},t))$

    and we can define it with the primitive recursion:

    $\sigma_p(\vec{x}, 0) = \pi(\prod\limits_{i=1}^k p_i^{x_i}, 1)$

    $\sigma_p(\vec{x}, t+1) = \pi(C_p(\vec{x},t+1), J_p(\vec{x},t+1))$

    with

    \[
      C_p(\vec{x},t+1) = \begin{cases}
        qt(p_n^{(C_p(\vec{x},t))_n}, C_p(\vec{x},t)) & \quad \text{if }1 \leq J_p(\vec{x},t) \leq s \; \And \; I_{J_p(\vec{x},t)} = z(n) \\
        p_n \cdot C_p(\vec{x},t) & \quad \text{if }1 \leq J_p(\vec{x},t) \leq s \; \And \; I_{J_p(\vec{x},t)} = s(n) \\
        qt(p_n^{(C_p(\vec{x},t))_n}, C_p(\vec{x},t)) \cdot p_n^{(C_p(\vec{x},t))_m} & \quad \text{if }1 \leq J_p(\vec{x},t) \leq s \; \And \; I_{J_p(\vec{x},t)} = T(m,n) \\
        C_p(\vec{x}, t) & \quad \text{otherwise}
      \end{cases}
    \]

    \[
      J_p(\vec{x}, t+1) = \begin{cases}
        J_p(\vec{x}, t) + 1 & \quad \text{if } 1 \leq J_p(\vec{x}, t) < s \And I_{J_p(\vec{x}, t)} = z(n), s(n), T(m,n) \\
        & \quad \quad \text{ or } J(m,n,q) \text{ with } (\sigma_p(\vec{x}, t))_m \neq (\sigma_p(\vec{x}, t))_n \\
        q & \quad \text{if } 1 \leq J_p(\vec{x}, t) < s \And I_{J_p(\vec{x}, t)} = J(m,n,q) \text{ with } q\leq 5 \\
        & \quad \quad \And (C_p(\vec{x}, t))_m = (C_p(\vec{x}, t))_n \\
        0 & \quad \text{otherwise}
      \end{cases}
    \]

    this way we define by primitive recursion $\sigma_p$, even if not in
    detail, starting from functions in $\mathcal{R}$.

    $\sigma_p \in \mathcal{R}$

    \textbf{Observation:} all the functions we used are in  $\mathcal{PR} \Rightarrow \sigma_p \in \mathcal{PR}$

    $C_p^1(\vec{x}, t) = (\pi_1(\sigma_p(\vec{x}, t)))_1 \in \mathcal{R}$

    $J_p(\vec{x}, t) = \pi_2(\sigma_p(\vec{x}, t)) \in \mathcal{R}$

    so $f$, defined by composition and minimization by $C_p^1, J_p$ is in $\mathcal{R}$ as we wanted.

  \end{proof}
\end{theorem}

\chapter{Pimitive recursive functions}
Let's consider again the class of primitivi recursive functions.

\textbf{Definition} (primitive recursive funcitons).  The class of
primitive recursive functions is the smallest set
$\mathcal{PR} \subseteq \bigcup\limits_k(\mathbb{N}^k \rightarrow
\mathbb{N})$ containing.

\begin{enumerate}[label=(\alph*)]
\item zero
\item successor
\item projections
\end{enumerate}

and closed under

\begin{enumerate}[label=(\arabic*)]
\item composition
\item primitive recursion
\end{enumerate}

There are many interesting point of $\mathcal{PR}$. For example the
fact that primitive recursion are similar to \texttt{for} loops, while
minimalization is similar to \texttt{while} loops. This fact can be
formalized into a new variant on the URM, with \textit{structured
  programs}, where the jump instruction is substituted with
\texttt{for} and \texttt{while} loops. We'll call this machine
$\text{URM}_{\text{for,while}}$ (This corresponds to consider the
subset of URM programs where the jump instructions are used in a
``disciplinated'' way).

We can deonstrate that this model has the same expressive power of the
URM model. This means that the class $\mathcal{C}_{\text{for,while}}$
fo functions computable in this model is

\[
  \mathcal{C}_{\text{for,while}} = \mathcal{C} = \mathcal{R}
\]

while the class of functions computable using only the \texttt{for}
construct are partial recursive functions:

\[
  \mathcal{C}_{\text{for}} = \mathcal{PR}
\]
So studing the relation between $\mathcal{R}$ and $\mathcal{PR}$ is
the same as studying the relation between the expressive power of
\texttt{for} and \texttt{while} constructs.  We know that many
``arithmetic'' functions, like Pr$(x), (x)_y,$ qt, mcm($x,y$),
$x^y, \dots$ are in $\mathcal{PR}$ and that it is closed for sum,
product and minimalization (all of them limited). This class is very
extended, but it does not contain all computable functions, in oter
words $\mathcal{PR} \subsetneq \mathcal{R}$, because $\mathcal{PR}$
functions are always total (based on an inductive caratheristic,
$\mathcal{PR}$ functions are obtainable for base total functions by
composition and primitive recursion).

One could think that $\mathcal{PR}$ includes all total reursive
functions, in other words if Tot is the set of all total functions:
$\mathcal{PR} = \mathcal{R} \cap \text{Tot}$ [Hilbert, 1926].

This is also false
($\mathcal{PR} \subsetneq \mathcal{R} \cap \text{Tot}$). In fact even
if we restrain ourselves to total functions (programs that always
terminates) the \texttt{while} construct is essential.

\section{Ackermann function}
$ \psi: \nat^2 \rightarrow \nat $ defined as:

$ \psi(0,y) = y+1 $

$ \psi(x+1,0) = \psi(x,1) $

$ \psi(x+1,y+1) = \psi(x, \psi(x+1, y)) $

Formally, $\psi$ is the minimum fixed point of the operator associated
to the recursive function, on the cpo % not sure about this
of partial functions
$T: (\mathbb{N}^2 \rightarrow \mathbb{N}) \rightarrow (\mathbb{N}^2
\rightarrow \mathbb{N})$. More intuitively, the scheme univocally
guesses a function, because the value $\psi(x,y)$ is always defined
based on ``smaller'' values of $\psi$ itself. But what does it mean
``smaller''?

$\psi(x+1,0) = \psi(x,1)$; here the first component
diminishes. $\psi(x+1,y+1)=$ at first we calculate $\psi(x+1, y)$,
here the second component diminishes, and then $\psi(x,u)$. Whatever
$u$ is, the first component is smaller.

We can see that the arguments diminishes in a \emph{lexicographical}
order on $\mathbb{N}^2$

with $(\mathbb{N}^2, \leq_{lex} )$, we have that $(x,y) \leq (x', y')$
if $(x < x') \wedge (x=x'$ and $y \leq y')$, and so
$( \mathbb{N}^2, \leq_{lex} )$ is \emph{well ordered}, in the sense
that it does not allow for infinite descending sequences.

\paragraph{Parenthesis on partial orders}
\begin{definition}
  Let $(D, \leq)$ be a partial order. It is
  \begin{description}
  \item[well ordered] if each non-void subset has the minimum element:
    $$\forall X \subseteq D, X \neq \emptyset \quad \exists \min D$$
  \item[well based] if each non-void subset has a minimal element
    $$ \forall X \subseteq D, X \neq \emptyset \Rightarrow \exists d \in X \text{ minimal}$$
  \end{description}
\end{definition}

\textbf{Observation:} $D$ is well formed $\Rightarrow$ $D$ is well
based.

\textbf{Observation:} if $D$ is well based it does not allow for
infinite descending sequences:
\[
  d_0 > d_1 > d_2 > \dots > d_n > d_n+1 \dots
\]
\newcommand{\conf}{\text{conf}} This fact can be useful when dealing
with termination problems. If we can conclude that the set of
configurations is well based, to end our proof we just need to prove
that for each step \( \conf _i \rightarrow \conf_{i+1} \) and
\( \conf _{i+1} < \conf _i \). This way our computation follows a descending sequence of values, necessarely finite.

In fact, if we think the calculation of $\psi$ is based on the
calculation of $\psi$ with smaller values necessarely at some point it
will reach for sure the case $\psi(0,y) = y + 1$, terminating.

\textbf{Note:} $(\nat^2, \leq_{lex})$ is \emph{well ordered}

(given $\emptyset \neq X \subseteq \nat^2$ we can identify
$x_0 = \min\{x \; | \; \exists y.(x,y) \in X\}$ and
$y_0 = \min \{ y \; | \; (x_0,y) \in X\}$ we can say that
$\min X = (x_0, y_0)$. In this way we can prove that the product of
two well ordered sets is ordered.)

The order is also total (\emph{left as an exercise})
\begin{itemize}
\item ``infinitely many copies of $\nat$, one after the other''
\item eache element has a successor, but not always a predecessor
\item following the path back we have jumps of infinite length
\end{itemize}

Associated with a well based order there is an induction principle:

\paragraph{\textbf{Complete (strong) induction}}

Let $(D, \leq)$ be a well based order, and let $P \subseteq D$ a
property. If
\[
  \forall d \in D \; . \; (\forall d' < d \quad P(d)) \Rightarrow P(d') \Rightarrow P(d)
\]

To prove that $P$ is valid on $D$ we have to prove that
\begin{itemize}
\item is valid on minimal elements
\item for each other element $d$, if it holds on elements smaller than
  $d$, then it holds on $d$
\end{itemize}

The proof is left as an exercise.

\paragraph{\textbf{Back to the Ackerman function}}

We will see that $\psi$:
\begin{enumerate}
\item It is total;
\item $ \psi \in \mathcal{C} = \mathcal{R} $
\item $ \psi \not \in \mathcal{PR} $
\end{enumerate}

% This part was not on the original notes but the prof added it so im
% keeping it

% We prove by induction on pairs $ (x',y') \leq_{lex} (x,y) $
% lexicographically ordered.

% \begin{enumerate}
% \item We proceed by case. If x$ = 0$ then $y + 1$ holds then the function is defined;
% \item $ x>0, y=0 $, then it holds $ \psi(x-1,1) $ which is def. by inductive hypothesis;
% \item $ x>0, y>0 $, then $ \psi(x, \psi(x+1, y)) $ defined by inductive hypothesis, both inside and outside;
% \end{enumerate}

% So the function is total.

\newcommand{\nsqlex}{\( (\nat ^2, \leq_{lex})\) }
\begin{theorem}[$\psi$ is total]
  \[\forall (x,y) \in \nat^2 \quad \psi (x,y) \downarrow \]
  \begin{proof}
    We can prove it with complete induction on \nsqlex:

    \begin{itemize}
    \item \textbf{base elements}. The truth is that \nsqlex has just
      one base element $(0,0)$ and $\psi(0,0)=1\downarrow$
    \item \textbf{inductive step}.
      \[\forall (x,y) \in \nat^2 ((\forall (x', y') \in \nat^2 \;
        (x',y') \leq_{lex} (x,y) \; \psi(x,y)\downarrow ) \Rightarrow
        \psi(x,y)\downarrow)\] we have then 2 subcases:
      \begin{enumerate}
      \item[$(x=0)$] \(\psi(0,y) = y + 1 \downarrow\)
      \item[$(x>0)$] again two subcases
        \begin{enumerate}
        \item[$(y=0)$] $\psi(x,0) = \psi(x-1,1) \downarrow $ for
          inductive hypothesis, since $(x-1,1) \leq_{lex} (x,0)$
        \item[$(y>0)$] $\psi(x,y) = \psi(x-1,\psi(x,y-1)) \downarrow $
          and $\psi(x,y-1) \downarrow$ for inductive hypothesis. If we
          call $u = \psi(x,y-1)$ we have that
          $\psi(x,y) = \psi(x-1,u) \downarrow $ for inductive
          hypothesis
        \end{enumerate}
      \end{enumerate}
    \end{itemize}
  \end{proof}
\end{theorem}

\begin{exercise}
  Given a box with an arbitrary number of balls in it, each one with a
  number in $\nat$ do the following:
  \begin{itemize}
  \item extract a ball
  \item substitute the extracted ball with an arbitrary number of
    balls, each one with a label lower than the extracted one.
  \end{itemize}

  Prove that this process always terminates.
\end{exercise}

\begin{theorem}[$\psi$ is computable]
  \[\psi \in \mathcal{C} = \mathcal{R}\]
  \begin{proof}
    We can use the Church-Turing thesis to prove it: the computation
    of $\psi(x,y)$ is always reduced to the computation of $\psi$ on
    smaller input values, and for the base case ...

    To do so, we'll need the definition of a valid set. Intuitively a
    set $S \subseteq \nat$ is valid if

    \begin{itemize}
    \item \((x,y,z) \in S \quad \Rightarrow \quad z = \psi(x,y)\)
    \item \((x,y,z) \in S \quad \Rightarrow \quad S\) contains all the
      tuples \((x', y', \psi(x',y'))\) needed in order to calculate
      $\psi(x,y)$
    \end{itemize}

    \textbf{example:}
    $\psi(1,1) = \psi(0,\psi(1,0)) = \psi(0,\psi(0,1)) = \psi(0,2) = 3$

    $\Rightarrow S = {(1,1,3), (0,2,3), (1,0,2), (0,1,2)}$

    Formally:

    \begin{definition}[valid $S \subseteq \nat^3$]
      Let $S$ be a set of tuples such that $S \subseteq \nat^3$. We say
      that $S$ is \emph{valid} if:
      \begin{itemize}
      \item \((0,y,z) \in S \quad \Rightarrow \quad z = y+1\)
      \item \((x+1,0,z) \in S \quad \Rightarrow \quad (x,1,z) \in S\)
      \item
        \((x+1,y+1,z) \in S \quad \Rightarrow \quad \exists u . (x+1,
        y, u) \in S \wedge (x,u,z) \in S \)
      \end{itemize}
    \end{definition}

    We can prove that
    \(\psi(x,y) = z \Leftrightarrow \exists S \subseteq \nat^3\) valid
    and finite s.t. \((x,y,z) \in S\) by complete induction on
    $(x,y)$, leveraging the fact that the validity of a set is
    preserved under union (left as an exercise).

    \newcommand{\vnu}{\text{Val}(\nu)}

    A tuple $(x,y,z)$ can be encoded with an integer using the
    encoding function
    \[\Pi^3 \, : \, \nat^3 \rightarrow \nat \quad \quad (\Pi_i^3 \, : \,
      \nat \rightarrow \nat \text{ are the projections})\]. This way a
    set of tuples becomes a set of natuarl numbers
    $\{x_1, \dots, x_n\}$ that we can encode injectively:
    \[\{x_1, \dots, x_n\} \mapsto P_{x_1}, \dots, P_{x_n}\] Now we
    have $\nu$ the number that rapresents the set of tuples
    $S_\nu$. We can verify that
    \[(x,y,z) \in S_\nu \Longleftrightarrow P_{\Pi(x,y,z)} \; | \;
      \nu \] and the predicate $\vnu=$``$\nu$ encodes a
    set of valid tuples'' is decidable.

    In fact $\vnu$ is true if and only if:
    \begin{itemize}
    \item \(\forall i \leq \nu \quad (\nu)_i = 1\)
    \item{
        \( \forall \omega \leq \nu \quad P_{\omega \; | \; \nu}\) \\
        \[\Rightarrow \begin{cases}
            \Pi_1(\omega) = 0 \quad & \Rightarrow
            \quad \Pi_3(\omega) = \Pi_2(\omega) + 1 \\
            \Pi_1(\omega) > 0 \quad & \Rightarrow \quad
            \begin{cases}
              \Pi_2(\omega) = 0 & \Rightarrow
              \Pi(\Pi_1(\omega),0,\Pi_3(\omega)) \in S_\nu \\
              \Pi_2(\omega) > 0 & \Rightarrow
              \exists u \leq \omega \text{ s.t. } \\
              & \quad \Pi(\Pi_1(\omega), \Pi_2(\omega)-1,u) \in S_\omega \\
              & \quad \Pi(\Pi_1(\omega)-1, u,z) \in S_\omega 
            \end{cases}
            \\
          \end{cases}
        \]
      }
    \end{itemize}

    And the associated caratheristic function
    \[\chi_{\text{Val}} \in \mathcal{PR}\]

    We can also verify that:

    \[
      R(x,y,z) = \begin{cases}

        \chi_{\text{Val}(\omega)} & \text{if $\omega$ encodes for some
          valid $S$ that contains $(x,y,z)$ for some $z$} \\

        0 & \text{otherwise}
        
      \end{cases}
    \]

    \[
      R(x,y,z) = \chi_{\text{Val}(\omega)} \cdot \text{sg} (\omega + 1
      \dot{-} \mu z \leq \omega . \text{div}(P_{\Pi(x,y,z)}, \omega)
    \]

    this way we can write the Ackermann function as

    \[
      \psi(x,y) = \mu (z,y) \; \cdot \; \overline{\text{sg}}(R(x,y,\omega)
      \cdot \text{div}(P_{\Pi(x,y,z)}, \omega))
    \]

    So since it is computable,

    \[
      \psi \in \mathcal{R} = \mathcal{C}
    \]

  \end{proof}
\end{theorem}

\begin{theorem}
  \[ \psi \notin \mathcal{PR} \]
  \begin{proof}
    The fact that the definition of $\psi$ includes a recursion (and
    does not respect the scheme of primitive recursive functions for
    this reason) does not allow us to end our proof immediatly.

    The proof of the fact that $\psi$ is not a primitive recursive
    function is done by showing that $\psi$ ``grows'' faster than
    every function in $\mathcal{PR}$. We altread saw how
    \begin{itemize}
    \item we obtain the sum from the successor
    \item we obtain the product from the sum
    \item we obtain the exponential from the product
    \end{itemize}
    each one for primitive recursion, nested each time more.

    The idea of the Ackermann function is that to push to its limit
    this procedure and capture each possible level in a unique
    function that won't be possible to compute with a finite number of
    nested primitive recursions.

    In fact, by calling \[\psi_x(y) = \psi(x,y)\] we have
    that
    \[
      \psi_{x+1}(y) = \psi_x(\psi_{x+1}(y-1)) =
      \psi^2_{x}(\psi_{x+1}(y-2)) = \dots = \psi_x^{y+1}(1)
    \]

    \(\psi_0(y) = y+1 = \) succ$(x)$

    \(\psi_1(y) = \psi_0^{y+1}(1) = \text{ succ}^{y+1}(1) = y+2\)
    
    \(\psi_2(y) = \psi_1^{y+1}(1) = 2y+3\)

    \(\psi_3(y) = \psi_2^{y+1}(1) = 2^{y+3}-3\)

    e.g.

    \(\psi_0(1) = 2, \quad \psi_1(1) = 3, \quad \psi_2(1) = 5, \quad
    \psi_3(1) = 13, \quad \dots\)

    Intuitively, if x grows in $\psi_x$ so does the level of nesting
    in the functions, wich is equivalent to say that we need more and
    more \texttt{for} loops, and since $x$ can grow to infinity and we
    can't keep nesting \texttt{for} loops to infinity, we need a
    \texttt{while} loop. More precisely we can show that for each
    function $f \in \mathcal{PR}$ if we take $j$ the maximum number of
    nested \texttt{for} cycles, holds asymptotically that
    \[
      P(x_1, \dots, x_k)\downarrow \quad \text{with anumber of steps }
      < \psi_{j+1}(\max\{x_1, \dots, x_k\})
    \]

    This way we see that $\psi_j$ gives a bound to the time complexity
    as well to the number of increment operations, so
    \[f(\vec{x}) < \psi_{j+1}(\max(\vec{x})\]

    But for absurd, if $\psi \in \mathcal{PR}$, we can call $j$ the
    maximum number of nested \texttt{for} cycles in a program that
    computes it, and so \[\psi(x,y) < \psi_{j+1}(\max\{x,y\})\] should
    hold. In particular with $x=y=k>j+1$ big enough we have
    that \[\psi(k,k) < \psi_{j+1}(k) < \psi_k(k) = \psi(k,k)\] wich is
    absurd.
  \end{proof}
\end{theorem}

\newpage

\textbf{Observation:} initially Gödel and Kleene had studied a class
of functions $\mathcal{R}_0$ called $\mu$-recursive. This class
contained
\begin{enumerate}[label=\alph*]
\item zero
\item successor
\item projections
\end{enumerate}

and was closed with respect to

\begin{enumerate}
\item composition;
\item primitive recursion;
\item minimalization, restricted to the case in wich the function that
  produces it is \emph{total}.
\end{enumerate}

Is clear that \[\mathcal{R}_0 \subsetneq \mathcal{R}\] and the
inclusion is strict, since:
\begin{itemize}
\item the functions in $\mathcal{R}_0$ are total;
\item some functions in $\mathcal{R}$ are partial.
\end{itemize}

Also \[\mathcal{R}_0 \subseteq \mathcal{R} \cap \text{Tot}\] but is
not obvious that the equality holds. In fact a function
$f \in \mathcal{R} \cap \text{Tot}$ can be total, but obtained trough
minimalization of partial functions. For example:
\[
  f(x,y) = \begin{cases}
    x+1 & x<y \\
    0 & x=y \\
    \uparrow & x>y
  \end{cases}
\]

\[
  \mu y . f(x,y) = \lambda x . x 
\]

Generally speaking is not obvious that if $f(x,y)$ is partial and
$\mu y . f(x,y)$ is total then \[\mu y . f(x,y) \in \mathcal{R}_0\]
but it is true!

\begin{theorem}
  \[\mathcal{R}_0 = \mathcal{R}\cap \text{Tot}\]
  \begin{proof}
    \begin{enumerate}
    \item[$(\subseteq)$] obvious.
    \item[$(\supseteq)$] Let $f \in \mathcal{R} \cap \text{Tot}$ so
      $f \in \mathcal{C}$ because of the proof of $\mathcal{R = C}$ so
      we can observe that
      \[f(\vec{x}) = C_p^1 ( \vec{x} , \mu t . J_p(\vec{x}, t))\]
      since composed of total functions, $f$ is total.
    \end{enumerate}
  \end{proof}
\end{theorem}

\chapter{Enumeration of programs}

Review: $ A, B $ sets, $ |A| = |B| $ if $ \exists f:A\rightarrow B $ biunivocal. Furthermore $ |A| \leq |B| $ if $ \exists f:A\rightarrow B $ injective or there exists $g$ opposite direction surjective.

$A$ is countable if $ |A| \leq |\nat| $, that is, $ \exists g: \nat \rightarrow A $ surjective.

An enumeration is without repetitions if in addition to being surjective it is also injective. Let's say that an enumeration is effective when it is computable or made from pieces that are computable (e.g. ennuple of results).

\textbf{Observation:} $ \nat^2 $, $ \nat^3 $ and $ \bigcup_{k\geq 1} \nat^k $ can be numbered effectively.

In particular the couples in $ \nat^2 $ can be encoded as $ \pi(x,y) = 2^x(2y+1)-1 $ which is computable. The inverse is $ \pi^{-1}(n) = (\pi_1(n), \pi_2(n)) $

The triple instead is a pair of a pair and an element. $ \upsilon (x,y,z) = \pi (x, \pi(y,z))$. The inverse is also obtained from the inverse of the first.

For lists we need an encoding $ \tau . \bigcup_{K \geq 1} \nat^k \rightarrow \nat $ we exploit the uniqueness of the prime numbers: $ \tau(a_1,\dots,a_k) = \Pi_{i=1}^k p_i^{a_i}$ where $ p_i $ is the $i$-th prime number.\\This, however, leads us to lose any zeros, since the encodings for (1,1) and (1,1,0) would be the same because the exponential function in 0 = 1.

So we use something that works: $ (\Pi_{i=1}^{k-1} p_i^{a_i}) \times p_k^{a_k+1} - 2$. For decoding we can proceed as usual, but limited minimization can be used.\\ Hint: $ max \{z \leq x . P(z)\} x - min\{\delta \leq x . P(x-\delta)\}$

But I don't just want the length, I also need the list of items. I need a function: \begin{equation*}
  a(x,i) = \begin{cases}
    (x+2)_i   & 1 \leq i \leq k-1 \\
    (x+2)_k-1 & i = k
  \end{cases}
\end{equation*}
And this is the inverse function of $\tau$. And these functions are computable recursive primitives.

To compute instructions: Let us take $ \mathcal{F} $ set of URM instructions, $ \mathcal{P} $ URM programs. Let's take function $ \beta:\mathcal{F}\rightarrow\nat $
\begin{itemize}
\item $ \beta(z(n)) = 4 \times (n-1) $;
\item $ \beta(s(n)) = 4 \times (n-1)+1 $;
\item $ \beta(t(m,n)) = 4 \times \pi(m-1,n-1)+2 $;
\item $ \beta(j(m,n,t)) = 4 \times \upsilon(m-1,n-1,t-1)+3 $;
\end{itemize}

The decoding of this monstrosity is obtained from the previous inverse functions applied on the basis of the dimension and the rest of the number.

\textbf{Example:} P = T (1,2); S (2); T (2,1) = $ \tau(10,5,6) = 2^{10} 3^5 5^{6+1} -2 $

The two-way function between programs and numbers has been demonstrated.

\begin{notation} given an effective enumeration (in our case the one defined previously) we say that $ \gamma(P) $ is the code of P (also called G\"{o} of the number), if $ \gamma(P) = n $ then $P$ is the $n$-th program.
\end{notation}

\begin{notation} $ \Phi_n^{(k)}: \nat^k\rightarrow\nat $ function of $K$ arguments computed by program $n$, that is, by program $ \gamma^{-1}(n) $, if $k = 1$ is is omitted. The function domain  is $ W_n^{k} = dom(\Phi_k^{(k)}) = \{\vec{x} | \Phi_k^{(k)})(\vec{x})\downarrow  \} \subseteq \nat^k$

  The function codomain: $ E^{(k)}_n = \Phi_k^{(k)}) = \{\vec{x} | \vec{x} \in W_n^{(k)} \} \subseteq \nat^k$
\end{notation}

So for example the program $ \Phi_{19439999998} = x+1 $, $ W_{19439999998} = \nat $, $ E_{19439999998} = \nat \setminus \{0\} $

Now we have an enumeration of all the unary computable functions that is \{$ \Phi_n $ . $ n \in \nat $ \} each function with infinite repetitions.

Remember we have indicated the computable functions of $k$ arguments as $ \mathcal{C} ^ {(k)} $, where $ | \mathcal{C} ^ {(k)} | \leq (=) | \nat | $ and therefore being $ \mathcal{C} = \bigcup_{K \geq 1} \mathcal{C} ^ {(k)} $ a union of countable sets it is still countable.

% places to put matrices to build: ``complex matrix''
\chapter{Cantor diagonalization method}

The diagonalization technique, in a general sense, allows to build an
object that differs from an enumerable infinity of similar objects.
The idea behind it is that when given an enumerable et of objects
(functions, numbers, ...) with structure $\{x_1, x_2, x_3, \dots \}$
we can build another object $x$ based on the structure itself with the
same nature of all the other objects, but different from each of them,
by making it ``differ from $x_n$ on $n$''.

\begin{example}[$|2^\nat| > |\nat|$]
  This is the original use of the Cantor method, from Cantor himself,
  founder of the modern theory of sets, to prove that there are
  various ``levels of infinity''.
  \begin{proof}
    Lets suppose that $|2^\nat| = |\nat|$. This means that exists an
    enumeration of $2^\nat$: $x_0, x_1, x_3, \dots$

    Consider
    % Complex to draw matrix

    obviously $X \in 2^\nat \rightarrow \exists k$ s.t. $X = X_k$. But
    is $k \in X$?

    \(k \in X \quad \Rightarrow \quad k \notin X_k = X\)
    
    \(k \notin X \quad \Rightarrow \quad k \in X_k = X\)

    Which is absurd. Therefore $2^\nat$ is not enumerable.
  \end{proof}
\end{example}

\newcommand{\nattonat}{\nat \rightarrow \nat}
\begin{corollary}\label{corollary:nattonat}
  if $\nattonat = \{$ partial functions from $\nat$ to $\nat \}$
  then \[|\nattonat| > |\nat|\]

  \begin{proof}
    We can observe that
    \[2^\nat = |\{f: \nat \rightarrow \{0,1\} \; | \; f \mbox{ total }
      \} | \; \leq | \nattonat | \]

    and so \[|\nattonat| \geq |2^\nat| > |\nat|\]

    Alternatively we can prove it with the diagonalization
    technique. Let $f_1, f_2, f_3,$ be an enumeration of elements in
    $\nattonat$ and consider

    % again, complex matrice of elements

    we can define a functoin $f$ that differs from every other
    function on the diagonal based on the input:
    \[f(n) = \begin{cases}
        0 & \quad \mbox{if } f_n(n)\uparrow \\
        \uparrow & \quad \mbox{if } f_n(n) \downarrow
      \end{cases}
    \] \[f \in \nattonat\] so that
    $\forall n \quad f(n) \neq f_n(n) \quad \Rightarrow f \neq f_n$
  \end{proof}
\end{corollary}

\newcommand{\noc}{\bar{\mathcal{C}}}

\textbf{Note:} the set
$\noc = \{f : \nattonat \; | \; f \mbox{ not computable}\}$ is not
enumerable.

\begin{proof}
  We know that $|\mathcal{C}| = |\nat|$. if $\noc$ was enumerable then
  $\nattonat = \mathcal{C} \cup \noc$ would be enumerable, which is
  absurd for the corollary \ref{corollary:nattonat}.
\end{proof}

\begin{exercise}
  Exists a total function non computable. We lready knew that, but for
  cardinality reasons, now we are able to exibhit it.
  \[
    f(n) = \begin{cases}

      \varphi_n(n) + 1 & \quad \mbox{if } \varphi_n(n) \downarrow (n
      \in W_n) \\

       0 & \quad \mbox{if } \varphi_n(n)\uparrow (n \notin W_n)
    \end{cases}
  \]

  again, with the Cantor method we can build
  \[f(0) = \varphi_0(0), f(1) = \varphi_1(1), f(2) = \varphi_2(2),
    \dots \]
  is easy to see that
  \begin{itemize}
  \item $f$ is total
  \item $\forall n \; f \neq \varphi_n \; \Rightarrow f$ is not
    computable
  \end{itemize}
\end{exercise}

\textbf{Note:} there are infinitely many total non computable
functions in the form

\[
  f(n)  = \begin{cases}
    \varphi_n(n) + k & n \in W_n \\
    k & n \notin W_n
  \end{cases}
\]

% the prof here wrote "how many are them?", but....

\begin{exercise}
  Let $f: \nat \rightarrow \nat$ (partial), $m \in \nat$

  Define a function $g : \nat \rightarrow \nat$ non computable
  s.t. \[g(x) = f(x) \quad \forall x < m\]

  You can use a ``translated diagonal'';

  % again, complex matrix

  \[
    g(x) = \begin{cases}
      f(x) & x < m \\
      \varphi_n(x) + 1 & x \geq m \mbox{ and } x \in W_{x-m} \\
      0 & x \geq m \mbox{ and } x \notin W_{x-m}
    \end{cases}
  \]

  clearly $\forall x \quad g\neq \varphi_x$ since
  $g(x + m) \neq \varphi_x(x+m)$.

  \textbf{Note:} we could have defined

  \[
    g(x) = \begin{cases}
      f(x) & x < m \\
      \varphi_n(x) + 1 & x \geq m \mbox{ and } x \in W_{x} \\
      0 & x \geq m \mbox{ and } x \notin W_{x}
    \end{cases}
  \]

  why? Each function appears infinitely many times in the enumeration,
  and skip the first $m-1$ steps doesn't create any
  problem... Formally we know that
  $\forall x \geq m \quad g \neq \varphi_x$ so $\forall y$ since there
  are infinitely many indices for $\varphi_x$
  \[\exists x \geq m \mbox{ s.t. } \quad \varphi_y = \varphi_x \quad
    \mbox{and } \varphi_x \neq y\] so, $\forall y \; \varphi_y \neq g$
  (g is not computable).
\end{exercise}

\begin{exercise}
  Given a family of functions
  $\{f_i\}_{i\in \nat} \quad f_i : \nattonat$ define $g: \nattonat$
  s.t. $\dom{g} \neq \dom{f_i} \quad \forall i \in N$

  \[
    g(n) = \begin{cases}
      0 & \mbox{if } n \notin \dom{f_n} \\
      \uparrow & \mbox{if } n \in \dom{f_n}
    \end{cases}
  \]

  This way $\forall n \quad n \in \dom{g}$ if and only if
  $n \notin \dom{f_n}$.
\end{exercise}

\begin{exercise}
  Deifne a non computable total function that returns 1 when the input
  is even
  \[
    f(x) = \begin{cases}
      1 & x \mbox{ is even} \\
      \varphi_{\frac{x-1}{2}} + 1 & x \mbox{ is odd, and } x \in
      W_{\frac{x-1}{2}} \\
      0 & x \mbox{ is odd, and } x \notin W_{\frac{x-1}{2}}
    \end{cases}
  \]

  is easy to prove that the function $f$ is not computable. In fact,
  $\forall n \quad f(2n + 1) \neq \varphi_n(2n+1)$. This means that

  if
  $2n+1 \in W_n \Rightarrow f(2n+1) = \varphi(2n+1) + 1 \neq
  \varphi_n(2n+1)$

  if
  $2n+1 \notin W_n \Rightarrow f(2n+1) = 0 \neq \varphi_n(2n+1)
  \uparrow $

\end{exercise}
% Same as the previous chapter, these notes are very different form
% the italian version and not complete in the same way

% Given a set $X$ we know that $ |X| \geq |2^X| $ is never valid, that
% is, the set of its parts is always bigger. Suppose there is
% $ f:X\rightarrow2^X $ surjective. Hence $ R = \{y . y \in X $ and
% $ x \not \in f(y) \ \} \in 2^X $, since f is surjective then
% $ \exists y_R \in X . f(y_R) = R$. I consider the cases separately:
% \begin{itemize}
% \item $ y_R \in R $ then $f(y_R) \Rightarrow y_R \not \in R $
% \item $ y_R \not \in R $ then $ f(y_R) \Rightarrow y_R \in R$
% \end{itemize}
% Now we prove that the set of the parts of the natural numbers is not
% countable. We assume there is a surjective function
% $ \nat \rightarrow 2^\nat $. This means that I can enumerate subsets
% $ X_i $ s.t. $ i \in \nat $. At this point I create a matrix where
% each row $i$ of column $j = 1$ if $ x_i \in X_j $ and 0 otherwise. Now
% consider the inverted diagonal, that is if $ x_i \not \in X_i $, in
% this way it is different from all the columns, that is
% $ \forall i . R \not= X_i $ because
% $ n \in R \Leftrightarrow b \not \in X_n $ absurd since I assumed that
% $ \{X_0 \dots X_n \} = 2^\nat$.

% We conclude that $ |\nat| \not \geq |2^\nat| $

% But we want $ |\nat| < |2^\nat| $.

% But:
% $ |\nat| \leq |2^\nat| \land |\nat| \not\geq |2^\nat| \implies |\nat|
% < |2^\nat| $

% I take the set of the characteristic functions $g$ of $ 2^\nat $ and I
% call it $Y$, I take the set of all the functions $ \mathcal{F} $, of
% course it holds that $ Y \subseteq \mathcal{F} $ and therefore
% $ |Y| \leq |\mathcal{F}| $ but being that $ |\nat < |2^\nat| $ then I
% also have that $ |\nat| \leq |\mathcal{F}| $

% There is a total function that cannot be computed. We know how to
% enumerate computable functions because we can enumerate them in the
% form of numbers, repeating some of them. Now let's define $ f(n) $ as
% non-total computable.

% Matrix where the columns are the functions computed by the program
% $ i \in \nat $, that is $ \{\phi_i . i \in \nat \} $ and the rows are
% the arguments 0,1, \dots.

% \begin{equation*}
%   f(n) = \begin{cases}
%     \phi_n(n)+1 & $ se $\phi(n)\downarrow \\
%     0           & $ se $\phi(n)\uparrow
%   \end{cases}
% \end{equation*}

% we observe that $f$ is total by construction and
% $ f \not= \phi_n \forall n, n \in \nat $ Furthermore by construction
% it is different from all computable functions and therefore it is not
% computable.

\chapter {Parameterization theorem}
\newcommand{\smn}{$S_n^m$}

We'll start by giving an intuition on what the theorem talks
about. Let $f : \nat^2 \rightarrow \nat$ be a computable
function. Surely there exists an indice (to be precise infinitely many
of them) $\ell$ s.t. \[f(x,y) = \varphi_\ell^{(2)}(x,y)\]

Now, if we fix the first argument to a certain value $x$, we obtain a
function of a single argument $f_x : \nattonat$
\[f_x(y) = \varphi_\ell^{(2)}(x,y)\] and $\forall x \in \nat, \; f_x$
is computable (since obtained from composition of computable
funcitons. This means that exists a $d \in \nat$ s.t.
\[f_x = \varphi_d\] in other words, $\forall y \in \nat$
\[f_x(y) = \varphi_\ell^{(2)}(x,y) = \varphi_d(y)\] Clearly $d$
depends on $\ell$ and $x$, since $d = S(\ell, x)$ with
$S : \nat^2 \rightarrow \nat$ total function. The \smn
(\textit{smn}) theorem tells us that $S$ is computable. Intuitively,
how can I compute $S(\ell, x)$?

\begin{itemize}
\item form $\ell$ I can determine the program
  $P_\ell = \gamma^{-1}(\ell)$ that computes $\varphi_\ell^{(2)}(x,y)$
\item the program that computes $f_x = \lambda y \; . \; f(x,y)$ with
  $x$ fixed we can easly obtain from $P_\ell$:
  \begin{itemize}
  \item move $y$ in the register 2;
  \item put $x$ in the register 1;
  \item execute $P_\ell$
  \end{itemize}
\item we take the code of the obtained program
\end{itemize}

Again, intuitively, funcitons on indices, like $S$ are functions that
transform programs. The \smn theorem states that the operation of
fixing an argument of a program is effective.

\begin{example}
  Consider the computable function \[f(x,y) = x^y\]. We know that an
  indices s.t. \(\varphi_ell = f\) must exist, in other
  words \[\varphi_\ell(x,y) = f(x,y) = x^y\] So, when $x$ varies we
  obtain computable functions
  \begin{itemize}
  \item[] \(f_0(y) = y^0 = 1 \rightarrow \mbox{ indice} \quad S(\ell,0)\)
  \item[] \(f_1(y) = y^1 = y \rightarrow \mbox{ indice} \quad S(\ell,1)\)
  \item[] \(f_2(y) = y^2 = y^2 \rightarrow \mbox{ indice} \quad S(\ell,2)\)
  \item[] \(\dots\)
  \end{itemize}
  thanks to the \smn theorem we can determine those indices in an
  effective way. The thoerem also does this in general for functions
  in the form $f(\vec{x}, \vec{y}) : \nat^{(m+n)} \rightarrow \nat$,
  hence the name.
\end{example}

\begin{theorem}[\smn theorem]
  Given $m, n \geq 1$ exists a computable total function
  \[S_{m,n} : \nat^{m+1} \rightarrow \nat\] s.t.

  \[
    \varphi_\ell^{(m+n)}(\vec{x},\vec{y}) = \varphi_{s_{m,n}(\ell,
      \vec{x})}^{(n)}(\vec{y}) \quad \forall \ell \in \nat, \; x \in
    \nat^m, y \in \nat^n
  \]

  \begin{proof}
    intuitively, given $\ell \in \nat$, $\vec{x}\in \nat^m$ we can:
    \begin{itemize}
    \item with $\gamma^{-1}$ obtain the program
      $P_\ell = \gamma^{-1}(\ell)$ in standard form that computes the
      function $\varphi_\ell^{(m+n)}$, that is to say that starting
      from ( $\vec{x}, \vec{y}$ occupying respectively $m$ and $n$
      registers)
      \[
        \begin{tabu}{|c|c|c|c|c}
          \hline
          \vec{x} & \vec{y} & 0 & 0 & \dots \\ \hline
        \end{tabu}
        \quad \quad \mbox{it computes }
        \varphi_\ell^{(m+n)}(\vec{x},\vec{y})
      \]
    \item starting from $P_\ell$ we can build a new program $P$ tht
      starting from
      \[
        \begin{tabu}{|c|c|c|c}
          \hline
          \vec{y} & 0 & 0 & \dots \\ \hline
        \end{tabu}
        \quad \quad \mbox{it computes }
        \varphi_\ell^{(m+n)}(\vec{x},\vec{y})
      \]
    \end{itemize}

    In fact, it is sufficent
    \begin{itemize}
    \item move $\vec{y}$ ``forward'' of $m$ registers
    \item load $\vec{x}$ in the free $m$ registers
    \item execute $P_\ell$
    \end{itemize}
    
    The program P can be

    \begin{center}
      \begin{tabular}{lr}
        $T([1,n], [m+1, m+n])$    &          \\
        $z(n)$                    &          \\
        $s(n)$                    &          \\
        $\dots$                   & \comment{$x_1$ times} \\
        $z(m)$                    &          \\
        $s(m)$                    &          \\
        $\dots$                   & \comment{$x_m$ times}
      \end{tabular}
    \end{center}
    (\textbf{Reminder:} the concatenation has to update all the jump
    instrutions in $P_\ell$,
    $j(m^\prime, n^\prime, t) \rightsquigarrow j(m^\prime, n^\prime, t
    + m + n + \sum_{i=1}^mx_i)$

    Once we have build $P$ we can say that
    \[S(\ell, \vec{x}) = \gamma(P)\] Each function and construction
    method used are effective (in particular $\gamma, \gamma^{-1}$
    are) and so the existence, totality and computability of $S$ are
    proved (even if informally, by appealing to the Church-Turing
    thesis.
  \end{proof}
\end{theorem}
% Again, here he goes very fast, where in the notes he approaches it
% with more calm

% Suppose we have a computable function $ f:\nat^2\rightarrow\nat$
% therefore $ \exists e \in \nat . f = \phi_e^{(2)} $, now consider for
% the first fixed argument: $ f(x,y) = f_x(y) $, where
% $ f_x:\nat\rightarrow\nat $ computable, therefore there is a program
% that calculates it.

% For example let's take $ f(x,y) = y^x $ then $ f_0(y) = 1 $, \dots all
% these functions are computable.

% We had the program that computed $ f(x,y) = \phi_e^{(x)}(x,y) $, so
% there is some program $d$ that calculates $ f_x = \phi_d $, but
% obviously they are all different programs for each $d$. We observe
% that this program depends on \textit{e} and on \textit{x}.

% What we are saying is that there exists total
% $ s:\nat^2\rightarrow\nat$ such that
% $ \phi_{s(e,x)}(y) = \phi_e^{(2)}(x,y)$, the theorem says it is
% computable. In general we can take a function of $ m+n $ arguments,
% $ m,n \in \nat $, $ m,n \geq 1 $ there exists
% $ S_{m,n} : \nat^{m+1}\rightarrow\nat \in \mathcal{PR}$
% s.t.

% \begin{equation*}
%   \phi_e^{m+n}(\vec{x},\vec{y}) = \phi_{s_{m,n}(e, \vec{x})}^{n}(\vec{y})
% \end{equation*}

% But now let's see how to calculate $ \gamma $. First we define
% function $ agg:\nat^2\rightarrow\nat $ where $ agg(e,t) =$ program
% obtained by \textit{e} adding \textit{t} to the destination of each
% jump.

% Now let's take $ \overline{agg} $ where $ \overline{agg}(i,t) = $
% update of instruction i (meaning $ \beta^{-1}(i) $)

% \dots

% \section {Corollary SMN theorem}
% Given the function
% $ f:\nat^{n+m}\rightarrow\nat,\\ \exists S:\nat^m\rightarrow\nat $
% total computable s.t.
% $ f(\vec{x},\vec{y}) = \phi_{s(\vec{x})}^{(n)}(\vec{y}) \forall
% \vec{x},\vec{y}$

% \section {Exercise}

% Show that there exists $s$ total computable s.t.
% $ \phi_{s(n)}(x) = \sqrt[n]{x} $

% \textbf{Execution:} I take the function $ f(n,x) = \sqrt[n]{x} $ so I
% have to search \\ $ max(y) . y^n \leq x $, but I can only use the
% minimum, so I try $ min(y). (y+1)^n > x $

% that is: $ \mu y \leq x . x+1 - (y+1)^n $. Even without bound it
% worked, but so we also show that it is recursive primitive (it was not
% required).

% We see that it is computable, therefore by corollary of the smn
% theorem there exists the total computable function $s$ such that
% $ f(n,x) = \phi_{s(n)}(x) $.

\chapter {Universal Function}
Program that takes input $ e $ and $ \vec{x} $ and returns $ \phi_e(\vec{x}) \forall e$ so I find all the computable functions.

More specifically, the function $ \psi_u(x,y) = \phi_x(y)$, $ \psi : \nat^2 \rightarrow \nat $

In other words, the interpreter exists.

\textbf{Theorem}: This is computable. $ \forall k \geq 1  \quad  \psi_u^{(k)} $ is computable.

Let's prove it. Suppose we have a fixed $ k \geq 1 $.

How do I calculate $ e \in \nat, \vec{x} \in \nat^k, \psi_u^{k}(e, \vec{x}) $?

Fact $e$, calculation $ \gamma^{-1}(e) = P_e $

But we don't want to decode the program.

\chapter{Recursive sets}
In previous chapters, we saw many computable functions and decidable problems, but
only in few cases we gave examples of the large class of
non-computable functions and undecidable predicates. For this reason,
we want to start a mathematical study of
\begin{itemize}
  \item classes of undecidable predicates
  \item techniques to prove the undecidability of some predicates
\end{itemize}
In this way we can highlight the limits of computers abilities and give a
structure to problems classes (the majority of interesting
predicates are undecidable).

We'll focus on \emph{numerical sets} $X \subseteq \nat$ and try to find
an answer to the problem ``$x \in X$?''. We'll get
\begin{itemize}
\item recursive sets
\item recursively enumerable sets
\end{itemize}

\section{Recursive sets}
\begin{definition}
  A set $A \subseteq \nat$ is \emph{recursive} if its characteristic
  function
  \begin{gather*}
    \chi_A : \nattonat \\
    \chi_A(x) = \begin{cases}
      1 & x\in A \\
      0 & x \notin A
    \end{cases}
  \end{gather*}
  is computable.
\end{definition}
In other words, if the predicate ``$x \in A$'' is decidable.

\begin{observation}
  \begin{itemize}
    \item if $\chi_A \in \pr$ we'll say that $A$ is \emph{primitively}
      recursive.
    \item the notion can be extended to subsets of $\nat^k$, but we'll
      stick to $\nat$ subsets, since every $\nat^k$ subset can be encoded
      into $\nat$
    \end{itemize}
\end{observation}


\begin{example}
  These are recursive:
\begin{enumerate}[label=(\alph*)]
\item $\nat$, since $\chi_\nat = \mathbf{1}$ is computable;
\item $\varnothing$, because $\chi_\varnothing = \mathbf{0}$ is computable;
\item prime numbers $\mathbb{P}$, since
  \[
    Pr(x) = \begin{cases}
      1 & \mbox{if $x$ is prime} \\
      0 & \mbox{otherwise}
    \end{cases}
  \]
  is computable;
\item each and every finite set. In fact, given $A \subset \nat$ with
  $|A| < \infty$, $A = \{x_1, x_2, \dots, x_n\}$, we have that
  \[
    \chi_A(x) = \overline{sg}\left( \prod_{i=1}^n|x - x_i| \right)
  \]
  is computable.
\end{enumerate}
\end{example}


On the other hand, these are definitely not recursive:
\begin{enumerate}[label=(\alph*)]
\item $K = \left\{ x \; | \; x \in W_x \right\} $, since
  \[
    \chi_{K}(x) = \begin{cases}
      1 & x \in W_x \\
      0 & x \notin W_x
    \end{cases}
  \]
  is not computable;
\item $\left\{ x \; | \; \varphi_x \mbox{ total} \right\} $
\end{enumerate}

\begin{observation}
  If $A,B \subseteq \nat$ are recursive, then
\begin{enumerate}[label=\arabic*)]
\item $\overline{A} = \nat - A$
\item $A \cap B$
\item $A \cup B$
\end{enumerate}
are recursive.
\end{observation}


\subsection{Reduction}
Reduction is a process used to study decidability
problems. It formalizes the intuition of a problem
$\mathcal{A}$ being ``easier'' then another one, $\mathcal{B}$.

\newcommand{\red}{\ensuremath{\leq_m}}
\begin{definition}
  Let $A,B \subseteq \nat$. We say that the problem $x \in A$
  \emph{reduces} to the problem $x \in B$ ($A$ reduces to $B$), 
  written $A \red B$ if there exists
  $f : \nattonat$ computable and total such that, for every $x \in \nat$
  \[x \in A \quad  \Leftrightarrow \quad f(x) \in B\]
\end{definition}
In this case, $f$ is the \emph{reduction function}.

\begin{observation}
  Let $A,B \subseteq \nat$ such that $A \red B$ then
\begin{enumerate}[label=\arabic*]
\item if $B$ is recursive, then $A$ is recursive
\item of $A$ is not recursive, then $B$ is not recursive
\end{enumerate}

\begin{proof}
Just observe that $\chi_A = \chi_B \circ f$
\end{proof}
\end{observation}

We know that $K = \{ x \mid x \in W_x \}$ is not recursive. Let's see how
the recursiveness of other sets can be proven by reduction to this
set which we know for certain is not recursive.
\begin{example}
  $K \red T = \{x \mid \varphi_x $ total$ \}$
  \begin{proof}
    we need to prove that there exists $s : \nattonat$ computable and total
    such that $x \in k \Leftrightarrow s(x) \in T$. In other words
    \[ x\in W_x \Leftrightarrow  \varphi_{f(x)} \mbox{ is total} \]
    To do so, we can define
    \[
      g(x,y) = \begin{cases}
        1 & x\in W_x \\
        \uparrow & \mbox{otherwise}
      \end{cases}
    \]
    which is computable. This fact is easily proven by rewriting it as
    \[
      g(x,y) = \mathbf{1}(\varphi_x(x)) = \mathbf{1}(\univ(x,x))
    \]
    then, by \emph{smn} theorem we have that there exists $s: \nattonat$
    computable and total such that 
    \[\varphi_{s(x)}(y) = g(x,y)\] and 
    \[x \in K \Rightarrow x \in W_x \Rightarrow \forall y\
      \varphi_{s(x)}(y) = g(x,y) = 1 \Rightarrow \varphi_{s(x)} \mbox{
        total } \Rightarrow s(x) \in T\]
    \[x \notin K \Rightarrow x \notin W_x \Rightarrow \forall y\
      \varphi_{s(x)}(y) = g(x,y) \uparrow \Rightarrow \varphi_{s(x)}
      \mbox{ not total } \Rightarrow s(x) \notin T\]
  \end{proof}
\end{example}

\begin{example}[Input problem]
  For every $ n \in \nat$
  \[
    A_n = \{x  \mid \varphi_x (n) \downarrow\}
  \]
  is not recursive.

  \begin{proof}
    We will prove that $K \leq A_n$. We have to define a function $f$
    s.t.
    \begin{gather*}
      x \in K \Leftrightarrow f(x) \in A_n \\
      x \in W_x \Leftrightarrow \varphi_{f(x)}(n) \downarrow
    \end{gather*}
    Define
    \begin{align*}
      g(x,y) &= \begin{cases}
        1 & x \in W_x \\
        \uparrow & \mbox{otherwise}
      \end{cases} \\
      &= \mathbf{1}(\univ(x,x))
    \end{align*}
    which is computable, and therefore by the \emph{smn} theorem, there exists
    $f: \nattonat$ computable and total such that
    $g(x,y) = \varphi_{f(x)}(y)$, and
    \begin{gather*}
      x \in k \Rightarrow f(x) \in A_n \\
      x \notin  k \Rightarrow f(x) \notin A_n
    \end{gather*}
  \end{proof}
\end{example}

\begin{example}[The output problem]
  For every $ n \in \nat$, $B_n\{ x \mid n \in E_x\}$ is not recursive
  \begin{proof}
    We show that $K \leq_m B_n$
    \[
      \begin{split}
        g(x,y) &= \begin{cases}
          n & x \in W_x \\
          \uparrow & \mbox{otherwise}
        \end{cases} \\
        &= n \cdot \mathbf{1}(\univ(x,x))
      \end{split}
    \]
    computable, by \emph{smn} theorem, there exists a function
    $s : \nattonat$ such that
    \[
      \forall x,y \quad g(x,y) = \varphi_{s(x)}(y)
    \]
  moreover
  \begin{gather*}
    x \in k \Rightarrow s(x) \in B_n \\
    x \notin k \Rightarrow s(x) \notin B_n
  \end{gather*}
\end{proof}
\end{example}

\begin{observation}
Let $A,B \subseteq \nat$ with $A \leq_m B$
through an injective reduction function $f : \nattonat$ (total and computable). 
One could think that, since $f^{-1}$ is computable, then
also $B \leq_m A$. This does not happen, since $f^{-1}$ is
not total and so reduces $A$ to a ``subproblem'' of $B$ even though
it has no complexity relationships with $B$.
\end{observation}

\chapter {Rice theorem}

Rice thorem gives an undecidability result very general. In broad
terms it states that \emph{no property} of computable functions
(besides obvois ones) are decidable.

To state it in terms of of subsets of $\nat$, we need the following
notion

\section{Saturated sets}
\begin{definition}[Saturated set]
  A subset $A \subseteq \nat$ is \emph{saturated} if
  \[
    \forall x \in A \wedge \varphi_x = \varphi_y \Rightarrow y \in A \}
  \]
\end{definition}

In other words $A$ is saturated if it expresses a property of
functions, independently from indices
\[A = \{x \mid P(\varphi_x)\}\]
or, again, if exists $\mathcal{A} \subseteq \mathcal{C}$ s.t.
\[A = \{ x \mid \varphi_x \in \mathcal{A}\]

\begin{example}
  \[
    T_2 = \{ e | P_e(e)\downarrow \mbox{ in two steps } \} =
    \{e|\phi_e \in \mathcal{T}_2 \}
  \]

  But two programs can calculate the same function and finish one in
  less than 2 steps and the other in more than 2, so the set is not
  saturated.
\end{example}

\begin{example}
  
  \begin{center}
    $ K = \{e \mid e\in W_e \} = \{e \mid \phi_e\in \mathcal{K} \}
  \mathcal{K} = \{f \mid ? \}$ I don't know what to write in this.
  \end{center}

  It is not actually saturated. Difficult to prove this, but we can
  show that there is a program $e$ such that:
  \begin{equation*}
    \phi_e(x) = \begin{cases}
      0 & x = e \\
      \uparrow & $ else $
    \end{cases}
  \end{equation*}
  And if you try to write such a program you realize that it is
  impossible.
\end{example}

\section {Rice theorem}

\begin{theorem}[Rice theorem]
  Let $ A \in \nat $ be saturated. $ A \neq \emptyset, A \neq \nat $
  then it is non-recursive.
\end{theorem}

\begin{proof}
  We wanto to prove that $ K \leq A $. We will show that it reduces,
  that is, we'll find $f$ total and computable s.t. all the elements
  of $K$ go to $A$ and all the elements of the complenent of $K$ go to
  the complement of $A$.

  \begin{itemize}
  \item[($e_0 \notin A$)]
    Let $ e_0$ s.t. $ \phi_{e_0}(x)\uparrow\forall x $ suppose
    $ e_0\notin A $ and let $ e_1\in A $
    
    Now let's define the following function:
    \begin{equation*}
      g(x,y) = \begin{cases}
        \phi_{e1}(y) & x \in K \\
        \phi_{e0}(y) & x \notin K
      \end{cases}
    \end{equation*}

    Which is equal to:

    \begin{equation*}
      g(x,y) =
      \begin{cases}
        \phi_{e1}(y) & x \in K \\
        \uparrow & x \notin K
      \end{cases}
    \end{equation*}

    and to calculate if $x$ is in $K$, we just need to run $ \phi_e(x) $
    and see if it ends.

    For the \smn theorem there exists $f$ s.t.
    $ \phi_{f(x)} = \phi_{e1} \Rightarrow f(x)\in A$

    \begin{equation}\label{eq:one14}
      x \in K
      \Rightarrow x \in W_x
      \Rightarrow \varphi_{f(x)}(y) = g(x,y) = \varphi_{e_1}(x)
      \Rightarrow e_1 \in A
    \end{equation}
    \begin{equation}\label{eq:two14}
      x \notin K
      \Rightarrow x \notin W_x
      \Rightarrow \varphi_{f(x)}(y) = g(x,y)\uparrow \forall y
      \Rightarrow e_0 \notin A
    \end{equation}

    And since $A$ is saturated, for (\ref{eq:one14})
    $x \in K \Rightarrow S(x) \in A$, for (\ref{eq:two14})
    $x\notin K \Rightarrow S(x) \notin A$.

    This way we proved that $K \leq_m A$, and since $K$ is not
    recursive, also $A$ is not recursive.
    
  \item[($e_0 \in A$)] if $ e_0 \in A $ then $ e_0 \not \in \bar{A}
    $. But we remember that
    $ \bar{A} \subseteq \nat, \bar{A} \neq \emptyset, \bar{A} \neq
    \nat $ then for the last point $ \bar{A} $ non-recursive and
    therefore neither $A$ is recursive.
  \end{itemize}
\end{proof}

\begin{example}
  \[ T = \{e \mid \varphi_e \mbox{ total } \} \mbox{ non-recursive} \]

  $T$ is saturated: $ T = \{e \mid \varphi_e \in \mathcal{T} \} $, $
  \mathcal{T} = \{f \mid f $ total $ \} $, $ T\neq\emptyset $, $ e_1 $\\
  s.t. $ \varphi_{e_1}(x) = 0 \quad \forall x $

  $ e_0 \notin \mathcal{T} $. Saturated, other than empty and $ \nat $
  therefore for Rice theorem it is non-recursive.
\end{example}

\begin{example}[output problem]
  \[ B_n = \{e \mid n \in E_e \} \quad \mbox{is not saturated} \]

  In this case we can use Rice theorem, since
  \begin{itemize}
  \item $B_n$ is saturated;
  \item $B_n \neq \emptyset$
  \item $B_n \neq \nat$ 
  \end{itemize}

  Hence $ B_n $ non-recursive.
\end{example}
\chapter{Recursively enumerable sets}

\begin{definition}[Recursively enumerable set]
  We say that $ A \subseteq \nat $ is \emph{recursively enumerable} if
  the semi-characteristic function 
  \begin{equation*}
    sc_A(x) = \begin{cases} 1 & x \in A \\ \uparrow & $
      otherwise $
    \end{cases}
  \end{equation*}
  is computable.
\end{definition}

\begin{definition}[Semi-decidable predicate]
  A predicate $ Q(x) \subseteq \nat $ is\\
   semi-decidable if 
  \( \{ x \in \nat \mid Q(x) \} \)
  is r.e.
\end{definition}

Thus, saying that $A$ is r.e. is like saying that the predicate $ Q(x)=``x \in A"
$ is semi-decidable. This notion is also easily generalisable to
\begin{itemize}
\item subsets of $\nat^k$
\item $k$-ary predicates
\end{itemize}

\begin{observation}\label{th:aiffanota}
  \[A \subseteq \nat \mbox{ recursive } \Leftrightarrow A, \bar{A} \mbox{
      are r.e.} \]
  \begin{proof}
    \begin{itemize}
    \item[($\Rightarrow$)]
      If $A$ recursive,
      \begin{equation*}
        \mathcal{X}_A(x)= \begin{cases}
          1 & x\in A \\
          0 & $otherwise $
        \end{cases}
      \end{equation*}
      is computable.
      Then $ sc_A(x) = \mathbf{1}(\mu z. | \chi_A(x)- 1 | )$
      is computable, therefore ${A}$ is r.e. Since $A$ is recursive, then
      $\bar{A}$ is recursive, thus, r.e. 

    \item[($\Leftarrow$)] Let $A, \bar{A}$ be r.e., then by definition
      $sc_A$ and $sc_{\bar{A}}$ are computable, and we can define
      \[
        \mathbf{1} - sc_{\bar{A}}(x) = \begin{cases}
          0 & x \in \bar{A} \\
          \uparrow & \mbox{otherwise}
        \end{cases}
      \]
      that is computable. This means that $\exists e_0, e_1 \in \nat$ such that
      \[
        \varphi_{e_0} = sc_A \quad \varphi_{e_1} = \mathbf{1} -
        sc_{\bar{A}}
      \]
      therefore we can ``combine two machines'' and wait until one
      of the two terminates. Since either $x \in A$ or $x \in \bar{A}$,
      then the process will terminate for sure. We can build the characteristic function of $A$ as
      \[
          \chi_A(x) = (\mu \omega . |\chi_{S(e_0, x,
            (\omega)_1, (\omega)_2 \wedge S(e_1, x, (\omega)_1,
            (\omega)_2))}-1|)_1
        \]
        which is computable, therefore $A$ is recursive.
    \end{itemize}
  \end{proof}
\end{observation}

\begin{observation}
  the set $K = \{x \mid x \in W_x\}$ is r.e. In fact
  \[
    sc_K(x) = \begin{cases}
      1 & x \in K \\
      \uparrow & \mbox{otherwise}
    \end{cases}
    = \mathbf{1}(\varphi_x(x)) = \mathbf{1}(\univ(x,x))
  \]
  is computable by definition and by \ref{th:aiffanota}
  \[
    \bar{K} = \{x \mid x \notin W_x\}
  \]
  is \emph{not} r.e, otherwise $K,\bar{K}$ would have been both r.e.,
  and therefore $K$ would have been recursive, which is a
  contradiction.
\end{observation}

\begin{theorem}[Structure of semi-decidable predicates]\label{th:structure}
  Let $ P(\vec{x}) \subseteq \nat^k $ be a predicate.
  $ P(\vec{x})$ is decidable if and only if there is a decidable predicate $
  Q(t,\vec{x}) \subseteq \nat^{k+1} $ such that $ P(\vec{x}) = \exists
  t. Q(t,\vec{x}) $
  \begin{proof}
    \begin{itemize}
    \item[($\Rightarrow$)] Let $P(\vec{x})$ be
      semi-decidable. It has a computable semi characteristic function
      $sc_P$ so
      \[
        P(\vec{x}) \equiv \exists t . H(e,\vec{x}, t)
      \]
      therefore if we can rewrite $H$ as
      $Q(t, \vec{x}) = H(e,\vec{x}, t)$, in this way $Q$ is decidable as
      we wanted and \[P(\vec{x}) \equiv \exists t . Q(t, \vec{x})\]

    \item[($\Leftarrow$)] Let
      \(P(\vec{x}) \equiv \exists t . Q(t, \vec{x})\) with $Q(t, \vec{x})$
      decidable. Observe that
      \[
        sc_P(\vec{x}) = \mathbf{1}(\mu t . |\chi_Q(t,\vec{x}) - 1|)
      \]
      which is computable by definition, and therefore $P(\vec{x})$ is
      semi-decidable.
    \end{itemize}
  \end{proof}
\end{theorem}

\section {Projection theorem}

From the last theorem we had a hint about the fact that the class of
semi-decidable predicates is closed under \emph{existential
  quantification}. The projection theorem states this:
\begin{theorem}[Projection theorem]
  Let $ P(x,\vec{y}) $ be semi-decidable; then
  \[
    \exists x.P(x,\vec{y}) = P'(\vec{y})
  \]
  is semi-decidable.

  \begin{proof}
    Let $P(x,\vec{y})$ be semi-decidable. The by
    (\ref{th:structure}), there exists $Q(t,x,\vec{y})$ decidable such that
    \[
      P(x, \vec{y}) \equiv \exists t . Q(t,x,\vec{y})
    \]
    Thus 
    \begin{align*}
      P^\prime(\vec{y}) &\equiv \exists x . P(x, \vec{y}) \\
        &\equiv \exists x .\exists t .
        Q(t,x,\vec{y}) \\
        &\equiv \exists \omega . Q((\omega)_1, (\omega)_2, \vec{y})
    \end{align*}    
    since $Q((\omega)_1, (\omega)_2, \vec{y})$ is decidable,
    by (\ref{th:structure}) $P^\prime(\vec{y})$ is
    semi-decidable.
  \end{proof}
\end{theorem}

\begin{theorem}[Closure property]
  Let $ P_1(\vec{x}), P_2(\vec{x}) $ be semi-decidable predicates. Then
  \begin{itemize}
  \item $  P_1(\vec{x}) \lor P_2(\vec{x}) $;
  \item $ P_1(\vec{x}) \land P_2(\vec{x}) $
  \end{itemize}
  are semi-decidable.
  \begin{proof}
    Let $ P_1(\vec{x}), P_2(\vec{x}) $ be semi-decidable
    predicates. Then by (\ref{th:structure}) there are two
    decidable predicates \( Q_1(t, \vec{x}), Q_2(t, \vec{x})\) such that
    \begin{gather*}
      P_1(\vec{x}) \equiv \exists t . Q_1(t, \vec{x}) \\
      P_2(\vec{x}) \equiv \exists t . Q_2(t, \vec{x})
    \end{gather*}
    Hence
    \begin{enumerate}[label=(\arabic*)]
    \item
      \[
        \begin{split}
          P_1(\vec{x}) \lor P_2(\vec{x}) &\equiv \exists t .
          Q_1(t, \vec{x}) \lor \exists t . Q_2(t, \vec{x}) \\
          &\equiv \exists \omega . (Q_1((\omega)_1, \vec{x})
          \lor Q_2((\omega)_2, \vec{x}))
        \end{split}
      \]
      This means that by (\ref{th:structure}),
      $P_1(\vec{x}) \lor P_2(\vec{x})$ is semi-decidable.
    \item Analogously
      \[
        P_1(\vec{x}) \land P_2(\vec{x}) \equiv \exists t .
        (Q_1(t, \vec{x}) \land Q_2(t, \vec{x}))
      \]
    \end{enumerate}
  \end{proof}
\end{theorem}

\begin{observation}
  The set of semi-decidable predicates is closed under $\land, \lor$
  and $\exists$. It is not closed under $\forall$ and $\lnot$.
\end{observation}

\begin{exercise}
  Prove that if $ P(\vec{x}) $ is semi-decidable and is not decidable
  then $ \lnot P(\vec{x}) $ is not semi-decidable.
\end{exercise}

\begin{observation}
  \begin{enumerate}[label=(\arabic*)]
  \item $A \subseteq \nat$ is recursive if and only if $A, \bar{A}$ are r.e.
  \item if $A \subseteq \nat$ r.e. and $f : \nattonat$ computable
    $\Rightarrow f^{-1}(A)$ is r.e. (projection)
  \item $A,B \subseteq \nat$ r.e. $\Rightarrow A \cup B, A \cap B$
    are r.e.
  \end{enumerate}
\end{observation}

\subsection{r.e. sets and reducibility}
Properties similar to those already seen for recursive sets hold for r.e. sets:
\begin{observation}
  Given $ A,B \subseteq \nat, A\leq_m B $, then
  \begin{itemize}
  \item B is r.e. $ \Rightarrow $ A is r.e.
  \item A is not r.e. $ \Rightarrow $ B not r.e.
  \end{itemize}
  \begin{proof}
    \begin{enumerate}
      \item If $B$ r.e., then
      \begin{equation*}
        sc_B(x) = \begin{cases}
          1 & x \in B \\
          \uparrow & $ otherwise $
        \end{cases}
      \end{equation*}
      is computable.  Let $ f:\nat\rightarrow\nat $ be a total
      computable reduction function for $ A\red B $. Then
      $ sc_A(x) = sc_B(f(x)) $, therefore $ sc_A $ is computable by
      composition and $A$ is r.e.
      \item equivalent.
    \end{enumerate}
  \end{proof}
\end{observation}

\chapter {Rice-shapiro theorem}
Rice-shapiro states that a property of the functions computed by
programs can be semi-decidable \textbf{only if} it depends on a finite
part of the function (behavior on finite inputs). 
\begin{theorem}[Rice-shapiro theorem]
  Let $\mathcal{A} \subseteq \mathcal{C}$ be a set of computable
  functions. If the set $A = \{x \mid \varphi_x \in \mathcal{A}\}$ is
  r.e., then
  \[
    \forall f (f \in \mathcal{A} \Leftrightarrow \exists \theta \mbox{
      finite function, } \theta \subseteq f \land \theta \in
    \mathcal{A})
  \]

In order for us to prove this theorem, we'll need some more tools:
\begin{definition}[Finite function]
  A finite function is a function $ \theta: \nat\rightarrow\nat $
  such that $ dom(\theta) $ is finite.  This means that the set of
  input-output pairs is finite; in other words
  $ \theta = \{(x_1,y_1),\dots(x_n,y_n) \} $ and it is undefined on other inputs.
\end{definition}

\begin{definition}
  Given $ f:\nat\rightarrow\nat $, $ \theta $ is a sub-function of
$f$ if $ \theta \subseteq f $
\end{definition}


\begin{notation}
  \begin{itemize}
  \item $ W_e $ is the domain of the function $ \varphi_e $;
  \item $ E_e = \{\varphi_e(x) \mid x\in W_e \}$;
  \item $ H(x,y,t) = $ "$ P_x(y)\downarrow $ in $t$ steps or les";
  \item $ s(x,y,z,t) =$ "$ P_x(y)\downarrow z$ in $t$ steps or les";
  \item $ K = \{x \mid x\in W_x \} = \{x\mid \varphi_x(x)\downarrow \} =
    \{x\mid P_x(x) \mbox{ terminates} \}$
  \end{itemize}
\end{notation}

\begin{proof}
  We'll prove the following
  \begin{enumerate}
    \item $ \exists f \in \mathcal{C} . f \not\in \mathcal{A} \land
    \exists\theta\subseteq f \mbox{ finite}, \theta\in\mathcal{A} \Rightarrow
    A$ not r.e
    \item $ \exists f \in \mathcal{C}. f\in\mathcal{A} \land
    \forall\theta\subseteq f \mbox{ finite}, \theta\not\in\mathcal{A}\Rightarrow
    A$ not r.e.
  \end{enumerate}

  so
  \begin{enumerate}
    \item
    Let $ f\not\in \mathcal{A}$ and $\theta \subseteq f$ finite with $\theta \in \mathcal{A}$.
    We show that $
    \bar{K}\leq A $.

    Define
    \begin{equation*}
      \begin{aligned}
        g(x,y) & = \begin{cases}
          \theta(y) & x \in \bar{K} \\
          f(y) & x \in K
        \end{cases} \\
               & = \begin{cases}
                 \uparrow & x \in \bar{K} \land x \not\in dom(\theta) \\
                 \theta(y) = f(y) & x \in \bar{K} \land x \in dom(\theta) \\
                 f(y) & x\in K
               \end{cases}\\
               &= \begin{cases}
                 f(y) & x \in K\lor y\in dom(\theta) \\
                 \uparrow & $otherwise $
               \end{cases}
      \end{aligned}
    \end{equation*}

    But $ x\in K\lor y\in dom(\theta) = Q(x,y)$ predicate. $ x\in K $
    semi-decidable; $ y \in dom(\theta) $ decidable; therefore $ Q(x,y) $
    semi-decidable.

    \begin{equation*}
    sc_Q(x,y) = \begin{cases}
        1 & Q(x,y) \\
        0 & $ altrimenti $
      \end{cases}
    \end{equation*}

    This is computable $ = f(y) \times sc_Q(x,y) $ computable.

    For sMN, given that \textit{g} is computable, $ \exists
    s:\nat\rightarrow\nat $ s.t.
    \begin{equation*}
      g(x,y) = \begin{cases}
        \theta(y) & x \in \bar{K} \\
        f(y) & x \in K
      \end{cases}
    \end{equation*}

    $ g(x,y) = \varphi_{s(x)}(y)$

    $s$ is the reduction function for $ \bar{K}\leq A $

    \begin{itemize}
    \item $ x\in\bar{K} \Rightarrow \forall y. \varphi_{s(x)}(y) = g(x,y) =
      \theta(y) \Rightarrow \varphi_{s(x)} = \theta \Rightarrow s(x) \in A $
    \item $ x\not\in\bar{K}\Rightarrow x\in K\Rightarrow\forall
      y\varphi_{s(x)}(y) = g(x,y)=f(y)\Rightarrow\varphi_{s(x)}=f\Rightarrow
      s(x)\in\bar{A}$
    \end{itemize}

    Hence $A$ is reduced to $ \bar{K} $ which is not r.e. therefore A is
    not r.e.

    \item
    let $ f\in\mathcal{A}\land\theta\subseteq f $ be with $ \theta $
    finished, $ \theta\not\in\mathcal{A} $

    We want it to be in quotes because it's not formal: \begin{equation*}
      g(x,y) = \begin{cases}
        f(y) & x \in\bar{K} $ cioè $ \varphi_x(x)\uparrow \\
        \theta(y) & $ for some $ \theta\subseteq f $ finite, otherwise ($ x\in K $) $
      \end{cases}
    \end{equation*}

    This is computable, meaning $f(y) \times sc_Q(x,y) $ is
    computable.

    for sMN there exists $ s:\nat\rightarrow\nat $ total computable s.t. $
    \forall x,y. \varphi_{s(x)}(y) = g(x,y) $

    We have to prove that s is a reduction function for $ \bar{K}\leq A $

    \begin{itemize}
    \item $ x\in\bar{K}\\ \Rightarrow\varphi_{s(x)}\uparrow \\
      \Rightarrow\forall y\lnot H(x,x,y)\\ \Rightarrow \forall
      y.\varphi_{s(x)}(y) = g(x,y) = f(y)\\ \Rightarrow f = \varphi_{s(x)}\\
      \Rightarrow s(x)\in A$
    \item
      $ x\not\in\bar{K}\\ \Rightarrow x\in K\\ \Rightarrow
      \varphi_x(x)\downarrow\\ \Rightarrow \exists t_0 $ s.t.
      $ \forall
      t>t_0. H(x,x,t), \forall t<t_0. \lnot H(x,x,y)\\
      \Rightarrow\varphi_{s(x)}(y) = g(x,y)\\ \Rightarrow
      \varphi_{s(x)}\subseteq f$ finite
      $\\ \Rightarrow s(x) \in \bar{A} $
    \end{itemize}
  \end{enumerate}

  Rice-shapiro's theorem proved.
\end{proof}
\end{theorem}

\begin{example}\label{exe:rice1}
  $A = \{ x \mid \varphi_x \mbox{ total}\}$ is not r.e.

  \begin{proof}
    Clearly $A$ is saturated since $A = \{x \mid \varphi_x \in
    \mathcal{A}\}$, and $\mathcal{A} = \{f \in \mathcal{C} \mid f $ total$\}$. 
    Given any function $f \in \mathcal{A}$ (total by
    definition) we know that $\forall \theta \subseteq f$
    is finite $\theta \notin \mathcal{A}$, since each and every finite
    function is partial, then by Rice-shapiro's theorem, $A$ is
    not r.e.
  \end{proof}
\end{example}

\begin{example}\label{exe:rice2}
  $\bar{A} = \{x \mid \varphi_x $ not total$\}$ is not r.e.

  \begin{proof}
    Let $\bar{\mathcal{A}} = \{f \in \mathcal{C} \mid f $ not total$\}$. We observe that each $\theta$ finite is in
    $\bar{\mathcal{A}}$, but no total extension of such $\theta$ can
    be included in $\bar{\mathcal{A}}$. Again, by Rice-shapiro
    $\bar{A}$ is not r.e.
  \end{proof}
\end{example}

The examples \ref{exe:rice1} and \ref{exe:rice2} are esential to
understand two core situations in which we can apply the theorem:
\begin{observation}
  Let $\mathcal{A} \subseteq \mathcal{C}$ be a set of computable
  functions s.t. $A = \{ x \mid \varphi_x \in \mathcal{A}\}$ is
  r.e. Then
  \begin{enumerate}[label=(i)]
  \item if \(\forall \theta \) finite
    \( \theta \notin \mathcal{A} \Rightarrow \mathcal{A} = \emptyset\)
  \item
    \(\emptyset \in \mathcal{A} \Rightarrow \mathcal{A} =
    \mathcal{C}\)
  \end{enumerate}

  \begin{proof}
    \begin{enumerate}[label=(i)]
    \item given $f \in \mathcal{C}$ we know that $f \in \mathcal{A}$
      iff $\exists \theta \subseteq f$ finite
      $\theta \in \mathcal{A} \rightarrow f \notin \mathcal{A}$
    \item given $f \in \mathcal{C}$, since $\emptyset \subseteq f$ and
      $\emptyset \in \mathcal{A} \Rightarrow f \in \mathcal{A}$
    \end{enumerate}
  \end{proof}
\end{observation}

\begin{exercise}
  study the recursivenes of $A = \{x \mid \varphi_x = \mathds{1}\}$

  \begin{enumerate}
  \item[(*)] $A$ is not r.e.

    In fact the set of functions is in this case
    $\mathcal{A} = \{\mathds{1}\}$, which
    \begin{itemize}
    \item does not contain finite functions
    \item is not empty
    \end{itemize}
    And therefore is not r.e.

  \item[(**)] $\bar{A}$ is not r.e.

    In fact $\bar{\mathcal{A}} = \mathcal{C} - \{\mathds{1}\}$, and we
    have that
    \begin{itemize}
    \item $\emptyset \in \bar{\mathcal{A}}$
    \item $\bar{\mathcal{A}} \neq \mathcal{C}$
    \end{itemize}
    And therefore $\bar{A}$ is not r.e.
  \end{enumerate}
\end{exercise}

\begin{observation}
  Rice-shapiro theorem is a necesary condition, but not sufficient to
  be r.e., that is to say that does not hold that
  \begin{equation}\label{eq:star}
    \forall f \; (\; f \in \mathcal{A} \mbox{ iff } \exists \theta \mbox{
      finite, } \theta \subseteq f, \; \theta \in \mathcal{A} \; ) \quad
    \Rightarrow \quad A \mbox{ r.e. }
  \end{equation}

  In other words, Rice-shapiro can be used to prove that a set is
  \emph{not} r.e., \underline{not} to prove that a set is r.e.
\end{observation}

\begin{counterexample}
  Let
  $\mathcal{A} = \{ f \in \mathcal{C} \mid dom(f) \cap \bar{k} \neq
  \emptyset\}$ , $A = \{x \mid \varphi_x \in \mathcal{A}\}$

  \begin{enumerate}
  \item[(*)] $\mathcal{A}$ satisfies (\ref{eq:star})
    \begin{itemize}
    \item[] \[
        \begin{aligned}
          \mbox{if } f \in \mathcal{A} & \Rightarrow dom(f) \cap \bar{k} \neq \emptyset \\
                                       & \Rightarrow \mbox{ called } x \in dom(f) \cap \bar{k} \mbox{ we have that } \theta = \{(x, f(x))\} \\
                                       & \quad \mbox{ is finite, } \theta \subseteq f \mbox{ and } dom(\theta)\cap \bar{k} = \{x\} \neq \emptyset
        \end{aligned}
      \]

    \item[] \[
        \begin{aligned}
          \mbox{if $\theta$ finite, } \theta \subseteq f, \theta \in \mathcal{A} & \Rightarrow \mbox{ since } \theta \subseteq f \; dom(\theta) \subseteq dom(f) \\
          & \Rightarrow dom(f) \cap \bar{k} \supseteq dom(\theta) \subseteq dom(f) \neq \emptyset \\
          & \Rightarrow f \in \mathcal{A}
        \end{aligned}
      \]
    \end{itemize}

  \item[(**)] $A$ is not r.e., since $\bar{k} \leq_m A$

    we can define
    \[
      g(x,y) = \begin{cases}
        0        & x=y \\
        \uparrow & \mbox{otherwise}
      \end{cases}
    \]
    which is $= \mu z . |x-y|$, and therefore computable. Again,
    for the \smn theorem $\exists s : \nattonat$ computable and total
    s.t.
    \[
      g(x,y) = \varphi_{s(x)}(y)
    \]
    and therefore $dom(\varphi_{s(x)}) = \{x\}$. so
    \begin{itemize}
    \item
      \(x \in \bar{k} \Rightarrow dom(\varphi_{s(x)}) \cap \bar{k} =
      \{x\} \neq \emptyset \Rightarrow s(x) \in A\)
    \item
      \(x \notin \bar{k} \Rightarrow dom(\varphi_{s(x)}) \cap \bar{k}
      = \{x\} = \emptyset \Rightarrow s(x) \notin A\)
    \end{itemize}
  \end{enumerate}
\end{counterexample}

\chapter{First recursion theorem}

In programming languages, we have functions that use other functions as arguments,
e.g. in ML, the function \texttt{succ} that
given a function $f$ returns $f+1$ can be defined as
\begin{equation*}
    fun\ succ\ f\ x = f\ x + 1
\end{equation*}
From the computability point of view it is still somewhat natural to ask
how effective/computable operations can be characterized on
functions. We'll later see that this idea leads to the concept of
\emph{recursive functional}.

\begin{definition}
  We'll call $\mathcal{F}(\nat^k)$ the set of all the functions
  (computable and not) of k arguments $\nat^k \to \nat^k$.

A \emph{functional} is just a function
\[
  \Phi : \mathcal{F}(\nat^k) \to \mathcal{F}(\nat^h)
\]
(only total functions will be considered).
\end{definition}

When can we say that a functional is effective (computable)?
Given $\Phi : \mathcal{F}(\nat^k) \to \mathcal{F}(\nat^h)$
\begin{itemize}
\item a function $f \in \mathcal{F}$ and its image
  $\Phi(f) \in \mathcal{F}(\nat^k)$ are both infinite objects in
  general.
\item we can not ask for $\Phi(f)$ to be computable in a finite time
  from $f$
\end{itemize}
We need a way to see functions as numbers
\subsection{Encoding of finite functions}
For each finite function its' encoding $\tilde{\theta} \in \nat$ is defined
as
\begin{itemize}
\item if $\theta = \emptyset$ then $\tilde{\theta} = 0$
\item if $\theta = \{(x_1, y_1), \dots, (x_2, y_2)\}$ then
  $\tilde{\theta} = \prod_{i=1}^n p_{x_1}^{y_i+1}$
\end{itemize}
which is both injective and effective. Given the encoding of a function
$z= \tilde{\theta}$,
\[
  x \in \dom{\theta} \quad \mbox{iff} \quad (z)_1 \neq \emptyset
\]
\[
  \begin{aligned}
    app(z,x) = \theta(x) & = \begin{cases}
      (z)_x = 1 & x\in\dom{\theta} \\
      \uparrow & \mbox{otherwise}
    \end{cases} \\
    &= ((z)_x \dotdiv 1) \mathds{1}(\mu \omega \; . \; |1 - (z)_x|)
  \end{aligned}
\]

In this way we can give the following definition of recursive functional
\begin{definition}
  A functional
  $\Phi : \mathcal{F}(\nat^k) \to \mathcal{F}(\nat^h)$ is
  \emph{recursive} if there exists $\varphi : \nat^{h+1} \to \nat$
  computable such that, for every
  $f \in \mathcal{F}(\nat^k), \vec{x} \in \nat^h,
  y \in \nat$
  \[
    \Phi(f)(y) = y \quad \mbox{iff} \quad \exists \theta \subseteq f
    \mbox{ finite s.t. } \varphi(\tilde{\theta}, \vec{x}) = y
  \]
\end{definition}

\begin{example}
  The functional
  \newcommand{\fib}[1]{\ensuremath{\mbox{fib}(#1)}}
  \[
    \mbox{fib} : \mathcal{F}(\nat) \to \mathcal{F}(\nat)
  \]
  \[
    \fib{f}(x) = \begin{cases}
      1 & x=0 \lor x=1 \\
      f(x-2) + f(x-1) & x \geq 2
    \end{cases}
  \]
  is recursive, the function $\varphi: \nat^2 \to \nat$ can be
  \[
    \begin{aligned}
      \varphi(z, x) &= \begin{cases}
        1 & x=0 \lor x=1 \\
        \theta(x-2) + \theta(x-1) & x \geq 2 \land z = \tilde{\theta}
      \end{cases} \\
      &= \begin{cases}
        1 & x = 0 \lor x = 1 \\
        app(z, x-2) + app(z, x-1) & x > 2
      \end{cases}
    \end{aligned}
  \]
  which is computable.
\end{example}

\begin{example}
The functional associated to the Ackermann's function is
\[
  \Psi_{ack} : \mathcal{F}(\nat^2) \to \mathcal{F}(\nat^2)
\]
\[
  \begin{cases}
    \Psi_{ack}(f)(0,y)   =  y+1 \\
    \Psi_{ack}(f)(x+1, 0)  =  f(x,1) \\
    \Psi_{ack}(f)(x+1, y+1)   =  f(x, f(x+1, y)) 
  \end{cases}
\]
which is computable.
\end{example}

\begin{theorem}\label{th:unknown}
  Let $\Phi : \mathcal{F}(\nat^k) \to \mathcal{F}(\nat^h)$
  be a recursive functional and let $f \in \mathcal{F}(\nat^k)$ be
  computable. Then $\Phi(f) \in \mathcal{F}(\nat^h)$ is computable
\end{theorem}

\section{Myhill-Sheperdson theorems}
Given a recursive functional $\Phi$, by (\ref{th:unknown})
\[
  f \mbox{ computable } \rightsquigarrow \Phi(f) \mbox{ computable }
\]
\[
  f = \varphi_e \rightsquigarrow \Phi(f) = \varphi_{e'}
\]

so we can see a recursive functional as a function that transforms
indices (programs) in indices (other programs), but with the property
that the transformation depends on the firstly indexed function and
\emph{not} on the index itself.

\begin{definition}[Extensional function]
  Let $h : \nattonat$ a total function. It is \emph{extensional} if
  \[
    \forall e,e' \quad \varphi_e = \varphi_{e'} \to
    \varphi_{h(e)} = \varphi_{h(e')}
  \]
\end{definition}

\begin{theorem}[Myhill-Shepherdson (I)]
  If $\Phi : \mathcal{F}(\nat^k) \to \mathcal{F}(\nat^h)$ is a
  recursive functional then there exists a total computable function
  $h_\Phi : \nattonat$ s.t.
  \[
    \forall e\in \nat \quad \Phi(\varphi_e) = \varphi_{h_\Phi(e)}^{(k)}
  \]
\end{theorem}

Intuitively, the behaviour of the recursive functional on computable
functions is captured by a total extensional function on the
indices.

\begin{theorem}[Myhill-Shepherdson (II)]\label{th:myhill-shepherdson2}
  If $h : \nattonat$ is a total computable extensional function, then
  there is a unique recursive functional $\Phi_h$ sych that
  \[
    \forall e \in \nat \quad \Phi_h(\varphi_e) = \varphi_{h(e)}
  \]
\end{theorem}

A total computable extensional function identifies
only one recursive functional. It is interesting to observe that
$\Phi_n$ exists also for non computable functions.

\begin{theorem}[First recursion theorem (Kleene)]\label{th:first-recursion}
  Let $\Phi : \mathcal{F}(\nat^k) \to \mathcal{F}(\nat^h)$ be
  a \textbf{recursive functional}. Then $\Phi$ has a \textbf{least fixed point}
  $f_\Phi$ which is \textbf{computable}, i.e.
  \begin{enumerate}
  \item $\Phi(f_\Phi) = f_\Phi$
  \item $\forall g \in \mathcal{F}(\nat^k) \quad \Phi(g) = g \Rightarrow f_\Phi \subseteq g$
  \item $f_\Phi$ is computable
  \end{enumerate}
  and we can see that $f_\Phi = \bigcup\limits_n
  \Phi^{n}(\emptyset)$, since a functional can be associated to a
  fixed point; the theorem proves the closure of the set of computable
  functions with respect to extremely general forms of recursion.
\end{theorem}

\begin{example}[Primitive recursion]
Given $f : \nat^h \to \nat$ and
$g : \nat^{h+2} \to \nat$, the function defined by primitive
recursion is the minimum fixed point of
$\Phi_r \in \mathcal{F}(\nat^{h+1})$, defined by
\begin{align*}
    \Phi_r(h)(\vec{x}, 0) & =  f(\vec{x}) \\
    \Phi_r(h)(\vec{x}, y+1) & =  g(\vec{x}, y h(\vec{x},y))
\end{align*}
and if $f,g$ are computable, then $\Phi_r$ is a recursive
functional. The theorem assures that
\begin{itemize}
\item there exists a minimal fixed point;
\item it is computable.
\end{itemize}
\end{example}

\begin{example}[Minimalisation]
Given a function $f: \nat^{k+1} \to \nat$, we can see the
minimization $\mu y \; . \; f(\vec{x}, y)$ as a fixed point. Let us
consider, for a fixed f
\[
  \Phi_\mu \in \mathcal{F}(\nat^{k+1})
\]
\[
  \Phi_\mu(h)(\vec{x}, y) = \begin{cases}
    y & f(\vec{x},y) = 0 \\
    h(\vec{x}, y+1) & f(\vec{x}, y)\downarrow \land f(\vec{x}, y) \neq 0 \\
    \uparrow & \mbox{otherwise}
  \end{cases}
\]
it is recursive and has a minimum fixed point:
\[
  f_{\Phi_\mu}(\vec{x}, y) = \mu (z \geq y) \; . \; f(\vec{x}, y)
\]
\end{example}

\begin{example}[Ackermann's function]
We saw that $\Phi_{ack}$ is recursive, therefore it has a computable
minimum fixed point (the Ackermann function itself $\Psi$). The fact
that $\Psi$ is total, implies that such fixed point is the
\emph{only} fixed point.
\end{example}

\begin{observation}
  Generally speaking, the fixed point is not unique. Counter-example:
  \[
    \Phi(f)(x) = \begin{cases}
      0 & x=0 \\
      f(x+1) & \mbox{otherwise}
    \end{cases}
  \]
  is recursive, and therefore has a minimum fixed point
  \[
    f_\Phi(x) = \begin{cases}
      0 & x=0 \\
      \uparrow & \mbox{otherwise}
    \end{cases}
  \]
  but it has other fixed points, for example:
  \[
    f(x) = \begin{cases}
      0 & x=0 \\
      k & x>0
    \end{cases}
  \]
\end{observation}

\chapter{Second recursion theorem}
Let h be total computable and extensional function. Then, for the
theorem (\ref{th:myhill-shepherdson2}) exists only one recursive
functoinal $\Phi$ such that
\[ \forall e \quad \Phi(\varphi_e) = \varphi_{h(e)} \] Since $\Phi$ is
recursive, for the first recursion theorem (\ref{th:first-recursion})
it has a minimum fixed point $f_\Phi$ computale, and therefore exists
$e\in \nat$ s.t. $f_\Phi = \varphi$. This means that
\[
  \varphi_e = \Phi(\varphi_e) = \varphi_{h(e)}
\]
That is to say that if $h$ is computable total and extensional then it
exists $n$ s.t. \[\varphi_n = \varphi_{h(n)}\] The second recusrsion
theorem states that this is valid also when h is not extensinal.  This
means that a function $h : \nattonat$ cannot be thought as a
functional on computable functions. Instead, we can think that it is a
\emph{function} that is defined on \emph{programs}.

\begin{theorem}[Second recusrsion theorem]\label{th:second-recursion}
  If $h : \nattonat$ a function computable and total, then exists
  $n \in \nat$ s.t.
  \[
    \varphi_{h(n)} = \varphi_n
  \]
  \begin{proof}
    Lets consider the function
    $g(x,y) = \varphi_{h(\varphi_x(x))}(y)$. It is obviously
    computable:
    \[ g(x,y) = \univ(h(\univ(x,x)), y) \] This means that for the
    \smn theorem exists $S : \nattonat$ computable and total s.t.
    \[
      g(x,y) = \varphi_{S(x)}(y) \quad \forall y
    \]
    Since $S$ is computable $\exists m$ s.t. $S = \varphi_m$ and so,
    we can substitute the variables
    \[
      \varphi_{h(\varphi_x(x))}(y) = \varphi_{\varphi_m(x)}(y) \quad \forall y
    \]
    therefore, if $m=x$
    \[
      \varphi_{h(\varphi_m(m))}(y) = \varphi_{\varphi_m(m)}(y) \quad \forall y
    \]
    and if we consider $n = \varphi_m(m)$ (wich is defined since
    $S = \varphi_m$ is total)
    \[
      \varphi_{h(n)}(y)  = \varphi_n(y) \quad \forall y
    \]
    that is to say
    \[
      \varphi_{h(n)} = \varphi_n
    \]
  \end{proof}
\end{theorem}

This theorem can therefore be interpreted in the following manner:
``given any effective procedute to transform programs, exists a
program such that when properly modified does exactly what it did
before''.

Or again ``it is impossible to write a program that edits the core of
all programs''.

\textbf{Note:} the proof can appear misterious, but to a closer
inspection clearly appears to be a simple diagonalization.

\textbf{Note again:} the result of this theorem is extremely
deep. This way, many theorems we've seen up until now are just
corollaries of it:

\begin{corollary}[Rice theorem]
  Let $\emptyset \neq A \subsetneq \nat$. If $A$ is saturated, then
  $A$ is not rich.
  \begin{proof}
    Let $\emptyset \subsetneq A \subsetneq \nat$ and suppone that $A$
    (and therefore $\bar{A}$) is recursive. This means that
    $\chi_A, \chi_{\bar{A}}$ are computable. Now, let
    $a_0 \in \bar{A}, a_1 \in A$ be fixed elements (they clearly
    exist, since A is not trivial). We can define
    \[
      f(x) = \begin{cases}
        a_0 & x \in A \\
        a_1 & x \in \bar{A} 
      \end{cases} = a_0\chi_A(x) + a_1\chi_{\bar{A}}(x)
    \]
    it's  easy to see that
    \begin{equation}\label{eq:xinafxinbara}
      x\in A \mbox{ iff } f(x) \in \bar{A}
    \end{equation}
    Since $f$ is computable and total, for the second recursion
    theorem (\ref{th:second-recursion}) $\exists n$ s.t.
    $\varphi_n = \varphi_{f(n)}$ and $A$ is saturated, wich means that
    either $n, f(n) \in A$ or $n, f(n) \in \bar{A}$, wich contraddicts
    \ref{eq:xinafxinbara}.
  \end{proof}
\end{corollary}

\begin{corollary}
  $K$ is not saturated
  \begin{proof}
    Let $k = \{ x \mid x \in W_x\}$ recursive for the sake of the
    argument. and let $e_0, e_1$ be indexes s.t.
    $\varphi_{e_0} = \emptyset$ and
    $\varphi_{e_1} = \lambda x \; . \; x$. We can define
    \[
      f(x) = \begin{cases}
        e_0 & x \in K \\
        e_0 & x \notin K
      \end{cases}
    \]
    $f$ is both computable and total, therefore according to
    (\ref{th:second-recursion})
    $\exists n \; . \; \varphi_n = \varphi_{f(n)}$, but:
    \[
      \begin{tabu}{l l l l l l l}
        n \in K & \Rightarrow & \varphi_n = \varphi_{f(n)} = \varphi_{e_0} & \Rightarrow & \varphi_n(n) = \varphi_{e_0} (n) \uparrow & \Rightarrow & n \notin K \\
        n \notin K & \Rightarrow & \varphi_n = \varphi_{f(n)} = \varphi_{e_1} & \Rightarrow & \varphi_n(n) = \varphi_{e_1} (n) = n & \Rightarrow & n \in K \\
      \end{tabu}
    \]
    wich is absurd.
  \end{proof}
\end{corollary}

\begin{corollary}
  $K$ is not saturated
  \begin{proof}
    Let $n_0$ s.t. $\varphi_{n_0} = \{(n_0, n_0)\}$. We know that
    there are infinitely many indeces for the same function; so let
    $n \neq n_0$ s.t. $\varphi_n = \varphi_{n_0}$.
    \[
      \varphi_n(n) \uparrow \Rightarrow n \notin K
    \]
  \end{proof}
\end{corollary}

\printbibliography

\end{document}
%%% Local Variables:
%%% mode: latex
%%% TeX-master: t
%%% End: