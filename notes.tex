\documentclass{amsbook}
\usepackage[utf8]{inputenc}
\usepackage[english]{babel}
\usepackage{csquotes}
\usepackage{hyperref}
\usepackage{amsmath}
\usepackage{dsfont}
\usepackage{amsfonts}
\usepackage{amssymb}
\usepackage{graphicx}
\usepackage{amsfonts}
\usepackage{parskip}
\usepackage{enumitem}
\usepackage{tabu}
\usepackage{listings}
\usepackage{xcolor}
\usepackage{mathabx}
\usepackage{quiver}

\hypersetup{
	colorlinks=false,
	linkcolor={blue!50!black}
}
\usepackage{mathtools}
\usepackage[
backend=biber,
style=alphabetic,
sorting=ynt
]{biblatex}
\addbibresource{bibliography.bib}

\DeclarePairedDelimiter\ceil{\lceil}{\rceil}
\DeclarePairedDelimiter\floor{\lfloor}{\rfloor}


% macros

% natural numbers
\newcommand{\nat}{\ensuremath{\mathbb{N}}}

% domain
\newcommand{\dom}[1]{\ensuremath{\mathit{dom}({#1})}}

% comment (in programs)
\newcommand{\comment}[1]{{\texttt{// #1}}}

% theorems
\newtheorem{theorem}{Theorem}[chapter]
\newtheorem{lemma}[theorem]{Lemma}
\newtheorem{corollary}[theorem]{Corollary}
\newtheorem{proposition}[theorem]{Proposition}

\theoremstyle{definition}
\newtheorem{definition}[theorem]{Definition}
\newtheorem{example}[theorem]{Example}
\newtheorem{exercise}[theorem]{Exercise}
\newtheorem{observation}[theorem]{Observation}

\theoremstyle{remark}
\newtheorem{remark}[theorem]{Remark}
\newtheorem{notation}[theorem]{Notation}
\newtheorem{counterexample}[theorem]{Counterexample}

\numberwithin{section}{chapter}
\numberwithin{equation}{chapter}

%    For a single index; for multiple indexes, see the manual
%    "Instructions for preparation of papers and monographs:
%    AMS-LaTeX" (instr-l.pdf in the AMS-LaTeX distribution).
\makeindex

\begin{document}

\frontmatter

\title{Computability}
  
\author{Prof. Paolo Baldan}
\address{}
\curraddr{}
\email{}
\thanks{}

\keywords{}

\date{}

\begin{abstract}
\end{abstract}

\maketitle

%    Dedication.  If the dedication is longer than a line or two,
%    remove the centering instructions and the line break.
%\cleardoublepage
%\thispagestyle{empty}
%\vspace*{13.5pc}
%\begin{center}
%  Dedication text (use \\[2pt] for line break if necessary)
%\end{center}
%\cleardoublepage

%    Change page number to 6 if a dedication is present.
\setcounter{page}{4}

\tableofcontents


\mainmatter

\chapter{Introduction}\label{chap:intro}

In this chapter, we informally discuss the notion of \textbf{effective procedure} and \textbf{computable function} by means of an effective
procedure. This will lead us to single out the main features of an
algorithm/computational model.  Despite being informal, these
considerations will allow us to derive the existence of non-computable
functions for any effective computational model chosen.
%
Subsequently these notions and
considerations will be formalized by setting a specific computational model,
a kind of idealized computer.

\section{Algorithm or effective procedure}

Even though we do not always refer to them by their technical terms when we apply them, \textbf{effective procedure}s and \textbf{algorithms} are a part of our everyday life.

For example; at the primary school we are not only taught that given two numbers their sum exists, but we are also provided a procedure to compute the sum of two numbers!

In general terms, an \emph{algorithm} can be defined as the
description of a sequence of \emph{elementary steps} (where
``elementary'' means that they can be performed mechanically, without
any intelligence) which allows one reach some objective.  Typically,
the aim is transforming some input into a corresponding output,
suitably related to the input.
%
This could be transforming ingredients into a cake, although normally
we are interested in computational problems.

\begin{example}
  Some examples are:
  \begin{enumerate}

  \item given $n \in \nat$, establishing whether $n$ is prime;
  \item finding the $n^{th}$ prime number;
  \item differentiating a polynomial;
  \item performing the square root $\sqrt{n}$;
  \item finding least common multiple $lcm$ and greatest common divisor $GCD$.
  \end{enumerate}
\end{example}

Therefore we can think of an algorithm as a black box
\begin{center}
  in $\rightarrow$ \boxed{black box} $\rightarrow$ out
\end{center}
where the transformation is performed by executing a sequence of
mechanical instructions.

If each step is \emph{deterministic} (i.e., for each state of the
system, the instruction to execute and the new state it produces are
uniquely determined), then each possible input will uniquely determine
the corresponding output (the procedure might not terminate, in which case we will have no output).

In mathematical terms the algorithm induces a (partial) function

\begin{center}
  $f : \mathit{input} \rightarrow \mathit{output}$.
\end{center}

We say that $f$ is the function \emph{computed} by the algorithm and
that $f$ is effectively computable. Thus, we can give the following
first definition of an algorithm (still informal since it refers to a
generic notion of algorithm).

\begin{definition}[Computable function]
  A function $f$ is \emph{computable} if there exists an algorithm
  that computes $f$.
\end{definition}

We stress that for $f$ to be computable, it is not important to know the algorithm that computes $f$, but rather we need to know that some algorithm that computes $f$ exists.

\begin{example}
  According to the above definition, we expect the the following
  functions to be computable.

  \begin{itemize}

  \item GCD (greatest common divisor), e.g., exploiting Euclid's
    algorithm.

  \item the function $f : \nat \to \nat$ defined as

    \begin{equation*}
      f(n)=
      \begin{cases}
        1 & n \mbox{ prime} \\
        0 &   \mbox{otherwise}
      \end{cases}
    \end{equation*}

  \item
    $g(n) = p_n$\\
    where $p_n$ is the $n$-th prime number.

    This is computable by generating numbers
    and testing for primality until the $n$-th prime is found.


  \item
    $h(n) = \pi_n$
    where $\pi_n$ is the $n$-th digit of the decimal representation of $\pi$.

    Indeed there are
    \begin{itemize}
    \item series that converge to $\pi$
    \item techniques to estimate (by excess) the error caused by
      \begin{itemize}
      \item truncating a series
      \item rounding in the calculation of the truncated series value
      \end{itemize}
    \end{itemize}
  \end{itemize}
\end{example}

What about the function below?

\begin{equation*}
  g(n) = \begin{cases}
    1 & $ there is a sequence of exactly  \textit{n} consecutive $5$'s in $ \pi \\
    0 & $ otherwise $
	\end{cases}
\end{equation*}

For example $g(3) = 1$ if and only if  $\pi = 3.14 \dots k 555 h \dots $, with $k, h \neq 5$.

A naive algorithm could be generate the digits of $\pi$ until a sequence of $5$'s of the desired length $n$ is found.
Clearly, if such a sequence exists, it will be eventually found and the answer $1$ will be returned. 
However, at no time, can we say that a desired sequence of $n$ $5$'s will never appear again later; hence, we currently have no way of returning $0$.

\begin{remark}
  Thus, the following
   \begin{itemize}
   \item generate all digits in the decimal representation of $\pi$;
   \item if they include a sequence of $n$ consecutive 5's then
    $g(n) = 1$
  \item otherwise $g(n) = 0$
\end{itemize}
is \emph{not} an effective procedure.
\end{remark}

Note that this doesn't mean that $g$ is not computable, i.e., that an
effective procedure couldn't be found, but at the moment this procedure is not known!

We don't really know if it's computable, but there might be a property
of $\pi$ that allows us to conclude. In particular, there is a conjecture that all
finite sequences of digits appear in $\pi$, which would imply that $g$
is the constant $1$, whence computable.

\medskip

Consider now a slightly different function $h: \nat \to \nat$, defined by
\begin{equation*}
  h(n) = \begin{cases}
    1 & $ there is a sequence of at least  \textit{n} consecutive $5$'s in $ \pi \\
    0 & $ otherwise $
  \end{cases}
\end{equation*}

The function seems very similar to the one considered before. However, note that if $\pi = 3.14 \dots k 555 h \dots $, then we deduce, not only that $h(3)=5$, but also $h(2)=h(1)=h(0)=1$. 
More generally, whenever $h(n) =1$ then $h(m)=1$ for all $m < n$. This suggests that $h$ could be quite simple.

More precisely, consider $K = sup\{ n \mid \pi\ \text{contains}\ n\ \text{consecutive digits}\ 5 \}$. Then we have 2 possibilities:
\begin{enumerate}
\item $K$ is finite, and thus $h(n) = 1$ if $ n\leq K$, $0$ otherwise
\item $K$ is infinite, and thus $ h(n) = 1$ for all $n \in \nat$
\end{enumerate}

This implies that $h$ is computable because it is either a step
function or a constant function, that are computed by simple
programs. One could object that we don't know which shape the function
has and thus we do not know the program that calculates the
function. Fine. This doesn't mean that it's not computable.

Trying to repeat the same argument for function $g$ fails. In fact, one could think of defining $A = \{n \mid \mbox{$\pi$ contains exactly $n$ consecutive 5's}\}$. Then

\begin{equation*}
  g(n) =
    \begin{cases}
      0 & n \in A     \\
      1 & n \not\in A
    \end{cases}
\end{equation*}

This does not suggest that $g$ is computable. Set $A$ is possibly infinite and we do not see a way of providing a finite representation of $A$ which can be included in a program.

Bringing the argument to the extreme, one could consider the function
$G : \nat \to \nat$ defined by
\begin{center}
  $G(x) = \begin{cases}
    1 & $ if God exists $ \\
    0 & $ otherwise $
  \end{cases}
  $
\end{center}
Since the condition does not depend on the variable $x$, the function is either the constant $0$ or the constant $1$. 
Independently of which of the two cases applies, the function is computable.



\chapter{Algorithms and existence of non-computable functions}

\section{Characteristics of an algorithm}
\label{se:alg-char}

We present a list of features that an algorithm should satisfy in
order to capture the intuitive idea of \textbf{effective procedure}. Roughly,
what we will ask is that it is ``implementable'' on some sort of
idealised machine, the computational model. Hence, in turn, we will
list some requirements that the computational model should respect to be
considered effective.

An \textbf{algorithm} is a sequence of instructions with the following characteristics:
\begin{enumerate}[label=\alph*)]
\item
  \label{as:prog_fin}
  it is of \textbf{finite length};
\item there exists a \textbf{computing agent} able to execute its instructions;
\item the agent has available \textbf{memory} (to store input, intermediate results to be used in subsequent steps and output);
\item the computation consists of discrete steps;
\item the computation is neither non-deterministic nor probabilistic (we model a digital computer);

\item there is no limit to the size of the input data\\
  (we want to be able to define algorithms that work on any possible
  input, e.g. $+$ operating on summands of any size);

\item there is no limit to the memory that can be used.\\
  This requirement may seem less natural, but having unlimited memory is essential to avoid the notion of computability being dependent
  from available resources. In fact, for many
  functions the space required for intermediate results depends on
  the size of the input,

  e.g. $f(n) = n^2$ $(1000)^2 = 1000000 \leftarrow$ I must add a
  number of zeroes that depends on $n \rightarrow n$ must be stored
  (the states are finite);

\item
  \label{as:istr_fin}
  there exist a finite limit to the number of the
  instructions and to their complexity.

  This is intended to capture the intrinsic finiteness of the
  calculation device (justified by Turing with the limits of the human
  mind/memory),

  e.g. for a computer, the memory that can be accessed with a single
  instruction must be finite (even if by (g), the memory is unlimited);

\item computations might
  \begin{enumerate}

  \item  end and return a result after a finite, but unlimited number of steps
    (e.g. the square function requires a number of steps proportional to the argument);

  \item continue forever, and not return a result.
  \end{enumerate}
\end{enumerate}

\section{Existence of non-computable functions}

Later on, we will focus on a concrete computational model that will allow
us to give a completely formal definition of computable function. Simply on the basis of the assumptions above, we observe that we can infer the existence of non-computable functions for every ``effective'' computational model.

We start by recalling some basic notions and introducing useful notation.

\begin{itemize}
\item We will consider the set of \emph{natural numbers}
  $\nat = \{ 0, 1, 2, \dots \}$;

\item Given the sets $A, B$ their \emph{Cartesian product} is
  $A \times B = \{ (a,b) \mid a \in A\ \land\ b \in B\}$. We will
  write $A^n$ for $A \times A \times A \times \ldots \times A$ $n$
  times. Thus, we have $A^1 = A$ and $ A^{n+1} = A \times A^n$.

\item A (binary) \emph{relation} or \emph{predicate} is
  $r \subseteq A \times B$.

\item A \emph{(partial) function} $f : A \to B$ is a special relation $f \subseteq A\times B$ such that if $(a, b_1), (a, b_2) \in f$ then  $b_1 = b_2$.  Following the standard convention, we will write $f(a) = b$ instead
  of $(a, b)\in f$
  \begin{itemize}
  \item the \emph{domain} of $f$ is
    $\dom{f} = \{a \mid \exists b \in B.\ f(a) = b \}$;

  \item we write $f(a) \downarrow$ for $a \in dom (f)$ and
    $f(a) \uparrow$ for $a \not\in dom (f)$;
  \end{itemize}

\item Given a set $A$ we indicate with $|A|$ its \emph{cardinality}
  (intuitively, the number of elements of $A$, but the notion extends
  to infinite sets). Given the sets $A$ and $B$ we have
  \begin{itemize}
  \item $|A| = |B|$ if there exists a bijective function $f : A \to B$;
  \item $|A| \leq |B|$ if there exists an injective function
    $f: A \to B$ or equivalently\footnote{Stritly speaking,
      the equivalence requires the axiom of choice.} a surjective
    function $g : B \to A$.
  \end{itemize}
  Observe that if $A \subseteq B$ then $|A| \leq |B|$ as witnessed by
  the inclusion, which is an injective function
  \begin{center}
    $\begin{array}{cc}
       i: & A \to B  \\
          & a \mapsto a
     \end{array}$
   \end{center}

\item We say that $A$ is \emph{countable} or \emph{denumerable} when
  $|A| \leq |\nat|$, i.e., there is a surjective function
  $f: \nat \to A$. Note that, when this is the case, we can
  list (enumerate, whence the name) the elements of $A$ as
  \begin{center}
    $\begin{array}{cccc}
      f(0) & f(1) & f(2) & \dots\\
      a_0  & a_1  & a_2 & \dots
    \end{array}
    $
  \end{center}

\item When $A, B$ are countable then $A\times B$ is countable.

  Idea of the proof:
  \begin{itemize}
  \item Since $A$ and $B$ are countable, we can consider the
    corresponding enumerations

    \begin{quote}
      $
      \begin{array}{cccc}
        A & a_0 & a_1 & a_2 \\
        B & b_0 & b_1 & b_2
      \end{array}
      $
    \end{quote}
    and place the elements of $A \times B$ in a matrix
    \begin{center}
      $
      \begin{tabu}{c|ccc}
        & b_0       & b_1       & b_2       \\
        \hline
        a_0 & (a_0,b_0) & (a_0,b_1) & (a_0,b_2) \\
        a_1 & (a_1,b_0) & (a_1,b_1) & (a_1,b_2) \\
        a_2 & (a_2,b_0) & (a_2,b_1) & (a_2,b_2)
      \end{tabu}
      $
    \end{center}
    so that they can be enumerated along the diagonals
    as follows:\\
    $(a_0,b_0), (a_0,b_1), (a_1,b_0), (a_0,b_2), (a_1,b_1), (a_2,b_0),
    \dots$ (this is referred to as \emph{dove tail} enumeration)
  \end{itemize}


\item A countable union of countable sets is countable: if
  $\{A_i\}_{i\in\nat}$ is a collection of countable sets then
  $\bigcup \limits_{i \in \nat} A_i$ is countable.
\end{itemize}

\section{Existence of non-computable functions in each computational model}\label{se:existence-non-2}

Let us consider some fixed computational model satisfying the
assumptions in \S\ref{se:alg-char}. We want to show that there are
functions which are not computable in such a model.

We focus on unary functions over the natural numbers. Let
$\mathcal{F} = \{f \mid f:\nat\rightarrow\nat\}$ be the set of all the
(partial) unary functions on $\nat$.

Let $\mathcal{A}$ be the set of all algorithms in our fixed
computational model.
%
Every algorithm $A \in \mathcal{A}$ computes a function
$f_A: \nat \to \nat$ and a function is said to be computable in our model if
there exists an algorithm that computes it. Hence the set
$\mathcal{F}_\mathcal{A}$ set of computable functions in the given computational
model is
\begin{equation*}
  \mathcal{F}_{\mathcal{A}} = \{ f_A \mid A \in \mathcal{A} \}
\end{equation*}
Certainly $\mathcal{F}_\mathcal{A} \subseteq \mathcal{F}$. But, is the inclusion
strict (i.e., is there a non-computable function)?

The answer is yes. Basically, the algorithms are too few to compute all the
functions for combinatorial reasons.

In fact, an algorithm $A \in \mathcal{A}$ will be a finite sequence of instructions taken from
some instruction set $I$, by
assumption \ref{as:prog_fin}. Moreover, by assumption \ref{as:istr_fin},
$I$ must be finite. Hence:
\begin{equation*}
  \mathcal{A} \subseteq \bigcup_{i \in \nat} I^n
\end{equation*}
Since a countable union of finite (hence countable) sets is countable, we have
\begin{equation*}
  |\mathcal{A}| \leq |\bigcup_{n\in\nat} I^n| \leq |\nat|
\end{equation*}
and since the function
\begin{align*}
  \mathcal{A} & \to \mathcal{F}_\mathcal{A}\\
  A & \mapsto f_A
\end{align*}

is surjective by definition, we have that
\begin{equation*}
    |\mathcal{F}_\mathcal{A}| \leq |\mathcal{A}| \leq |\nat|
\end{equation*}

On the other hand the set of all functions, $\mathcal{F}$, is not countable. Let $\mathcal{T}$ the subset of $\mathcal{F}$ consisting of the total functions $\mathcal{T} = \{ f \mid f \in \mathcal{F}\ \land\ \dom{f} = \nat\}$. We show that
\begin{equation*}
  |\mathcal{F}| \geq |\mathcal{T}| > |\nat|
\end{equation*}

We prove that $|\mathcal{T}| > |\nat|$ by contradiction. Let us suppose that $\mathcal{T}$ is countable. 
Then we can consider an enumeration $f_0, f_1, f_2, \ldots$ of $\mathcal{F}$ such as the following matrix
\begin{center}
  \begin{tabular}{c|ccc}
    & $f_0$    & $f_1$    & $f_2$\\
    \hline
    0 & $f_0(0)$ & $f_1(0)$ & $f_2(0)$ \\
    1 & $f_1(0)$ & $f_1(1)$ & $f_1(2)$ \\
    2 & $f_2(0)$ & $f_2(1)$ & $f_2(2)$
  \end{tabular}
\end{center}
and build a function $d$ using the diagonal values 
\begin{align*}
  d :\ & \nat \to \nat\\
  & n \mapsto f_n(n)+1
\end{align*}

We can observe that
\begin{itemize}
\item $d$ is total, by definition;
\item $d \neq f_n$ for all $n \in \nat$, since $d(n) = f_n(n)+1 \neq f_n(n)$.
\end{itemize}
This is absurd, since $f_0, f_1, f_2, \ldots$ is an enumeration of all the total functions.

\medskip

Summing up
\begin{gather*}
  \mathcal{F}_\mathcal{A} \subseteq \mathcal{F}\\
  |\mathcal{F}_\mathcal{A}| \leq |\nat| < |\mathcal{T}| = |\mathcal{F}|
\end{gather*}
we get $\mathcal{F}_\mathcal{A} \subset \mathcal{F}$, as desired.

Note that the set of non-computable functions is not countable
\begin{equation*}
  |\mathcal{F} \setminus \mathcal{F}_{\mathcal{A}}| > |\nat|
\end{equation*}
In fact, $\mathcal{F} = \mathcal{F}_{\mathcal{A}} \cup (\mathcal{F} \setminus \mathcal{F}_{\mathcal{A}})$. 
Thus, if it were to be $|\mathcal{F} \setminus \mathcal{F}_{\mathcal{A}}| \leq |\nat|$, we would have had $|\mathcal{F}| \leq |\nat|$ because union of countable sets is countable.

We conclude that
\begin{enumerate}
\item no computational model can compute all functions;
\item there are more non-computable than computable functions.
\end{enumerate}

\chapter{URM computability}

\section {Which model?}

To give a formal notion of computability we must choose a concrete model of computation that induces a class of algorithms and therefore of computable functions. 
Despite the fact that we focus on an abstract ideal model, there are still a lot of possibilities. Many models have been considered in the literature:

\begin{enumerate}
\item Turing machine (Turing, 1936)
\item $\lambda$-calculus (Church, 1930)
\item Partial recursive functions (Godel-Kleene 1930)
\item Canonical deductive systems (Post, 1943)
\item Markov systems (Markov, 1951)
\item Unlimited register machine (URM) (Shepherdson - Sturgis, 1963)
\end{enumerate}

In principle, each computational model determines a class of computable functions.
We may be concerned thinking that the developed theory is valid only for a specific one
model. In fact, it can be verified that the class of computable functions for all
models cited (for all the models "sufficiently expressive" considered
in literature) is always the same. This leads to the so-called Church-Turing
thesis:

\textbf{Church-Turing thesis}: A function is computable by an
effective procedure (i.e., in a finitary computational model, obeying the conditions (a)-(e) from the chapter before) if and only if it is computable
by a Turing machine.


This means that the notion of ``computable function'' is robust (i.e. independent of the specific computational model), and we can choose our favorite one for developing our theory.

\begin{remark}
  The \emph{Church-Turing thesis} is called a thesis and not a theorem due
  of its informal nature. 
  It cannot be proved w.r.t effective procedures, but is supported only by evidence: 
  several computational models have been considered and all respect the thesis 
  (e.g. Yuri Gurevich, argues that it should be proved on the basis of a formal
  axiomatization of conditions (a) - (e)).
\end{remark}

Sometimes we resort to the Church-Turing thesis to shorten the proof that a certain entity is computable, however it can only be used when it is not strictly necessary, i.e. when it could be replaced by a formal proof
(however, this could hide the intuitive idea under a bunch of technical details).

\section{URM (Unlimited register machine)}

We will formalise the notion of \textbf{computable function} by using an \textbf{abstract machine} 
called \textbf{URM-machine} (Unlimited Register Machine), 
which is an abstraction of a computer based on the Von Neumann's model. It is characterized by

\begin{itemize}
\item \textbf{unbounded memory} that consists of a infinite sequence of \textbf{registers}, each of which can store a  natural number


  $\begin{tabu}{|c|c|c|c|c|}
    \hline
    R_1 & R_2 & \dots & R_n & \dots \\
    \hline
    r_1 & r_2 & \dots & r_n & \dots \\
    \hline
  \end{tabu}$

  the $n$-th register is indicated with $R_n$, its content with $r_n$

  the sequence $r_1, r_2,\dots, r_n, \dots \in \nat^w$ is called
  \textbf{configuration} of the URM;

\item a \textbf{computing agent} capable of executing an URM program;

\item an execution of a \textbf{URM program}, i.e. a finite sequence of instructions
  $I_1, I_2, \dots, I_s$ that can ``locally'' alter the configuration
  of the URM.
\end{itemize}


Program instructions can be the following

\begin{itemize}

\item \textbf{zero} $Z(n)$ sets the content of the register $R_n$ to zero: $r_n \leftarrow 0$;

\item \textbf{successor} $S(n)$ increments the content of the $R_n$ register by 1: $r_n \leftarrow r_n+1$;

\item \textbf{transfer} $T(m,n)$ transfers the content of the register $R_m$ in the register $R_n$, $R_m$ stays untouched: $r_n\leftarrow r_m$.
\end{itemize}
These are called arithmetic functions, where the instruction to be executed in the next step
 follows the current instruction in the program.
Then there is another instruction
\begin{itemize}
\item \textbf{conditional jump} $J(m,n,t)$ compares the content of the registers $R_m$ and $R_n$
  \begin{itemize}
  \item if $r_m = r_n$ it jumps to the $t$-th instruction;
  \item otherwise, it continues with the next instruction.
  \end{itemize}
\end{itemize}


\begin{example}
  An example of program is the following:
  \begin{quote}
    \begin{tabular}{llr}
      $I_1$: & J(2,3,5) &                       \\
      $I_2$: & S(1)     &                       \\
      $I_3$: & S(3)     &                       \\
      $I_4$: & J(1,1,1) &  \comment{unconditional jump}
    \end{tabular}
  \end{quote}

  Disregard what this program computes for the moment. The computation starting from the configuration below is:

  \begin{center}
    $\begin{tabu}{|c|c|c|c|}
      \hline
      R_1 & R_2 & R_3 & \dots \\
      \hline
      1   & 2   & 0   & \dots \\
      \hline
    \end{tabu}
    %
    \xrightarrow{I_1, I_2}
    %
    \begin{tabu}{|c|c|c|c|}
      \hline
      R_1 & R_2 & R_3 & \dots \\
      \hline
      2   & 2   & 0   & \dots \\
      \hline
    \end{tabu}
    %
    \xrightarrow{I_3}
    %
    \begin{tabu}{|c|c|c|c|}
      \hline
      R_1 & R_2 & R_3 & \dots \\
      \hline
      2   & 2   & 1   & \dots \\
      \hline
    \end{tabu}
    %
    \xrightarrow{I_4, I_1, I_2}
    %
    \begin{tabu}{|c|c|c|c|}
      \hline
      R_1 & R_2 & R_3 & \dots \\
      \hline
      3   & 2   & 1   & \dots \\
      \hline
    \end{tabu}
    %
    \xrightarrow{I_3}
    %
    \begin{tabu}{|c|c|c|c|}
      \hline
      R_1 & R_2 & R_3 & \dots \\
      \hline
      3   & 2   & 2   & \dots \\
      \hline
    \end{tabu}
    \xrightarrow{I_4, I_1, I_5}
    $
  \end{center}
\end{example}


The \textbf{state} of the URM machine in which it executes a program $P = I_1 \dots I_s$ is given by a pair $\langle c, t \rangle$ that consists of;

\begin{itemize}
\item \emph{register configuration} $c: \nat \rightarrow \nat$\\
  a total function $c : \nat \to \nat$ such that $c(n)$ is the content
  of register $R_n$;

\item \emph{program counter} $t$, i.e., index of the current instruction.
\end{itemize}

An \emph{operational semantics} could easily be defined via a set of deduction rules axiomatising the state transitions  $\langle c, t \rangle \rightarrow \langle c', t' \rangle$. However we do not need this degree of formality, and we will rely on an informal description of the program execution.


\begin{remark}[non termination]
  A computation might \textbf{not terminate}! Consider for instance the program:

  \begin{quote}
    \begin{tabular}{ll}
      $I_1$: & S(1)     \\
      $I_2$: & J(1,1,1)
    \end{tabular}
  \end{quote}

  Then none of the computations will terminate. For instance:
  \begin{center}
    $\begin{tabu}{|c|c|c|c|}
      \hline
      R_1 & R_2 & R_3 & \dots \\
      \hline
      0  & 0   & 0   & \dots \\
      \hline
    \end{tabu}
    %
    \xrightarrow{I_1, I_2}
    %
    \begin{tabu}{|c|c|c|c|}
      \hline
      R_1 & R_2 & R_3 & \dots \\
      \hline
      1   & 0   & 0   & \dots \\
      \hline
    \end{tabu}
    %
    \xrightarrow{I_1, I_2}
    %
    \begin{tabu}{|c|c|c|c|}
      \hline
      R_1 & R_2 & R_3 & \dots \\
      \hline
      2   & 0  & 0   & \dots \\
      \hline
    \end{tabu}
    %
    \xrightarrow{\ldots}
    %
    $
  \end{center}
\end{remark}


\begin{notation}
  Let $P$ be an URM program, and $a_1,a_2,a_3,\dots \in \nat^w$ a sequence
  of natural numbers. We indicate the computation of $P$ starting from the
  initial configuration by $P(a_1,a_2,\dots)$:

  \begin{center}
    $\begin{tabu}{|c|c|c|c|}
      \hline
      R_1 & R_2 & R_3 & \dots \\
      \hline
      a_1 & a_2 & a_3 & \dots \\
      \hline
    \end{tabu}$
  \end{center}

  and

  \begin{itemize}
  \item $P(a_1,a_2,\dots) \downarrow$ if the computation \textbf{halts}.
  \item $P(a_1,a_2,\dots) \uparrow$ if the computation \textbf{never
      halts} (i.e. it \textbf{diverges}).
  \end{itemize}


  We will work on computations that start from an initial configuration
  where only a \textbf{finite number of registers contain a non-zero value} for
  the majority of the time (almost always for obvious reasons of input
  finiteness). Hence; given $a_1,a_2,\dots,a_k \in \nat$ we write:

  \begin{quote}
    $P(a_1,\dots,a_k)$ for the computation
    $P(a_1,\dots,a_k,0,\dots,0)$
  \end{quote}
  The notation extend to $P(a_1,\dots,a_k)\downarrow$ or
  $P(a_1,\dots,a_k)\uparrow$.
\end{notation}

\section{URM-computable functions}

Let $f : \nat^k \rightarrow \nat$ be a partial function. What does it mean for  $f$ to be computable by an URM machine?

Intuitively, it means that there exists a program $P$ s.t. for each $(a_1,\dots,a_k) \in \nat^k \quad P(a_1,\dots,a_k)$ computes the value of $f$. i.e. when $(a_1,\dots,a_k)  \in \dom{f}$, $P$ terminates and outputs $f(a_1, \ldots, a_k)$. However, $P$ does not terminate if $(a_1,\dots,a_k) \not\in \dom{f}$.

A doubt could still concern the place where the output is stored. We conventionally decide that the output will be in the first register $R_1$ (at the end of the computation, any registers other than the first register are basically just garbage to us \dots). For this reason we introduce the following notation.

\begin{notation}
  Let $P$ be a program, $a_1,\dots,a_k \in \nat^k$, we write
  $P(a_1,\dots,a_k)\downarrow a$ if $P(a_1,\dots,a_k) \downarrow$ and
  the final configuration contains $a$ in $R_1$
\end{notation}

\begin{definition}[URM-computable function]
  A function $f:\nat^k\rightarrow\nat$ is said to be
  \textbf{URM-computable} if there exists a URM program $P$ s.t.
  $\forall (a_1,\dots,a_k) \in \nat^k, a\in\nat$,
  $P(a_1,\dots,a_k)\downarrow$ iff $(a_1,\dots,a_k)\in dom(f)$ and
  $f(a_1,\dots,a_k) = a$. In this case we say that $P$ computes $f$.

  We denote by $\mathcal{C}$ the class of all URM-computable
  functions and by $\mathcal{C}^{(k)}$ the class of the k-ary
  URM-computable function.
  Therefore we have
  $\mathcal{C} = \bigcup_{k\geq 1} \mathcal{C}^{(k)}$
\end{definition}

FIN QUA

\section{Some examples of URM-computable functions}

\begin{enumerate}
\item $f:\nat^2 \rightarrow \nat$\\
  $ f(x,y) = x+y$

  \begin{quote}
    \begin{tabular}{lll}
      $I_1$: & J(2,3,5) &                    \\
      $I_2$: & S(1)     &                    \\
      $I_3$: & S(3)     &                    \\
      $I_4$: & J(1,1,1) &  \comment{unconditional jump}
    \end{tabular}
  \end{quote}

  \begin{center}
    $\begin{tabu}{|c|c|c|c|}
      \hline
      R_1 & R_2 & R_3 & \dots \\
      \hline
      x   & y   & 0   & \dots \\
      \hline
    \end{tabu}$
  \end{center}

  \emph{Idea}: Increment $R_1$ and $R_3$ until $R_2$ and $R_3$ contain
  the same value. This results in adding to $R_1$ the content of
  $R_2$.

\item $f:\nat \rightarrow \nat$\\
  $f(x) = x\dot{-}1 = \begin{cases} 0 & x=0 \\ x-1 & x>0 \end{cases}$

  \begin{center}
    $\begin{tabu}{|c|c|c|c|}
      \hline
      R_1 & R_2 & R_3 & \dots \\
      \hline
      x   & 0   & 0   & \dots \\
      \hline
    \end{tabu}$
  \end{center}

  \emph{Idea}: if $x=0$ ok, end; if $x>0$ keep a value $k-1$ in
  $R_2$ and $k$ in $R_5$, with $k>1$ ascending until $R_3=x$, at that
  point $R_2 = x-1$

  Here's the program

  \begin{quote}
    \begin{tabular}{lll}
      $I_1$: & J(1,4,8) \\
      $I_2$: & S(3)     \\
      $I_3$: & J(1,3,7) \\
      $I_4$: & S(2)     \\
      $I_5$: & S(3)     \\
      $I_6$: & J(1,1,3) \\
      $I_7$: & T(2,1)   \\
    \end{tabular}
  \end{quote}


\item $f:\nat \rightarrow \nat$\\
  $f(x) = \begin{cases}
    \frac{1}{2} x & \mbox{if $x$ even}\\
    \uparrow      & \mbox{if $x$ odd}
  \end{cases}$

  \emph{Idea:} Store and increasing even number in $R_2$ and store its' half in
  $R_3$.
  \begin{center}
    $\begin{tabu}{|c|c|c|c|}
      \hline
      R_1 & R_2 & R_3 & \dots \\
      \hline
      x   &  2k  & k   & \dots \\
      \hline
    \end{tabu}$
  \end{center}

  \begin{quote}
    \begin{tabular}{lll}
      $I_1$: & J(1,2,6) \\
      $I_2$: & S(2)     \\
      $I_3$: & S(2)     \\
      $I_4$: & S(3)     \\
      $I_5$: & J(1,1,1) \\
      $I_6$: & T(3,1)   \\
    \end{tabular}
  \end{quote}

\end{enumerate}

\section {Function computed by a program}
Given a program $P$, for some fixed number $k \geq 1$ of parameters, there exists a unique \textbf{function computed by $P$} that we denote by $f_p^{(k)} : \nat^k \to \nat$ defined by:

\begin{equation*}
  f_p^{(k)}(a_1, \dots, a_k) = \begin{cases}
    a        & $ if $ P(a_1, \dots, a_k) \downarrow a  \quad \\
    \uparrow & $ if $ P(a_1, \dots, a_k) \uparrow
  \end{cases}
\end{equation*}

\begin{remark}
  The same function can be computed by different programs, essentially for the following two reasons;

  \begin{itemize}
  \item we can add useless instructions to a program (dead code, $T(n,n)$, etc.)

  \item the same function can be computed via different algorithms
    (e.g., for sorting we have quicksort, mergesort, heapsort, etc.)
  \end{itemize}

  A function can be computed either by no program or by infinitely many programs.
\end{remark}


SO FAR

\section {Exercises}

\subsection{Reduced URM}

Let URM be reduced without transfer instruction $T(m, n)$. We indicate the class
of functions that can be computed with the reduced machine $ \mathcal{C}' $ and
compare it with $ \mathcal{C} $. Obviously $ \mathcal{C}' \subseteq \mathcal{C}
$. Let's see if $ \mathcal{C} \subseteq \mathcal{C}^R$? \\
(The answer is ``yes because $T(m, n)$ can be replaced with other instructions.)

\begin{quote}
  \begin{tabular}{lll}
    $I_t$: & T(m,n) \\
  \end{tabular}
\end{quote}

can be replaced with a subroutine at the right place

\begin{quote}
  \begin{tabular}{lll}
    $I_{t'}$:   & J(m,n,t+1)  \\
    $I_{t'+1}$: & Z(n)        \\
    $I_{t'+2}$: & J(m,n,t+1)  \\
    $I_{t'+3}$: & S(n)        \\
    $I_{t'+4}$: & J(1,1,t'+2) \\
  \end{tabular}
\end{quote}

But let's prove it: ($ \mathcal{C} \subseteq \mathcal{C}' $), $ f \in \mathcal{C} $, $ f: \nat^K \rightarrow \nat $ There is an URM $P$ s.t. $ f_P^{(K)}  = f$ the program $P$ can be transformed into $P ^R $ of the reduced URM machine s.t. $ f_{P^R}^{(K)}  = f_{P}^{(K)}$.

There is a demonstration by induction. We show that $P$ can be transformed into $P' $ s.t. $ f_{P'}^{(K)}  = f_{P}^{(K)} $ by induction on $h$ = number of transfer instructions $T$ in $P$.

$h = 0$ trivial.

$h \rightarrow  h + 1$:

$P$ contains $h + 1 \quad T$ instructions.

Transform $P$ into $P''$ where all instructions from 1 to \textit{l} are as same as before, while instead of $T$ we put a jump $J(1,1, SUB)$ where the subroutine is written before. We assume that if $P$ ends, it does so at instruction $l + 1$. And at position $l + 1$ we put a $J (1,1, END)$. After these replacements, we have $h$ instructions $T$ and therefore we can say that we have the program of the reduced URM s.t. the computed function is the same by inductive assumption.

\subsection{Exercises}

\subsection{URM with swap instruction}
Let $URM^S $ be the model obtained by removing transfer function and inserting the swap function $ T_S(m,n) $, what relationship is there between the class of this model and the other?

Proof that $ \mathcal{C}^S \subseteq \mathcal{C} $ The exchange $T (m, n)$ is equivalent to:

\begin{lstlisting}
  T(n,i)
  T(m,n)
  T(i,m)
\end{lstlisting}

Formalization:

Let $ f \in \mathcal{C}^S f:\nat\rightarrow \nat $. There exists $P$  $URM^S $ s.t. $ f_P^{(K)} = f $. Let's proceed by induction on the number of transfer functions $h$.

If $h = 0$ the program is already URM. therefore $ P' = P $

Otherwise if there is at least one transfer instruction we show that the inductive case is valid $ h \rightarrow h+1 $, if $P$ ends it does so in $l + 1$, the program has $h + 1$ exchange instructions.

Let $i$ be a register not used by $P$ found by inspecting the program. We replace at step $t \quad J (1,1, SUB)$ and add a subroutine as the one above, thus creating $ P'' $, at the end of the subroutine we return to the starting point with $J (1,1, t + 1)$. By inductive hypothesis there is $ P' $ URM s.t. $ f_{P'}^{(K)} = f_{P''}^{(K)} = f_{P}^{(K)}$. \\
\textbf{But all of this is wrong!}

Why is it wrong? Because with the replacement that we made we have 1 transfer instruction, but $n-1$ exchange instructions.

Let's prove something stronger: Given $P$ program that uses both URM instructions and  $URM^S $ instructions, there is $ P'' $ program that uses $URM$ instructions s.t. $ f_{P}^{(K)} = f_{P'}^{(K)} $.

The proof procedure is the same but we are demonstrating something stronger, the inductive case is now correct. This proves that $ \mathcal{C}^S \subseteq \mathcal{C} $

To show $  \mathcal{C} \subseteq \mathcal{C}^S $ we know that we have shown that $ \mathcal{C} \subseteq \mathcal{C}^R $ and therefore given that $ \mathcal{C}^R \subseteq \mathcal{C}^S $ is demonstrated by transitivity.

So we can say that $ \mathcal{C}^S = \mathcal{C} $

\subsection{URM without jump instructions}

URM$ ^{nj} $ is a model without jumps, meaning without $J(m,n,t)$ instructions.

Demonstrate that $ \mathcal{C}^{nj} \subset \mathcal{C} $ where the first is the set of computable functions without a jump instruction. We know that $ f: \nat \rightarrow \nat, f(x)\uparrow \forall x $ is computable in URM, but it is not computable in URM $ ^{nj} $

Which functions $ f: \nat \rightarrow \nat $ can be computed without jumping? Keep in mind that we also removed the only conditional statement, so they always end. Then we have the following cases:

$f(x) = c  \forall x $ or $ f(x) = x + c \forall x $

We see $ r_1(h,x) $ the contents of register 1 after h steps with initial content x. Given $P$ program, the function computed by him $ f_p(x) = r_1(l(P), x) $ where $ l(P) $ is the length of $P$;

We show by induction on $h$ that after $h$ execution steps on $P \quad  r_1(h,x) $ is equal to $x + c$ or to $c$.

By induction $h = 0: r_1(0,x) = x $ OK

Case $ h \rightarrow h+1 $: We know $ r_1(h,x) = x+c $ or $ c $ by inductive hypothesis. The next instruction can be one of three cases:
\begin{itemize}
\item The instruction is $Z (n)$. Then if $n = 1  \quad  r_1(h+1,x) = 0 $, otherwise $ r_1(h+1,x) = r_1(h,x) $ in both cases it's OK. The first is constant, the second is fine by inductive hypothesis.
\item The instruction is $S (n)$. Then if $n = 1$ we find that $ r_1(h+1,x) = r_1(h,x)+1 $ which by induction hypothesis is fine. Even if $n$ is other than 1 (see the previous bullet point)
\item The instruction is $T (m, n)$. In cases where $ n>1 $ or $ n=m=1 $ then $ r_1(h+1,x) = r_1(h,x) $ and that's fine. Otherwise $ n = 1, m > 1 $ we do not know what $ r_1(h+1,x) $ is worth. So who knows?
\end{itemize}

Actually we should prove this not only for register $ r_1 $ but for a generic
register $ r_j $.

\chapter{Decidable Predicates}

In mathematics we often want to establish \textbf{properties}, for example \emph{$m$ is a divisor of $n$}. 
We can define it using a relation such as
\begin{align*}
    div & \subseteq \nat \times \nat\\
    div & = \{(m,k \cdot m) \mid m \in \nat, k \in \nat \}
\end{align*}

We can also characterise $div$ as 
\begin{align*}
    div & : \nat \times \nat \rightarrow \{true, false\}\\
    div & = \begin{cases}
                true & \mbox{if $m$ is a divisor of $n$}\\
                false & \mbox{otherwise}
            \end{cases}
\end{align*}

In the field of calculability theory we talk about \textbf{predicates}.

A \textbf{k-ary predicate} on $\nat $ indicated with $Q(x_1,\dots,x_k)$ is a property that can be true or false, formally we can see it as

\begin{itemize}
\item a function $Q: \nat^k\rightarrow \{true,false\}$
\item a set $Q \subseteq \nat^k$
\end{itemize}

we write $Q(x_1,\dots,x_k)$ to denote $(x_1,\dots,x_k) \in Q$ or $Q(x_1,\dots,x_k) = true$

When is $Q$ computable? When there exists a URM such that given a k-tuple $(x_1,\dots,x_k)$ in input, it returns $true$ if $Q(x_1,\dots,x_k)$ and $false$ otherwise. 

To represent $true$ and $false$ we conventionally use values $1$ and $0$.

\begin{definition}
    A predicate $Q \subseteq \nat^k$ is said to be \textbf{decidable} if its \textbf{characteristic function}
\begin{equation*}
\mathcal{X}_Q(x_1,\dots,x_k) = \begin{cases}
1 & $ if $ Q(x_1,\dots,x_k) \\
0 & $ otherwise $
\end{cases}
\end{equation*}

is (URM) computable.
\end{definition}


\begin{remark}
    $\mathcal{X}_Q$ is a \textbf{total} function (dealing with decidability of predicates, involves only total functions).
\end{remark}


\section {Examples of decidable predicates}
\begin{enumerate}
    \item Equality\\
    $ Q \subseteq \nat^2 $, $ Q(x,y) \equiv x = y $

    The characteristic function
    \begin{equation*}
    \mathcal{X}_Q(x,y) = \begin{cases}
    1 & $ if $ x = y  \\
    0 & $ otherwise $
    \end{cases}
    \end{equation*}
    is computed, for instance, by the program
    \begin{quote}
    \begin{tabular}{lll}
    $I_1$ & J(1,2,3)  \\
    $I_2$ & J(1,1,4)       \\
    $I_3$ & S(3) \\
    $I_4$ & T(3,1)
    \end{tabular}
    \end{quote}

    \item $ Q(x) \equiv x \text{ is even} $

    \begin{quote}
    \begin{tabular}{lll}
    $I_1$ & J(1,2,6)   \\
    $I_2$ & S(2)        \\
    $I_3$ & J(1,2,7)   \\
    $I_4$ & S(2)        \\
    $I_5$ & J(1,1,1) \\
    $I_6$ & S(3)        \\
    $I_7$ & T(3,1)
    \end{tabular}
    \end{quote}
    
    \begin{tabu}{|c|c|c|}
    \hline
    x & k & r \\
    \hline
    \end{tabu} in memory where k is a growing index and r is the result.
    
    \item $Q(x,y) \equiv x \leq y$\\
    We can either increment both $x$ and $y$ until $x+k=y$, so that $x\leq y$, or until $y+k=x$, so that $x>y$ (not equal for the order of comparisons).
    
    \begin{quote}
    \begin{tabular}{lll}
    & T(1,3)      &        \\
    & T(2,4)      &        \\
    LOOP: & J(2,3,SI)   & \comment{x+k=y?} \\
    & J(1,4,NO)   & \comment{y+k=x?} \\
    & S(3)        &        \\
    & S(4)        &        \\
    & J(1,1,LOOP) &        \\
    SI:   & S(5)        &        \\
    NO:   & T(5,1)      &
    \end{tabular}
    \end{quote}
    
    Memory: $\begin{tabu}{|c|c|c|c|c|}
    \hline
    x & y & x+k & y+k & r \\
    \hline
    \end{tabu}$ where $r$ is the result.
    
    Another approach is to increment a register starting from 0. If we reach $x$ first then $x \leq y$, otherwise $x > y$.
    
    \begin{quote}
    \begin{tabular}{lll}            
    LOOP: & J(1,3,SI)   & \\
    & J(2,3,NO)   & \\
    & S(3)        & \\
    & J(1,1,LOOP) & \\
    SI:   & S(4)        & \\
    NO:   & T(4,1)      &
    \end{tabular}
    \end{quote}
    
    $\begin{tabu}{|c|c|c|c|}
    \hline
    x+k & y & k & r \\
    \hline
    \end{tabu}$ where $r$ is the result.
    
    Example of $div(x,y)$, suppose $x \not= 0$:
    
    \begin{quote}
    \begin{tabular}{lll}            
    LOOP: & J(2,3,SI)   &                                   \\
    & Z(4)        & \comment{sum $x$ to $R_2$}         \\
    ADDX: & J(1,4,LOOP) &                                   \\
    & J(2,3,NO)   & \comment{if for $h<x$  $kx+h=y$ then no!} \\
    & S(3)        &                                   \\
    & S(4)        &                                   \\
    & J(1,1,ADDX) &                                   \\
    SI:   & S(5)        &                                   \\
    NO:   & T(5,1)      &
    \end{tabular}
    \end{quote}
    
    $\begin{tabu}{|c|c|c|c|c|}
    \hline
    x & y & kx+h & h & r \\
    \hline
    \end{tabu}$ where $r$ is the result.
\end{enumerate}



\chapter{Computability on other domains}
Since the URM is only able to manipulate natural numbers, our definition of computability concerns only functions and predicates on $\nat$.

The concept of computability can be extended to other domanins referencing a notion of effective encoding.

Suppose that we are interested in computability on an object domain $D$. 
Does our concept of computability extend to this domain? 
One of the necessary conditions is that it is possible to encode the elements of $D$ as natural numbers. 
Suppose there exists $ \alpha: D \rightarrow \nat $, which is biunivocal and that $ \alpha, \alpha^{-1} $ are ``effective''. 
We can't have a formal notion of effectiveness; it means that everyone would agree on its calculability (if there's any justice in this world).

$D$ must be countable. For example, take the strings of a certain alphabet $ \Sigma $, $ D = \Sigma^* $. 
The set of rational numbers $ \mathbb{Q} $ is also countable, and so is the set of integers $\mathbb{Z}$, while $D$ can't be  $ \mathbb{R} $ or $A^\omega$ (streams).

At this point we must ask ourselves when a function $ f: D \rightarrow D $ is URM-computable. 
And the answer to this question is: when its encoding $ f: \nat \rightarrow \nat $ is computable.

$ f^*: \nat \rightarrow \nat $\\
$ f^* = \alpha . f . \alpha^{-1} $

We will see that if $\alpha$ is effective, its' inverse is also effective.

\textbf{Note:} The pb realtive to the informality of the effectivity notion remains, but once this has been accepted, the rest is formal.

\begin{example}
  To work on integers we need a function $ \alpha: \mathbb{Z} \rightarrow \nat $, one way to define it is
  \begin{equation*}
    \alpha(z) = \begin{cases}
      2z    & z \geq 0 \\
      -2z-1 & z < 0
    \end{cases} 
  \end{equation*}
which is an effective function with inverse
\begin{equation*}
  \alpha^{-1}(n) = \begin{cases}
    \dfrac{n}{2}    & n $ is even $ \\
    -\dfrac{(n+1)}{2} & n $ is odd $
  \end{cases}
\end{equation*}

An example of a computable function is $f (z) =  |z| $. 
It is computable if $ f^*=\alpha\circ f\circ \alpha^{-1} : \nat \rightarrow \nat $ is URM-computable
\begin{align*}
  f^*(n) &= (\alpha\circ f\circ \alpha^{-1})(n)\\ 
         &=  
\begin{cases}
  (\alpha\circ f)\left(\dfrac{n}{2}\right) & n $ even $ \\
  (\alpha\circ f)\left(-\dfrac{n+1}{2}\right) & $ otherwise $
\end{cases}\\
&=
\begin{cases}
  \alpha\left(\dfrac{n}{2}\right) & n $ even $\\
  \alpha\left(\dfrac{n+1}{2}\right) & $ otherwise $
\end{cases}\\
&= \begin{cases}
  n   & n $ even $ \\
  n+1 & $ otherwise $
\end{cases} 
\end{align*}
that is URM-computable, so $f$ is computable.
\end{example}


\chapter {Generation of computable functions}

We can prove that certain functions are computable if they are simpler combinations of functions that are computable. 
We have the set $\mathcal{C}$ of the computable functions and if we take $f_1, f_2 \in \mathcal{C}$ and compose them with a suitable operation $ op(f_1, f_2) $ we are still in the set $\mathcal{C}$.

More precisely we will prove that the $\mathcal{C}$ class is closed with respect to the following operations:
\begin{itemize}
\item (generalized) composition 
\item primitive recursion
\item (unbounded) minimization 
\end{itemize}

So to prove that the function $f:\nat^k\rightarrow \nat$ is computable 
we could write the URM program P that computes $f$ (such that $f_P^{(k)} = f$), 
or we could use the theorems of the closure of $\mathcal{C}$.

Actually the three operations we consider are not chosen randomly; 
the long term objective is to show that $\mathcal{C}$ coincides with the class of functions which can be obtained
through composition, primitive recursion and minimization, 
starting from a restricted core of basic functions (\textbf{partial recursive functions} of Godel-Kleene).

\section {Basic computable functions}
Consider the following basic functions
\begin{enumerate}
\item zero constant 
      \begin{align*}
        z: &\ \nat^k \rightarrow \nat\\
           & (x_1,\dots, x_k) \mapsto 0
      \end{align*}
\item successor 
      \begin{align*}
        s: &\ \nat \rightarrow \nat\\
        & x \mapsto x + 1
      \end{align*}
\item projection 
      \begin{align*}
        U_i^k: &\ \nat^k \rightarrow \nat\\
        & (x_1,\dots, x_k) \mapsto x_i
      \end{align*}
      
\end{enumerate}

\begin{remark}
  Identity is a sub-case of projection.
\end{remark}


The functions that compute these basic functions are the arithmetic instructions
\begin{enumerate}
\item $z$ computed by $Z(1)$;
\item $s$ computed by $S(1)$;
\item $U_i^k$ computed by $T(i, 1)$.
\end{enumerate}

To prove the properties of closure we will need to ``combine'' programs so we need a bit of notation.
\begin{notation}
  Given a URM program $P$
  \begin{itemize}
    \item $\rho(P)$ is the \textbf{maximum register index} used by $P$
    \item $ l(P) $ is the \textbf{number of instructions} in P;
    \item $P$ is in \textbf{standard form} if, for each $J(m,n,t)$ instruction, $t\leq l(P)+1$ (if it ends, it does so at the instruction $l(P)+1$).
  \end{itemize}
\end{notation}



Considering only standard form programs is not limitative, as stated by the following lemma:
\begin{lemma}
  For each URM program $P$ there exists an equivalent program $P'$ in standard form, i.e. for all $ k$, $f_p^{(k)} = f_{P'}^{(k)}$
\begin{proof} It is enough to replace every instruction $J(m,n,t)$ in $P$ such that $t>l(P)+1$ with $J(m,n,l(P)+1)$
\end{proof}
\end{lemma}

Often we will have to \textbf{concatenate} programs, let $P, Q$ be programs, 
the concatenation consists in executing $P$ and when it terminates executing $Q$, 
that is, every jump instruction $J(m,n,t)$ in $Q$ is replaced with $J(m,n,t+l(P))$

\begin{remark}
  If $P$ and $Q$ are in standard form then $PQ$ is in standard form; moreover $(PQ)R = P(QR)$. We will assume every program is in standard form and we will use concatenation freely.
\end{remark} 
It will be useful to take the input and give the output in arbitrary registers. 
Given a program $P$, we want a program $P[i_1,\dots,i_k \rightarrow h]$ that takes input from $R_{i1},\dots,R_{ik}$, puts the output in $ R_h $ without assuming that the rest of the registers are set to 0. 
This is easily obtainable with transfer and reset operations to move the contents of registers from $i_1,\dots,i_k$ to $1,\dots,k$ and the output from $h$ to 1.

$P[i_1,\dots,i_k \rightarrow h]$ is as follows:

$\begin{tabu}{l}
  T(i_1, 1)\\
  \dots    \\     
  T(i_k, k)\\          
  Z(k+1) \\
  \dots  \\
  Z(\rho(P))     \\     
  P \\
  T(1, l)
\end{tabu}$

\section {Generalized composition}
\begin{definition}
  Given a function
$f: \nat^k \rightarrow \nat$ and functions
$g_1,\dots,g_k: \nat^n \rightarrow \nat$
we define the \textbf{composition} $h: \nat^n \rightarrow \nat$ by 
  \begin{equation*}
    h(\vec{x}) = \begin{cases}
      f(g_1(\vec{x}), \dots, g_k(\vec{x})) & \mbox{if } g_1(\vec{x})\downarrow, \dots, g_k(\vec{x})\downarrow \mbox{ and } f(g_1(\vec{x}), \dots, g_k(\vec{x}))\downarrow\\
      \uparrow & \mbox{otherwise}
    \end{cases}
  \end{equation*}
\end{definition}

\begin{example}
  Consider
  \begin{equation*}
    z(x)=0\quad \forall x \qquad \varnothing(x)\uparrow\quad \forall x
  \end{equation*}
  then
  \begin{equation*}
    z(\varnothing(x))\uparrow \quad \forall x
  \end{equation*}
\end{example}

\begin{example}
  Consider $\varnothing$ and $U^2_1$, then
  \begin{equation*}
    U^2_1(x_1, x_2)=x_1 \quad \text{but} \quad U^2_1(x_1, \varnothing(x_2))\uparrow
  \end{equation*}
\end{example}

\begin{proposition}
  $\mathcal{C}$ is closed under generalised composition
  \begin{proof}
    Let
    \begin{align*}
      f& :\nat^k\rightarrow\nat\\
      g_1,\dots,g_k& : \nat^n\rightarrow\nat
    \end{align*}
    in $\mathcal{C}$, consider the composition
    \begin{align*}
      h :\ & \nat^k\rightarrow\nat\\
      & \vec{x} \mapsto f(g_1(\vec{x}), \dots, g_k(\vec{x}))
    \end{align*}
    Since $f, g_1,\dots,g_k\in\mathcal{C}$,
    we can take $F, G_1, \dots, G_k$ programs in standard form for them.

    Let us consider a register index $m = max\{\rho(F),\rho(G_1), \dots \rho(G_k),k,n\}$ not used, the program for the composition can be


$\begin{tabu}{|c|c|c|c|c|c|c|c|c|}
  \hline
  1                      & \dots                        & m                                 & m+1   & \dots & m+n   & m+n+1 & \dots & m+n+k \\
  \hline
  \multicolumn{3}{|c|}{\dots} & x_1 & \dots & x_n & g_1(\vec{x}) & \dots & g_k(\vec{x})                                                 \\
  \hline
\end{tabu}$

$\begin{tabu}{l}
  T(1, m+1)\\ 
  \dots\\
  T(n, m+n)\\
  G_1 [m+1,\dots,m+n \rightarrow m+n+1] \\
  \dots                                 \\
  G_k [m+1,\dots,m+n\rightarrow m+n+k]  \\
  F[m+n+1,\dots,m+n+k \rightarrow 1]
\end{tabu}$

then $h\in\mathcal{C}$.
  \end{proof}
\end{proposition}

\iffalse

SISTEMARE

\textbf{Corollary:} Let $f:\nat^k\rightarrow \nat$ be computable. Then $g:\nat^n\rightarrow \nat$, where $g(x_1,\dots,x_n) = f(x_{i1},\dots,x_{ik})$ is computable, where $(x_{i1},\dots,x_{ik})$ is a sequence of variables in $x_1,\dots,x_n$ with repetitions and missing variables.

\textbf{proof}: if $\vec{x} = (x_1,\dots, x_n)$,

\begin{equation*}
  g(\vec{x}) = f(\cup_{i1}^n(\vec{x}),\dots,\cup_{ik}^n\vec{x})
\end{equation*}

\fi


\begin{example}
  if $f:\nat^2 \rightarrow \nat $ is computable, then the following are also computable
\begin{itemize}
\item $f_1(x,y) = f(y,x)$;
\item $f_2(x) = f(x,x)$;
\item $f_3(x,y,z) = f(x,y)$.
\end{itemize}
\end{example}


\begin{remark}
  On the basis of this result we can use generalized composition when the $g_i$ are not functions of all the variables or are functions with repetitions.
\end{remark}

\begin{example}
  Given $f: \nat^2 \rightarrow \nat$ where $ f(x_1,x_2) = x_1 + x_2 $ is computable, we can derive that $g: \nat^3 \rightarrow \nat$ where $ g(x_1,x_2,x_3) = x_1 + x_2 + x_3 $ is also computable.
  In fact $g(x_1,x_2,x_3) = f(f(x_1,x_2),x_3) $. Then, we can think about $x_1,x_2,x_3$ as functions $\nat^3\rightarrow\nat$ on $\vec{x}$, so we get to $f(f(U_1^3(\vec{x}),U_2^3(\vec{x})), U_3^3(\vec{x}))$, that is computable.
\end{example}

\begin{example}
  The following functions are computable
\begin{itemize}
\item \textbf{constant} $\lambda \vec{x}.m$, as $m(\vec{x}) = s(s(\dots s(z(\vec{x}))))$, $s$ applied $m$ times;
\item \textbf{addition} $g(x_1,\dots,x_k) = x_1 + \dots + x_k$, seen before;
\item \textbf{product by constant} $g(x,\dots,x) = k \cdot x$, where $g$ is the function at the previous step;
\item if $f(x,y)$ is computable, then also $f'(x) = f(x,m)$ is computable.
  In fact $f'(x) = f(x,m) = f(U_1^1(x)), m(x))$, that is computable;
\item if $f:\nat\rightarrow\nat$ is total computable, the predicate $Q(x,y)\equiv f(x) = y$ is decidable.

  In fact, we know that \begin{equation*}
    \mathcal{X}_{Eq}(x,y) = \begin{cases}
    1 & x=y         \\
    0 & $otherwise$
  \end{cases}
\end{equation*}
  is computable.

  Therefore $\mathcal{X}_Q(x,y) = \mathcal{X}_{Eq}(f(x),y) = \mathcal{X}_{Eq}(f((U_1^2(x,y)), (U_2^2(x,y))$, so $\mathcal{X}_Q$ is computable.
\end{itemize}
\end{example}


\section {Primitive recursion}

\textbf{Recursion} is a familiar concept; it allows to define a function specifying the values in terms of other values of the function itself (and possibly using other already defined functions).

\begin{example}[Factorial]
  \[
    \begin{cases}
      0! = 1\\
      (n+1)! = n! \cdot (n+1)
    \end{cases}
  \]
\end{example}

\begin{example}[Fibonacci]
  \[
    \begin{cases}
      f(0) = 1 \\
      f(1) = 1 \\
      f(n+2) = f(n) + f(n+1)
    \end{cases}
  \]
\end{example}

There are many types of recursion, here we use a ``controlled'' version of recursion.
\begin{definition}[Primitive recursion]
  Given $f:\nat^k\rightarrow\nat$ and $g:\nat^{k+2}\rightarrow\nat$ functions,
we define $h:\nat^{k+1}\rightarrow\nat$ by \textbf{primitive recursion} as follows
\begin{equation*}
  \begin{cases}
    h(\vec{x},0) = f(\vec{x})\\
    h(\vec{x}, y+1) = g(\vec{x},y,h(\vec{x},y))
  \end{cases}
\end{equation*}
\end{definition}

\begin{remark}
  The function $h$ is defined in an equational manner, with $h$ that appears on both sides: 
  it is an implicit definition, not obvious that such $h$ exists or that it is unique, 
  but actually it does exist and it is unique. However, a general theory that supports this observation is not trivial.

  The argument proceeds as follows

\begin{enumerate}
\item let $\nat^n\rightarrow\nat$ the set of functions on natural numbers with $n$ arguments
\item we define an operator
  \begin{align*}
    &T: (\nat^{k+1}\rightarrow\nat) \rightarrow (\nat^{k+1}\rightarrow\nat)\\
    &T(h)(\vec{x},0) = f(\vec{x})\\
    &T(h)(\vec{x},y+1) = g(\vec{x},y,h(\vec{x},y))
  \end{align*}
\item the wanted functions are fixed points of $T$, i.e. $h$ such that $T(h) = h$;
\item the existence of the fixed point follows from these properties
  \begin{itemize}
  \item $\nat^{k+1}\rightarrow\nat$ is a chain partial ordered set;
  \item $T$ is continuous;
  \item Scott functions have a least fixed point.
  \end{itemize}
\item uniqueness follows inductively, so if $h,h'$ fixed points then $h=h'$.
\end{enumerate}
\end{remark}

\begin{example}
  Consider the sum function $ h(x,y) = x+y $ with $ h(x,0) = x = f(x) $ and $ h(x,y+1) = h(x,y) + 1 = g(h(x,y)) $.
  
  $f$ is the identity and $g$ is the successor.
   Both are computable, so the sum is computable by primitive recursion.
\end{example}



\begin{proposition}
  Functions obtained from total functions by
\begin{enumerate}
\item generalized composition
\item primitive recursion
\end{enumerate}
are total.
\begin{proof}
  \begin{enumerate}
    \item obvious by definition;
    \item Let $f:\nat^k\rightarrow\nat, g:\nat^{k+2}\rightarrow\nat$ be total functions and define
    $h$ by primitive recursion.
    
    It can be proved by induction on $y$ that 
    \begin{equation*}
      \forall \vec{x}\in\nat^k  \ (\vec{x},y) \in dom(h)
    \end{equation*}
    \begin{itemize}
      \item $(y=0)$: for all $\vec{x}\in\nat^k$, $h(\vec{x},0) = f(\vec{x})\downarrow$;
      \item $(y\rightarrow y+1)$: for all $\vec{x}\in\nat^k$, $h(\vec{x},y+1) = g(\vec{x},y,h(\vec{x},y))\downarrow$ by inductive hypothesis.
    \end{itemize}
  \end{enumerate}
\end{proof}
\end{proposition}





\begin{example}
  \begin{itemize}
    \item \textbf{sum} $x+y$\\
      $x+0 = x\\
      x+(y+1) = (x+y)+1\\\\
      h(x,0) = x\\
      h(x,y+1) = h(x,y)+1\\\\
      f(x) = x\\
      g(x,y,z) = z+1$
    \item \textbf{product}
      $x\cdot y\\
      x\cdot 0 = 0\\
      x\cdot (y+1) = (x\cdot y)+x\\
      \\
      h(x,0) = 0\\
      h(x,y+1) = h(x,y)+x\\
      \\
      f(x) = 0\\
      g(x,y,z) = z+y$
    \item \textbf{factorial}
      $y!\\
      0! = 1\\
      (y+1)! = y!\cdot (y+1)\\
      \\
      h(0) = 1\\
      h(y+1) = h(y)\cdot(y+1)\\
      \\
      f(0) = 1\\
      g(y,z) = z\cdot (y+1)$
    \end{itemize}
  
\end{example}


\begin{proposition}
  $\mathcal{C}$ is closed under primitive recursion.

\begin{proof}
Let $f:\nat^k\rightarrow\nat$ and 
$g:\nat^{k+2}\rightarrow\nat$ be computable functions.
We want to prove that $h:\nat^{k+1}\rightarrow\nat$ defined through primitive recursion
\begin{equation*}
  \begin{cases}
    h(\vec{x},0) = f(\vec{x})\\
    h(\vec{x}, y+1) = g(\vec{x},y,h(\vec{x},y))
  \end{cases}
\end{equation*}
is computable.

Let $F,G$ programs in standard form for $f,g$. We want a program $H$ for $h$.
We proceed as suggested by the definition.

We start from $\begin{tabu}{|c|c|c|c|c|c|}
  \hline
  x_1 & \dots & x_k & y & 0 & \dots \\
  \hline
\end{tabu}$

we save the parameters and we start to compute $h(\vec{x},0)$ using $F$.

If $y=0$ we are done, otherwise we save $h(\vec{x},0)$ and compute $h(\vec{x},1) = g(\vec{x},0,h(\vec{x},0))$ with $G$. Do the same for $h(\vec{x},i)$ until we arrive to $i=y$

As usual we need registers not used by $F$ and $G$, $m = max\{\rho(F),\rho(G),k+2\}$ and we build the program for $h$ as follows:

$\begin{tabu}{|c|c|c|c|c|c|c|c|c|}
  \hline
  1                     & \dots                                  & m+1                    & \dots   & m+k   & m+k+1 & \dots        & m+k+3 &   \\
  \hline
  \dots                 & \dots                                  & \dots                  & \vec{x} & \dots & i     & h(\vec{x},2) & y     & 0                                                \\
  \hline
\end{tabu}$

$\begin{tabu}{lll}
  & T(1,m+1)                              &                            \\
  & \dots                                 &                           \\
  & T(k,m+k)                              &                            \\
  & T(k+1,m+k+3)                         &                                 \\
  & F[m+1,\dots,m+k\rightarrow m+k+2]    & h(\vec{x},0)                               \\
  LOOP: & J(m+k+1,m+k+3,END)                   & i=y?                                       \\
  & G[m+1,\dots,m+k+2 \rightarrow m+k+2] & h(\vec{x},i+1) = g(\vec{x},i,h(\vec{x},i)) \\
  & S(m+k+1)                             & i = i+1                                    \\
  & J(1,1,LOOP)                          &                                            \\
  END:  & T(m+k+2,1)
\end{tabu}$
\end{proof}
\end{proposition}


\textbf{Note:} We do nothing more than implementing recursion through iteration!

\begin{observation}
  The following functions are computable.
\begin{enumerate}
\item \textbf{sum} $x+y$, see above;
\item \textbf{product} $x \cdot y$ see above;
\item \textbf{exponential} $x^y$\\
  $x^0 = 1, h(x,0) = 1, f(x) = 1$\\
  $x^{y+1} = x^y\cdot x, h(x,y+1) = h(x,y)\cdot x, g(x,y,z) = z\cdot x$;
\item \textbf{predecessor} $x \dot - 1$\\
  $0 \dot -1 = 0, h(0) = 0, f \equiv \underline{0}$\\
  $(x+1)\dotdiv  1 = x, h(x+1) = x, g(y,z) = y$;
\item \textbf{subtraction} $x\dotdiv  y = \begin{cases}
    x-y & x \geq y    \\
    0   & $otherwise$
  \end{cases}$\\
  $x\dotdiv  0 = x, f(x) = x\\
  x\dotdiv (x+1) = (x\dotdiv  y)\dotdiv  1, g(x,y,z) = z\dotdiv  1$;
\item \textbf{sign} $sg(x) = \begin{cases}
    0 & x=0   \\
    1 & x > 0
  \end{cases}\\
  sg(0) = 0, f \equiv \underline{0}\\
  sg(x+1) = 1, g(y,z) = 1$;
\item \textbf{complement sign} $\bar{sg}(x) = \begin{cases}
    0 & x=0 \\
    1 & x>0
  \end{cases}\\
  \bar{sg}(x) = 1 \dotdiv  sg(x), $ composition and (6);
\item $ |x - y| = \begin{cases}
    x-y & x\geq y \\
    y-x & x < y
  \end{cases}$\\
  $ |x - y| = (x\dotdiv y)+(y\dotdiv x)$ from (1), (6) and composition;
\item \textbf{factorial} $y!\\
  0! = 1, f \equiv 1
  (y+1)! = y!(y+1), g(y,z) = (y+1)z $;
\item \textbf{minimum} $min(x,y) = x\dotdiv  (x\dotdiv  y)$;
\item \textbf{maximum} $max(x,y) = (x \dotdiv  y) + y$;
\item \textbf{remainder} $rm(x,y) = \begin{cases}
    y mod y & x \not= 0 \\
    y       & x=0
  \end{cases}$ \\ rest of the integer division of $y$ by $x$ (convention! reasonable if the rest $rm(x,y)$ must be such that $\exists k \mid kx + rm(x,y) = y$)\\
  $rm(x,0) = 0\\
  rm(x,y+1) = \begin{cases}
    rm(x,y)+1 & rm(x,y)+1 \not= x \\
    0         & $otherwise$
  \end{cases}\\
  = (rm(x,y)+1) sg((x\dotdiv  1)\dotdiv  rm(x,y))\\
  f(x) = 0, g(x,y,z) = z * sg(x\dotdiv 1\dotdiv z)$ OK!


  
\item \textbf{quotient} = $qt(x,y) = y$ div $x$, by convention $qt(0,y) = y$\\
  we define:\\
  $qt(x,0) = 0\\
  qt(x,y+1) = \begin{cases}
    qt(x,y)+1 & rm(x,y)+1=x  \\
    qt(x,y)   & $ otherwise$
  \end{cases}\\
  = qt(x,y) + sg((x\dotdiv 1)\dotdiv rm(x,y))$

\item $div(x,y) = \begin{cases}
    1 & x|y                                   \\
    0 & $otherwise, $0|0 $ and $ 0\not|y, y>0
  \end{cases}\\
  div(x,y) = \bar{sg}(rm(x,y))$
\end{enumerate}
\end{observation}

\begin{corollary}[Definition by cases]
  Given $f_1,\dots,f_n: \nat^k \rightarrow \nat$ total, computable and
  $Q_1,\dots,Q_n \subseteq \nat^k$ decidable, predicate and mutually
  exclusive (for each $\vec{x} \in \nat^k$, \textbf{exactly one} of
  $Q_1,\dots,Q_n$ holds) then $ f:\nat^k \rightarrow \nat $ is total
  computable where
  \begin{equation*}
    f(\vec{x}) = \begin{cases}
      f_1(\vec{x}) & Q_1(\vec{x}) \\
      f_2(\vec{x}) & Q_2(\vec{x}) \\
      \dots        &              \\
      f_n(\vec{x}) & Q_n(\vec{x})
    \end{cases}
  \end{equation*}
\end{corollary}

\begin{proof}
$f(\vec{x}) = f_1(\vec{x}) \cdot \mathcal{X}_{Q1}(\vec{x}) + \dots + f_n(\vec{x}) \cdot \mathcal{X}_{Qn}(\vec{x})$

We conclude using the computability of sum and product and the fact that composition preserves computability.
\end{proof}

\section{Algebra of decidability}

\begin{lemma}
  Let $Q, Q'$ be  decidable predicates.  Then also $\neg Q, Q \wedge Q', Q \vee Q'$ are decidable.
\end{lemma}

\begin{proof}
It is enough to observe that:
\begin{enumerate}
\item $\mathcal{X}_{\lnot Q}(\vec{x}) =  \overline{sg}(\mathcal{X}_Q(\vec{x}))$
\item $\mathcal{X}_{Q \vee Q'}(\vec{x}) = \mathcal{X}_{Q}(\vec{x}) \cdot \mathcal{X}_{Q'}(\vec{x})$
\item observe that $Q \wedge Q' = \lnot (\lnot Q \vee \lnot Q')$
\end{enumerate}
\end{proof}

We remind that $\{\neg, \wedge, \vee \}$ ($\{\neg, \vee \}$ is enough) is a functionally complete set of connectives (it allows to express any function $\{0,1\}^n \rightarrow \{0,1\}$). We deduce that:
\begin{corollary}
  Let $Q_1, \dots, Q_n \subseteq \nat^k$ decidable predicates and let $f:\{0,1\}^n \rightarrow \{0,1\}$ a function, let us consider:
  \begin{quote}
    $\mathcal{X}: \nat^k\rightarrow\{0,1\}$\\
    $\mathcal{X}(\vec{x}) = f(\mathcal{X}_{Q_1}(\vec{x}), \dots, \mathcal{X}_{Q_n}(\vec{x}) )$
  \end{quote}
  Then the predicate $Q$ which corresponds to $\mathcal{X}$ is
  decidable, and therefore $\mathcal{X}$ is computable.
\end{corollary}

\section{Sum, product, bounded quantification}

\begin{definition}[Bounded sum and product]
  Let $f:\nat^{k+1}\rightarrow\nat$ be a total function. Then
  
  \begin{itemize}
    
  \item 
    $\sum_{z<y}f(\vec{x},z)$ is defined by
    \begin{align*}
      \sum_{z<0}f(\vec{x},z) &= 0 \\
      \sum_{z<y+1}f(\vec{x},z) &= \sum_{z<y}f(\vec{x},z) + f(\vec{x},y)
    \end{align*}
    
    
  \item $\prod_{z<y}f(\vec{x},z)$ is defined by:
    \begin{align*}
      \prod_{z<1}f(\vec{x},z) &= 1 \\
      \prod_{z<y+1}f(\vec{x},z) &= \prod_{z<y}f(\vec{x},z) \cdot f(\vec{x},y)
    \end{align*}
  \end{itemize}
\end{definition}

\begin{lemma}
  If $f:\nat^{k+1}\rightarrow\nat$ is total computable then
  \begin{enumerate}
  \item $g(\vec{x},y) = \sum_{z<y}f(\vec{x},y)$
  \item $h(\vec{x},y) = \prod_{z<y}f(\vec{x},y)$
  \end{enumerate}
  are total computable.
\end{lemma}

\begin{proof}
  Just note that they are defined by primitive recursion!

  \begin{quote}
    $g(\vec{x},0) = 0$\\
    $g(\vec{x},y+1) = g(\vec{x},y) + f(\vec{x},y)$
  \end{quote}
  
  and $+,f$ are computable.

  Same for 2.
\end{proof}

Obviously, by closure under composition, the bound can be a total computable function.

Another immediate consequence concerns the decidability of the bounded quantification on the predicates.

\begin{lemma}
  Let $Q\subseteq \nat^{k+1}$ be a decidable predicate, then:
  \begin{enumerate}
  \item $Q_1(\vec{x},y) \equiv \forall z<y. Q(\vec{x},z)$
  \item $Q_2(\vec{x},y) \equiv \exists z<y. Q(\vec{x},z)$
  \end{enumerate}  
  are decidable.
\end{lemma}

\begin{proof}
  \begin{enumerate}
  \item observe that $\mathcal{X}_{Q_1}(\vec{x},y) = \prod_{z<y}\mathcal{X}_Q(\vec{x},z)$
  \item observe that $\mathcal{X}_{Q_2}(\vec{x},y) = sg(\sum_{z<y}\mathcal{X}_Q(\vec{x},z))$
  \end{enumerate}
\end{proof}

\section{Bounded minimalisation}
Given a function $f: \nat^{k+1} \rightarrow \nat$, we define a function $h: \nat^{k+1} \rightarrow \nat$ as follows:

\begin{equation*}
  h(\vec{x},y) = \mu z<y . f(\vec{x},z) =
  \begin{cases}
    $min.$z<y$ such that $ f(\vec{x},z) = 0 & $ if it exists$ \\
    y                                   & $ otherwise $
  \end{cases}
\end{equation*}

%(query-replace-regexp "\\$ \\([^\n$]*\\\\[^\n$]*\\) \\$" "$\\1$")

\begin{lemma}
  Let $f: \nat^{k+1} \rightarrow \nat$ total computable. Then also
  $h: \nat^{k} \rightarrow \nat$ defined by
  $h(\vec{x},y) = \mu z<y. f(\vec{x},z)$ is (total) computable.
\end{lemma}

\begin{proof}
  We observe that $h$ can be defined as:
  
  \begin{quote}
    $h(\vec{x},y) = \sum_{z<y}\prod_{w\leq z} sg(f(\vec{x},w))$
  \end{quote}
  
  The product value is $1$ on the intervals $[0,z]$ in which $f\not= 0$,
  i.e. if $z_0$ is the min $z<y$ where $f$ is null, they're equal to
  $z_0$, therefore the external sum counts them.

  Alternatively $h$ can be defined directly through primitive recursion:
  \[
  \begin{cases}
    h(\vec{x},0) = 0 \\
      h (\vec{x},y+1)
      & =
        \begin{cases}
          h(\vec{x},y)               & h(\vec{x},y)\not= y \\
          \begin{cases}
            y   & f(\vec{x},y) = 0 \\
            y+1 & $otherwise$
          \end{cases} & $ otherwise $
        \end{cases}\\[2mm]
        % 
      & =sg(y-h(\vec{x},y)) \cdot h(\vec{x},y) + \bar{sg}(y-h(\vec{x},y))(y+sg(f(\vec{x},y)))
  \end{cases}
  \]
\end{proof}

\begin{lemma}
  The following functions are computable:
\begin{enumerate}[label=\alph*)]
\item $D(x) = $ number of divisors of $x$
\item $Pr(x) = \begin{cases}
    1 & $ x is prime $ \\
    0 & $ otherwise $
  \end{cases}$ (x prime is decidable)
\item $p_x$ = $x$-th prime number (convention: $p_0=0, p_1=2,p_2=3\dots$)
\item $(x)_y = \begin{cases}
    $exponent of $p_y$ in the factorization of $x & x,y > 0      \\
    0                                             & x=0 \vee y=0
  \end{cases}$\\
  e.g. $72 = 2^3\cdot 3^2, (72)_1 = 3, (72)_2 = 2, (72)_3 = 0$
\end{enumerate}
\begin{proof}
\begin{enumerate}[label=\alph*)]
\item $D(x) = \sum_{y\leq x}div(y,x)$
\item $Pr(x)$ is $1$ if $x>1$ and is divided only by 1 and itself
      \begin{align*}
        Pr(x) &= \begin{cases}
          1 & D(x) = 2      \\
          0 & $ otherwise $
        \end{cases} \\
        &= \bar{sg}(|D(x)-2|)
      \end{align*}    

\item $P_x$ can be defined by primitive recursion
  \begin{align*}
    &P_0=0 \\
    &P_{x+1} = \mu z \leq (P_x!+1) . \bar{sg}(P_z(z)\cdot \mathcal{X}_{z>Px}(z))
  \end{align*}
  Certainly $P_{x+1} \leq P_x!+1$, in fact, 
  call $p$ a prime in the decomposition of $p_x!+1$, therefore $p\mid p_x!+1$, so $p>p_x$, otherwise $p \mid p_x!$ and therefore $p \mid 1$. 
  Thus $p_x < p_{x+1} \leq p$.

\item Note that
  \begin{align*}
       (x)_y & = \max \ z . p_y^z \mid x = \\
             & = \min  \  z . p_y^{z+2}\not \mid x\\
             & = \mu z\leq x . \lnot div((p_y)^{z+1},x)
  \end{align*}
\end{enumerate}
\end{proof}
\end{lemma}

\subsection{Exercises}
Prove that the following functions are computable:

\begin{enumerate}[label=\alph*)]
\item $\floor{\sqrt{x}}$

  $\floor{\sqrt{x}} = max\, y\leq x \quad y^2 \leq x\\
  = min \, y \leq x \quad (y+1)^2 > x\\
  \mu y\leq x. ((x+1)-(y+1)^2)$
\item $\mathit{lcm}(x,y)\\
  mxm(x,y) = \mu < \leq x\cdot y . (x|z $ and $ y|z)\\
  = \mu z \leq x\cdot y. \bar{sg}(dic(x,z)\cdot div(y,z))$
\item $\mathit{GCD}(x,y)$

  Certainly $\mathit{GCD}(x,y)\leq min\{x,y\}$ and it can be characterized using the minimum number that can be subtracted to $min\{x,y\}$ to obtain the divisor of $x,y$

  $\mathit{GCD}(x,y)\leq min(x,y)-\mu z\leq min(x,y).(1\dotdiv div(min(x,y)-z,x)\cdot div(min(x,y)-z, y))$
\item number of prime divisors of $x$

  $\sum_{z\leq x} pr(z)\cdot div(z,x)$
\end{enumerate}

\section{Encoding of pairs (and n-tuples)}

Let us see an encoding in $\nat$ of pairs (and n-tuples) of natural numbers that will later be proved useful for some considerations on recursion (and for the future\dots)

Let us define as a \textbf{pair encoding}:
\begin{quote}
  $\pi: \nat^2\rightarrow\nat$\\  
  $\pi(x,y) = 2^x(2y+1)-1$
\end{quote}

Notice that $\pi$ is bijective and effective (computable).

The inverse can be characterized in terms of two computable functions that give the first and second component of a natural $x$ seen as pair:

\begin{quote}
  $\pi^{-1}:\nat\rightarrow\nat^2$\\
  $\pi^{-1}(x) = (\pi_1(x),\pi_2(x))$
\end{quote}
%
where $\pi_1(x) = (x+1)_1$ and 
$\pi_2(x) = (\frac{x+1}{2\pi_1(x)}-1)/2$.

(the division is $qt(\_,\_)$)

It can be generalized to an encoding of $n$-tuples:
\begin{quote}
  $\pi^n: \nat^n\rightarrow\nat \quad n\geq2$
\end{quote}
defining inductively
\begin{quote}
  $\pi^2 = \pi$\\
  $\pi^{n+1}(\vec{x},y) = \pi(\pi^n(\vec{x},y)) \quad \vec{x} \in \nat^n, y \in \nat$
\end{quote}

and correspondingly we can define the projections $pi_j^n:\nat\rightarrow\nat^n$

\subsection{Considerations on recursion}

The Fibonacci function is defined by:

\begin{quote}
  $ fib(0) = fib(1) = 1$\\
  $fib(n+2) = fib(n) + fib(n+1)$
\end{quote}

This is not exactly a definition by primitive recursion. Given that $f(y+2)$ is defined in terms of $f(y)$ and $f(y+1)$, it does not respect the schema \dots

We can show that $f$ is computable by resorting to the encoding and defining:

\begin{quote}
  $g:\nat\rightarrow\nat$\\
  $g(y) = \pi(f(y),f(y+1))$\\
\end{quote}

therefore $g$ can be defined by primitive recursion:

\begin{quote}
  $\begin{array}{ll}
     g(0) & =  \pi(f(0),f(1)) = \pi(1,1)\\[2mm]
     %
     g(y+1)  & = \pi(f(y+1),f(y+2)) = \pi(f(y+1),f(y)+f(y+1))\\
              & = \pi(\pi_2(g(y)), \pi_1(g(y)) + \pi_2(g(y)))
   \end{array}
   $
\end{quote}
so $g$ is computable, by primitive recursion.

Finally, $f(y) = \pi_1(g(y))$ is computable by composition.

\medskip

In general we could have a function $f$ defined using $k$ previous values

$\begin{cases}
  f(0) = c_0   \\
  f(k-1) = c_k \\
  f(y+k) = h(f(y),\dots,f(y+k-1))
\end{cases}$

with $h:\nat^k\rightarrow\nat$ computable.

One can proceed like before and define

\begin{quote}
  $g:\nat\rightarrow\nat$\\
  $g(y) = \pi^k(f(y),\dots,f(y+k-1))$
\end{quote}

Then function $g$ can be defined by primitive recursion

\begin{quote}
  $g(0) = \pi^k(c_0,\dots,c_{k-1})$\\
  $g(y+1) = \pi^k(f(y+1),\dots,f(y+k-1),f(y+k))$
\end{quote}
where
\begin{quote}
  $\begin{array}{ll}
     f(y+1) & = \pi_2^k(g(y))\\
     f(y+k-1) & = \pi_k^k(g(y))\\
     f(y+k) & = h(f(y),\dots,f(y+k-1)) \\
            & = h(\pi_1^k(g(y)),\dots,\pi_k^k(g(y)))\\
            & = \pi^k(\pi_2^k(g(y)),\dots,\pi_k^k(g(y)),h(\pi_1^k(g(y)),\dots,\pi_k^k(g(y))))
   \end{array}
   $
 \end{quote}

 g is computable, so $f(y) = \pi_1(g(y))$ is computable.

\section{Unbounded minimalisation}
The operators to manipulate functions seen until now, generalized composition and primitive recursion, starting from \textbf{total} functions return total functions. 
Another essential operator, which allows to build partial functions is the \textbf{unbounded minimalisation} operator.

Similar to the bounded minimalisation, given $f(\vec{x},y)$ not necessarily total, it defines, informally, the following function:
\[
  \mu y . f(\vec{x},y)  = \mbox{ minimum } y \mbox{ s.t. } f(\vec{x},y) = 0.
\]
But there are two cases in which we have problems
\begin{enumerate}
\item if there is no $y$ s.t. $f(\vec{x},y) = 0 \uparrow$
\item if before finding a $y$ s.t. $f(\vec{x},y) = 0$, it happens that $f(\vec{x},z)\uparrow$
\end{enumerate}
This is intuitive if we think about the obvious algorithm to compute the minimalisation: start from 0, $f(\vec{x},0) = 0$? if yes then $out(0)$, otherwise $f(\vec{x},1) = 0$? until $f(\vec{x},y) = 0$.

\begin{definition}
  Let $f : \nat^{k+1}\rightarrow\nat$ be a function. Then the function
  $h:\nat^k\rightarrow\nat$ defined through \textbf{unbounded
    minimalisation} is:

  \begin{equation*}
    h(\vec{x}) = \mu y. f(\vec{x},y) = \begin{cases}
      $least $ z$ s.t. $ & \begin{cases}
        f(\vec{x},z) = 0 \\
        f(\vec{x},z)\downarrow \quad f(\vec{x},z') \not= 0 \quad $ for $ z<z'
      \end{cases} \\
      \uparrow           & $ otherwise $
    \end{cases}
  \end{equation*}
\end{definition}

\section{Closure under minimalisation}

\begin{theorem}
  Let $f:\nat^{k+1}\rightarrow\nat$ a computable function (not necessarily total). Then $h:\nat^k\rightarrow\nat$ defined by $h(\vec{x}) = \mu y. f(\vec{x},y)$ is computable.
\end{theorem}

\begin{proof}
  Let $F$ be a program in standard form for $f$.

  \textbf{idea:} for $z=0,1,2,\dots$ we compute $f(\vec{x},z)$ until we find a zero\dots
  
  We need to save the argument $\vec{x}$ in a zone that is not used by the program $F$.

$m = max\{\rho(F),k+1\}$

So the program for $h$ is obtained as follows:

$\begin{tabu}{cccccccc}
  1                            & \dots                  & k                             & \dots                  & m+1 & \dots & m+k & m+k+1 \\
  \hline
  \multicolumn{3}{|c}{\vec{x}} & \multicolumn{1}{|c|}{} & \multicolumn{3}{|c|}{\vec{x}} & \multicolumn{1}{c|}{z}                             \\
  \hline
\end{tabu}$

$\begin{tabu}{lll}
  & T([1,k],[m+1,m+k])
  & \mbox{saves $\vec{x}$}\\
  %
  LOOP: & F[m+1,\dots,m+k+1\rightarrow 1] & f(\vec{x},z) \rightarrow R_1                        \\
  & J(1,m+k+2,END)                  &  \mbox{$f(\vec{x},z) = 0$? (recall that $m+k+z$ contains $0$)}\\
  & S(m+k+1)                        & z=z+1                                               \\
  & J(1,1,LOOP)                     &                                                     \\
  END:  & T(m+k+1,1)
\end{tabu}$
\end{proof}

\textbf{Note}: Observe that $F$ may not terminate \dots this is correct! The entire program does not terminate and $\mu$ is undefined!

\textbf{Note:} This is nothing more than a \textbf{while} loop implemented with \textbf{goto}.

\textbf{Observation}: The $\mu$ operator allows us to obtain \textbf{non total} functions starting from total functions.

\begin{example}
  Given $f(x,y) = |x-y^2|$, we have that
  \begin{quote}
    $\mu y. f(x,y) =
    \begin{cases}
      \sqrt{x} & x $ is a perfect square $ \\
      \uparrow & $ otherwise $
    \end{cases}$
  \end{quote}
\end{example}

\begin{exercise}
  Let $f:\nat\rightarrow\nat$ be computable, total and injective. The
  the \textbf{inverse} $f^{-1} = \begin{cases}
    y        & f(y) = x                \\
    \uparrow & \not\exists y. f(y) = x
\end{cases}$

is computable. In fact, in our hypothesis' $f^{-1}(x) = \mu y. |f(y)-x|$
\end{exercise}

\textbf{Note} The above proof uses in an essential way the fact that $f$ is total, but the result is completely general \dots

Intuitively, when $f$ is not total, to find $f^{-1}(x)$ we consider a program $P$ for $f$ and I execute it as follows:
\begin{itemize}
\item 0 steps of the program on argument 0
\item 1 step on 0
\item 0 steps on 1
\item 2 steps on 0\\
  \dots
\end{itemize}
in a dove tail execution pattern.

Every time for a certain number of steps $k$ on argument $y$, the program \textbf{terminates} we check the output $f(y)$, if $f(y) = x$ we stop, otherwise we continue.

Informal \dots we will see how to formalize it.

\begin{exercise}
  Prove that the following function is computable.
  \[f(x,y) = \begin{cases}
    \frac{x}{y} & y\not= 0 \land y|z \\
    \uparrow    & $ otherwise $
  \end{cases}\]
  \begin{proof}
    \[ f(x,y) = \mu z. (|yz-x| + \mathcal{X}_{x=0\land y=0}(x,y)) \]
  \end{proof}
\end{exercise}


\begin{lemma}
  All the functions with finite domain are computable, i.e. let
  $\theta: \nat\rightarrow\nat$ with $dom(\theta)$ finite, then
  $\theta$ is computable.
\end{lemma}
  
\begin{proof}
  Let $\theta:\nat\rightarrow\nat$ a finite domain function
  \[
    \theta=\{(x_1,y_1),\dots,(x_n,y_n)\}
  \]
  i.e.
  
  \[
    \theta(x) = \begin{cases}
      y_1      & x=x_1         \\
      \dots                    \\
      y_n      & x=x_n         \\
      \uparrow & $ otherwise $
    \end{cases}
  \]
  then
  \[
    \theta(x) = \sum_{i=1}^{n}y_i \cdot \bar{sg}(|x-x_i) + \mu z. (\prod_{i=1}^{n}|x-x_i|)
  \]
  The minimalisation is needed only to make the function $\uparrow$ when $x\not= x_1,\dots,x_n$, it is $0$ otherwise.
\end{proof}
\chapter{Other approaches to computability}
We already observed that the URM machine is just one of the many possible computational models that allow us to formalize the notion of computable functions.

We could have used:
\begin{itemize}
\item Turing machine
\item Canonical deduction systems of Post
\item $\lambda$-calculus of Church
\item Partial recursive functions of Gödel-Kleene
\end{itemize}

All of these approaches define the \textbf{same class of computable functions}, leading to the

\textbf{Church-Turing thesis}: a function is computable through an effective procedure 
if and only if it is URM-computable

Now, we introduce another formalism for the definition of computable functions, the set $\mathcal{R}$ of \textbf{partial recursive functions} of Gödel-Kleene and prove that it is equivalent to the URM, meaning it defines the same class of functions: $\mathcal{R} = \mathcal{C}$.

\section{Partially recursive functions}

\begin{definition}
  The class $ \mathcal{R} $ of \textbf{partially recursive functions} is the least class of partial functions on the natural numbers which contains
  \begin{enumerate}[label=(\alph*)]
    \item zero function;
    \item successor;
    \item projections
    \end{enumerate}
    
    and \textbf{closed} under
    \begin{enumerate}
    \item composition;
    \item primitive recursion;
    \item minimalisation.
    \end{enumerate}
\end{definition}

It is a well given definition.

\begin{definition}[Rich class]
    A class of functions $\mathcal{A}$ is said to be \textbf{rich} if it includes (a),(b) and (c) and it's closed under (1), (2) and (3).
\end{definition}

$\mathcal{R}$ is rich and for all $\mathcal{A}$, we have $\mathcal{R}\subseteq\mathcal{A}$

We observe that the property of being ``rich class'' is \textbf{closed for intersection}:

Let $\{\mathcal{A}_i\}i\in I$ a family of rich classes, then $\bigcap_{i\in I}\mathcal{A}_i$ rich.

And finally we define:

\textbf{Definition} The set of the partially recursive functions is:

$\mathcal{R} = \bigcap_{\mathcal{A} \subseteq \bigcup_k \nat^k\rightarrow\nat
  \land
  \mathcal{A}rich} \mathcal{A}$

\textbf{Note} $\mathcal{R}$ admits an inductive characteristic\\
$\mathcal{R}_0 = \{a,b,c\}$\\
$\mathcal{R}_{i+1} = \mathcal{R}_i \cup \{1,2,3 $ applied to $ R_i\}\\
R = \bigcup_i \mathcal{R}_i$

Another interesting class, on which we will come back:

\textbf{Definition} (Primitive recursive functions).

The class of the primitive recursive functions is the least class of functions $\mathcal{PR}\subseteq \bigcup_k\nat^k\rightarrow\nat$ that contains (a),(b) and (c) and is closed for (1) and (2).

\begin{theorem}[$\mathcal{R} = \mathcal{C}$]\label{reqc} The partially recursive
  functions coincide with those URM-computable
  
  \begin{proof}
    $ \mathcal{R} \subseteq \mathcal{C} $ because $\mathcal{R}$ is the
    least rich class, the other is a rich class
    $\Rightarrow \mathcal{R}\subseteq\mathcal{C}$.

    Let us now consider $ \mathcal{C} \subseteq \mathcal{R} $. Let
    $ f:\nat^k\rightarrow\nat \in \mathcal{C} $ be a function and show
    that $ f \in \mathcal{R} $. We know that $ \exists P \in URM $
    programs $ f_P^{(k)} = f$.

    Considering the following functions (dependant on $P$)

    With $ C_p^1(\vec{x}, t) $ we indicate contents of register 1 after $t$ steps of $ P(\vec{x}) $. A computation step is the execution of an instruction.

    It is understood that if $P(\vec{x})$ terminates in less than $t$ steps, $ C_p^1(\vec{x}, t) $ gives the content of $R_1$ in the final configuration.

    With $ J_P(\vec{x},t) $ we indicate instruction to be executed at the $t+1$-th step (after $t$ steps) of $P(\vec{x})$ (program counter) If the program has already ended, it is worth 0.

    Clearly $C_p^1$ and $J_p$ are total functions (we will need this later\dots)

    If $ f(\vec{x})\downarrow $ then $ P(\vec{x})\downarrow $ after $ t_0 $ pass, $ t_0 = \mu t. J_P(\vec{x},t) \Rightarrow f(\vec{x}) = C_p^1(\vec{x},t_0) = C_P^1(\vec{x}, \mu t.J_P(\vec{x},t)) $.
    Otherwise, if $ f(\vec{x})\uparrow $ then $P(\vec{x})\uparrow$ too and $ \mu t.J_P(\vec{x},t)\uparrow $

    Therefore:
    $f(\vec{x}) = c_p^1(\vec{x}),\mu t.J_p(\vec{x},t)$

    % =======================================================

    If you knew that $ C_P^1, J_P \in \mathcal{R} $ then here it is shown that $ f \in \mathcal{R} $

    We prove that they are even in $ \mathcal{PR} $

    The idea of the proof is the following:

    \begin{itemize}
    \item we can work on sequences encodings taht rapresents the registers and program counter configuraiton
    \item we then manipulate such sequences  with the funcitons (\( p_x, q_t, \text{div}, \dots \) ) that we built by:
      \begin{itemize}
      \item composition
      \item primitive recursion
      \end{itemize}
    \item this way we obtain $C_p^1, J_p$ through primitive recursion
    \end{itemize}
    More precisely, the computation state is a tuple
    $\langle \vec{R}, t \rangle$ where $\vec{R}$ are the registers of the
    machine and $t$ is the current instruction.

    To a register configuration in which a finite number of registers
    contains a value other than 0 can be encoded with

    \begin{center}
      $\text{cod}(\vec{R}) = \prod\limits_{i \geq 1}p_i^{r_i}$
    \end{center}

    where just a finite number of factors is $\neq 1$. At this point,
    given $x \geq 1$ interpreted as $\text{cod}(\vec{R})$

    \begin{center}
      $r_i = (x)_i$
    \end{center}

    Using this encoding, we can consider the function $C_p(\vec{x},t)$
    (the registers' configuration after t steps of $P(\vec{x})$) as a
    function $C_p : \nat^{k+1} \rightarrow \nat$. This way we can define:

    \begin{center}
      $\sigma_p(\vec{x},t) = \langle C_p(\vec{x},t), J_p(\vec{x},t) \rangle$
    \end{center}

    the state of the computation of $P(\vec{x})$ after $t$ steps. And
    using the encoding function for the $\pi$ couples we can view
    $\sigma_p$ as:

    $\sigma_p : \nat^{k+1} \rightarrow \nat$

    $\sigma_p(\vec{x}, t) = \pi(C_p(\vec{x},t), J_p(\vec{x},t))$

    and we can define it with the primitive recursion:

    $\sigma_p(\vec{x}, 0) = \pi(\prod\limits_{i=1}^k p_i^{x_i}, 1)$

    $\sigma_p(\vec{x}, t+1) = \pi(C_p(\vec{x},t+1), J_p(\vec{x},t+1))$

    with

    \[
      C_p(\vec{x},t+1) = \begin{cases}
        qt(p_n^{(C_p(\vec{x},t))_n}, C_p(\vec{x},t)) & \quad \text{if }1 \leq J_p(\vec{x},t) \leq s \; \And \; I_{J_p(\vec{x},t)} = z(n) \\
        p_n \cdot C_p(\vec{x},t) & \quad \text{if }1 \leq J_p(\vec{x},t) \leq s \; \And \; I_{J_p(\vec{x},t)} = s(n) \\
        qt(p_n^{(C_p(\vec{x},t))_n}, C_p(\vec{x},t)) \cdot p_n^{(C_p(\vec{x},t))_m} & \quad \text{if }1 \leq J_p(\vec{x},t) \leq s \; \And \; I_{J_p(\vec{x},t)} = T(m,n) \\
        C_p(\vec{x}, t) & \quad \text{otherwise}
      \end{cases}
    \]

    \[
      J_p(\vec{x}, t+1) = \begin{cases}
        J_p(\vec{x}, t) + 1 & \quad \text{if } 1 \leq J_p(\vec{x}, t) < s \And I_{J_p(\vec{x}, t)} = z(n), s(n), T(m,n) \\
        & \quad \quad \text{ or } J(m,n,q) \text{ with } (\sigma_p(\vec{x}, t))_m \neq (\sigma_p(\vec{x}, t))_n \\
        q & \quad \text{if } 1 \leq J_p(\vec{x}, t) < s \And I_{J_p(\vec{x}, t)} = J(m,n,q) \text{ with } q\leq 5 \\
        & \quad \quad \And (C_p(\vec{x}, t))_m = (C_p(\vec{x}, t))_n \\
        0 & \quad \text{otherwise}
      \end{cases}
    \]

    this way we define by primitive recursion $\sigma_p$, even if not in
    detail, starting from functions in $\mathcal{R}$.

    $\sigma_p \in \mathcal{R}$

    \textbf{Observation:} all the functions we used are in  $\mathcal{PR} \Rightarrow \sigma_p \in \mathcal{PR}$

    $C_p^1(\vec{x}, t) = (\pi_1(\sigma_p(\vec{x}, t)))_1 \in \mathcal{R}$

    $J_p(\vec{x}, t) = \pi_2(\sigma_p(\vec{x}, t)) \in \mathcal{R}$

    so $f$, defined by composition and minimization by $C_p^1, J_p$ is in $\mathcal{R}$ as we wanted.

  \end{proof}
\end{theorem}

\chapter{Primitive recursive functions}

We define the primitive recursive functions as follows
\begin{definition}[Primitive recursive functions]
  The class of
primitive recursive functions is the smallest set
$\mathcal{PR}$ containing
\begin{enumerate}[label=(\alph*)]
\item zero function
\item successor
\item projections
\end{enumerate}
and closed under
\begin{enumerate}[label=(\arabic*)]
\item composition
\item primitive recursion
\end{enumerate}
\end{definition}

One of the interesting property of $\mathcal{PR}$ is the
fact that primitive recursive functions are similar to \texttt{for} loops, while
function defined also by minimalisation are similar to \texttt{while} loops. This fact can be
formalized into a new variant on the URM, with \emph{structured
  programs}, where the jump instruction is substituted by
\texttt{for} and \texttt{while} loops. We'll call this machine
$\text{URM}_{\text{for,while}}$.

We can prove that this model has the same expressive power as the
URM model. This means that the class $\mathcal{C}_{\text{for,while}}$
coincides with $\mathcal{C}$ and $\mathcal{R}$,
while the class $\mathcal{C}_{\text{for}}$ of functions computable using only the \texttt{for}
construct coincides with $\mathcal{PR}$.

Thus, studying the relation between $\mathcal{R}$ and $\mathcal{PR}$ is
the same as studying the relation between the expressive power of
\texttt{for} and \texttt{while} constructs.  We know that many
``arithmetic'' functions, like $Pr(x), (x)_y, qt, mcm(x,y),
x^y$ are in $\mathcal{PR}$ and $\mathcal{PR}$ is closed under sum,
product and minimalisation. This class is very
extended, but it does not contain all computable functions, in other
words $\mathcal{PR} \subsetneq \mathcal{R}$, because $\mathcal{PR}$
functions are always total, since
$\mathcal{PR}$ functions are obtainable from base total functions by
composition and primitive recursion.

One could think that $\mathcal{PR}$ includes all the total recursive
functions, in other words if $Tot$ is the set of all total functions:
$\mathcal{PR} = \mathcal{R} \cap Tot$ [Hilbert, 1926].

This is false
($\mathcal{PR} \subsetneq \mathcal{R} \cap \text{Tot}$) because, even
if we restrain ourselves to total functions (programs that always
terminates), the \texttt{while} construct is essential.

\section{Ackermann's function}
Consider the function $\psi: \nat^2 \rightarrow \nat$ defined as
\begin{equation*}
  \begin{cases}
    \psi(0,y) = y+1\\
    \psi(x+1,0) = \psi(x,1)\\
    \psi(x+1,y+1) = \psi(x, \psi(x+1, y))\\
  \end{cases}
\end{equation*}

This scheme uniquely determine a function, because the value $\psi(x,y)$ is always defined
based on \emph{smaller} values of $\psi$ itself. But what does \emph{smaller} mean?

\begin{itemize}
  \item in $\psi(x+1,0) = \psi(x,1)$ the first argument
  diminishes.
  \item in $\psi(x+1,y+1)=\psi(x, \psi(x+1, y))$ at first we compute $\psi(x+1, y)$
  where the second argument diminishes and then $\psi(x,u)$ in which the first argument
  $u$ is, the first argument is smaller.
\end{itemize}
 

We can see that the arguments diminish in a \emph{lexicographical}
order on $\nat^2$, so we can define a partial ordered set 
$(\nat^2, \leq_{lex} )$ with $(x,y) \leq (x', y')$
if $(x < x') \wedge (x=x'$ and $y \leq y')$ and then
$( \nat^2, \leq_{lex} )$ is also \emph{well ordered}, i.e.
it does not allow for infinite descending sequences.

\begin{definition}[Partially ordered set]
  A set $D$ with a binary relation $\leq$ is a partially ordered set (poset) $(D, \leq)$ if $\leq$ is a partial order, i.e., for all $x,y,z\in D$, it is
  \begin{enumerate}
    \item reflexive: $x\leq x$;
    \item antisymmetric: if $x\leq y$ and $y\leq x$, then $ x=y $;
    \item transitive: if $x\leq y$ and $y\leq z$, then $x \leq z$.
  \end{enumerate}
\end{definition}

\begin{definition}[Well founded poset]
  $(D, \leq)$ is well-founded if every non-empty $X \subseteq D$ has a minimal element $d$, i.e.
  \[  
  \forall d' \in X \quad d' \leq d \Rightarrow d' = d
  \]
\end{definition}

\begin{observation}
  $(D, \leq)$ is well-founded iff it does not allow for
  infinite descending chains
  \[
    d_0 > d_1 > d_2 > \dots > d_n > d_n+1 \dots
  \]
\end{observation}
\newcommand{\conf}{\text{conf}} 
This fact can be useful when dealing
with termination problems. If we can conclude that the set of
configurations is well-founded, we simply need to prove
that for each step \( \conf _i \rightarrow \conf_{i+1} \) and
\( \conf _{i+1} < \conf _i \) to end our proof. This way our computation follows a descending sequence of values, necessarily finite.
In fact, if we think the computation of $\psi$ is based on the
computation of $\psi$ with smaller values, at some point it
will for sure reach the case $\psi(0,y) = y + 1$, terminating.

\begin{example}
  $(\nat^2, \leq_{lex})$ is \emph{well-founded}.
Let $\emptyset \neq X \subseteq \nat^2$ and define
\begin{align*}
  x_0 &= \min\{x \mid \exists y \in \nat . (x,y) \in X\} \\
  y_0 &= \min \{ y \; | \; (x_0,y) \in X\}
\end{align*}
then we can state that
$\min X = (x_0, y_0)$. In this way we can prove that the product of
two well-ordered sets is well-ordered.
\end{example}
\begin{observation}
  $\leq_{lex}$ is total.
\end{observation}

A stronger induction principle can be defined on natural numbers:
\begin{definition}[Complete induction]
  To prove that $\forall n\in\nat \;.\; P(n)$, we can assume 
  \[
    \forall n'<n\;.\; P(n')
    \]
  and prove $P(n)$.
\end{definition}
We can generalize the complete induction on $D$ well-founded poset:
\begin{definition}[Well-founded induction]
  Let $(D, \leq)$ be a well-founded poset and let $P(x)$ a
  property on elements of $D$. If for all $d \in D$, assuming $P(d')$ for
  $d' < d$, we can conclude that $P(d)$ holds, then
  \[
    \forall d\in D. P(d)
  \]
\end{definition}

\newcommand{\nsqlex}{\( (\nat ^2, \leq_{lex})\)}

\begin{theorem}
  $\psi$ is total, i.e.
  \[\forall (x,y) \in \nat^2 \quad \psi (x,y) \downarrow \]
  \begin{proof}
    We proceed by well-founded induction on \nsqlex.
    Let $(x,y) \in \nat^2$, assume 
    \[
    \forall (x', y')\leq_{lex} (x,y) \;.\; \psi(x',y')\downarrow
    \]    
    we want to prove $\psi(x',y')\downarrow$.
    We have 3 cases:
    \begin{enumerate}
      \item[$(x=0)$] \(\psi(0,y) = y + 1 \downarrow\)
      \item[$(x>0,y=0)$] $\psi(x,0) = \psi(x-1,1) \downarrow $ for
          inductive hypothesis, since $(x-1,1) \leq_{lex} (x,0)$
      \item[$(x>0,y>0)$] $\psi(x,y) = \psi(x-1,\psi(x,y-1))$
          where $\psi(x,y-1) \downarrow$ by inductive hypothesis. Let
           $u = \psi(x,y-1)$, so
          $\psi(x,y) = \psi(x-1,u) \downarrow $ by inductive
          hypothesis.
      \end{enumerate}
  \end{proof}
\end{theorem}

\begin{exercise}
  Given a box with an arbitrary number of balls in it, each one with a
  number in $\nat$, do the following:
  \begin{itemize}
  \item extract a ball;
  \item substitute the extracted ball with an arbitrary number of
    balls, each one with a label lower than the extracted one.
  \end{itemize}
  Prove that this process always terminates.
\end{exercise}

\begin{theorem}
  $\psi$ is computable, i.e.
  \[\psi \in \mathcal{C} = \mathcal{R}\]
  \begin{proof}
    We can use the Church-Turing thesis to prove it: the computation
    of $\psi(x,y)$ is always reduced to the computation of $\psi$ on
    smaller input values.

    To do so, the definition of a valid set is needed. Intuitively a
    set $S \subseteq \nat^3$ is considered valid if, for all $(x,y,z) \in S$, we have
    \begin{itemize}
    \item $z = \psi(x,y)$
    \item $S$ contains all the
      triples needed to compute
      $\psi(x,y)$
    \end{itemize}
    \begin{example}
      $\psi(1,1) = \psi(0,\psi(1,0)) = \psi(0,\psi(0,1)) = \psi(0,2) = 3$

      $\Rightarrow S = {(1,1,3), (0,2,3), (1,0,2), (0,1,2)}$
    \end{example}
    Formally:
    \begin{definition}[Valid set]
      Let $S$ be a set of triples such that $S \subseteq \nat^3$. We say
      that $S$ is \emph{valid} if:
      \begin{enumerate}
      \item \((0,y,z) \in S \quad \Rightarrow \quad z = y+1\)
      \item \((x+1,0,z) \in S \quad \Rightarrow \quad (x,1,z) \in S\)
      \item \((x+1,y+1,z) \in S \quad \Rightarrow \quad \exists u \;.\; (x+1,
        y, u) \in S \wedge (x,u,z) \in S \)
      \end{enumerate}
    \end{definition}

    We can prove that for every $(x,y,z)\in\nat^3$ we have
   $\psi(x,y) = z$ if and only if there exists a valid \textbf{finite} set of triples $S \subseteq \nat^3$ 
    such that \((x,y,z) \in S\) by complete induction on
    $(x,y)$, knowing that the validity of a set is
    preserved under union (left as an exercise).

    \newcommand{\vnu}{\text{Val}(\nu)}

    A tuple $(x,y,z)$ can be encoded into an integer using the
    encoding function
    \[\Pi^3 \, : \, \nat^3 \rightarrow \nat \quad \quad (\Pi_i^3 \, : \,
      \nat \rightarrow \nat \text{ are the projections})\] 
    In this way a
    set of tuples becomes a set of natural numbers
    $\{x_1, \dots, x_n\}$ that we can encode injectively:
    \[\{x_1, \dots, x_n\} \mapsto P_{x_1}, \dots, P_{x_n}\] 
    Now we have $\nu$ the number that represents the set of tuples
    $S_\nu$. 
    We can verify that
    \[(x,y,z) \in S_\nu \Longleftrightarrow P_{\Pi(x,y,z)} \; | \;
      \nu \] and the predicate $\vnu=$``$\nu$ encodes a
    set of valid tuples'' is decidable.

    In fact $\vnu$ is true if and only if:
    \begin{itemize}
    \item \(\forall i \leq \nu \ (\nu)_i = 1\)
    \item{
        \( \forall \omega \leq \nu \ P_{\omega \; | \; \nu}\)
        \[\Rightarrow \begin{cases}
            \Pi_1(\omega) = 0 \quad & \Rightarrow
            \quad \Pi_3(\omega) = \Pi_2(\omega) + 1 \\
            \Pi_1(\omega) > 0 \quad & \Rightarrow \quad
            \begin{cases}
              \Pi_2(\omega) = 0 & \Rightarrow
              \Pi(\Pi_1(\omega),0,\Pi_3(\omega)) \in S_\nu \\
              \Pi_2(\omega) > 0 & \Rightarrow
              \exists u \leq \omega \text{ s.t. } \\
              & \quad \Pi(\Pi_1(\omega), \Pi_2(\omega)-1,u) \in S_\omega \\
              & \quad \Pi(\Pi_1(\omega)-1, u,z) \in S_\omega
            \end{cases}
            \\
          \end{cases}
        \]
      }
    \end{itemize}
    with associated characteristic function
    \[\chi_{\text{Val}} \in \mathcal{PR}\]

    We can also verify that
    \begin{align*}
      R(x,y,z) &= \begin{cases}
        \chi_{\text{Val}(\omega)} & \text{if $\omega$ encodes some
          valid $S$ that contains $(x,y,z)$ for some $z$} \\
        0 & \text{otherwise}
      \end{cases} \\
      &= \chi_{\text{Val}(\omega)} \cdot \text{sg} (\omega + 1
      \dot{-} \mu z \leq \omega . \text{div}(P_{\Pi(x,y,z)}, \omega)
    \end{align*}

    Thus we can write the Ackermann function as
    \[
      \psi(x,y) = \mu (z,y) \; \cdot \; \overline{\text{sg}}(R(x,y,\omega)
      \cdot \text{div}(P_{\Pi(x,y,z)}, \omega))
    \]

    Since it is computable,
    \[
      \psi \in \mathcal{R} = \mathcal{C}
    \]

  \end{proof}
\end{theorem}

\begin{theorem}
  \[ \psi \notin \mathcal{PR} \]
  \begin{proof}[Informal idea of the proof]
    The proof of the fact that $\psi$ is not a primitive recursive
    function is done by showing that $\psi$ \emph{grows} faster than
    every function in $\mathcal{PR}$. We already saw how we obtain
    \begin{itemize}
    \item sum from successor
    \item product from sum
    \item exponential from product
    \end{itemize}
    each one by nested primitive recursion.

    The idea of the Ackermann function is that it won't be possible to compute it with a finite number of
    nested primitive recursions.

    In fact, by calling \[\psi_x(y) = \psi(x,y)\] we have
    that
    \[
      \psi_{x+1}(y) = \psi_x(\psi_{x+1}(y-1)) =
      \psi^2_{x}(\psi_{x+1}(y-2)) = \dots = \psi_x^{y+1}(1)
    \]
    \begin{align*}
      \psi_0(y) &= y+1 = succ(x)\\
      \psi_1(y) &= \psi_0^{y+1}(1) = y+2\\
      \psi_2(y) &= \psi_1^{y+1}(1) = 2y+3\\
      \psi_3(y) &= \psi_2^{y+1}(1) = 2^{y+3}-3
    \end{align*}

    e.g.

    \(\psi_0(1) = 2, \quad \psi_1(1) = 3, \quad \psi_2(1) = 5, \quad
    \psi_3(1) = 13, \quad \dots\)

    Intuitively, if $x$ grows so does the level of nesting
    in the functions, which is equivalent to say that we need more\texttt{for} loops. 
    Since $x$ can grow to infinity and
    \texttt{for} loops cannot be nested to infinity, a \texttt{while}
    loop is needed. More precisely, given a
    function $f :\nat^n \to \nat \in \mathcal{PR}$ and a program $P$ computing $f$ using only
    \emph{for-loops} (primitive recursion),
    if $j$ is the maximum level of nesting of \texttt{for}-loops, then
    \[
      P(x_1, \dots, x_k)\downarrow \quad \text{in a number of steps }
      < \psi_{j+1}(\max\{x_1, \dots, x_k\})
    \]
    in this way we see that $\psi_j$ gives a bound to the time complexity
    as well to the number of increment operations, so
    \[f(\vec{x}) < \psi_{j+1}(\max\{x_1, \dots, x_k\})\]

    Now, assume $\psi \in \mathcal{PR}$, let $j$ be the
    level of nesting of \texttt{for}-loops for computing $\psi$, so 
    \[\forall (x,y)\;.\;\psi(x,y) < \psi_{j+1}(\max\{x,y\})\]
    Let $x=y=j+1$ big enough we have
    that \[\psi(j+1,j+1) < \psi_{j+1}(j+1) = \psi(j+1,j+1)\] which is
    absurd, so $\psi \notin \mathcal{PR}$.
  \end{proof}
\end{theorem}

\begin{observation}
Initially, Gödel and Kleene studied a class
of functions $\mathcal{R}_0$ called $\mu$-recursive. This class
contained
\begin{enumerate}[label=\alph*]
\item zero function
\item successor
\item projections
\end{enumerate}
and was closed under
\begin{enumerate}
\item composition;
\item primitive recursion;
\item minimization, restricted to the case in which the function that
  produces is \emph{total}.
\end{enumerate}
$\mathcal{R}_0 \subset \mathcal{R}$ trivially holds, since:
\begin{itemize}
\item functions in $\mathcal{R}_0$ are total;
\item some functions in $\mathcal{R}$ are partial.
\end{itemize}

Also \[\mathcal{R}_0 \subseteq \mathcal{R} \cap \text{Tot}\] but is
not obvious that the equality holds. In fact, a function
$f \in \mathcal{R} \cap \text{Tot}$ can be total, but obtained through
minimization of partial functions. For example:
\begin{align*}
  f(x,y) &= \begin{cases}
    x+1 & x<y \\
    0 & x=y \\
    \uparrow & x>y
  \end{cases} \\
  \mu y . f(x,y) &= \lambda x . x
\end{align*}
thus, $f(x,y)$ is partial and
$\mu y . f(x,y)$ is total, then \[\mu y . f(x,y) \in \mathcal{R}_0\]
\begin{theorem}
  \[\mathcal{R}_0 = \mathcal{R}\cap \text{Tot}\]
  \begin{proof}
    \begin{enumerate}
    \item[$(\subseteq)$] trivial.
    \item[$(\supseteq)$] Let $f \in \mathcal{R} \cap \text{Tot}$, then
      $f \in \mathcal{C}$.
      We can observe that
      \[f(\vec{x}) = c_P^1 ( \vec{x} , \mu t . j_P(\vec{x}, t))\]
      but $c_P^1, j_P$ are total, so $f$ is total.
    \end{enumerate}
  \end{proof}
\end{theorem}
\end{observation}

\chapter{Enumeration of programs}
\newcommand{\pr}{\mathcal{PR}}
The objective here is to define an \emph{effective enumeration} of URM
programs and URM-computable functions. These results will be
fundamental for our theory, and in particular to
\begin{itemize}
\item prove the existence of non computable functions
\item the \textit{smn} theorem
\item the universal function/machine.
\end{itemize}

\begin{remark}
  Let $ A, B $ be two sets,
\begin{itemize}
\item $ |A| = |B| $ if there exists $  f:A\to B $ biunivocal.

\item $ |A| \leq |B| $ if there exists $ f:A\to B $ injective or
  there exists $g:B\to A$ surjective.

\item if $|A| \leq |B|$ and $|B| \leq |A|$, then $|A|=|B|$.
  Assuming the axiom of choice, if we have $\{A_i\}_{i \in I}$ family
  of non-empty sets, then
  there exists a function $$f:I \to \bigcup_{i \in I}A_i \quad
  \text{ s.t. }  \forall i \in I\; f(i) \in A_i$$
\end{itemize}
\end{remark}

\begin{definition}[Countable set]
  $A$ is \textbf{countable} if $ |A| \leq |\nat| $, i.e. we have
$ f: \nat \to A $ surjective. We say that
$f$ is an enumeration of $X$, because we can enumerate all
elements in $X$ with \[f(0), f(1), f(2), \dots \]
\end{definition}

\begin{definition}[Enumeration without repetitions]
  An enumeration is \textbf{without repetitions} if it is injective.
  It is also called \textbf{effective}.
\end{definition}

\begin{lemma}
  There are effective enumerations of
  \begin{enumerate}[label=(\arabic*)]
  \item $ \nat^2 $
  \item $ \nat^3 $
  \item $\bigcup_{k\geq 1} \nat^k $
  \end{enumerate}
  \begin{proof}
    \begin{enumerate}[label=(\arabic*)]
    \item we already saw that 
      \begin{align*}
        &\pi : \nat^2 \to
        \nat \\
        &\pi(x,y) = 2^x(2y+1)-1
      \end{align*}  
     is computable
       with inverse
       \begin{align*}
        &\pi^{-1} : \nat \to \nat^2\\
        &\pi^{-1}(x) = (\pi_1(x), \pi_2(x))
       \end{align*}
      where $\pi_1,\pi_2 : \nat \to \nat$
      \begin{align*}
        &\pi_1 (n) = (n+1)_1\\
        &\pi_2(n) = \left(\left(\dfrac{n+1}{2^{\pi_1(n)}}\right)-1\right)
       \end{align*}
       are computable.
    \item{
        consider 
        \begin{align*}
          &\nu : \nat^3 \to
          \nat\\
          &\nu(x,y,z)= \pi(\pi(x,y),z)
        \end{align*}
        with inverse built upon projections
        \begin{align*}
          &\nu^{-1} : \nat \to \nat^3\\
          &\nu^{-1}(x) = (\nu_1(x), \nu_2(x), \nu_3(x))
        \end{align*}
        with $\nu_1, \nu_2, \nu_3$ are computable.}
    \item{ The following tuple encoding
    \begin{align*}
      &\tau : \bigcup_{k \geq 1} \nat^k \to \nat\\
      &\tau(x_1, \dots, x_k) = \prod_{i=1}^{k}p_i^{x_i}-1
    \end{align*}
    does not work, since it is not injective.
    The idea is that we can \emph{increment} the last component, in this way
    \[
    \tau(x_1, \dots, x_k) = \left(\prod_{i=1}^{k-1}p_i^{x_i}\right) \cdot p_k^{x_k+1} - 2
    \]
    with inverse $\tau^{-1} : \nat \to \bigcup_{k \geq 1} \nat^k$ defined out of the following functions:
    \begin{itemize}
      \item $l :\nat \to \nat$
            \[
            l(n) =   \max\{k : p_k | (x+2)\} = x - \mu (h \leq x) \;
            . \; p_{x-h} | (x+2)
            \]
      \item $a :\nat^2 \to \nat$
            \[
              a(n,i)=\begin{cases}
                (n+2)_i & i = 1, \dots, \ell(x)-1 \\
                (n+2)_i - 1 & i = \ell(x)
              \end{cases}
              \]
    \end{itemize}
    An alternative encoding is the following
    \begin{itemize}
      \item $\tau(x_1, \dots, x_k) = \pi(\prod_{i=1}^k p_i^{a_i}, k)$
      \item $l(n) = \pi_2(n)$
      \item $a(n,i) = (\pi_1(n))_i$
    \end{itemize}
      
    
    
    
    
    
    
    
    
    
    
    
    
    
    
    
    
    \iffalse
    Observe that the encoding
        \[(a_1, \dots, a_k) \mapsto \prod^k_{i=1}p_i^{a_i+1}\] can't
        be used since it is injective, but not surjective (the
        co-domain does not contain $0,1$ and any number $x$
        s.t. \(\exists i,j \; i<j \quad p_i \times x \wedge p_j | x\))

        The idea is to leverage the uniqueness of the binary
        representation of natural numbers (positional).
        \[\tau : \cup_{k \geq 1} \nat^k \to \nat \]

        \[\tau (a_1, \dots, a_k ) = 2^{a_1}+2^{a_1+a_2+1}+ \dots +
          2^{a_1+\dots + a_k + (k-1)} - 1\]

        \[\tau (a_1, \dots, a_k ) = \sum^k_{i=1}2^{(\sum_{j=1}^i a_j)
            + (i-1)}-1\]

        \begin{itemize}
        \item $\tau$ is intuitively effective (it uses only
          exclusively functions), but the domain doesn't allow us to
          prove its' computability.
        \item $\tau$ is bijective: let $x \in \nat$, we need to find
          $(a_1, \dots, a_k)$ s.t. $\tau(a_1,\dots,a_k) = x$. Thanks
          to the uniqueness of binary representation of $x+1$
          \[x = (\alpha_m \alpha_{m-1} \alpha_1 \alpha_0)_2 = \alpha
            2^m + \dots + \alpha_1 2^1 + \alpha_0 - 1 \quad \text{with
            } \alpha \in \{0,1\} \] We can then easily define a
          function $\alpha(x,j)$ s.t. $\alpha(x,j) = \alpha_j$, in
          fact
          \[\alpha(x, j) = \text{rm}(2, \text{qt}(2^j, x+1)) \quad \in
            \pr \subseteq \mathcal{C}\]

          Now, based on the definition of $\tau$
          \[\tau(a_1, \dots, a_k) = (1 0 0 1 0 0 0 1 \dots 1 0 0 1 0
            0)\] where the first digit is $a_1+\dots+a_k+k-1 =
          b_k$. We can already express the length $k$ of the tuple
          as \[\ell (x) = \sum_{j \leq x}\alpha(x,j)\]. Now considering
          only the digits 1 and noting their position with $b_k$ we
          can define $b(x,i)$ s.t. $b(x,i) = b_i$ in fact

          \[b(x,1) = \mu(j \leq x) \; . \; |\alpha(x,j) - 1|\]

          \[b(x,i+1) = \mu(j \leq x) \; . \; |\alpha(x,j)-1| +
            (b(x,i)+1 \dotdiv j)\]

          This means that $b(x,i)\in \pr$.  Eventually we can write a
          function $a(x,i)$ s.t. $a(x,i) = a_i$.
          \[a(x,1) = b(x,1)\]
          \[a(x,i+1) = (b(x,i+1) \dotdiv b(x,i))-1\]

          We can finally express $\tau^{-1}$ for $x \in \nat$ as
          \[\tau^{-1} = (a(x,1), \dots, a(x, \ell(x)))\] and is
          effective.
        \end{itemize}
    \fi
    
    
       
      }
    \end{enumerate}
  \end{proof}
\end{lemma}

\begin{theorem}
  Let $\mathcal{P}$ the set of all URM programs. 
  Then there exists an effective bijective enumeration of $\mathcal{P}$.
  \[
    \gamma : \mathcal{P} \to \mathbb{N}
    \]
  \begin{proof}
    Let $\mathcal{F}$ the set of all URM instructions.
    First, we'll prove that there exists
    \[\beta : \mathcal{F} \to \nat \]
    a bijective effective correspondence. The idea is to use the
    enumeration of couples and triples, sending
    \begin{itemize}
    \item $Z(n)$ instructions to multiples of 4
    \item $S(n)$ instructions to numbers equal $ 1 \mod 4$
    \item $T(m,n)$ instructions to numbers equal $2 \mod 4$
    \item $J(m,n,t)$ instructions to numbers equal $3 \mod 4$
    \end{itemize}
    this means that
    \begin{equation*}
      \begin{cases}
        \beta(Z(n)) = 4*(n-1)\\
        \beta(S(n)) = 4*(n-1) + 1\\
        \beta(T(m,n)) = 4*\pi(m-1, n-1) + 2\\
        \beta(J(m,n,t)) = 4*\nu(m-1, n-1, t-1) + 3
      \end{cases}
    \end{equation*}
    with inverse $\beta^{-1} : \nat \to \mathcal{F}$ such
    that, let $r = rm(4,x)$ and
    $q = qt(4,x)$,
    \[
      \beta^{-1}(x) = \begin{cases}
        Z(q+1) & \mbox{if } r=0 \\
        S(q+1) & \mbox{if } r=1 \\
        T(\pi_1(q)+1, \pi_2(q)+1) & \mbox{if } r=2 \\
        J(\nu_1(q)+1, \nu_2(q)+1, \nu_3(q)+1) & \mbox{if } r=3
      \end{cases}
    \]
    so both $\beta$ and $\beta^{-1}$ are effective. 
    Now
    $\gamma : \mathcal{P} \to \nat$ can be defined as follows:
    if 
    $P = I_1 \dots I_s$, then
    \[\gamma(P) = \tau(\beta(I_1), \dots, \beta(I_s))\]
    with inverse $\gamma^{-1}(x) = P = I_1 \dots I_{l(x)}$, where
    $I_i = \beta ^ {-1} (a (n, i))$.
    Thus, $\gamma$ is bijective because is composition of bijective
    functions. Since $\gamma, \gamma^{-1}$ are effective,
    $\mathcal{P}$ is effectively enumerable.
  \end{proof}
\end{theorem}

\begin{definition}[Gödel number]
  Given $P \in \mathcal{P}$ the value $\gamma(P)$ is called code (or
  Gödel number) of $P$. Usually we'll write $P_n$ to represent
  $\gamma^{-1}(n)$ (In other words, the $n^{\mbox{th}}$ program of the
  enumeration).
\end{definition}

\begin{observation}
  From now on we will consider a fixed
enumeration $\gamma$ of programs, which
determines the meaning of $P_n$. This fixed enumeration can be
defined in another way, but it is absolutely necessary. In other words:
\begin{itemize}
\item given a program $P$ we can compute in an effective way its code
  $\gamma(P)$;
\item given a number is possible to find the $n^{\mbox{th}}$ program.
  $P_n = \gamma^{-1}(n)$
\end{itemize}
\end{observation}

\begin{example}
  Let's consider the program $P$
  \begin{itemize}
  \item[] $T(1,2)$
  \item[] $S(2)$
  \item[] $T(2,1)$
  \end{itemize}
  encoded by
  \begin{itemize}
  \item[] $\beta(T(1,2)) = 4 * \pi(1-1,2-1) + 2 = 4 * \pi(0,1) + 2 = 10$
  \item[] $\beta(S(2)) = 4 * (2-1) + 1 = 5$
  \item[] $\beta(T(2,1)) = 4 * \pi(2-1,1-1) + 2 = 4 * \pi(1,0) + 2 = 6$
  \end{itemize}
  then
      \begin{align*}
        \gamma(P) & = \tau(10,5,6) \\
                  & = p_1^{10} \cdot p_2^5 \cdot p_3^{6+1} - 2 \\
                  & = 2^{10} \cdot 3^5 \cdot 5^7 -2 \\
                  & = 19439999998
      \end{align*}
  or with an alternative encoding
  \begin{align*}
    \gamma(P) & = \tau(10,5,6) \\
              & = 2^{10} + 2^{10+5+1} + 2^{10+5+6+2} - 1 \\
              & = 2^{10} + 2^{16} + 2^{23} - 1 \\
              & = 8455167
  \end{align*}

  What does this program compute? $\lambda x . x+1$.

  The program $P^\prime : S(1)$ computes the same function. 
  In this case the encoding is
  \[\beta(S(1)) = 4 * (1-1) + 1 = 1\] and so
  \[\gamma(P^\prime) = \tau(1) = 2^{1+1} - 2 = 2\]
  with the alternative encoding
  \[\gamma(P^\prime) = \tau(1) = 2^1 - 1 = 1\]
\end{example}

\begin{example}
  Show what $P_{100} = \gamma^{-1}(100)$ is.

  We observe that $100 = 1100101 - 1$. 100 is the encoding of a
  quadruple (program with 4 instructions). So
  $\tau^{-1}(100) = (a_1 a_2 a_3 a_4)$ and the components can be
  expressed as before.

  \[
    \begin{split}
      a_1 = b_1 = 0 & \to Z(1) \\
      a_2 = b_2 - b_1 - 1 = 1 & \to S(1) \\
      a_3 = b_3 - b_2 - 1 = 2 & \to T(1,1) \\
      a_4 = b_4 - b_3 - 1 = 0 & \to Z(1)
    \end{split}
  \]

  \[\gamma^{-1}(100) = (Z(1), S(1), T(1,1), Z(1))\]

  With the alternative encoding:
  \[100 + 2 = 2^1 * 3^1 * 17^1 = p_1^1 \cdot p_2^1 \cdot p_3^1 \cdot
    p_4^0 \cdot p_5^0 \cdot p_6^0 \cdot p_7^1 \]

  \[
    \begin{split}
      \beta^{-1}(1) & \quad S(1) \\
      \beta^{-1}(1) & \quad S(1) \\
      \beta^{-1}(0) & \quad Z(1) \\
      \vdots & \quad\vdots \\
      \beta^{-1}(0) & \quad Z(1)
    \end{split}
  \]
\end{example}

Clearly, an enumeration of URM programs induces an enumeration of computable functions

\begin{definition}
  Fixed an effective enumeration
  $\gamma : \mathcal{P} \to \nat$ we define:
  \begin{enumerate}[label=\arabic*.]
  \item $\varphi_n^{(k)}$: the function of k arguments (k-ary
    function) computed by the program $P_n = \gamma^{-1}(n)$ (with
    the notation presently introduced: $\varphi_n^{(k)} = f_{P_n}^{(k)}$)
  \item $W_n^{(k)} = \mbox{dom}(\varphi_n^{(k)}) \subseteq \nat^k $
  \item $E_n^{(k)} = \mbox{cod}(\varphi_n^{(k)}) \subseteq \nat$
  \end{enumerate}

  usually if $k=1$, it is omitted. $\varphi_n = \varphi_n^{(1)}$
\end{definition}

\textbf{Observation:} The function \[
  \begin{split}
    \varphi^{(k)} : & \quad \nat \to \mathcal{C}^{(k)} \\
    & \quad n \mapsto \varphi_n^{(k)}
  \end{split}
\]

is obviously surjective (each computable function is computed by a
program!), and so $\mathcal{C}^{(k)}$ is enumerable:
\[|\mathcal{C}^{(k)}| = |\nat|\]

The existence of a surjective function $\nat \to \mathcal{C}$
follows that $|\mathcal{C}^{(k)}| \leq |\nat|$, but obviously there
exists infinitely many computable functions, for example constants
$\lambda x_1 \dots x_k . c$, and so $|\mathcal{C}^{(k)}| \geq |\nat|$
is also true.

Clearly $\varphi^{(k)} : \nat \to \mathcal{C}^{(k)}$ is not
injective. In fact, for each computable function there are infinitely
many programs that computes it
\[\forall f \in \mathcal{C} \quad \quad |(\varphi^{(k)})^{-1}(f)| =
  |\nat|\] which means
\(\varphi_0^{(k)}, \varphi_1^{(k)}, \varphi_2^{(k)}, \dots\) is an
enumeration of $\mathcal{C}$ with infinitely many repetitions. An
enumeration without repetitions can be defined as:
\begin{itemize}
\item[] $\chi(0) = 0$
\item[]
  $\chi(n+1) = \mu z \; . \; (\varphi_z \notin \{\varphi_{\chi(0)},
  \dots, \varphi_{\chi(n)}\})$
\end{itemize}
which rises the enumeration
$\varphi_{\chi(0)}, \varphi_{\chi(1)}, \varphi_{\chi(2)}, \dots$

But this enumeration is highly ineffective.

(Can be proved that there exists an enumeration
$h : \nat \to \nat$ which is total and computable s.t.
$\varphi_{h(0)}, \varphi_{h(1)}, \varphi_{h(2)}, \dots$ is an
enumeration without repetitions \cite{firedberg:1958}. However, an
enumeration with repetitions is sufficient for us).

\begin{theorem}[$|\mathcal{C}| = |\nat|$]
  The class $\mathcal{C}$ of computable functions is enumerable.
  \begin{proof}
    \[ \mathcal{C} = \bigcup_{k \geq 1}\mathcal{C}^{(k)} \] Since the
    union of enumerable sets is enumerable, $\mathcal{C}$ is
    enumerable.
  \end{proof}
\end{theorem}

\textbf{Note:} from now on we will implicitly use the enumeration of
programs $\gamma$. The meaning of
$\varphi_n^{(k)}, W_n^{(k)}, E_n^{(k)}$ is fixed and determined
starting from $\gamma$.
% \textbf{Observation:}m

% In particular the couples in $ \nat^2 $ can be encoded as
% $ \pi(x,y) = 2^x(2y+1)-1 $ which is computable. The inverse is
% $ \pi^{-1}(n) = (\pi_1(n), \pi_2(n)) $

% The triple instead is a pair of a pair and an element.
% $ \upsilon (x,y,z) = \pi (x, \pi(y,z))$. The inverse is also obtained
% from the inverse of the first.

% For lists we need an encoding
% $ \tau . \bigcup_{K \geq 1} \nat^k \to \nat $ we exploit the
% uniqueness of the prime numbers:
% $ \tau(a_1,\dots,a_k) = \pi_{i=1}^k p_i^{a_i}$ where $ p_i $ is the
% $i$-th prime number. \\
% This, however, leads us to lose any zeros, since
% the encodings for (1,1) and (1,1,0) would be the same because the
% exponential function in 0 = 1.

% So we use something that works:
% $ (\pi_{i=1}^{k-1} p_i^{a_i}) \times p_k^{a_k+1} - 2$. For decoding we
% can proceed as usual, but limited minimization can be used.\\ Hint:
% $ max \{z \leq x . P(z)\} x - min\{\delta \leq x . P(x-\delta)\}$

% But I don't just want the length, I also need the list of items. I
% need a function:
% \begin{equation*}
%   a(x,i) = \begin{cases}
%     (x+2)_i   & 1 \leq i \leq k-1 \\
%     (x+2)_k-1 & i = k
%   \end{cases}
% \end{equation*}
% And this is the inverse function of $\tau$. And these functions are
% computable recursive primitives.

% To compute instructions: Let us take $ \mathcal{F} $ set of URM
% instructions, $ \mathcal{P} $ URM programs. Let's take function
% $ \beta:\mathcal{F}\to\nat $
% \begin{itemize}
% \item $ \beta(z(n)) = 4 \times (n-1) $;
% \item $ \beta(s(n)) = 4 \times (n-1)+1 $;
% \item $ \beta(t(m,n)) = 4 \times \pi(m-1,n-1)+2 $;
% \item $ \beta(j(m,n,t)) = 4 \times \upsilon(m-1,n-1,t-1)+3 $;
% \end{itemize}

% The decoding of this monstrosity is obtained from the previous inverse
% functions applied on the basis of the dimension and the rest of the
% number.

% \textbf{Example:} P = T (1,2); S (2); T (2,1) =
% $ \tau(10,5,6) = 2^{10} 3^5 5^{6+1} -2 $

% The two-way function between programs and numbers has been demonstrated.

% \begin{notation} given an effective enumeration (in our case the one
%   defined previously) we say that $ \gamma(P) $ is the code of P (also
%   called G\"{o} of the number), if $ \gamma(P) = n $ then $P$ is the
%   $n$-th program.
% \end{notation}

% \begin{notation} $ \Phi_n^{(k)}: \nat^k\to\nat $ function of
%   $K$ arguments computed by program $n$, that is, by program
%   $ \gamma^{-1}(n) $, if $k = 1$ is is omitted. The function domain is
%   $ W_n^{k} = dom(\Phi_k^{(k)}) = \{\vec{x} |
%   \Phi_k^{(k)})(\vec{x})\downarrow \} \subseteq \nat^k$

%   The function codomain:
%   $ E^{(k)}_n = \Phi_k^{(k)}) = \{\vec{x} | \vec{x} \in W_n^{(k)} \}
%   \subseteq \nat^k$
% \end{notation}

% So for example the program $ \Phi_{19439999998} = x+1 $,
% $ W_{19439999998} = \nat $, $ E_{19439999998} = \nat \setminus \{0\} $

% Now we have an enumeration of all the unary computable functions that is \{$ \Phi_n $ . $ n \in \nat $ \} each function with infinite repetitions.

% Remember we have indicated the computable functions of $k$ arguments as $ \mathcal{C} ^ {(k)} $, where $ | \mathcal{C} ^ {(k)} | \leq (=) | \nat | $ and therefore being $ \mathcal{C} = \bigcup_{K \geq 1} \mathcal{C} ^ {(k)} $ a union of countable sets it is still countable.

\chapter{Cantor diagonalization method}

The diagonalization technique allows to build an
object that differs from an enumerable infinity of similar objects.
The idea behind is: given an enumerable set of objects
$\{x_1, x_2, x_3, \dots \}$
we can build another object $x$ with the
same nature of all the other objects, but different from all of them
by making it ``differ from $x_n$ on $n$''.

\begin{proposition}[Cantor diagonalization]
  For all $X$ set, we have
  \[ 
    |X| < |2^X|
  \]
\end{proposition}

This is the original method used by Cantor, one of the founders of the classic theory of sets, to prove that there are
  various ``levels of infinity''.

\begin{corollary}
  $|\nat| < |2^\nat|$
  \begin{proof}
    By contradiction $|\nat| \geq |2^\nat| $, i.e. $|2^\nat| $ countable. This means that there exists an
    enumeration of $2^\nat$: $x_0, x_1, x_2, \dots$

    Consider
    \[
      \begin{tabu}{c | c c c c}
        & x_0 & x_1 & x_2 & \dots \\ \hline
        0 &  ?  & NO & \dots & \\
        1 &  NO  & ? & YES & \\
        2 &  YES  & NO & ? & \\
        \vdots & & & &
      \end{tabu}
    \]
    We can define $D = \{i \mid i \notin X_i\} \subseteq \nat$ which differs from
    $X_i$ on its $i^{\mbox{th}}$ element. Obviously $D \in 2^\nat $
    which means that there exists $k$ such that $D = X_k$. But is $k$ in $D$?
    \begin{gather*}
      k \in D \quad \Rightarrow \quad k \notin X_k = D \\
      k \notin D \quad \Rightarrow \quad k \in X_k = D
    \end{gather*}
    which is absurd. Therefore $|\nat| < |2^\nat|$.
  \end{proof}
\end{corollary}

\newcommand{\nattonat}{\nat \to \nat}
\begin{corollary}\label{corollary:nattonat}
  Consider $\nattonat = \{f \mid f : \nattonat \}$, 
  we have \[|\nattonat| > |\nat|\]
  \begin{proof}
    There are two approaches to prove the corollary:
    \begin{enumerate}
      \item Define
      \[
        \mathcal{F} = \{f \mid f : \nattonat \mbox{ total }, \forall x \ f(x)\in \{0, 1\}
        \} \subseteq \nattonat
        \]
      there is a bijection between $ \mathcal{F} $ and $ \nat \to 2$
      and so 
      \[|\nattonat| \geq |\nat \to 2| > |\nat|\]
      \item Let $f_1, f_2, f_3, \dots$ be an enumeration of elements in
      $\nattonat$ and consider
      \[
        \begin{tabu}{c | c c c c}
          & f_0 & f_1 & f_2 & \dots \\ \hline
          0 &  f_0(0)  & \dots & \dots & \\
          1 &  \dots  & f_1(1) & \dots & \\
          2 &  \dots  & \dots & f_2(2) & \\
          \vdots & & & &
        \end{tabu}
      \]
  
      we can define a function $f$ that differs from every other
      function on the diagonal based on the input:
      \[f(i) = \begin{cases}
          0 & \quad \mbox{if } f_i(i)\uparrow \\
          \uparrow & \quad \mbox{if } f_i(i) \downarrow
        \end{cases}
      \] so that
      \[
        \forall i \quad f(i) \neq f_i(i) \quad (f \neq f_i)
        \]
      thus no enumeration of functions in $\mathcal{F}$ can contain all $\mathcal{F}$,
      so $\mathcal{F}$ is not countable.
    \end{enumerate}
    

    
  \end{proof}
\end{corollary}

\newcommand{\noc}{\bar{\mathcal{C}}}

\begin{corollary}
  The set
$\noc = \{f : \nattonat \; | \; f \mbox{ not computable}\}$ is not
enumerable.
\begin{proof}
  We know that $|\mathcal{C}| = |\nat|$. If $\noc$ were enumerable, then
  $\nattonat = \mathcal{C} \cup \noc$ would be enumerable, which is
  absurd for the previous corollary.
\end{proof}
\end{corollary}

\begin{observation}
  There exists a total non-computable function $f : \nattonat$ defined by
  \[
    f(n) = \begin{cases}

      \varphi_n(n) + 1 & \quad \mbox{if } \varphi_n(n) \downarrow  \\

       0 & \quad \mbox{if } \varphi_n(n)\uparrow 
    \end{cases}
  \]
  $f$ is not computable because
  \begin{itemize}
    \item if $\varphi_n(n) \downarrow$, then $f(n) = \varphi_n(n) + 1 \neq \varphi_n(n) $
    \item if $\varphi_n(n)\uparrow$, then $f(n) = 0 \neq \varphi_n(n)$
  \end{itemize}
  so \[\forall n \; f \neq \varphi_n\]
\end{observation}


\begin{observation}
  There are infinitely many total non-computable
functions of the followin shape
\[
  f(n)  = \begin{cases}
    \varphi_n(n) + k & n \in W_n \\
    k & n \notin W_n
  \end{cases}
\]

\end{observation}

\begin{exercise}
  Let $f: \nat \rightarrow \nat$, $m \in \nat$.
  Show that there exists a non-computable function $g : \nat \rightarrow \nat$ 
  such that \[g(x) = f(x) \quad \forall x < m\]

  Idea: use a ``translated diagonal'';
  \[
    g(x) = \begin{cases}
      f(x) & x < m \\
      \varphi_{x - m}(x) + 1 & x \geq m \mbox{ and } x \in W_{x-m} \\
      0 & x \geq m \mbox{ and } x \notin W_{x-m}
    \end{cases}
  \]
  $g$ is not computable since
  $g(x + m) \neq \varphi_x(x+m)$ for all $x$, so
  \[\forall x \ g\neq \varphi_x\]

  Another approach is to define $g$ in the following way
  \[
    g(x) = \begin{cases}
      f(x) & x < m \\
      \varphi_x(x) + 1 & x \geq m \mbox{ and } x \in W_{x} \\
      0 & x \geq m \mbox{ and } x \notin W_{x}
    \end{cases}
  \]
  because each function appears infinitely many times in the enumeration,
  and skipping the first $m-1$ steps doesn't create any
  problem. Formally
  \[
    \forall x \geq m \quad g \neq \varphi_x
    \]
    so for all $y$
  \[
    \forall y\ \exists x \geq m \ \varphi_y = \varphi_x
  \]
    thus \[
      \forall y \; \varphi_y \neq g
      \]
  then $g$ is not computable.
\end{exercise}

\begin{exercise}
  Given a family of functions
  $\{f_i\}_{i\in \nat}$ with $ f_i : \nattonat$, construct $g: \nattonat$
  such that $\dom{g} \neq \dom{f_i}$ for all $i \in N$

  Idea:
  \[
    g(n) = \begin{cases}
      0 & \mbox{if } n \notin \dom{f_n} \\
      \uparrow & \mbox{if } n \in \dom{f_n}
    \end{cases}
  \]
  In this way 
  \[
    \forall n \ n \in \dom{g} \Leftrightarrow 
  n \notin \dom{f_n}
    \]
\end{exercise}

\begin{exercise}
  Define a non-computable total function that returns $0$ when the input
  is even

  Idea:
  \[
    f(x) = \begin{cases}
      0 & x \mbox{ is even} \\
      \varphi_{\frac{x-1}{2}}(x) + 1 & x \mbox{ is odd, and } x \in
      W_{\frac{x-1}{2}} \\
      0 & x \mbox{ is odd, and } x \notin W_{\frac{x-1}{2}}
    \end{cases}
  \]
  it is total not computable. In fact
  \begin{itemize}
    \item if
    $2n+1 \in W_n \Rightarrow f(2n+1) = \varphi_{n}(2n+1) + 1 \neq
    \varphi_n(2n+1)$
    \item if
    $2n+1 \notin W_n \Rightarrow f(2n+1) = 0 \neq \varphi_n(2n+1)
    \uparrow $
  \end{itemize}
  so \[
  \forall n \ f(2n + 1) \neq \varphi_n(2n+1)\]

  

  

\end{exercise}
% Same as the previous chapter, these notes are very different form
% the italian version and not complete in the same way

% Given a set $X$ we know that $ |X| \geq |2^X| $ is never valid, that
% is, the set of its parts is always bigger. Suppose there is
% $ f:X\rightarrow2^X $ surjective. Hence $ R = \{y . y \in X $ and
% $ x \not \in f(y) \ \} \in 2^X $, since f is surjective then
% $ \exists y_R \in X . f(y_R) = R$. I consider the cases separately:
% \begin{itemize}
% \item $ y_R \in R $ then $f(y_R) \Rightarrow y_R \not \in R $
% \item $ y_R \not \in R $ then $ f(y_R) \Rightarrow y_R \in R$
% \end{itemize}
% Now we prove that the set of the parts of the natural numbers is not
% countable. We assume there is a surjective function
% $ \nat \rightarrow 2^\nat $. This means that I can enumerate subsets
% $ X_i $ s.t. $ i \in \nat $. At this point I create a matrix where
% each row $i$ of column $j = 1$ if $ x_i \in X_j $ and 0 otherwise. Now
% consider the inverted diagonal, that is if $ x_i \not \in X_i $, in
% this way it is different from all the columns, that is
% $ \forall i . R \not= X_i $ because
% $ n \in R \Leftrightarrow b \not \in X_n $ absurd since I assumed that
% $ \{X_0 \dots X_n \} = 2^\nat$.

% We conclude that $ |\nat| \not \geq |2^\nat| $

% But we want $ |\nat| < |2^\nat| $.

% But:
% $ |\nat| \leq |2^\nat| \land |\nat| \not\geq |2^\nat| \implies |\nat|
% < |2^\nat| $

% I take the set of the characteristic functions $g$ of $ 2^\nat $ and I
% call it $Y$, I take the set of all the functions $ \mathcal{F} $, of
% course it holds that $ Y \subseteq \mathcal{F} $ and therefore
% $ |Y| \leq |\mathcal{F}| $ but being that $ |\nat < |2^\nat| $ then I
% also have that $ |\nat| \leq |\mathcal{F}| $

% There is a total function that cannot be computed. We know how to
% enumerate computable functions because we can enumerate them in the
% form of numbers, repeating some of them. Now let's define $ f(n) $ as
% non-total computable.

% Matrix where the columns are the functions computed by the program
% $ i \in \nat $, that is $ \{\phi_i . i \in \nat \} $ and the rows are
% the arguments 0,1, \dots.

% \begin{equation*}
%   f(n) = \begin{cases}
%     \phi_n(n)+1 & $ se $\phi(n)\downarrow \\
%     0           & $ se $\phi(n)\uparrow
%   \end{cases}
% \end{equation*}

% we observe that $f$ is total by construction and
% $ f \not= \phi_n \forall n, n \in \nat $ Furthermore by construction
% it is different from all computable functions and therefore it is not
% computable.

\chapter {Parametrisation theorem}
\newcommand{\smn}{$S_n^m$}

We'll start by giving an intuition on what the theorem talks
about. Let $f : \nat^2 \rightarrow \nat$ be a computable
function, there exists $e \in \nat$ such that
 \[f(x,y) = \varphi_e^{(2)}(x,y)\]

Now, if we fix the first argument to a certain value $x \in \nat$, we obtain a
function of a single argument $f_x : \nattonat$
\[f_x(y) = f(x,y) = \varphi_e^{(2)}(x,y)\] and for all $ x \in \nat, \; f_x$
is computable (since it is obtained as composition of computable
functions). This means that there exists a $d \in \nat$ such that
\[f_x = \varphi_d\] in other words, for all $ y \in \nat$
\[f_x(y) = \varphi_e^{(2)}(x,y) = \varphi_d(y)\] Clearly $d$
depends on $e$ and $x$, so we can consider a with 
total function $s : \nat^2 \rightarrow \nat$
\[
  s(e, x) = d
  \]
The \textit{smn} theorem tells us that $s$ is computable. Intuitively,
how can we compute $s(e, x)$?
\begin{itemize}
\item get the program
  $P_e = \gamma^{-1}(e)$ that computes $\varphi_e^{(2)}(x,y)$
\item get the program that computes $f_x = \lambda y \; . \; f(x,y)$ with
  fixed $x$, from $P_e$:
  \begin{itemize}
  \item move $y$ to $R_2$;
  \item write $x$ on $R_1$;
  \item execute $P_e$
  \end{itemize}
\item take the code of the obtained program
\end{itemize}
Functions on indices, like $s$, are functions that
transform programs. The \emph{smn} theorem states that the operation of
fixing an argument of a program is effective.

\begin{example}
  Consider the computable function 
  \[f(x,y) = x^y\]
  We know that an
  index such that \(\varphi_d = f_x\) must exist, in other
  words \[\varphi_d(x,y) = f(x,y) = x^y\] So, when $x$ varies we
  obtain computable functions
  \begin{itemize}
  \item[] \(f_0(y) = y^0 = 1 \quad \mbox{ with index} \quad s(e,0)\)
  \item[] \(f_1(y) = y^1 = y \quad \mbox{ with index} \quad s(e,1)\)
  \item[] \(f_2(y) = y^2  \quad \mbox{ with index} \quad s(e,2)\)
  \item[] \(\dots\)
  \end{itemize}
  by \emph{smn} theorem we can determine those indices in an
  effective way. The theorem also does this in general for functions
  of the form $f(\vec{x}, \vec{y}) : \nat^{m+n} \rightarrow \nat$,
  which give the name to the theorem.
\end{example}

\section{\emph{smn} Theorem}
\begin{theorem}[\emph{smn} theorem]
  Given $m, n \geq 1$ there is a computable total function
  \[s_{m,n} : \nat^{m+1} \rightarrow \nat\] such that 
  \[
    \varphi_e^{(m+n)}(\vec{x},\vec{y}) = \varphi_{s_{m,n}(e,
      \vec{x})}^{(n)}(\vec{y}) \quad \forall e \in \nat, \; \vec{x} \in
    \nat^m, \vec{y} \in \nat^n
  \]
  \begin{proof}
    intuitively, given $e \in \nat$, $\vec{x}\in \nat^m$
    \begin{itemize}
    \item we get the program
      $P_e = \gamma^{-1}(e)$ in standard form that computes $\varphi_e^{(m+n)}$,
      so starting from
      \[
        \begin{tabu}{|c|c|c|c|c}
          \hline
          \vec{x} & \vec{y} & 0 & 0 & \dots \\ \hline
        \end{tabu}
        \quad \quad \mbox{it computes }
        \varphi_e^{(m+n)}(\vec{x},\vec{y})
      \]
    \item from $P_e$ we can build a new program $P'$.
      Starting from
      \[
        \begin{tabu}{|c|c|c|c}
          \hline
          \vec{y} & 0 & 0 & \dots \\ \hline
        \end{tabu}
        \quad \quad \mbox{it computes }
        \varphi_e^{(m+n)}(\vec{x},\vec{y})
      \]
    \end{itemize}

    In fact, it is sufficient to
    \begin{itemize}
    \item move $\vec{y}$ forward of $m$ registers
    \item load $\vec{x}$ in the free $m$ registers
    \item execute $P_e$
    \end{itemize}

    The program $P'$ can be
    \begin{center}
      \begin{tabular}{lr}
        $T(1, m+1)$               &          \\
        $\dots$                   &          \\
        $T(n, m+n)$               &          \\
        $z(1)$                    &          \\
        $s(1)$                    &          \\
        $\dots$                   & \comment{$x_1$ times} \\
        $s(1)$                    &          \\[1mm]
        $z(m)$                    &          \\
        $s(m)$                    &          \\
        $\dots$                   & \comment{$x_m$ times} \\
        $s(m)$                    &          \\[1mm]
        $P_e$                     & 
      \end{tabular}
    \end{center}
    \begin{remark}
      The concatenation has to update all the jump
    instructions in $P_e$,
    $J(m^\prime, n^\prime, t) \rightsquigarrow J(m^\prime, n^\prime, t
    + m + n + \sum_{i=1}^mx_i)$
    \end{remark}

    Once $P$ has been built, we have
    \[s(e, \vec{x}) = \gamma(P')\] 
    Each function and construction
    method used are effective (so are $\gamma, \gamma^{-1}$).
    Thus, the existence, totality and computability of $s$ are
    informally proven, by appealing to the Church-Turing
    thesis.

    The formal proof of computability is long, but not difficult. We next provide some hints. We first discuss how to define some auxiliary functions and the we use them to construct the smn-function.

    
    \subsection*{Update function}
      \newcommand{\up}[1]{\ensuremath{\mathit{upd}({#1})}}
      Consider
      \[
        \mathit{upd} : \nat^2 \rightarrow \nat
      \]
      where $\up{e, h}$ is the code of a program obtained from
      $P_e = \gamma^{-1}(e)$
     by updating each jump instruction $J(m,n,t)$ to $J(m,n,t+h)$.
      
     It is useful to define a support function that works
      on each single instruction encoded with $\beta$
      \newcommand{\tup}[1]{\ensuremath{\tilde{\mathit{upd}}({#1})}}
      \[
        \tilde{\mathit{upd}} : \nat^2 \rightarrow \nat
      \]
      where $\tup{i, h}$ is the code of the instruction $\beta^{-1}(i)$,
       updated if it is jump instruction.

      \newcommand{\qt}[1]{\ensuremath{\mathit{qt}({#1})}}
      \newcommand{\rmf}[1]{\ensuremath{\mathit{rm}({#1})}}

      Given $i,h \in \nat$ and $q=\qt{4,i}, r=\rmf{4,i}$ it is formally defined in this way
      \begin{align*}
        \tup{i,t} &= \begin{cases}
          4 * \nu(\nu_1(q), \nu_2(q), \nu_3(q) + h) + 3 & r=3 \\
          i & r \neq 3
        \end{cases} \\
        &= sg(r - 3) \cdot i + \bar{sg}(r - 3) \cdot (4 * \nu(\nu_1(q), \nu_2(q), \nu_3(q) + h) + 3)
      \end{align*}
      Now
      \begin{align*}
        \up{e,t} &= \tau(\tup{a(e,1), h}, \dots, \tup{a(e, l(e)), h}) \\
        &= \left(\prod^{l(e)-1}_{i=1}p_i^{\tup{a(e,i), h}}\right) \cdot p_{l(e)}^{\tup{a(e,l(e)), h} + 1}-2
      \end{align*}

\subsection*{Concatenation of sequences}
We will need a function
    \newcommand{\conc}[1]{\ensuremath{\mathit{c}({#1})}}
    \[
      \mathit{c}: \nat^2 \rightarrow \nat
    \]
    to concatenate sequences
    \[
        \conc{e_1, e_2} = \tau(a(e_1,1), \dots, a(e_1,l(e_1)),
        a(e_2,1), \dots, a(e_2, e(e_2)))
    \]
\subsection*{Concatenation of programs}
    \newcommand{\seq}[1]{\ensuremath{\mathit{seq}({#1})}}
    \[
      \mathit{seq} : \nat^2 \rightarrow \nat
    \]
where
    \[
      \seq{e_1, e_2} = \gamma \left( \begin{array}{c}
                                P_{e_1} \\
                                P_{e_2}
                              \end{array} \right)
                              = \conc{e_1, \up{e_2, l(e_2)}}
    \]

\subsection*{Transfer}
    \newcommand{\tran}[1]{\ensuremath{\mathit{transf}({#1})}}
    \[
      \mathit{transf}: \nat^2 \rightarrow \nat
    \]
where
\begin{align*}
\tran{m,n} &= \gamma(T([1,n],[m+1, m+n])) \\
&= \tau(\beta(T(1,m+1)), \dots, \beta(T(n,n+m))) 
\end{align*}

\noindent
\subsection*{Set}
    \newcommand{\set}[1]{\ensuremath{\mathit{set}({#1})}}
    \[
      \mathit{set} : \nat^2 \rightarrow \nat
    \]
\begin{align*}
  \set{i,x} &= \gamma\left( \begin{array}{c}
    z(i) \\
    s(i) \\
    \vdots \\
    s(i)
  \end{array} \right) \\
&= \tau(\beta(z(i)), \beta(s(i)), \dots, \beta(s(i))) \\
\end{align*}

\subsection*{Proof of the fact that the smn function is computable}
    Just define:
    \newcommand{\pref}[1]{\ensuremath{\mathit{pref}_{m,n}({#1})}}
    \[
      \mathit{pref}_{m,n}: \nat^m \rightarrow \nat
    \]
where
    \[
      \pref{\vec{x}} = \seq{\tran{m,n}, \seq{\set{1,x_1},\dots, \seq{\dots , \set{m,x_m}}} \dots}
    \]
    Then we have that
    \[
      s_{m,n}: \nat^{m+1} \rightarrow \nat
    \]
    \[
      s_{m,n}(e, \vec{x}) = \seq{\pref{\vec{x}}, e}
    \]
    which is in $\pr$
  \end{proof}
\end{theorem}

\subsection*{\emph{smn} Theorem applications}
\begin{observation}
  We proved that \emph{smn} is not just computable and total,
but is also primitively recursive. 
\end{observation}

Furthermore, the theorem is usually presented
in the following simpler shape
\begin{corollary}[Simplified \emph{smn} theorem]\label{corollary:simple-smn}
  Let $f: \nat^{m+n} \rightarrow \nat$ be a computable function. There
  exists a total computable function $s: \nat^m\rightarrow \nat$ such that
  \[
    f(\vec{x}, \vec{y}) = \varphi_{s(\vec{x})}^{(n)}(\vec{y}) \quad \quad
    \forall x \in \nat^m \quad \forall y \in \nat^n
  \]
  \begin{proof}
    Since $f$ is computable, given $e \in \nat$ and $s(\vec{x}) = s_{m,n}(e, \vec{x})$
    \begin{align*}
      f(\vec{x}, \vec{y}) &= \varphi_e^{(m+n)}(\vec{x}, \vec{y}) \\
      &= \varphi_{s_{m,n}(e, \vec{x})}^{(n)}(\vec{y})\\
      &= \varphi_{s(\vec{x})}^{(n)}(\vec{y})
    \end{align*}
  \end{proof}
\end{corollary}

\begin{example}
  Prove that there exists a total computable function $k : \nattonat$ such that
  for all $n, x \in \nat$
  \[
    \varphi_{k(n)}(x) = \lfloor \sqrt[n]{x} \rfloor
  \]
  It means that $k$ is an enumeration of functions of the form
  $\lfloor \sqrt[n]{x} \rfloor$ or $k$ is a function that given
  $n$, it returns the program that computes $\lfloor \sqrt[n]{x}
  \rfloor$.
\begin{proof}
  We define $f : \nat^2 \to \nat$
\begin{align*}
  f(n,x) = \lfloor \sqrt[n]{x} \rfloor &= \mu y \leq x \quad \mbox{``}(y+1)^n > x\mbox{''} \\
      &= \mu y \leq x \; . \; (x+1 \dotdiv (y+1)^n)
\end{align*}
  the function $f$ is computable because it is a bounded minimalisation of a composition of known computable functions.
  By Corollary~\ref{corollary:simple-smn}, there exists $k: \nattonat$ total computable such that for all $n, x \in \nat$
  \[
    \varphi_{k(n)}(x) = f(n,x) = \lfloor \sqrt[n]{x} \rfloor
  \]
\end{proof}
\end{example}

\begin{example}
  There exists $k : \nattonat$ computable and total such that for all $n \in \nat$,
  $\varphi_{k(n)}$ is defined only on $n^{th}$
  powers, i.e.
  \[
    W_{k(n)} = \{ \; x \; | \; \exists y \in \nat \ . \ x = y^n \; \}
  \]
\begin{proof}
We define $f : \nat^2 \rightarrow \nat$ as
  \begin{align*}
    f(n,x) &= \begin{cases}
      \sqrt[n]{x} & \quad \mbox{if } \exists y \in \nat \ . \ x = y^n \\
      \uparrow &\quad \mbox{otherwise}
    \end{cases} \\
    & = \mu y \; . \; |y^n-x|
  \end{align*}
  It is computable. By Corollary~\ref{corollary:simple-smn}, there exists $k: \nattonat$ total computable such that for all $n, x \in \nat$
  \[
    \varphi_{k(n)}(x) = f(n,x)
  \]
  We claim 
  \[
    W_{k(n)} = \{ \; x \; | \; \exists y \in \nat \ . \ x = y^n \; \}
  \]
  in fact, $x \in W_{k(n)}$ iff
  $\varphi_{k(n)}(x)\downarrow$ iff $f(n,x)\downarrow$ iff $x$ is the
  $n^{th}$ power.
\end{proof}
\end{example}

\begin{exercise}
  Prove that there exists a function $s: \nattonat$ which is total and computable
  such that
  \[W_{s(x)}^{(k)} = \{(y_1, \dots, y_k) \; | \;
    \sum\limits_{i=1}^ky_i = x)\}\]

  \textbf{Idea:} define
  \begin{align*}
    f(x, \vec{y}) &=
      \begin{cases}
        0 & \quad \sum_{i=1}^ky_i = x \\
        \uparrow & \quad \mbox{otherwise}
      \end{cases} \\
      &= \mu z \; . \; \left|\left(\sum_{i=1}^ky_i\right) - x\right|
  \end{align*}
\end{exercise}

\chapter {Universal Function}
\newcommand{\Psiex}{\ensuremath{\Psi_{\mathcal{U}}^{(k)} (e, \vec{x})}}
\newcommand{\Psiuex}{\ensuremath{\Psi_e^{(k)} (\vec{x})}}
\newcommand{\univ}{\ensuremath{\Psi_{\mathcal{U}}}}
We'll now see how the theory which was described up until now allows us to
prove something a bit surprising. This is to say, the existence of
universal functions/programs which are able to reproduce the behaviour of each
and every other computable function/program. 
Let's consider the function $\Psi : \nat^{2} \rightarrow \nat$
\[
  \Psi(x,y) = \varphi_e(y)
\]
We can observe that $\Psi$ is a \emph{universal function}, i.e. it
captures all unary computable functions $\varphi_1, \varphi_2, \dots$. In fact,
for all $e \in \nat$
\[
  g(y) = \Psi(e,y) = \varphi_e(y) \quad \rightsquigarrow \quad g = \varphi_e
\]
so, thanks to $e$, $\Psi$ represents all the computable functions of the
form $\nattonat$.

\begin{definition}
  The universal function for $k$-ary functions (with $k \in \nat$) is
  defined as
  \[
    \univ^{(k)} : \nat^{k+1} \rightarrow \nat
  \]
  \[
    \Psiex = \varphi_e^{(k)}(\vec{x})
  \]
\end{definition}
Is it computable? If yes, then a program $P_{\mathcal{U}}$ computing $\univ$ would
be able to compute all $k$-ary functions. It includes in itself all
other computable programs, and we can call it a Universal Computer
\cite{davis:2011}.

At first glance this might seem odd, but thinking again it receives in
input:
\begin{itemize}
\item $e$ (the index of the program, a \textit{description} of the
  program $P_e$ to run)
\item $\vec{x}$ the arguments
\end{itemize}
We observe that it fits the description of an interpreter.

In fact, the following result applies

\begin{theorem}
  $\forall k \geq 1$ the universal function $\Psi_{\mathcal{U}}^{(k)}$
  is computable.

  \begin{proof}
    Informally, we can say that if we fix $k \geq 1$ and an index
    $e \in \nat$ and the arguments $\vec{x} \in \nat^k$ we can compute
    $\Psiex = \Psiuex$ as follows:
    \begin{itemize}
    \item we build the program $P_e = \gamma^{-1}(e)$;
    \item we simulate $P_e$ on input $\vec{x}$
    \item if $P_e(\vec{x})\downarrow$, the value of $\Psiex$ is in
      $R_1$, otherwise it is okay.
    \end{itemize}
    This way everything is effective, and for the Church-Turing
    thesis computable.

    But we want to be more formal than this. More precisely we'll need
    some programs in order to accomplish this:

    \paragraph{Configuration of registers}
    Given the registers of the machine $(r_1, r_2, \dots, r_n)$
    \[
      \begin{tabu}{|c|c|c|c|c}
        \hline
        r_1 & r_2 & r_3 & 0 & \dots \\ \hline
      \end{tabu}
    \]
    the \textit{configuration of registers} is given by
    \[ c = \prod_{i \geq 1} p_i^{r_i} \]
    This way the value of the registers $\forall i \quad r_i = (c)_i$

    \paragraph{The state of the machine}
    The state of the machine is encoded with \[ \sigma = \Pi(j, c) \]
    where $j$ is the next instruction to execute and $c$ is the
    configuration of registers.

    We want to show that the functions

    \[
      c_k : \nat^{k+2} \rightarrow \nat
    \]

    \[
      c_k(e, \vec{x}, t) = \begin{cases}
        % Insert the function definition we saw in class
        x & 0 \\
        y & 1
      \end{cases}
    \]

    \[
      j_k : \nat^{k+2} \rightarrow \nat
    \]

    \[
      j_k(e, \vec{x}, t) = \begin{cases}
        % Insert the function definition we saw in class
        x & 0 \\
        y & 1
      \end{cases}
    \]

    At this point we have that
    \[\Psiex = \Psiuex = (c_k(e,\vec{x}, \mu t \; . \; j_k(e, \vec{x}, t)))_1\]
    so if we prove that $c_k, j_k$ are computable, we can conclude that
    $\Psi_{\mathcal{U}}^{(k)}$ is also computable. We proceed in the
    same way we did in the proof \S\ref{reqc}, by proving that
    $c_k, j_k \in \pr$ (in fact, this can be seen as a more formal
    proof of the same fact, the only difference is that we defined
    $c_p, j_p$ with \emph{a fixed $P$} in the latter demonstration.
    Nevertheless, $P$ is instead a parameter here).

    Explicitly, each step:

    \newcommand{\uarg}{{\mbox{arg}}}
    \newcommand{\uargh}{{\mbox{arg}_h}}
    \begin{enumerate}[label=(\alph*)]
    \item arguments of an URM instruction $( i = \beta(\mbox{Instruction}))$

      $Z_\uarg (i) = qt(4, i) + 1$

      $S_\uarg (i) = qt(4,i) + 1$

      $T_\uargh(i) = \Pi_h(qt(4,i)) + 1 \quad h \in \{1,2\}$

      $J_\uargh(i) = \nu_h(qt(4,i)) + 1 \quad h \in \{1,2,3\}$

    \item effect of executing an instruction on the configuration $C$

      \[
        \begin{tabu}{l l}
          \mbox{zero}(c,n) = qt(p_n^{(c)_n}, c) & Z(n) \\
          \mbox{succ}(c,n) = p_n \cdot c & S(n) \\
          \mbox{tfr}(c,m,n) = qt(p_n^{(c)_n}, x) \cdot p_n^{(c)_m} & T(m,n)
        \end{tabu}
      \]

    \item effect on the configuration of registers of the execution of
      the instruction $i=\beta(\mbox{Instruction})$

      \newcommand{\change}{\mbox{change}}
      \[
        \change(c,i) = \begin{cases}
          \mbox{zero}(c, Z_\uarg(i)) & rm(4,i) = 0 \\
          \mbox{succ}(c, S_\uarg(i)) & rm(4,i) = 1 \\
          \mbox{tfr}(c, T_{\uarg_1}(i), T_{\uarg_2}(i)) & rm(4,i) = 2 \\
          c & rm(4,i) = 3
      \end{cases}
    \]

  \item configuration of the registers if the current one ($c$) and
    the $t$ instruction of $P_e$ is executed

    \newcommand{\nextconf}{\mbox{nextconf}}
    \[
      \nextconf(e,c,t) = \begin{cases}
        \change(c, a(e,t)) & 1 \leq t \leq \ell(e) \\
        c & \mbox{otherwise}
      \end{cases}
    \]

  \item next configuration of registers if the $t^{\mbox{th}}$
    instruction $i=\beta(\mbox{Instruction})$ is executed

    \newcommand{\instr}{\mbox{instr}}
    \[
      \instr(c, i, t) = \begin{cases}

        t+1 & (rm(4,1) \neq 3) \vee (rm(4,i) = 3 \wedge (c)_{J_{\uarg_1(c)}} \neq (c)_{J_{\uarg_2(i)}}) \\
        J_{\uarg_3(c)} & \mbox{otherwise}
      \end{cases}
    \]

  \item number of the next instruction if the $t^{\mbox{th}}$
    instruction of $P_e$ is executed on the configuration $c$

    \newcommand{\nextinstr}{\mbox{nextinstr}}
    \[
      \nextinstr(e,c,t) = \begin{cases}
        \instr(c, a(e,t), t) & 1 \leq t \leq \ell(e) \wedge \instr(c, a(e,t), t) \leq \ell(e) \\
        0 & \mbox{otherwise}
      \end{cases}
    \]
  \end{enumerate}

  At this point we can define $c_k$ and $j_k$

  $c_k(e, \vec{x}, 0) = \prod_{i=1}^kp_i^{x_i}$

  $j_k(e, \vec{x}, 0) = 1$

  $c_k(e, \vec{x}, t+1) = \mbox{nextconf} (e, c_k(e, \vec{x}, t), j_k(e,\vec{x},t))$

  $j_k(e, \vec{x}, t+1) = \mbox{nextinstr} (e, c_k(e, \vec{x}, t), j_k(e,\vec{x},t))$

  Therefore

  \[
    \sigma_k(e,\vec{x},t) = \Pi(j_k(e,\vec{x},t), c_k(e,\vec{x},t))
  \]

  can be defined with primitive recursion, therefore $\sigma \in \pr \Rightarrow c_k, j_j \in \pr$

  \[
    \Psiex = c_k(e, \vec{x}, \mu t \; . \; j_k(e,\vec{x},t)) \quad \in \mathcal{R} = \mathcal{C}
  \]
  \end{proof}
\end{theorem}

As a corollary, we obtain the decidability of two statements that will
be really useful in the next chapters.

\begin{corollary}
  The following predicates are decidable:
  \begin{enumerate}[label=(\alph*)]
  \item $H_k(e, \vec{x}, t) \equiv$ ``$P_e(\vec{x})\downarrow$ in $t$
    or less steps''
  \item $S_k(e, \vec{x}, y, t) \equiv$ ``$P_e(\vec{x})\downarrow y$ in
    $t$ or less steps''
  \end{enumerate}
  \begin{proof}
    \begin{enumerate}[label=(\alph*)]
    \item We observe that
      $H_k(e, \vec{x}, t) \equiv (j_k(e,\vec{x}, t) = 0)$ and therefore
      we have

      $\chi_{H_k}(e, \vec{x}, t) = \overline{sg}(j_k(e,\vec{x},t))$
    \item We observe that
      $S_k(e, \vec{x}, y, t) \equiv ((j_k(e,\vec{x},t) = 0) \wedge ((c_k(e,\vec{x},t))_1)=y)$
    \end{enumerate}
  \end{proof}
\end{corollary}

Also, from the theorem we deduce the possibility to express every
computable function in the Kleene normal form (KNF).

\begin{corollary}[Kleene Normal Form]
  $\forall e,k \in \nat \quad \forall x \in \nat^k$
  \[
    \Psiuex = (\mu z \;.\; |\chi_{S_k}(e, \vec{x}, (z)_1, (z)_2) - 1|)_1
  \]
\end{corollary}

\textbf{Observations:}
\begin{enumerate}[label=\roman*.]
\item This corollary highlights how each computable function (or
  equivalently $\in\pr$) can be obtained from primitive recursion
  functions using minimization maximum one time (we can use just one
  \texttt{while})
\item Minimization allows us to ``search'' a single value that has a
  certain property. The one we used is a technique to search couples
  of values generalizable to tuples.
\end{enumerate}

\section{Applications of the universal function}
Reminding that we already observed that if $f : \nattonat$ is a total
computable injective function, then
\[
  f^{-1}(x) = \begin{cases}
    y & \quad \mbox{if exists $y$ s.t. } f(y) = x \\
    \uparrow & \quad \mbox{otherwise}
\end{cases}
\]
is computable since $f^{-1} = \mu y \; . \; |f(y) = x|$. We can verify
that the hypothesis of \emph{totalyty} can be omitted.

\begin{theorem}
  Let $f: \nattonat$ computable and injective. Then
  $f^{-1}: \nattonat$ is computable.
  \begin{proof}
    Since $f$ is computable, exists $e \in \nat$ s.t. $\varphi_e =
    f$. Now is sufficient to observe that
    \[
      \begin{split}
        f^{-1}(x) &= (\mu z \; . \;  "S(e, (z)_1, x, (z)_2)")_1 \\
        &= (\mu z \; . \; |\chi_S(e, (z)_1, x, (z)_2) - 1|)_1
      \end{split}
    \]
  \end{proof}
\end{theorem}

We can now find new non-computable functions and undecidable predicates:

\begin{theorem}
  The statement ``$\varphi_x$ is total'' is undecidable
  \begin{proof}
    Let $Tot(x)$ be the predicate
    \[ Tot(x) \equiv \mbox{ ``$\varphi_x$ is total''} \] and assume
    that it is decidable. Then the characteristic function
    \[
      \chi_{Tot}(x) = \begin{cases}
        1 & \varphi_x \mbox{ is total} \\
        0 & \mbox{otherwise}
      \end{cases}
    \]
    is computable. If this is the case, then the function
    \[
      g(x) = \begin{cases}
        \varphi_x (x) + 1 & \varphi_x \mbox{ total}\\
        1 & \mbox{otherwise}
      \end{cases}
    \]
    is both total and computable since by hypothesis ``$\varphi_x$
    total'' is decidable and $\varphi_x(x) + 1 = \Psi(x,x) + 1$ is
    computable. But by looking at the definition by cases that it is
    false, it is required for the used functions to be total:
    \[ g(x) \neq (\Psi(x,x) + 1) \cdot \chi_{Tot}(x) + 0 \cdot (1
      \dotdiv \chi_{Tot}(x))\] which is undefined if
    $\Psi(x,x) \uparrow$. The theorem of definition by cases continues
    to be valid for non total functions, but the proof must be
    changed. More in detail:
    \[
      \begin{split}
        g(x) &= (\mu z \;.\; \mbox{``}S(x,x,(z)_1, (z)_2) \wedge \lnot Tot(x)\mbox{''})_1 + 1 \\
        &= (\mu z \;.\; | \chi_{S(x,x,(z)_1, (z)_2) \wedge \lnot Tot(x)} -  1|)_1 + 1 \\
      \end{split}
    \]
    and $\forall x$ if $\varphi_x$ total
    $\Rightarrow g(x) = \varphi_x(x) + 1 \Rightarrow \varphi_x \neq
    g$. That is to say that $g$ is total and computable, but is
    absurdly different from every other total computable function
    Therefore $Tot(x)$ is not computable.
  \end{proof}
\end{theorem}

\textbf{Observation:} The same applies to prove that the following
statements are undecidable (Halting problem):
\begin{itemize}
\item
  $P_1(x) \equiv \mbox{``}x \ in W_x\mbox{''} \equiv
  \mbox{``}\varphi_x(x) \downarrow \mbox{''}$
\item
  $P_2(x,y) \equiv \mbox{``} y \in W_x \mbox{''} \equiv
  \mbox{``}\varphi_x(y) \downarrow\mbox{''}$
\end{itemize}

\section{Effective operations on computable functions}
The existence of the universal function, together with the \smn
theorem allows us to prove the effectivity of various
operations on indices of computable functions formally (in other words:
programs). For example:
\begin{description}
\item[product] given $x,y$ find $\omega$ s.t.
  $\varphi_\omega = \varphi_x \cdot \varphi_y \quad(\varphi_x\cdot
  \varphi_y(z) = \varphi_x(z) \cdot \varphi_y(z))$
\item[inverse] given $x$ find $\omega$ s.t.
  $\varphi_\omega = \varphi_x^{-1}$
\end{description}
It is more or less intuitive that these operations are effective on
programs, but is less obvious how they can be proven to be computable.
\subsection{Product}
Exists a function $s: \nat^2 \rightarrow \nat$ total and computable
s.t. \[\varphi_{s(x,y)} = \varphi_x \cdot \varphi_y\]
\begin{proof}
  we define a function where x and y are arguments, and then we
  transform them into parameters.
  \[
    \begin{split}
      g(x,y,z) &= \varphi_x(z) \cdot \varphi_y(z) \\
      &= \Psi_{\mathcal{U}}(x,z) \cdot \Psi_{\mathcal{U}}(y,z)
    \end{split}
  \]
  and that is computable by definition. For the \smn theorem there
  exists $s: \nat^2 \rightarrow \nat$ s.t.
  \[
    \varphi_{s(x,y)}(z) = g(x,y,z) = \varphi_x(z) \cdot \varphi_y(z)
  \]
  that is to say
  \[
    \varphi_{s(x,y)} = \varphi_x \cdot \varphi_y
  \]
\end{proof}

\subsection{Squaring}
Exists $k:\nattonat$ total and computable s.t.
$\varphi_{k(x)} = \varphi_x^2$
($\forall z \quad \varphi_{k(x)}(z) = (\varphi_x(z))^2$)
\begin{proof}
$k(x) = s(x,x)$
\end{proof}

\subsection{Effectivity of recursion}
remembering the notion of primitive recursion
\[h(\vec{x}, 0) = f(\vec{x})\]
\[h(\vec{x}, y+1) = g(\vec{x}, y, f(\vec{x},y))\] and knowing that
$f,g$ are computable $\Rightarrow h$ is computable, if
$f = \varphi_{e_1}^{(k)}$ and $g = \varphi_{e_2}^{(k+2)}$ exists an
index $ind = r(e_1, e_2)$ s.t. $h = \varphi_{ind}^{(k+1)}$. We want to
prove that $r: \nat^2 \rightarrow \nat$ is total and computable. In
other words, there exists $r: \nat^2 \rightarrow \nat$ total
computable s.t. $\forall e_1,e_2$ if we define
\[h(\vec{x}, 0) = f(\vec{x})\]
\[h(\vec{x}, y+1) = g(\vec{x}, y, f(\vec{x},y))\]
then
\[h = \varphi_{r(e_1, e_2)}^{(k+1)}\]

\begin{proof}
  We just need to define
  \[
    \begin{tabu}{l l l}
      h^\prime(e_1, e_2, \vec{x}, 0) &= \varphi_{e_1}^k(\vec{x}) &= \Psi_{\mathcal{U}}^{k}(e_1, \vec{x}) \\
      h^\prime(e_1, e_2, \vec{x}, y+1) &= \varphi_{e_2}^{k+2}(\vec{x},y,f(\vec{x}, y)) &= \Psi_{\mathcal{U}}^{k+2}(e_1, e_2, \vec{x}, y, h^\prime(e_1, e_2, \vec{x} , y))
    \end{tabu}
  \]
  $H^\prime$ defined by primitive recursion from computable functions,
  is computable, and for the \smn theorem there exists
  $r: \nat^2 \rightarrow \nat$ total and computable s.t.
  \[\varphi_{r(e_1, e_2)}(\vec{x}, y) = h^\prime (e_1, e_2, \vec{x}, y)\]
  as we wanted.
\end{proof}

\subsection{Effectiveness of the inverse function}
Exists $k: \nattonat$ total and computable s.t.
\[\forall x \in \nat \quad \mbox{if } \varphi_x \mbox{ is injetive }
  \Rightarrow \varphi_{k(x)} = (\varphi_x)^{-1}\]

\begin{proof}
  we can define a function $g(x,y)$ that for those $x$
  s.t. $\varphi_x$ is injective it:
  \[
    g(x,y) = \varphi_x^{-1}(y) = \begin{cases}
      z & \exists z \mbox{ s.t. } \varphi_x(z) = y \\
      \uparrow & \mbox{otherwise}
    \end{cases}
  \]
  \[
    g(x,y) = (\mu \omega \; . \; |\chi_{S(x, (\omega)_1, y, (\omega)_2)} - 1|)_1
  \]
  and for the \smn theorem exists a $k: \nattonat$ total and
  computable s.t.
  \[\varphi_{k(x)}(y) = g(x,y) = \varphi_x^{-1}(y) \mbox{ if }
    \varphi_x \mbox{ is injective}\]
\end{proof}

\section{manipulating domains and co-domains}
(a) Exists a total computable function $S : \nat^2 \rightarrow \nat$
s.t. \[W_{S(x,y)} = W_x \cup W_y\]
\begin{proof}
  we want a function s.t. $\varphi_{S(x,y)}(z)\downarrow$ iff
  $\varphi_x(z)\downarrow$ or $\varphi_y(z) \downarrow$. We can define
  a function with $x,y$ as arguments that captures both properties

  \[
    g(x,y,z) = \begin{cases}
      0 & z \in W_z \mbox{ or } z \in W_y \\
      \uparrow  & \mbox{otherwise}
    \end{cases}
  \]
  which is computable:
  \[
    g(x,y,z) = \underline{0}(\mu \omega \; . \; |\chi_{H(x,z,\omega)
      \wedge H(y,z,\omega)} - 1|)
  \]
  \textbf{Note:} one $\omega$ is enough, since
  $\exists \omega_1 \; . \; P_1(\omega_1) \wedge \exists \omega_2 \;
  . \; P_2(\omega_2) \equiv \exists \omega \; . \; P_1(\omega) \wedge
  P_2(\omega)$

  So, for the \smn theorem exists $s:\nat^2 \rightarrow \nat$
  computable and total s.t. \[\varphi_{S(x,y)}(z) = g(x,y,z)\]
\end{proof}

(b) Exists $k:\nat^2 \rightarrow \nat$ computable and total s.t.
\[\forall x,y \quad E_{K(x,y)} = E_x \cup E_y\]
\begin{proof}
  We want the value of $\varphi_{S(x,y)}$ to be the same of the
  functions $\varphi_x and \varphi_y$. In order to do this, we can simulate
  $\varphi_x$ on even numbers and $\varphi_y$ on odd numbers. We
  define a function where $x$ and $y$ are arguments
  \[
    g(x,y,z) = \begin{cases}
      \varphi_x(\frac{z}{2}) & \mbox{if } z \mbox{ even} \\
      \varphi_y(\frac{z-1}{2}) & \mbox{if } z \mbox{ odd}
    \end{cases}
  \]
  computable since
  \begin{multline*}
    g(x,y,z) = (\mu \omega \; . \; |\max\{\chi_S(x,qt(2,z),(\omega)_1,(\omega)_2) \; \cdot \; \overline{sg}(rm(2,z)), \\ \chi_S(y,qt(2,z),(\omega)_1, (\omega)_2) \; \cdot \; sg(rm(2,z))\}- 1|)_1
  \end{multline*}
  And for the \smn theorem exists $k: \nat^2 \rightarrow \nat$
  computable and total s.t. \[\varphi_{K(x,y)}(z) = g(x,y,z)\]
  So
  \[
    \begin{split}
      \nu \in E_{K(x,y)} & \Leftrightarrow \exists z \; . \; \varphi_{S(x,y)}(z) = g(x,y,z) = \nu \\
      & \Leftrightarrow \exists z \; . \; \begin{cases}
        z \mbox{ even and } & \varphi_x(\frac{z}{2}) = \nu \\
        z \mbox{ odd and } & \varphi_y(\frac{z-1}{2}) = \nu \\
      \end{cases} \\
      & \Leftrightarrow \exists z \; . \; \varphi_x(z) = \nu \wedge
      \varphi_y(z) = \nu \Leftrightarrow \omega \in E_x \cup E_y
    \end{split}
  \]
\end{proof}

(c) Exists $k : \nattonat$ computable and total s.t. $E_{k(x)} = W_x$
\begin{proof}
  We want
  \[(y \in W_x \Leftrightarrow y \in E_{k(x)}) \equiv
    (\varphi_x(y)\downarrow \Leftrightarrow \exists z \; . \;
    \varphi_{k(x)}(z) = y) \]
  and we can define
  \[
    \begin{split}
      g(x,y) &= \begin{cases}
        y & y \in W_x \\
        \uparrow & \mbox{otherwise}
      \end{cases} \\
      &= \mathds{1}( \univ (x,y)) \cdot y
    \end{split}
  \]
  which is computable, so for the \smn theorem exists $k : \nattonat$
  computable and total s.t. \[\varphi_{k(x)} = g(x,y)\]
  in other words
  \[y \in E_{k(x)} \Leftrightarrow \varphi_{k(x)}(y) = y
    \Leftrightarrow g(x,y) = y  \Leftrightarrow y \in W_x\]
\end{proof}

(d) Given $f : \nattonat$ computable, exists $k : \nattonat$
computable and total s.t. $\forall x \quad W_{k(x)} = f^{-1}(W_x)$
\begin{proof}
  we want a function s.t.
  \[y \in W_{k(x)} \Leftrightarrow f(y) \downarrow \mbox{ and } f(y)
    \in W_x \]
  in other words
  \[\varphi_{k(x)}(y) \downarrow \Leftrightarrow \varphi_x(f(y))
    \downarrow\]
  so we can define
  \[g(x,y) = \varphi_x(f(y)) = \univ(x, f(y))\] computable by
  definition. So for the \smn theorem exists $k: \nattonat$ computable
  and total s.t. \(\varphi_{k(x)}(y) = g(x,y)\). So
  \[
    \begin{split}
      y \in W_{k(x)} & \Leftrightarrow \varphi_{k(x)}(y) = g(x,y) = \varphi_x(f(y)) \downarrow \\
      & \Leftrightarrow f(y)\downarrow \mbox{ and } f(y) \in W_x \\
      & \Leftrightarrow y \in f^{-1}(W_x)
    \end{split}
  \]
\end{proof}

\section{Operations on predicates}
Exists $k : \nattonat$ computable and total s.t. if
$\varphi_x = \chi_a$ is the characteristic function of a decidable
predicate $Q$, then $\varphi_{k(x)} = \chi_{\neg Q}$
\begin{proof}
  we can define \[g(x,y) = 1 \dotdiv \varphi_x(y) = 1 - \univ(x,y) \]
  which is computable by definition. So, for the \smn theorem exists
  $k$ computable and total s.t. \[g(x,y) = \varphi_{k(x)}\] this way
  if $\varphi_x = \chi_Q$
  \[
    g(x,y) = 1-\varphi_x(y) = \varphi_{k(x)}(y) = 1 \Leftrightarrow
    \varphi_x(y) = 0 \Leftrightarrow \chi_Q(y) = 0
  \]
  therefore
  \[
    \varphi_{k(x)} = \chi_{\neg Q}
  \]
\end{proof}

\chapter{Recursive sets}
In previous chapters, we saw many computable functions and decidable problems, but
only in few cases we gave examples of the large class of
non-computable functions and undecidable predicates. For this reason,
we want to start a mathematical study of
\begin{itemize}
  \item classes of undecidable predicates
  \item techniques to prove the undecidability of some predicates
\end{itemize}
In this way we can highlight the limits of computers abilities and give a
structure to problems classes (the majority of interesting
predicates are undecidable).

We'll focus on \emph{numerical sets} $X \subseteq \nat$ and try to find
an answer to the problem ``$x \in X$?''. We'll get
\begin{itemize}
\item recursive sets
\item recursively enumerable sets
\end{itemize}

\section{Recursive sets}
\begin{definition}
  A set $A \subseteq \nat$ is \emph{recursive} if its characteristic
  function
  \begin{gather*}
    \chi_A : \nattonat \\
    \chi_A(x) = \begin{cases}
      1 & x\in A \\
      0 & x \notin A
    \end{cases}
  \end{gather*}
  is computable.
\end{definition}
In other words, if the predicate ``$x \in A$'' is decidable.

\begin{observation}
  \begin{itemize}
    \item if $\chi_A \in \pr$ we'll say that $A$ is \emph{primitively}
      recursive.
    \item the notion can be extended to subsets of $\nat^k$, but we'll
      stick to $\nat$ subsets, since every $\nat^k$ subset can be encoded
      into $\nat$
    \end{itemize}
\end{observation}


\begin{example}
  These are recursive:
\begin{enumerate}[label=(\alph*)]
\item $\nat$, since $\chi_\nat = \mathbf{1}$ is computable;
\item $\varnothing$, because $\chi_\varnothing = \mathbf{0}$ is computable;
\item prime numbers $\mathbb{P}$, since
  \[
    Pr(x) = \begin{cases}
      1 & \mbox{if $x$ is prime} \\
      0 & \mbox{otherwise}
    \end{cases}
  \]
  is computable;
\item each and every finite set. In fact, given $A \subset \nat$ with
  $|A| < \infty$, $A = \{x_1, x_2, \dots, x_n\}$, we have that
  \[
    \chi_A(x) = \overline{sg}\left( \prod_{i=1}^n|x - x_i| \right)
  \]
  is computable.
\end{enumerate}
\end{example}


On the other hand, these are definitely not recursive:
\begin{enumerate}[label=(\alph*)]
\item $K = \left\{ x \; | \; x \in W_x \right\} $, since
  \[
    \chi_{K}(x) = \begin{cases}
      1 & x \in W_x \\
      0 & x \notin W_x
    \end{cases}
  \]
  is not computable;
\item $\left\{ x \; | \; \varphi_x \mbox{ total} \right\} $
\end{enumerate}

\begin{observation}
  If $A,B \subseteq \nat$ are recursive, then
\begin{enumerate}[label=\arabic*)]
\item $\overline{A} = \nat - A$
\item $A \cap B$
\item $A \cup B$
\end{enumerate}
are recursive.
\end{observation}


\subsection{Reduction}
Reduction is a process used to study decidability
problems. It formalizes the intuition of a problem
$\mathcal{A}$ being ``easier'' then another one, $\mathcal{B}$.

\newcommand{\red}{\ensuremath{\leq_m}}
\begin{definition}
  Let $A,B \subseteq \nat$. We say that the problem $x \in A$
  \emph{reduces} to the problem $x \in B$ ($A$ reduces to $B$), 
  written $A \red B$ if there exists
  $f : \nattonat$ computable and total such that, for every $x \in \nat$
  \[x \in A \quad  \Leftrightarrow \quad f(x) \in B\]
\end{definition}
In this case, $f$ is the \emph{reduction function}.

\begin{observation}
  Let $A,B \subseteq \nat$ such that $A \red B$ then
\begin{enumerate}[label=\arabic*]
\item if $B$ is recursive, then $A$ is recursive
\item of $A$ is not recursive, then $B$ is not recursive
\end{enumerate}

\begin{proof}
Just observe that $\chi_A = \chi_B \circ f$
\end{proof}
\end{observation}

We know that $K = \{ x \mid x \in W_x \}$ is not recursive. Let's see how
the recursiveness of other sets can be proven by reduction to this
set which we know for certain is not recursive.
\begin{example}
  $K \red T = \{x \mid \varphi_x $ total$ \}$
  \begin{proof}
    we need to prove that there exists $s : \nattonat$ computable and total
    such that $x \in k \Leftrightarrow s(x) \in T$. In other words
    \[ x\in W_x \Leftrightarrow  \varphi_{f(x)} \mbox{ is total} \]
    To do so, we can define
    \[
      g(x,y) = \begin{cases}
        1 & x\in W_x \\
        \uparrow & \mbox{otherwise}
      \end{cases}
    \]
    which is computable. This fact is easily proven by rewriting it as
    \[
      g(x,y) = \mathbf{1}(\varphi_x(x)) = \mathbf{1}(\univ(x,x))
    \]
    then, by \emph{smn} theorem we have that there exists $s: \nattonat$
    computable and total such that 
    \[\varphi_{s(x)}(y) = g(x,y)\] and 
    \[x \in K \Rightarrow x \in W_x \Rightarrow \forall y\
      \varphi_{s(x)}(y) = g(x,y) = 1 \Rightarrow \varphi_{s(x)} \mbox{
        total } \Rightarrow s(x) \in T\]
    \[x \notin K \Rightarrow x \notin W_x \Rightarrow \forall y\
      \varphi_{s(x)}(y) = g(x,y) \uparrow \Rightarrow \varphi_{s(x)}
      \mbox{ not total } \Rightarrow s(x) \notin T\]
  \end{proof}
\end{example}

\begin{example}[Input problem]
  For every $ n \in \nat$
  \[
    A_n = \{x  \mid \varphi_x (n) \downarrow\}
  \]
  is not recursive.

  \begin{proof}
    We will prove that $K \leq A_n$. We have to define a function $f$
    s.t.
    \begin{gather*}
      x \in K \Leftrightarrow f(x) \in A_n \\
      x \in W_x \Leftrightarrow \varphi_{f(x)}(n) \downarrow
    \end{gather*}
    Define
    \begin{align*}
      g(x,y) &= \begin{cases}
        1 & x \in W_x \\
        \uparrow & \mbox{otherwise}
      \end{cases} \\
      &= \mathbf{1}(\univ(x,x))
    \end{align*}
    which is computable, and therefore by the \emph{smn} theorem, there exists
    $f: \nattonat$ computable and total such that
    $g(x,y) = \varphi_{f(x)}(y)$, and
    \begin{gather*}
      x \in k \Rightarrow f(x) \in A_n \\
      x \notin  k \Rightarrow f(x) \notin A_n
    \end{gather*}
  \end{proof}
\end{example}

\begin{example}[The output problem]
  For every $ n \in \nat$, $B_n\{ x \mid n \in E_x\}$ is not recursive
  \begin{proof}
    We show that $K \leq_m B_n$
    \[
      \begin{split}
        g(x,y) &= \begin{cases}
          n & x \in W_x \\
          \uparrow & \mbox{otherwise}
        \end{cases} \\
        &= n \cdot \mathbf{1}(\univ(x,x))
      \end{split}
    \]
    computable, by \emph{smn} theorem, there exists a function
    $s : \nattonat$ such that
    \[
      \forall x,y \quad g(x,y) = \varphi_{s(x)}(y)
    \]
  moreover
  \begin{gather*}
    x \in k \Rightarrow s(x) \in B_n \\
    x \notin k \Rightarrow s(x) \notin B_n
  \end{gather*}
\end{proof}
\end{example}

\begin{observation}
Let $A,B \subseteq \nat$ with $A \leq_m B$
through an injective reduction function $f : \nattonat$ (total and computable). 
One could think that, since $f^{-1}$ is computable, then
also $B \leq_m A$. This does not happen, since $f^{-1}$ is
not total and so reduces $A$ to a ``subproblem'' of $B$ even though
it has no complexity relationships with $B$.
\end{observation}

\chapter {Rice theorem}

Rice's theorem gives a general undecidability result. It states that \emph{no property} of computable functions
(besides the obvious ones) is decidable.

Formally, we'll need the notion of \emph{saturated} set.
\section{Saturated sets}
\begin{definition}[Saturated set]
  A subset $A \subseteq \nat$ is \emph{saturated} (or \emph{extensional}) if
  \[
     x \in A \wedge \varphi_x = \varphi_y \Rightarrow y \in A 
  \]
  In other words, $A$ is saturated if it expresses a property of
functions, independently from indices
\[A = \{x \mid P(\varphi_x)\}\]
or, again, if there exists $\mathcal{A} \subseteq \mathcal{C}$ such that
\[A = \{ x \mid \varphi_x \in \mathcal{A}\}\]
\end{definition}

\begin{example}
  The following set is saturated
  \begin{align*}
    T &= \{ n \mid P_n \mbox{ always terminate} \} \\
    &= \{n \mid \phi_n \in \mathcal{T} \}
  \end{align*}
  where
  \[ \mathcal{T} = \{ f \mid f \mbox{ is total} \} \]
\end{example}

\begin{example}
  The following set is saturated
  \begin{align*}
    ONE &= \{ n \mid P_n \mbox{ computes } \mathbf{1}\} \\
    &= \{n \mid \phi_n = \mathbf{1} \} \\
    &= \{n \mid \phi_n \in \{ \mathbf{1} \} \}
  \end{align*}
\end{example}

\begin{example}
  Consider
  \begin{align*}
    T_2 &= \{ e \mid P_e(e)\downarrow \mbox{ in two steps } \} \\
    &=
    \{e \mid \phi_e \in \mathcal{T}_2 \}
  \end{align*}
  two programs can calculate the same function, one terminates in
  less than 2 steps and the other in more than 2. Thus, the set is not
  saturated.
\end{example}

\begin{example}
  Consider
  \begin{align*}
    K &= \{e \mid e\in W_e \} \\ 
    &= \{e \mid \phi_e\in \mathcal{K} \}
  \end{align*}
  where we would like
  \[   \mathcal{K} = \{f \mid ? \} \]
  It is not saturated. It can be shown that there is a program $e$ such that
  \begin{equation*}
    \phi_e(x) = \begin{cases}
      0 & x = e \\
      \uparrow & $ otherwise $
    \end{cases}
  \end{equation*}
\end{example}

\section {Rice's theorem}

\begin{theorem}[Rice's theorem]
  Let $ A \in \nat, A \neq \emptyset, A \neq \nat$ be saturated.
  Then it is not recursive.
\end{theorem}

\begin{proof}
  We show that $ K \red A $. 
  Let $ e_0$ such that $ \phi_{e_0}(x)\uparrow\forall x $
  \begin{itemize}
  \item[($e_0 \notin A$)]
     Suppose
    $ e_0\notin A $ and let $ e_1\in A $.
    Now define
    \begin{align*}
      g(x,y) &= \begin{cases}
        \phi_{e_1}(y) & x \in K \\
        \phi_{e_0}(y) & x \notin K
      \end{cases} \\
      &=
      \begin{cases}
        \phi_{e_1}(y) & x \in K \\
        \uparrow & x \notin K
      \end{cases} \\
      &= \phi_{e_1}(y) \cdot \mathbf{1}(\univ(x,x))
    \end{align*}
    it is computable. By \emph{smn} theorem there is $s \nat \to \nat$ such that
    $ \phi_{s(x)}(y) = g(x,y)$.

    Now observe that $s$ is a reduction function for $K \red A$
    \begin{itemize}
      \item $x \in K
      \Rightarrow \forall y  \ \varphi_{s(x)}(y) = \varphi_{e_1}(y)
      \Rightarrow s(x) \in A$
      \item $x \notin K
      \Rightarrow \forall y  \ \varphi_{s(x)}(y) = \varphi_{e_0}(y)\uparrow 
      \Rightarrow s(x) \notin A$
    \end{itemize}
    Hence $K \leq_m A$, $K$ not recursive, thus $A$.
    
  \item[($e_0 \in A$)] If $ e_0 \in A $ then $ e_0 \not \in \bar{A}
    $. Then
    $ \bar{A} \subseteq \nat, \bar{A} \neq \emptyset, \bar{A} \neq
    \nat $, so $ \bar{A} $ is not recursive, and
    therefore $A$ is not recursive either.
  \end{itemize}
\end{proof}

\begin{example}[Output problem]
  \[ B_n = \{e \mid n \in E_e \} \quad \mbox{is not recursive} \]
  we showed that $K \red B_n$. We can conclude the same by observing
  \begin{itemize}
  \item $B_n$ is saturated;
  \item $B_n \neq \emptyset$;
  \item $B_n \neq \nat$.
  \end{itemize}
  By Rice's theorem $ B_n $ is not recursive.
\end{example}
\chapter{Recursively enumerable sets}

\begin{definition}[Recursively enumerable set]
  We say that $ A \subseteq \nat $ is \emph{recursively enumerable} if
  the semi-characteristic function 
  \begin{equation*}
    sc_A(x) = \begin{cases} 1 & x \in A \\ \uparrow & $
      otherwise $
    \end{cases}
  \end{equation*}
  is computable.
\end{definition}

\begin{definition}[Semi-decidable predicate]
  A predicate $ Q(x) \subseteq \nat $ is\\
   semi-decidable if 
  \( \{ x \in \nat \mid Q(x) \} \)
  is r.e.
\end{definition}

Thus, saying that $A$ is r.e. is like saying that the predicate $ Q(x)=``x \in A"
$ is semi-decidable. This notion is also easily generalisable to
\begin{itemize}
\item subsets of $\nat^k$
\item $k$-ary predicates
\end{itemize}

\begin{theorem}\label{th:aiffanota}
  \[A \subseteq \nat \mbox{ recursive } \Leftrightarrow A, \bar{A} \mbox{
      are r.e.} \]
  \begin{proof}
    \begin{itemize}
    \item[($\Rightarrow$)]
      If $A$ recursive,
      \begin{equation*}
        \mathcal{X}_A(x)= \begin{cases}
          1 & x\in A \\
          0 & $ otherwise $
        \end{cases}
      \end{equation*}
      is computable.
      Then $ sc_A(x) = \mathbf{1}(\mu z. | \chi_A(x)- 1 | )$
      is computable. Computable, therefore \textit{A} is r.e. Also
      $\bar{A}$ is r.e. In fact, since $A$ is r.e., by closure on
      negation, also $\bar{A}$ is r.e.

    \item[($\Leftarrow$)] Let $A, \bar{A}$ be r.e., then by definition
      $sc_a$ and $sc_B$ are computable, and we can define
      \[
        \mathds{1} - sc_{\bar{A}}(x) = 1 - sc_{\bar{A}}(x) = \begin{cases}
          0 & x \in \bar{A} \\
          \uparrow & \mbox{otherwise}
        \end{cases}
      \]
      is computable. This means that $\exists e_0, e_1 \in \nat$ s.t.
      \[
        \varphi_{e_0} = sc_A \quad \varphi_{e_1} = \mathds{1} -
        sc_{\bar{A}}
      \]
      therefore intuitively we can ``combine two machines'' until one
      of the two terminates. Note that since either $x \in A$ or $x \in \bar{A}$,
      it will terminate for sure. We can build the characteristic function of $A$:
      \[
          \chi_A(x) = (\mu \omega \; . \; |\chi_{s(e_0, x,
            (\omega)_1, (\omega)_2 \wedge s(e_1, x, (\omega)_1,
            (\omega)_2))}-1|)_1
        \]
        which is computable, therefore $A$ is recursive.
    \end{itemize}
  \end{proof}
\end{theorem}

\begin{observation}
  the set $k = \{x \mu x \in W_x\}$ is r.e. In fact
  \[
    sc_k(x) = \begin{cases}
      1 & x \in k \\
      \uparrow & \mbox{otherwise}
    \end{cases}
    = \mathds{1}(\varphi_x(x)) = \mathds{1}(\univ(x,x))
  \]
  is computable by definition and for the theorem \ref{th:aiffanota}
  \[
    \bar{K} = \{x \mid x \notin W_x\}
  \]
  is \emph{not} r.e. Otherwise $k,\bar{k}$ would have been both r.e.,
  and therefore $k$ would have been recursive, which is a
  contradiction.
\end{observation}

\section {Theorem of structure of semi-decidable predicates}

\begin{theorem}[Structure of semi-decidable predicates]\label{th:structure}
  Let $ Q(\vec{x}) \subseteq \nat^k $ be a predicate.

  This is decidable $ \Leftrightarrow $ there is a predicate $
  Q'(t,\vec{x}) \subseteq \nat^{k+1} $ s.t. $ Q(\vec{x}) = \exists
  t. Q'(t,\vec{x}) $
  \begin{proof}
    \begin{itemize}
    \item[($\Rightarrow$)] Let $P(\vec{x})$ be
      semi-decidable. Therefore it has a semi characteristic function
      $sc_P$ and by definition
      \[
        P(\vec{x}) \equiv \exists t \; . \; H(e,\vec{x}, t)
      \]
      therefore if we can rewrite $H$ as
      $Q(t, \vec{x}) = H(e,\vec{x}, t)$, this way $Q$ is decidable as
      we wanted. \[P(\vec{x}) \equiv \exists t \; . \; Q(t, \vec{x})\]

    \item[($\Leftarrow$)] Let
      \(P(\vec{x}) \equiv \exists t \; . \; Q(t, \vec{x})\) be
      decidable. We can build
      \[
        sc_P(\vec{x}) = \mathds{1}(\mu t \; . \; |\chi_Q(t,\vec{x}) - 1|)
      \]
      which is computable by definition, and therefore $P(\vec{x})$ is
      semi-decidable
    \end{itemize}
  \end{proof}
\end{theorem}

\section {Projection theorem}

From the last theorem we had a hint about the fact that the class of
semi-decidable predicates is closed under \emph{existential
  quantification}. The projection theorem further states that:

\begin{theorem}[Projection theorem]
  Let $ P(x,\vec{y}) $ be semi-decidable; then
  \[
    \exists x \mbox{ s.t. } P(x,\vec{y}) = P'(\vec{y})
  \]
  is semi-decidable.

  \begin{proof}
    Let $P(x,\vec{y})$ be semi-decidable. The by theorem
    (\ref{th:structure}) exists $Q(t,x,\vec{y})$ decidable s.t.
    \[
      P(\vec{x}, vec{y}) \equiv \exists t \; . \; Q(t,x,\vec{y})
    \]
    then if $P^\prime(\vec{y}) = \exists x \; . \; P(x, \vec{y})$ it
    holds that
    \[
      \begin{split}
        P^\prime(\vec{y}) &\equiv \exists x \exists t \; . \;
        Q(t,x,\vec{y}) \\
        &\equiv \exists \omega \; . \; Q((\omega)_1, (\omega)_2, \vec{y})
      \end{split}
    \]
    therefore since the last one is a decidable predicate, again for
    the theorem (\ref{th:structure}) $P^\prime(\vec{y})$ is
    semi-decidable
  \end{proof}
\end{theorem}

So if you use existential identifier, you go out of the set of the
decidable predicates and enter the semi-decidable set, but if you use
it twice you don't go outside the semi-decidable set.

\begin{theorem}[Closure property]
  Let $ P_1(\vec{x}), P_2(\vec{x}) $ be semi-decidable predicates. Then
  \begin{itemize}
  \item $  P_1(\vec{x}) \lor P_2(\vec{x}) $;
  \item $ P_1(\vec{x}) \land P_2(\vec{x}) $
  \end{itemize}
  are semi-decidable.

  \begin{proof}
    Let $ P_1(\vec{x}), P_2(\vec{x}) $ be semi-decidable
    predicates. Then for (\ref{th:structure}) there are two
    decidable predicates: \( Q_1(t, \vec{x}), Q_2(t, \vec{x})\) s.t.
    \begin{gather*}
      P_1(\vec{x}) \equiv \exists t \; . \; Q_1(t, \vec{x}) \\
      P_2(\vec{x}) \equiv \exists t \; . \; Q_2(t, \vec{x})
    \end{gather*}

    so
    \begin{enumerate}[label=(\arabic*)]
    \item
      \[
        \begin{split}
          P_1(\vec{x}) \lor P_2(\vec{x}) &\equiv \exists t \; . \;
          Q_1(t, \vec{x}) \lor \exists t \; . \; Q_2(t, \vec{x}) \\
          &\equiv \exists \omega \; . \; (Q_1((\omega)_1, \vec{x})
          \lor Q_2((\omega)_2, \vec{x}))
        \end{split}
      \]
      This means that for (\ref{th:structure}),
      $P_1(\vec{x}) \lor P_2(\vec{x})$ is decidable.

    \item
      \[
        P_1(\vec{x}) \land P_2(\vec{x}) \equiv \exists t \; . \;
        (Q_1(t, \vec{x}) \land Q_2(t, \vec{x}))
      \]
      which is decidable by definition.
    \end{enumerate}
  \end{proof}
\end{theorem}

\begin{observation}
  The set of semi-decidable predicates is closed under $\land, \lor$
  and $\exists$. It is not closed under $\forall$ and $\lnot$
\end{observation}

\begin{exercise}
  Prove that if $ P(\vec{x}) $ is semi-decidable and is not decidable
  then $ \lnot P(\vec{x}) $ is not semi-decidable.
\end{exercise}

\begin{observation}
  The latter results have an active translation in terms of r.e. sets
  properties:
  \begin{enumerate}[label=(\arabic*)]
  \item $A \subseteq \nat$ is recursive iff $A, \bar{A}$ are r.e.
  \item if $A \subseteq \nat$ r.e. and $f : \nattonat$ computable
    $\Rightarrow f^{-1}(A)$ is r.e. (projection).
  \item us $A,B \subseteq \nat$ r.e. $\Rightarrow A \cup B, A \cap B$
    are r.e.
  \end{enumerate}
\end{observation}

\subsection{r.e. sets and reducibility}

Properties similar to those already seen for recursive sets hold. That
is to say:

\begin{observation}
  $ A,B \subseteq \nat, A\leq_m B $ then:
  \begin{itemize}
  \item B is r.e. $ \Rightarrow $ A is r.e .;
  \item A is not r.e. $ \Rightarrow $ B not r.e.
  \end{itemize}
  \begin{proof}
    If $B$ r.e. then
    \begin{equation*}
      SC_B(x) = \begin{cases}
        1 & x \in B \\
        \uparrow & $ otherwise $
      \end{cases}
    \end{equation*}
    Which is computable.  Let $ f:\nat\rightarrow\nat $ be a total
    computable reduction function\\ $ A\leq B $ Then
    $ SC_A(x) = SC_B(f(x)) $, therefore $ SC_A $ is computable by
    composition. Therefore $A$ is r.e.

    To prove that $B$ is also r.e., we just need to reverse the names in
    the last proof (exercise).
  \end{proof}
\end{observation}

\chapter {Rice-shapiro theorem}
Rice-shapiro states that a property of the functions computed by
programs can be semi-decidable \textbf{only if} it depends on a finite
part of the function (behavior on finite inputs). 
\begin{theorem}[Rice-shapiro theorem]
  Let $\mathcal{A} \subseteq \mathcal{C}$ be a set of computable
  functions. If the set $A = \{x \mid \varphi_x \in \mathcal{A}\}$ is
  r.e., then
  \[
    \forall f (f \in \mathcal{A} \Leftrightarrow \exists \theta \mbox{
      finite function, } \theta \subseteq f \land \theta \in
    \mathcal{A})
  \]

In order for us to prove this theorem, we'll need some more tools:
\begin{definition}[Finite function]
  A finite function is a function $ \theta: \nat\rightarrow\nat $
  such that $ dom(\theta) $ is finite.  This means that the set of
  input-output pairs is finite; in other words
  $ \theta = \{(x_1,y_1),\dots(x_n,y_n) \} $ and it is undefined on other inputs.
\end{definition}

\begin{definition}
  Given $ f:\nat\rightarrow\nat $, $ \theta $ is a sub-function of
$f$ if $ \theta \subseteq f $
\end{definition}


\begin{notation}
  \begin{itemize}
  \item $ W_e $ is the domain of the function $ \varphi_e $;
  \item $ E_e = \{\varphi_e(x) \mid x\in W_e \}$;
  \item $ H(x,y,t) = $ "$ P_x(y)\downarrow $ in $t$ steps or les";
  \item $ s(x,y,z,t) =$ "$ P_x(y)\downarrow z$ in $t$ steps or les";
  \item $ K = \{x \mid x\in W_x \} = \{x\mid \varphi_x(x)\downarrow \} =
    \{x\mid P_x(x) \mbox{ terminates} \}$
  \end{itemize}
\end{notation}

\begin{proof}
  We'll prove the following
  \begin{enumerate}
    \item $ \exists f \in \mathcal{C} . f \not\in \mathcal{A} \land
    \exists\theta\subseteq f \mbox{ finite}, \theta\in\mathcal{A} \Rightarrow
    A$ not r.e
    \item $ \exists f \in \mathcal{C}. f\in\mathcal{A} \land
    \forall\theta\subseteq f \mbox{ finite}, \theta\not\in\mathcal{A}\Rightarrow
    A$ not r.e.
  \end{enumerate}

  so
  \begin{enumerate}
    \item
    Let $ f\not\in \mathcal{A}$ and $\theta \subseteq f$ finite with $\theta \in \mathcal{A}$.
    We show that $
    \bar{K}\leq A $.

    Define
    \begin{equation*}
      \begin{aligned}
        g(x,y) & = \begin{cases}
          \theta(y) & x \in \bar{K} \\
          f(y) & x \in K
        \end{cases} \\
               & = \begin{cases}
                 \uparrow & x \in \bar{K} \land x \not\in dom(\theta) \\
                 \theta(y) = f(y) & x \in \bar{K} \land x \in dom(\theta) \\
                 f(y) & x\in K
               \end{cases}\\
               &= \begin{cases}
                 f(y) & x \in K\lor y\in dom(\theta) \\
                 \uparrow & $otherwise $
               \end{cases}
      \end{aligned}
    \end{equation*}

    But $ x\in K\lor y\in dom(\theta) = Q(x,y)$ predicate. $ x\in K $
    semi-decidable; $ y \in dom(\theta) $ decidable; therefore $ Q(x,y) $
    semi-decidable.

    \begin{equation*}
    sc_Q(x,y) = \begin{cases}
        1 & Q(x,y) \\
        0 & $ altrimenti $
      \end{cases}
    \end{equation*}

    This is computable $ = f(y) \times sc_Q(x,y) $ computable.

    For sMN, given that \textit{g} is computable, $ \exists
    s:\nat\rightarrow\nat $ s.t.
    \begin{equation*}
      g(x,y) = \begin{cases}
        \theta(y) & x \in \bar{K} \\
        f(y) & x \in K
      \end{cases}
    \end{equation*}

    $ g(x,y) = \varphi_{s(x)}(y)$

    $s$ is the reduction function for $ \bar{K}\leq A $

    \begin{itemize}
    \item $ x\in\bar{K} \Rightarrow \forall y. \varphi_{s(x)}(y) = g(x,y) =
      \theta(y) \Rightarrow \varphi_{s(x)} = \theta \Rightarrow s(x) \in A $
    \item $ x\not\in\bar{K}\Rightarrow x\in K\Rightarrow\forall
      y\varphi_{s(x)}(y) = g(x,y)=f(y)\Rightarrow\varphi_{s(x)}=f\Rightarrow
      s(x)\in\bar{A}$
    \end{itemize}

    Hence $A$ is reduced to $ \bar{K} $ which is not r.e. therefore A is
    not r.e.

    \item
    let $ f\in\mathcal{A}\land\theta\subseteq f $ be with $ \theta $
    finished, $ \theta\not\in\mathcal{A} $

    We want it to be in quotes because it's not formal: \begin{equation*}
      g(x,y) = \begin{cases}
        f(y) & x \in\bar{K} $ cioè $ \varphi_x(x)\uparrow \\
        \theta(y) & $ for some $ \theta\subseteq f $ finite, otherwise ($ x\in K $) $
      \end{cases}
    \end{equation*}

    This is computable, meaning $f(y) \times sc_Q(x,y) $ is
    computable.

    for sMN there exists $ s:\nat\rightarrow\nat $ total computable s.t. $
    \forall x,y. \varphi_{s(x)}(y) = g(x,y) $

    We have to prove that s is a reduction function for $ \bar{K}\leq A $

    \begin{itemize}
    \item $ x\in\bar{K}\\ \Rightarrow\varphi_{s(x)}\uparrow \\
      \Rightarrow\forall y\lnot H(x,x,y)\\ \Rightarrow \forall
      y.\varphi_{s(x)}(y) = g(x,y) = f(y)\\ \Rightarrow f = \varphi_{s(x)}\\
      \Rightarrow s(x)\in A$
    \item
      $ x\not\in\bar{K}\\ \Rightarrow x\in K\\ \Rightarrow
      \varphi_x(x)\downarrow\\ \Rightarrow \exists t_0 $ s.t.
      $ \forall
      t>t_0. H(x,x,t), \forall t<t_0. \lnot H(x,x,y)\\
      \Rightarrow\varphi_{s(x)}(y) = g(x,y)\\ \Rightarrow
      \varphi_{s(x)}\subseteq f$ finite
      $\\ \Rightarrow s(x) \in \bar{A} $
    \end{itemize}
  \end{enumerate}

  Rice-shapiro's theorem proved.
\end{proof}
\end{theorem}

\begin{example}\label{exe:rice1}
  $A = \{ x \mid \varphi_x \mbox{ total}\}$ is not r.e.

  \begin{proof}
    Clearly $A$ is saturated since $A = \{x \mid \varphi_x \in
    \mathcal{A}\}$, and $\mathcal{A} = \{f \in \mathcal{C} \mid f $ total$\}$. 
    Given any function $f \in \mathcal{A}$ (total by
    definition) we know that $\forall \theta \subseteq f$
    is finite $\theta \notin \mathcal{A}$, since each and every finite
    function is partial, then by Rice-shapiro's theorem, $A$ is
    not r.e.
  \end{proof}
\end{example}

\begin{example}\label{exe:rice2}
  $\bar{A} = \{x \mid \varphi_x $ not total$\}$ is not r.e.

  \begin{proof}
    Let $\bar{\mathcal{A}} = \{f \in \mathcal{C} \mid f $ not total$\}$. We observe that each $\theta$ finite is in
    $\bar{\mathcal{A}}$, but no total extension of such $\theta$ can
    be included in $\bar{\mathcal{A}}$. Again, by Rice-shapiro
    $\bar{A}$ is not r.e.
  \end{proof}
\end{example}

The examples \ref{exe:rice1} and \ref{exe:rice2} are esential to
understand two core situations in which we can apply the theorem:
\begin{observation}
  Let $\mathcal{A} \subseteq \mathcal{C}$ be a set of computable
  functions s.t. $A = \{ x \mid \varphi_x \in \mathcal{A}\}$ is
  r.e. Then
  \begin{enumerate}[label=(i)]
  \item if \(\forall \theta \) finite
    \( \theta \notin \mathcal{A} \Rightarrow \mathcal{A} = \emptyset\)
  \item
    \(\emptyset \in \mathcal{A} \Rightarrow \mathcal{A} =
    \mathcal{C}\)
  \end{enumerate}

  \begin{proof}
    \begin{enumerate}[label=(i)]
    \item given $f \in \mathcal{C}$ we know that $f \in \mathcal{A}$
      iff $\exists \theta \subseteq f$ finite
      $\theta \in \mathcal{A} \rightarrow f \notin \mathcal{A}$
    \item given $f \in \mathcal{C}$, since $\emptyset \subseteq f$ and
      $\emptyset \in \mathcal{A} \Rightarrow f \in \mathcal{A}$
    \end{enumerate}
  \end{proof}
\end{observation}

\begin{exercise}
  study the recursivenes of $A = \{x \mid \varphi_x = \mathds{1}\}$

  \begin{enumerate}
  \item[(*)] $A$ is not r.e.

    In fact the set of functions is in this case
    $\mathcal{A} = \{\mathds{1}\}$, which
    \begin{itemize}
    \item does not contain finite functions
    \item is not empty
    \end{itemize}
    And therefore is not r.e.

  \item[(**)] $\bar{A}$ is not r.e.

    In fact $\bar{\mathcal{A}} = \mathcal{C} - \{\mathds{1}\}$, and we
    have that
    \begin{itemize}
    \item $\emptyset \in \bar{\mathcal{A}}$
    \item $\bar{\mathcal{A}} \neq \mathcal{C}$
    \end{itemize}
    And therefore $\bar{A}$ is not r.e.
  \end{enumerate}
\end{exercise}

\begin{observation}
  Rice-shapiro theorem is a necesary condition, but not sufficient to
  be r.e., that is to say that does not hold that
  \begin{equation}\label{eq:star}
    \forall f \; (\; f \in \mathcal{A} \mbox{ iff } \exists \theta \mbox{
      finite, } \theta \subseteq f, \; \theta \in \mathcal{A} \; ) \quad
    \Rightarrow \quad A \mbox{ r.e. }
  \end{equation}

  In other words, Rice-shapiro can be used to prove that a set is
  \emph{not} r.e., \underline{not} to prove that a set is r.e.
\end{observation}

\begin{counterexample}
  Let
  $\mathcal{A} = \{ f \in \mathcal{C} \mid dom(f) \cap \bar{k} \neq
  \emptyset\}$ , $A = \{x \mid \varphi_x \in \mathcal{A}\}$

  \begin{enumerate}
  \item[(*)] $\mathcal{A}$ satisfies (\ref{eq:star})
    \begin{itemize}
    \item[] \[
        \begin{aligned}
          \mbox{if } f \in \mathcal{A} & \Rightarrow dom(f) \cap \bar{k} \neq \emptyset \\
                                       & \Rightarrow \mbox{ called } x \in dom(f) \cap \bar{k} \mbox{ we have that } \theta = \{(x, f(x))\} \\
                                       & \quad \mbox{ is finite, } \theta \subseteq f \mbox{ and } dom(\theta)\cap \bar{k} = \{x\} \neq \emptyset
        \end{aligned}
      \]

    \item[] \[
        \begin{aligned}
          \mbox{if $\theta$ finite, } \theta \subseteq f, \theta \in \mathcal{A} & \Rightarrow \mbox{ since } \theta \subseteq f \; dom(\theta) \subseteq dom(f) \\
          & \Rightarrow dom(f) \cap \bar{k} \supseteq dom(\theta) \subseteq dom(f) \neq \emptyset \\
          & \Rightarrow f \in \mathcal{A}
        \end{aligned}
      \]
    \end{itemize}

  \item[(**)] $A$ is not r.e., since $\bar{k} \leq_m A$

    we can define
    \[
      g(x,y) = \begin{cases}
        0        & x=y \\
        \uparrow & \mbox{otherwise}
      \end{cases}
    \]
    which is $= \mu z . |x-y|$, and therefore computable. Again,
    for the \smn theorem $\exists s : \nattonat$ computable and total
    s.t.
    \[
      g(x,y) = \varphi_{s(x)}(y)
    \]
    and therefore $dom(\varphi_{s(x)}) = \{x\}$. so
    \begin{itemize}
    \item
      \(x \in \bar{k} \Rightarrow dom(\varphi_{s(x)}) \cap \bar{k} =
      \{x\} \neq \emptyset \Rightarrow s(x) \in A\)
    \item
      \(x \notin \bar{k} \Rightarrow dom(\varphi_{s(x)}) \cap \bar{k}
      = \{x\} = \emptyset \Rightarrow s(x) \notin A\)
    \end{itemize}
  \end{enumerate}
\end{counterexample}

\chapter{First recursion theorem}

In programming languages, we have functions that use other functions as arguments,
e.g. in ML, the function \texttt{succ} that
given a function $f$ returns $f+1$ can be defined as
\begin{equation*}
    fun\ succ\ f\ x = f\ x + 1
\end{equation*}
From the computability point of view it is still somewhat natural to ask
how effective/computable operations can be characterized on
functions. We'll later see that this idea leads to the concept of
\emph{recursive functional}.

\begin{definition}
  We'll call $\mathcal{F}(\nat^k)$ the set of all the functions
  (computable and not) of k arguments $\nat^k \to \nat^k$.

A \emph{functional} is just a function
\[
  \Phi : \mathcal{F}(\nat^k) \to \mathcal{F}(\nat^h)
\]
(only total functions will be considered).
\end{definition}

When can we say that a functional is effective (computable)?
Given $\Phi : \mathcal{F}(\nat^k) \to \mathcal{F}(\nat^h)$
\begin{itemize}
\item a function $f \in \mathcal{F}$ and its image
  $\Phi(f) \in \mathcal{F}(\nat^k)$ are both infinite objects in
  general.
\item we can not ask for $\Phi(f)$ to be computable in a finite time
  from $f$
\end{itemize}
We need a way to see functions as numbers
\subsection{Encoding of finite functions}
For each finite function its' encoding $\tilde{\theta} \in \nat$ is defined
as
\begin{itemize}
\item if $\theta = \emptyset$ then $\tilde{\theta} = 0$
\item if $\theta = \{(x_1, y_1), \dots, (x_2, y_2)\}$ then
  $\tilde{\theta} = \prod_{i=1}^n p_{x_1}^{y_i+1}$
\end{itemize}
which is both injective and effective. Given the encoding of a function
$z= \tilde{\theta}$,
\[
  x \in \dom{\theta} \quad \mbox{iff} \quad (z)_1 \neq \emptyset
\]
\[
  \begin{aligned}
    app(z,x) = \theta(x) & = \begin{cases}
      (z)_x = 1 & x\in\dom{\theta} \\
      \uparrow & \mbox{otherwise}
    \end{cases} \\
    &= ((z)_x \dotdiv 1) \mathds{1}(\mu \omega \; . \; |1 - (z)_x|)
  \end{aligned}
\]

In this way we can give the following definition of recursive functional
\begin{definition}
  A functional
  $\Phi : \mathcal{F}(\nat^k) \to \mathcal{F}(\nat^h)$ is
  \emph{recursive} if there exists $\varphi : \nat^{h+1} \to \nat$
  computable such that, for every
  $f \in \mathcal{F}(\nat^k), \vec{x} \in \nat^h,
  y \in \nat$
  \[
    \Phi(f)(y) = y \quad \mbox{iff} \quad \exists \theta \subseteq f
    \mbox{ finite s.t. } \varphi(\tilde{\theta}, \vec{x}) = y
  \]
\end{definition}

\begin{example}
  The functional
  \newcommand{\fib}[1]{\ensuremath{\mbox{fib}(#1)}}
  \[
    \mbox{fib} : \mathcal{F}(\nat) \to \mathcal{F}(\nat)
  \]
  \[
    \fib{f}(x) = \begin{cases}
      1 & x=0 \lor x=1 \\
      f(x-2) + f(x-1) & x \geq 2
    \end{cases}
  \]
  is recursive, the function $\varphi: \nat^2 \to \nat$ can be
  \[
    \begin{aligned}
      \varphi(z, x) &= \begin{cases}
        1 & x=0 \lor x=1 \\
        \theta(x-2) + \theta(x-1) & x \geq 2 \land z = \tilde{\theta}
      \end{cases} \\
      &= \begin{cases}
        1 & x = 0 \lor x = 1 \\
        app(z, x-2) + app(z, x-1) & x > 2
      \end{cases}
    \end{aligned}
  \]
  which is computable.
\end{example}

\begin{example}
The functional associated to the Ackermann's function is
\[
  \Psi_{ack} : \mathcal{F}(\nat^2) \to \mathcal{F}(\nat^2)
\]
\[
  \begin{cases}
    \Psi_{ack}(f)(0,y)   =  y+1 \\
    \Psi_{ack}(f)(x+1, 0)  =  f(x,1) \\
    \Psi_{ack}(f)(x+1, y+1)   =  f(x, f(x+1, y)) 
  \end{cases}
\]
which is computable.
\end{example}

\begin{theorem}\label{th:unknown}
  Let $\Phi : \mathcal{F}(\nat^k) \to \mathcal{F}(\nat^h)$
  be a recursive functional and let $f \in \mathcal{F}(\nat^k)$ be
  computable. Then $\Phi(f) \in \mathcal{F}(\nat^h)$ is computable
\end{theorem}

\section{Myhill-Sheperdson theorems}
Given a recursive functional $\Phi$, by (\ref{th:unknown})
\[
  f \mbox{ computable } \rightsquigarrow \Phi(f) \mbox{ computable }
\]
\[
  f = \varphi_e \rightsquigarrow \Phi(f) = \varphi_{e'}
\]

so we can see a recursive functional as a function that transforms
indices (programs) in indices (other programs), but with the property
that the transformation depends on the firstly indexed function and
\emph{not} on the index itself.

\begin{definition}[Extensional function]
  Let $h : \nattonat$ a total function. It is \emph{extensional} if
  \[
    \forall e,e' \quad \varphi_e = \varphi_{e'} \to
    \varphi_{h(e)} = \varphi_{h(e')}
  \]
\end{definition}

\begin{theorem}[Myhill-Shepherdson (I)]
  If $\Phi : \mathcal{F}(\nat^k) \to \mathcal{F}(\nat^h)$ is a
  recursive functional then there exists a total computable function
  $h_\Phi : \nattonat$ s.t.
  \[
    \forall e\in \nat \quad \Phi(\varphi_e) = \varphi_{h_\Phi(e)}^{(k)}
  \]
\end{theorem}

Intuitively, the behaviour of the recursive functional on computable
functions is captured by a total extensional function on the
indices.

\begin{theorem}[Myhill-Shepherdson (II)]\label{th:myhill-shepherdson2}
  If $h : \nattonat$ is a total computable extensional function, then
  there is a unique recursive functional $\Phi_h$ sych that
  \[
    \forall e \in \nat \quad \Phi_h(\varphi_e) = \varphi_{h(e)}
  \]
\end{theorem}

A total computable extensional function identifies
only one recursive functional. It is interesting to observe that
$\Phi_n$ exists also for non computable functions.

\begin{theorem}[First recursion theorem (Kleene)]\label{th:first-recursion}
  Let $\Phi : \mathcal{F}(\nat^k) \to \mathcal{F}(\nat^h)$ be
  a \textbf{recursive functional}. Then $\Phi$ has a \textbf{least fixed point}
  $f_\Phi$ which is \textbf{computable}, i.e.
  \begin{enumerate}
  \item $\Phi(f_\Phi) = f_\Phi$
  \item $\forall g \in \mathcal{F}(\nat^k) \quad \Phi(g) = g \Rightarrow f_\Phi \subseteq g$
  \item $f_\Phi$ is computable
  \end{enumerate}
  and we can see that $f_\Phi = \bigcup\limits_n
  \Phi^{n}(\emptyset)$, since a functional can be associated to a
  fixed point; the theorem proves the closure of the set of computable
  functions with respect to extremely general forms of recursion.
\end{theorem}

\begin{example}[Primitive recursion]
Given $f : \nat^h \to \nat$ and
$g : \nat^{h+2} \to \nat$, the function defined by primitive
recursion is the minimum fixed point of
$\Phi_r \in \mathcal{F}(\nat^{h+1})$, defined by
\begin{align*}
    \Phi_r(h)(\vec{x}, 0) & =  f(\vec{x}) \\
    \Phi_r(h)(\vec{x}, y+1) & =  g(\vec{x}, y h(\vec{x},y))
\end{align*}
and if $f,g$ are computable, then $\Phi_r$ is a recursive
functional. The theorem assures that
\begin{itemize}
\item there exists a minimal fixed point;
\item it is computable.
\end{itemize}
\end{example}

\begin{example}[Minimalisation]
Given a function $f: \nat^{k+1} \to \nat$, we can see the
minimization $\mu y \; . \; f(\vec{x}, y)$ as a fixed point. Let us
consider, for a fixed f
\[
  \Phi_\mu \in \mathcal{F}(\nat^{k+1})
\]
\[
  \Phi_\mu(h)(\vec{x}, y) = \begin{cases}
    y & f(\vec{x},y) = 0 \\
    h(\vec{x}, y+1) & f(\vec{x}, y)\downarrow \land f(\vec{x}, y) \neq 0 \\
    \uparrow & \mbox{otherwise}
  \end{cases}
\]
it is recursive and has a minimum fixed point:
\[
  f_{\Phi_\mu}(\vec{x}, y) = \mu (z \geq y) \; . \; f(\vec{x}, y)
\]
\end{example}

\begin{example}[Ackermann's function]
We saw that $\Phi_{ack}$ is recursive, therefore it has a computable
minimum fixed point (the Ackermann function itself $\Psi$). The fact
that $\Psi$ is total, implies that such fixed point is the
\emph{only} fixed point.
\end{example}

\begin{observation}
  Generally speaking, the fixed point is not unique. Counter-example:
  \[
    \Phi(f)(x) = \begin{cases}
      0 & x=0 \\
      f(x+1) & \mbox{otherwise}
    \end{cases}
  \]
  is recursive, and therefore has a minimum fixed point
  \[
    f_\Phi(x) = \begin{cases}
      0 & x=0 \\
      \uparrow & \mbox{otherwise}
    \end{cases}
  \]
  but it has other fixed points, for example:
  \[
    f(x) = \begin{cases}
      0 & x=0 \\
      k & x>0
    \end{cases}
  \]
\end{observation}

\chapter{Second recursion theorem}
Let $f:\nat \to \nat$ be total, computable and extensional i.e.
\[
\forall e, e'\ \varphi_e=\varphi_{e'}\Rightarrow\varphi_{f(e)}=\varphi_{f(e')}
\] 
Then, for the
theorem (\ref{th:myhill-shepherdson2}) there exists a unique recursive
functional $\Phi$ such that
\[ \forall e \in \nat \quad \Phi(\varphi_e) = \varphi_{f(e)} \] Since $\Phi$ is
recursive, for the first recursion theorem (\ref{th:first-recursion})
it has a least fixed point $f_\Phi :\nat \to \nat$ computable, therefore exists
$e_0\in \nat$ such that
\[
  \varphi_{e_0} = f_\Phi = \Phi(f_\Phi) = \Phi(\varphi_{e_0}) = \varphi_{f(e_0)}
\]
This means that if $f$ is total computable and extensional, then there
exists $e_0$ such that \[\varphi_{e_0} = \varphi_{f(e_0)}\]

The second recursion
theorem states that this is valid also when $f$ is not extensional. The
consequence is that a function $f : \nattonat$ cannot be thought as a
functional on computable functions.
\begin{theorem}[Second recursion theorem (Kleene)]\label{th:second-recursion}
  Let $f : \nattonat$ a total computable function. Then there exists
  $e_0 \in \nat$ such that
  \[
    \varphi_{e_0} = \varphi_{f(e_0)}
  \]
  \begin{proof}
    Let $f : \nattonat$ a total computable.
    Take
    \begin{align*}
      g(x,y) &= \varphi_{f(\varphi_x(x))}(y)\\
             &= \univ(f(\varphi_x(x)), y)     \\
             &= \univ(f(\univ(x,x)), y) 
    \end{align*}
    it is computable. This means that for the
    $smn$ theorem exists $s : \nattonat$ total computable such that
    \[
      \varphi_{s(x)}(y) = g(x,y) = \varphi_{f(\varphi_x(x))}(y) \quad \forall x,y
    \]
    Since $s$ is computable there exists $m\in\nat$ such that 
    \[
    S = \varphi_m
    \]
    hence
    \[
      \varphi_{\varphi_m(x)}(y) = \varphi_{f(\varphi_x(x))}(y) \quad \forall x,y
    \]
    For $x=m$
    \[
      \varphi_{\varphi_m(m)}(y) = \varphi_{f(\varphi_m(m))}(y) \quad \forall y
    \]
    We set $e_0 = \varphi_m(m)\downarrow$ and we replace in the previous equation
    \[
       \varphi_{e_0}(y) = \varphi_{f(e_0)}(y) \quad \forall y
    \]
    i.e.
    \[
      \varphi_{e_0} = \varphi_{h(e_0)}
    \]
  \end{proof}
\end{theorem}

This theorem can therefore be interpreted in the following manner
\emph{given any effective procedure to transform programs, there exists a
program such that when properly modified does exactly what it did
before} or \emph{it is impossible to write a program that edits the core of
all programs}.

The proof of the theorem can appear mysterious, but after a closer
inspection, it clearly appears to be a simple diagonalization.
Nevertheless, the result of this theorem is extremely
deep; in this way, many theorems we've seen up until now are just
corollaries.
\begin{corollary}[Rice's theorem]
  Let $\emptyset \neq A \subset \nat$ saturated, then
  $A$ is not recursive.
  \begin{proof}
    Let $\emptyset \neq A \subset \nat$ saturated.
    Take $e_1\in A$ and $e_0\notin A$ and
    assume by contradiction that $A$ is recursive.
    Define $f : \nattonat$
    \begin{align*}
      f(x) &= \begin{cases}
        e_0 & x \in A \\
        e_1 & x \notin A
      \end{cases} \\
            &= e_0 \cdot \chi_A(x) + e_1 \cdot \chi_{\bar{A}}(x)
    \end{align*}
    Since $A$ is recursive then also $\bar{A}$ is recursive, thus $\chi_A$ and $\chi_{\bar{A}}$ are computable.
    Thus, $f$ is computable and total, then for the second recursion
    theorem (\ref{th:second-recursion}) there exists $e\in\nat$ such that
    $\varphi_e = \varphi_{f(e)}$; there are two possibilities
    \begin{itemize}
      \item if $e\in A$, then $f(e)=e_0 \notin A$ and since $A$ saturated, $\varphi_e \neq \varphi_{e_0} = \varphi_{f(e)}$
      \item if $e\notin A$, then $f(e)=e_1 \in A$ and since $A$ saturated, $\varphi_e \neq \varphi_{e_1} = \varphi_{f(e)}$
    \end{itemize}
    that is absurd, so $A$ can't be recursive.
  \end{proof}
\end{corollary}

\begin{corollary}
  The halting set $K = \{ x \mid \varphi_x(x)\downarrow\}$ is not recursive.
  \begin{proof}
    Let $k = \{ x \mid x \in W_x\}$ recursive for the sake of the
    argument. and let $e_0, e_1$ be indexes s.t.
    $\varphi_{e_0} = \emptyset$ and
    $\varphi_{e_1} = \lambda x \; . \; x$. 
    
    Define $f: \nattonat$
    \begin{align*}
      f(x) &= \begin{cases}
        e_0 & x \in K \\
        e_0 & x \notin K
      \end{cases}\\
      &= e_0 \cdot \chi_K(x) + e_1 \cdot \chi_{\bar{K}}(x)
    \end{align*}
    If $K$ were recursive, then $\chi_K$ and $\chi_{\bar{K}}$ would be computable, thus
    $f$ would be both computable and total, then by
    (\ref{th:second-recursion}), there would be
    $e\in\nat$ such that $\varphi_e = \varphi_{f(e)}$, but
    \begin{itemize}
      \item if $e\in K$, then $f(e)=e_0$, so $\varphi_e(e)\downarrow \neq \varphi_{f(e)}(e) = \varphi_{e_0}(e)\uparrow $
      \item if $e\in \bar{K}$, then $f(e)=e_1$, so $\varphi_e(e)\uparrow \neq \varphi_{f(e)}(e) = \varphi_{e_1}(e)=e\downarrow  $
    \end{itemize}
    which is absurd, so $K$ can't be recursive.
  \end{proof}
\end{corollary}

\begin{corollary}
  $K = \{ x \mid \varphi_x(x)\downarrow\}$ is not saturated.
  \begin{proof}
    Let $n_0$ s.t. $\varphi_{n_0} = \{(n_0, n_0)\}$. We know that
    there are infinitely many indices for the same function; so let
    $n \neq n_0$ s.t. $\varphi_n = \varphi_{n_0}$.
    \[
      \varphi_n(n) \uparrow \Rightarrow n \notin K
    \]
  \end{proof}
\end{corollary}


\printbibliography

\end{document}
%%% Local Variables:
%%% mode: latex
%%% TeX-master: t
%%% End: