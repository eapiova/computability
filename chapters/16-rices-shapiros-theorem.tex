\chapter {Rice-Shapiro theorem}
Rice-Shapiro states that a property of the functions computed by
programs can be semi-decidable \textbf{only if} it depends on a finite
part of the function (behavior on finite inputs). 
\begin{theorem}[Rice-shapiro theorem]
  Let $\mathcal{A} \subseteq \mathcal{C}$ be a set of computable
  functions. If the set $A = \{x \mid \varphi_x \in \mathcal{A}\}$ is
  r.e., then
  \[
    \forall f (f \in \mathcal{A} \Leftrightarrow \exists \theta \mbox{
      finite function, } \theta \subseteq f \land \theta \in
    \mathcal{A})
  \]

\end{theorem}

In order to prove the above theorem, we will need some more tools.

\begin{definition}[Finite function]
  A finite function is a function $\theta: \nat\rightarrow\nat$
  such that $dom(\theta)$ is finite.
\end{definition}

The fact that a function is finite means that the set of
input-output pairs is finite, i.e.,

\[
  \theta(x) =
  \begin{cases}
    y_1 & \mbox{if $x=x_1$}\\
    y_2 & \mbox{if $x=x_2$}\\
    \ldots\\
    y_n & \mbox{if $x=x_m$}\\
    \uparrow & \mbox{otherwise}
  \end{cases}
\]In other words
  $\theta = \{(x_1,y_1),\dots(x_n,y_n) \}$, i.e., seen a set the function is finite.

\begin{definition}
  Given $f:\nat\rightarrow\nat$, $\theta$ is a sub-function of
$f$ if $\theta \subseteq f$
\end{definition}


\begin{notation}
  We recall some notation:
  \begin{itemize}
  \item $ W_e $ is the domain of the function $\varphi_e$;
  \item $ E_e = \{\varphi_e(x) \mid x\in W_e \}$;
  \item $ H(x,y,t) = $ "$P_x(y)\downarrow$ in $t$ steps or less";
  \item $ s(x,y,z,t) =$ "$ P_x(y)\downarrow z$ in $t$ steps or less";
  \item $ K = \{x \mid x\in W_x \} = \{x\mid \varphi_x(x)\downarrow \} =
    \{x\mid P_x(x) \mbox{ terminates} \}$
  \end{itemize}
\end{notation}

\begin{proof}
  We'll prove the following
  \begin{enumerate}
    \item $ \exists f \in \mathcal{C} . f \not\in \mathcal{A} \land
    \exists\theta\subseteq f \mbox{ finite}, \theta\in\mathcal{A} \Rightarrow
    A$ not r.e
    \item $ \exists f \in \mathcal{C}. f\in\mathcal{A} \land
    \forall\theta\subseteq f \mbox{ finite}, \theta\not\in\mathcal{A}\Rightarrow
    A$ not r.e.
  \end{enumerate}

  so
  \begin{enumerate}
    \item
    Let $ f\not\in \mathcal{A}$ and $\theta \subseteq f$ finite with $\theta \in \mathcal{A}$.
    We show that $
    \bar{K}\red A $.

    Define
    \begin{equation*}
      \begin{aligned}
        g(x,y) & = \begin{cases}
          \theta(y) & x \in \bar{K} \\
          f(y) & x \in K
        \end{cases} \\
               & = \begin{cases}
                 \uparrow & x \in \bar{K} \land x \not\in dom(\theta) \\
                 \theta(y) = f(y) & x \in \bar{K} \land x \in dom(\theta) \\
                 f(y) & x\in K
               \end{cases}\\
               &= \begin{cases}
                 f(y) & x \in K\lor y\in dom(\theta) \\
                 \uparrow & $otherwise $
               \end{cases}
      \end{aligned}
    \end{equation*}

    Since $ x\in K\lor y\in dom(\theta) = Q(x,y)$ predicate, $x\in K$
    semi-decidable and $y \in dom(\theta)$ decidable, then $ Q(x,y) $
    semi-decidable.
    Then, since
    \begin{equation*}
    sc_Q(x,y) = \begin{cases}
        1 & Q(x,y) \\
        0 & $ otherwise $
      \end{cases}
    \end{equation*}
  is computable, we have $g(x, y) = f(y) \cdot sc_Q(x,y)$ computable.

    By \emph{smn} theorem, there is a total computable function $
    s:\nat\rightarrow\nat $ such that, for every $x, y$
    \begin{equation*}
      \varphi_{s(x)}(y) = g(x,y) = \begin{cases}
        \theta(y) & x \in \bar{K} \\
        f(y) & x \in K
      \end{cases}
    \end{equation*}
    We show that $s$ is the reduction function for $\bar{K}\red A$

    \begin{itemize}
    \item $ x\in\bar{K} \Rightarrow \forall y\ \varphi_{s(x)}(y) = g(x,y) =
      \theta(y) \Rightarrow \varphi_{s(x)} = \theta \in \mathcal{A} \Rightarrow s(x) \in A $
    \item $ x\not\in\bar{K}\Rightarrow x\in K\Rightarrow\forall
      y\ \varphi_{s(x)}(y) = g(x,y)=f(y)\Rightarrow\varphi_{s(x)}=f \notin \mathcal{A}\Rightarrow
      s(x)\notin{A}$
    \end{itemize}
    Since $ \bar{K} \red A$ and $\bar{K}$ is not r.e. we conclude that $A$ is
    not r.e.

    \item
    Let $f\in\mathcal{A}\land\theta\subseteq f$ be with $\theta$
    finite, $\theta\not\in\mathcal{A}$

    Informally, we want
    \begin{equation*}
      g(x,y) = \begin{cases}
        f(y) & x \in\bar{K} $ $(\varphi_x(x)\uparrow) \\
        \theta(y) & $for some $\theta\subseteq f$ finite, otherwise ($x\in K$) $
      \end{cases}
    \end{equation*}

    More formally
    \begin{align*}
      g(x,y) &= \begin{cases}
        f(y) & \mbox{if } \neg H(x,x,y) \\
        \uparrow & \mbox{if } H(x,x,y)
      \end{cases} \\
      &= f(y) + \mu z . \chi_H(x,x,y)
    \end{align*}
    is computable.

    By \emph{smn} there exists $s:\nat\rightarrow\nat$ total computable such that
    \[\varphi_{s(x)}(y) = g(x,y) \]
    We show that $s$ is a reduction function for $\bar{K}\red A$
    \begin{itemize}
    \item $ x\in\bar{K}\\ 
      \Rightarrow\varphi_{x}(x)\uparrow \\
      \Rightarrow\forall y\ \lnot H(x,x,y)\\ 
      \Rightarrow \forall y\ \varphi_{s(x)}(y) = g(x,y) = f(y)\\ 
      \Rightarrow f = \varphi_{s(x)} \in \mathcal{A} \\
      \Rightarrow s(x)\in A$
    \item
      $ x\not\in\bar{K}\\ 
      \Rightarrow x\in K\\ 
      \Rightarrow \varphi_x(x)\downarrow\\ 
      \Rightarrow \exists t_0 \ (\forall
      t>t_0\ H(x,x,t) \wedge \forall t<t_0 \ \lnot H(x,x,t))\\
      \Rightarrow\varphi_{s(x)}(y) = g(x,y)\\ 
      \Rightarrow \varphi_{s(x)}\subseteq f$ finite
      $\\ \Rightarrow s(x) \in \bar{A} $
    \end{itemize}
  \end{enumerate}
\end{proof}


\begin{example}\label{exe:rice1}
  $A = \{ x \mid \varphi_x \mbox{ total}\}$ is not r.e.

  \begin{proof}
    Clearly $A$ is saturated since $A = \{x \mid \varphi_x \in
    \mathcal{A}\}$, and $\mathcal{A} = \{f \in \mathcal{C} \mid f $ total$\}$. 
    Given any function $f \in \mathcal{A}$ (total by
    definition) we know that $\forall \theta \subseteq f$
    is finite $\theta \notin \mathcal{A}$, since each and every finite
    function is partial, then by Rice-Shapiro's theorem, $A$ is
    not r.e.
  \end{proof}
\end{example}

\begin{example}\label{exe:rice2}
  $\bar{A} = \{x \mid \varphi_x $ not total$\}$ is not r.e.

  \begin{proof}
    Let $\bar{\mathcal{A}} = \{f \in \mathcal{C} \mid f $ not total$\}$. We observe that each $\theta$ finite is in
    $\bar{\mathcal{A}}$, but no total extension of such $\theta$ can
    be included in $\bar{\mathcal{A}}$. Again, by Rice-Shapiro
    $\bar{A}$ is not r.e.
  \end{proof}
\end{example}

The examples \ref{exe:rice1} and \ref{exe:rice2} are essential to
understand two core situations in which we can apply the theorem:
\begin{observation}
  Let $\mathcal{A} \subseteq \mathcal{C}$ be a set of computable
  functions s.t. $A = \{ x \mid \varphi_x \in \mathcal{A}\}$ is
  r.e. Then
  \begin{enumerate}
  \item if, for every \(\theta \) finite,
    \( \theta \notin \mathcal{A} \Rightarrow \mathcal{A} = \emptyset\)
  \item
    \(\emptyset \in \mathcal{A} \Rightarrow \mathcal{A} =
    \mathcal{C}\)
  \end{enumerate}
  \begin{proof}
    \begin{enumerate}
    \item given $f \in \mathcal{C}$ we know that $f \in \mathcal{A}$
      if and only if there exists $\theta \subseteq f$ finite
      $\theta \in \mathcal{A}$ then $f \notin \mathcal{A}$
    \item given $f \in \mathcal{C}$, since $\emptyset \subseteq f$ and
      $\emptyset \in \mathcal{A} \Rightarrow f \in \mathcal{A}$
    \end{enumerate}
  \end{proof}
\end{observation}

\begin{example}
  Consider $A = \{x \mid \varphi_x = \mathbf{1}\}$
  \begin{enumerate}
  \item $A$ is not r.e.

    The set of functions is
    $\mathcal{A} = \{\mathbf{1}\}$, which
    \begin{itemize}
    \item does not contain finite functions
    \item is not empty
    \end{itemize}
    therefore $A$ is not r.e.

  \item $\bar{A}$ is not r.e.

    $\bar{\mathcal{A}} = \mathcal{C} - \{\mathbf{1}\}$, and we
    have that
    \begin{itemize}
    \item $\emptyset \in \bar{\mathcal{A}}$
    \item $\bar{\mathcal{A}} \neq \mathcal{C}$
    \end{itemize}
    therefore $\bar{A}$ is not r.e.
  \end{enumerate}
\end{example}

\begin{observation}
  The converse implication of the Rice-Shapiro theorem does not hold, i.e. the following does not hold
  \begin{equation}\label{eq:star}
    \forall f \; (\; f \in \mathcal{A} \mbox{ iff } \exists \theta \mbox{
      finite, } \theta \subseteq f, \; \theta \in \mathcal{A} \; ) \quad
    \Rightarrow \quad A \mbox{ r.e. }
  \end{equation}

  In other words, Rice-Shapiro can be used to prove that a set is
  \emph{not} r.e., but \underline{not} to prove that a set is r.e.
  \begin{proof}
    Let
    $\mathcal{A} = \{ f \in \mathcal{C} \mid dom(f) \cap \bar{K} \neq
    \emptyset\}$ , $A = \{x \mid \varphi_x \in \mathcal{A}\}$
  
    \begin{enumerate}
    \item $\mathcal{A}$ satisfies (\ref{eq:star})
      \begin{itemize}
      \item[] 
          \begin{align*}
            f \in \mathcal{A} & \Rightarrow dom(f) \cap \bar{K} \neq \emptyset \\
                                         & \Rightarrow \mbox{let } x \in dom(f) \cap \bar{k} \mbox{ we have that } \theta = \{(x, f(x))\} \\
                                         & \quad \mbox{ is finite, } \theta \subseteq f \mbox{ and } dom(\theta)\cap \bar{k} = \{x\} \neq \emptyset \\
                                         & \Rightarrow \theta \in \mathcal{A}
          \end{align*}
        
  
      \item[]
      \[
          \begin{aligned}
            \mbox{if $\theta$ finite, } \theta \subseteq f, \theta \in \mathcal{A} & \Rightarrow dom(\theta) \subseteq dom(f) \\
            & \Rightarrow dom(f) \cap \bar{K} \supseteq dom(\theta) \cap \bar{K} \neq \emptyset \\
            & \Rightarrow f \in \mathcal{A}
          \end{aligned}
        \]
      \end{itemize}
  
    \item $A$ is not r.e., since $\bar{K} \leq_m A$
  
      Define
      \begin{align*}
        g(x,y) &= \begin{cases}
          0        & x=y \\
          \uparrow & \mbox{otherwise}
        \end{cases} \\
        &= \mu z . |x-y|
      \end{align*}
      is computable. Again,
      by \emph{smn} theorem $\exists s : \nattonat$ computable and total
      such that
      \[
        g(x,y) = \varphi_{s(x)}(y)
      \]
      and therefore $dom(\varphi_{s(x)}) = \{x\}$. so
      \begin{itemize}
      \item
        \(x \in \bar{K} \Rightarrow dom(\varphi_{s(x)}) \cap \bar{K} =
        \{x\} \neq \emptyset \Rightarrow s(x) \in A\)
      \item
        \(x \notin \bar{K} \Rightarrow dom(\varphi_{s(x)}) \cap \bar{K}
        = \{x\} = \emptyset \Rightarrow s(x) \notin A\)
      \end{itemize}
    \end{enumerate}
  \end{proof}
\end{observation}
