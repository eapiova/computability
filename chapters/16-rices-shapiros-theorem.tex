\chapter {Rice-Shapiro theorem}
Rice-Shapiro states that a property of the functions computed by
programs can be semi-decidable \textbf{only if} it depends on a finite
part of the function (behavior on finite inputs). 
\begin{theorem}[Rice-Shapiro theorem]
  Let $\mathcal{A} \subseteq \mathcal{C}$ be a set of computable
  functions. If the set $\{x \mid \varphi_x \in \mathcal{A}\}$ is
  r.e., then
  \[
    \forall f (f \in \mathcal{A} \Leftrightarrow \exists \theta \mbox{
      finite function, } \theta \subseteq f \land \theta \in
    \mathcal{A})
  \]

In order for us to prove this theorem, we'll need some more tools:
\begin{definition}[Finite function]
  A finite function is a function $ \theta: \nat\rightarrow\nat $
  such that $ dom(\theta) $ is finite.  This means that the set of
  input-output pairs is finite; in other words
  $ \theta = \{(x_1,y_1),\dots(x_n,y_n) \} $ and it is undefined on other inputs.
\end{definition}

\begin{definition}
  Given $ f:\nat\rightarrow\nat $, $ \theta $ is a sub-function of
$f$ if $ \theta \subseteq f $
\end{definition}


\begin{notation}
  \begin{itemize}
  \item $ W_e $ is the domain of the function $ \varphi_e $;
  \item $ E_e = \{\varphi_e(x) \mid x\in W_e \}$;
  \item $ H(x,y,t) = $ "$ P_x(y)\downarrow $ in $t$ steps or less";
  \item $ S(x,y,z,t) =$ "$ P_x(y)\downarrow z$ in $t$ steps or less";
  \item $ K = \{x \mid x\in W_x \} = \{x\mid \varphi_x(x)\downarrow \} =
    \{x\mid P_x(x) \mbox{ terminates} \}$
  \end{itemize}
\end{notation}

\begin{proof}

  $ \forall f \in \mathcal{C} (f \in \mathcal{A} \Leftrightarrow
  \exists\theta\subseteq f \mid \theta $ finite and $ \theta\in A)$

  1)
  $ \exists f \in \mathcal{C} \; . \; f \not\in \mathcal{A} \land
  \exists\theta\subseteq f, \theta\in\mathcal{A} \Rightarrow
  \mathcal{A}$ not r.e

  2)
  $ \exists f \in \mathcal{C}. f\in\mathcal{A} \land
  \forall\theta\subseteq f, \theta\not\in\mathcal{A}\Rightarrow
  \mathcal{A} $ not r.e.

  \begin{proof}[proof of 1)]

    $ f\not\in \mathcal{A} \land \theta\subseteq \mathcal{A}$

    that $
    \bar{K}\leq A $ ($ \mathcal{A} $ set of functions, $A$ set of
    programs)

    We define the function:
    \begin{equation*}
      \begin{aligned}
        g(x,y) & = \begin{cases}
          \theta(y) & x \in \bar{K} \\
          f(y) & x \in K
        \end{cases} \\
               & = \begin{cases}
                 \uparrow & x \in \bar{K} \land x \not\in dom(\theta) \\
                 \theta(y) = f(y) & x \not\in \bar{K} \land x \in dom(\theta) \\
                 f(y) & x\in K
               \end{cases}\\
               &= \begin{cases}
                 f(y) & x \in K\lor(x\not\in\bar{K}\land y\in dom(\theta)) \\
                 \uparrow & $ altrimenti $
               \end{cases}
      \end{aligned}
    \end{equation*}

    But $ x\in K\lor y\in dom(\theta) = Q(x,y)$ predicate. $ x\in K $
    semi-decidable; $ y \in dom(\theta) $ decidable; therefore $ Q(x,y) $
    semi-decidable.

    \begin{equation*}
      SC_Q(x,y) = \begin{cases}
        1 & Q(x,y) \\
        0 & $ altrimenti $
      \end{cases}
    \end{equation*}

    This is computable $ = f(y) \times SC_Q(x,y) $ computable.

    For SMN, given that \textit{g} is computable, $ \exists
    S:\nat\rightarrow\nat $ s.t.
    \begin{equation*}
      g(x,y) = \begin{cases}
        \theta(y) & x \in \bar{K} \\
        f(y) & x \in K
      \end{cases}
    \end{equation*}

    $ g(x,y) = \varphi_{S(x)}(y)$

    $S$ is the reduction function for $ \bar{K}\leq A $

    \begin{itemize}
    \item $ x\in\bar{K} \Rightarrow \forall y. \varphi_{S(x)}(y) = g(x,y) =
      \theta(y) \Rightarrow \varphi_{S(x)} = \theta \Rightarrow S(x) \in A $
    \item $ x\not\in\bar{K}\Rightarrow x\in K\Rightarrow\forall
      y\varphi_{S(x)}(y) = g(x,y)=f(y)\Rightarrow\varphi_{S(x)}=f\Rightarrow
      S(x)\in\bar{A}$
    \end{itemize}

    Hence $A$ is reduced to $ \bar{K} $ which is not r.e. therefore A is
    not r.e.

  \end{proof}

  \begin{proof}[proof of 2)]
    let $ f\in\mathcal{A}\land\theta\subseteq f $ be with $ \theta $
    finished, $ \theta\not\in\mathcal{A} $

    We want it to be in quotes because it's not formal: \begin{equation*}
      g(x,y) = \begin{cases}
        f(y) & x \in\bar{K} $ cioè $ \varphi_x(x)\uparrow \\
        \theta(y) & $ for some $ \theta\subseteq f $ finite, otherwise ($ x\in K $) $
      \end{cases}
    \end{equation*}

    This is computable, meaning $f(y) \times SC_Q(x,y) $ is
    computable.

    for SMN there exists $ S:\nat\rightarrow\nat $ total computable s.t. $
    \forall x,y. \varphi_{S(x)}(y) = g(x,y) $

    We have to prove that S is a reduction function for $ \bar{K}\leq A $

    \begin{itemize}
    \item $ x\in\bar{K}\\ \Rightarrow\varphi_{S(x)}\uparrow \\
      \Rightarrow\forall y\lnot H(x,x,y)\\ \Rightarrow \forall
      y.\varphi_{S(x)}(y) = g(x,y) = f(y)\\ \Rightarrow f = \varphi_{S(x)}\\
      \Rightarrow S(x)\in A$
    \item
      $ x\not\in\bar{K}\\ \Rightarrow x\in K\\ \Rightarrow
      \varphi_x(x)\downarrow\\ \Rightarrow \exists t_0 $ s.t.
      $ \forall
      t>t_0. H(x,x,t), \forall t<t_0. \lnot H(x,x,y)\\
      \Rightarrow\varphi_{S(x)}(y) = g(x,y)\\ \Rightarrow
      \varphi_{S(x)}\subseteq f$ finite
      $\\ \Rightarrow S(x) \in \bar{A} $
    \end{itemize}
  \end{proof}

  Rice-Shapiro's theorem proved.
\end{proof}
\end{theorem}

\begin{example}\label{exe:rice1}
  $A = \{ x \mid \varphi_x \mbox{ total}\}$ is not r.e.

  \begin{proof}
    Clearly $A$ is saturated since $A = \{x \mid \varphi_x \in
    \mathcal{A}\}$, and $\mathcal{A} = \{f \in \mathcal{C} \mid f $ total$\}$. 
    Given any function $f \in \mathcal{A}$ (total by
    definition) we know that $\forall \theta \subseteq f$
    is finite $\theta \notin \mathcal{A}$, since each and every finite
    function is partial, then by Rice-Shapiro's theorem, $A$ is
    not r.e.
  \end{proof}
\end{example}

\begin{example}\label{exe:rice2}
  $\bar{A} = \{x \mid \varphi_x $ not total$\}$ is not r.e.

  \begin{proof}
    Let $\bar{\mathcal{A}} = \{f \in \mathcal{C} \mid f $ not total$\}$. We observe that each $\theta$ finite is in
    $\bar{\mathcal{A}}$, but no total extension of such $\theta$ can
    be included in $\bar{\mathcal{A}}$. Again, by Rice-Shapiro
    $\bar{A}$ is not r.e.
  \end{proof}
\end{example}

The examples \ref{exe:rice1} and \ref{exe:rice2} are essential to
understand two core situations in which we can apply the theorem:
\begin{observation}
  Let $\mathcal{A} \subseteq \mathcal{C}$ be a set of computable
  functions s.t. $A = \{ x \mid \varphi_x \in \mathcal{A}\}$ is
  r.e. Then
  \begin{enumerate}[label=(i)]
  \item if \(\forall \theta \) finite
    \( \theta \notin \mathcal{A} \Rightarrow \mathcal{A} = \emptyset\)
  \item
    \(\emptyset \in \mathcal{A} \Rightarrow \mathcal{A} =
    \mathcal{C}\)
  \end{enumerate}

  \begin{proof}
    \begin{enumerate}[label=(i)]
    \item given $f \in \mathcal{C}$ we know that $f \in \mathcal{A}$
      iff $\exists \theta \subseteq f$ finite
      $\theta \in \mathcal{A} \rightarrow f \notin \mathcal{A}$
    \item given $f \in \mathcal{C}$, since $\emptyset \subseteq f$ and
      $\emptyset \in \mathcal{A} \Rightarrow f \in \mathcal{A}$
    \end{enumerate}
  \end{proof}
\end{observation}

\begin{exercise}
  Study the recursiveness of $A = \{x \mid \varphi_x = \mathds{1}\}$

  \begin{enumerate}
  \item[(*)] $A$ is not r.e.

    In fact the set of functions is in this case
    $\mathcal{A} = \{\mathds{1}\}$, which
    \begin{itemize}
    \item does not contain finite functions
    \item is not empty
    \end{itemize}
    And therefore is not r.e.

  \item[(**)] $\bar{A}$ is not r.e.

    In fact $\bar{\mathcal{A}} = \mathcal{C} - \{\mathds{1}\}$, and we
    have that
    \begin{itemize}
    \item $\emptyset \in \bar{\mathcal{A}}$
    \item $\bar{\mathcal{A}} \neq \mathcal{C}$
    \end{itemize}
    And therefore $\bar{A}$ is not r.e.
  \end{enumerate}
\end{exercise}

\begin{observation}
  Rice-Shapiro theorem is a necessary condition, but not sufficient to
  be r.e., that is to say that does not hold that
  \begin{equation}\label{eq:star}
    \forall f \; (\; f \in \mathcal{A} \mbox{ iff } \exists \theta \mbox{
      finite, } \theta \subseteq f, \; \theta \in \mathcal{A} \; ) \quad
    \Rightarrow \quad A \mbox{ r.e. }
  \end{equation}

  In other words, Rice-Shapiro can be used to prove that a set is
  \emph{not} r.e., \underline{not} to prove that a set is r.e.
\end{observation}

\begin{counterexample}
  Let
  $\mathcal{A} = \{ f \in \mathcal{C} \mid dom(f) \cap \bar{k} \neq
  \emptyset\}$ , $A = \{x \mid \varphi_x \in \mathcal{A}\}$

  \begin{enumerate}
  \item[(*)] $\mathcal{A}$ satisfies (\ref{eq:star})
    \begin{itemize}
    \item[] \[
        \begin{aligned}
          \mbox{if } f \in \mathcal{A} & \Rightarrow dom(f) \cap \bar{k} \neq \emptyset \\
                                       & \Rightarrow \mbox{ called } x \in dom(f) \cap \bar{k} \mbox{ we have that } \theta = \{(x, f(x))\} \\
                                       & \quad \mbox{ is finite, } \theta \subseteq f \mbox{ and } dom(\theta)\cap \bar{k} = \{x\} \neq \emptyset
        \end{aligned}
      \]

    \item[] \[
        \begin{aligned}
          \mbox{if $\theta$ finite, } \theta \subseteq f, \theta \in \mathcal{A} & \Rightarrow \mbox{ since } \theta \subseteq f \; dom(\theta) \subseteq dom(f) \\
          & \Rightarrow dom(f) \cap \bar{k} \supseteq dom(\theta) \subseteq dom(f) \neq \emptyset \\
          & \Rightarrow f \in \mathcal{A}
        \end{aligned}
      \]
    \end{itemize}

  \item[(**)] $A$ is not r.e., since $\bar{k} \leq_m A$

    we can define
    \[
      g(x,y) = \begin{cases}
        0        & x=y \\
        \uparrow & \mbox{otherwise}
      \end{cases}
    \]
    which is $= \mu z \; . \; |x-y|$, and therefore computable. Again,
    for the \smn theorem $\exists s : \nattonat$ computable and total
    s.t.
    \[
      g(x,y) = \varphi_{s(x)}(y)
    \]
    and therefore $dom(\varphi_{s(x)}) = \{x\}$. So
    \begin{itemize}
    \item
      \(x \in \bar{k} \Rightarrow dom(\varphi_{s(x)}) \cap \bar{k} =
      \{x\} \neq \emptyset \Rightarrow s(x) \in A\)
    \item
      \(x \notin \bar{k} \Rightarrow dom(\varphi_{s(x)}) \cap \bar{k}
      = \{x\} = \emptyset \Rightarrow s(x) \notin A\)
    \end{itemize}
  \end{enumerate}
\end{counterexample}
