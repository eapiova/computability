\chapter{Recursively enumerable sets}
This is a wider set that contains recursive sets. To see this we can
already see the definition:

\begin{definition}[recursively enumerable set]
  We say that $ A \subseteq \nat $ is \emph{recursive} if
  the semi-characteristic function is computable.
  \begin{equation*}
    sc_A(x) = \begin{cases} 1 & x \in A \\ \uparrow & $
      otherwise $
    \end{cases}
  \end{equation*}

  Predicate $ Q(\vec{x}) \subseteq \nat^k $ semi-decidable

  \begin{equation*} sc_Q(\vec{x}) = \begin{cases} 1 & Q(\vec{x}) \\
      \uparrow & $ altrimenti $
    \end{cases}
  \end{equation*}
\end{definition}

So say $A$ is r.e. is like saying that the predicate $ Q(x)=``x \in A"
$ is semi-decidable. This notion is also easily extensible to
\begin{itemize}
\item subsets of $\nat^k$
\item $k$-ary predicates
\end{itemize}
trough encoding or direct deifinition.

\begin{theorem}\label{th:aiffanota}
  \[A \subseteq \nat \mbox{ r.e. } \Leftrightarrow A, \bar{A} \mbox{
      are r.e.} \]
  \begin{proof}\mbox{}\\*
    \begin{itemize}
    \item[($\Rightarrow$)]
      If $A$ recursive,
      \begin{equation*}
        \mathcal{X}_A(x)= \begin{cases}
          1 & x\in A \\
          0 & $ otherwise $
        \end{cases}
      \end{equation*}
      
      Then $ sc_A(x) = \mathds{1}(\mu z. | \chi_A(x)- z | )$
      computable. Computable, therefore \textit{A} is r.e. Also
      $\bar{A}$ is r.e. In fact, since $A$ is r.e., by closure on
      negation, also $\bar{A}$ is r.e.

    \item[($\Leftarrow$)] Let $A, \bar{A}$ be r.e., then by definition
      $sc_a$ and $sc_B$ are computable, and we can define
      \[
        \mathds{1} - sc_{\bar{A}}(x) = 1 - sc_{\bar{A}}(x) = \begin{cases}
          0 & x \in \bar{A} \\
          \uparrow & \mbox{otherwise}
        \end{cases}
      \]
      is computable. This means that $\exists e_0, e_1 \in \nat$ s.t.
      \[
        \varphi_{e_0} = sc_A \quad \varphi_{e_1} = \mathds{1} -
        sc_{\bar{A}}
      \]
      therefore intuitively we can ``combine two machines'' until one
      of the two terminates. Note that it will terminate for sure,
      since either $x \in A$ or $x \in \bar{A}$. We can build the
      charateristic function of $A$:
      \[
          \chi_A(x) = (\mu \omega \; . \; |\chi_{s(e_0, x,
            (\omega)_1, (\omega)_2 \wedge s(e_1, x, (\omega)_1,
            (\omega)_2))}-1|)_1
        \]
        wich is computable, therefore $A$ is recursive.
    \end{itemize}
  \end{proof}
\end{theorem}

\begin{observation}
  the set $k = \{x \mu x \in W_x\}$ is r.e. In fact
  \[
    sc_k(x) = \begin{cases}
      1 & x \in k \\
      \uparrow & \mbox{otherwise}
    \end{cases}
    = \mathds{1}(\varphi_x(x)) = \mathds{1}(\univ(x,x))
  \]
  is computable by deifinition and for the theorem \ref{th:aiffanota}
  \[
    \bar{K} = \{x \mid x \notin W_x\}
  \]
  is \emph{not} r.e. Otherwise $k,\bar{k}$ would have been both r.e.,
  and therefore $k$ would have been recursive, wich is a
  contraddiction.
\end{observation}

\section {Theorem of structure of semidecidable predicates}

\begin{theorem}\label{th:structure}
  Let $ Q(\vec{x}) \subseteq \nat^k $ be a predicate.

  This is decidable $ \Leftrightarrow $ there is a predicate $
  Q'(t,\vec{x}) \subseteq \nat^{k+1} $ s.t. $ Q(\vec{x}) = \exists
  t. Q'(t,\vec{x}) $
  \begin{proof}
    \begin{itemize}
    \item[($\Rightarrow$)] Let $P(\vec{x})$ be
      semi-decidable. Therefore it has a semi characteristic function
      $sc_P$ and by definition
      \[
        P(\vec{x}) \equiv \exists t \; . \; H(e,\vec{x}, t)
      \]
      therefore if we can rewrite $H$ as
      $Q(t, \vec{x}) = H(e,\vec{x}, t)$, this way $Q$ is decidable as
      we wanted. \[P(\vec{x}) \equiv \exists t \; . \; Q(t, \vec{x})\]

    \item[($\Leftarrow$)] Let
      \(P(\vec{x}) \equiv \exists t \; . \; Q(t, \vec{x})\) be
      decidable. We can build
      \[
        sc_P(\vec{x}) = \mathds{1}(\mu t \; . \; |\chi_Q(t,\vec{x}) - 1|)
      \]
      wch is computable by definition, and therefore $P(\vec{x})$ is
      semidecidable
    \end{itemize} 
  \end{proof}
\end{theorem}

\section {Projection theorem}

From the last theorem we had a hint about the fact that the class of
semi-decidable predicates is closed unde \emph{existential
  quantification}. The projection theorem states this:

\begin{theorem}
  Let $ P(x,\vec{y}) $ be semi-decidable; then
  \[
    \exists x \mbox{ s.t. } P(x,\vec{y}) = P'(\vec{y})
  \]
  is semi-decidable.

  \begin{proof}
    Let $P(x,\vec{y})$ be semi-decidable. The by theorem
    (\ref{th:structure}) exists $Q(t,x,\vec{y})$ decidable s.t.
    \[
      P(\vec{x}, vec{y}) \equiv \exists t \; . \; Q(t,x,\vec{y})
    \]
    then if $P^\prime(\vec{y}) = \exists x \; . \; P(x, \vec{y})$ it
    holds that
    \[
      \begin{split}
        P^\prime(\vec{y}) &\equiv \exists x \exists t \; . \;
        Q(t,x,\vec{y}) \\
        &\equiv \exists \omega \; . \; Q((\omega)_1, (\omega)_2, \vec{y})
      \end{split}
    \]
    therefore since the last one is a decidable predicate, again for
    the theorem (\ref{th:structure}) $P^\prime(\vec{y})$ is
    semi-decidable
  \end{proof}
\end{theorem}

So if you use existential identifier you go out of the set of the
decidable predicates and enter the semi-decidable set, but if you use
it twice you don't go outside the semi-decidable set.

\begin{theorem}
  Let $ P_1(\vec{x}), P_2(\vec{x}) $ be semi-decidable predicates. Then
  \begin{itemize}
    \item $  P_1(\vec{x}) \lor P_2(\vec{x}) $;
    \item $ P_1(\vec{x}) \land P_2(\vec{x}) $
    \end{itemize}
    are semi-decidable.
\end{theorem}

Because you quantify existentially a single number whose components
are equivalent to quantifying the two numbers of the individual
predicates. In the \textit{OR} case and in the \textit{AND} case you
look for the same number directly.

\textbf{Exercize:} If $ P(\vec{x}) $ is semi-decidable and is not
decidable then $ \lnot P(\vec{x}) $ is not semi-decidable.

\textbf{Observation:} $ A,B \subseteq \nat, A\leq_m B $ then:
\begin{itemize}
\item B is r.e. $ \Rightarrow $ A is r.e .;
\item A is not r.e. $ \Rightarrow $ B not r.e.
\end{itemize} Demonstration: If B r.e. then
\begin{equation*} SC_B(x) = \begin{cases} 1 & x \in B \\ \uparrow & $
    otherwise $
  \end{cases}
\end{equation*} This is computable. Let $ f:\nat\rightarrow\nat $ be a
total computable reduction function\\ $ A\leq B $ Then $ SC_A(x) =
SC_B(f(x)) $, therefore $ SC_A $ is computable by composition.
