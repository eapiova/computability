\chapter{Recursively enumerable sets}

\begin{definition}[Recursively enumerable set]
  We say that $ A \subseteq \nat $ is \emph{recursively enumerable} if
  the semi-characteristic function 
  \begin{equation*}
    sc_A(x) = \begin{cases} 1 & x \in A \\ \uparrow & $
      otherwise $
    \end{cases}
  \end{equation*}
  is computable.
\end{definition}

\begin{definition}[Semi-decidable predicate]
  A predicate $ Q(x) \subseteq \nat $ is\\
   semi-decidable if 
  \( \{ x \in \nat \mid Q(x) \} \)
  is r.e.
\end{definition}

Thus, saying that $A$ is r.e. is like saying that the predicate $ Q(x)=``x \in A"
$ is semi-decidable. This notion is also easily generalisable to
\begin{itemize}
\item subsets of $\nat^k$
\item $k$-ary predicates
\end{itemize}

\begin{observation}\label{th:aiffanota}
  \[A \subseteq \nat \mbox{ recursive } \Leftrightarrow A, \bar{A} \mbox{
      are r.e.} \]
  \begin{proof}
    \begin{itemize}
    \item[($\Rightarrow$)]
      If $A$ recursive,
      \begin{equation*}
        \mathcal{X}_A(x)= \begin{cases}
          1 & x\in A \\
          0 & $otherwise $
        \end{cases}
      \end{equation*}
      is computable.
      Then $ sc_A(x) = \mathbf{1}(\mu z. | \chi_A(x)- 1 | )$
      is computable, therefore ${A}$ is r.e. Since $A$ is recursive, then
      $\bar{A}$ is recursive, thus, r.e. 

    \item[($\Leftarrow$)] Let $A, \bar{A}$ be r.e., then by definition
      $sc_A$ and $sc_{\bar{A}}$ are computable, and we can define
      \[
        \mathbf{1} - sc_{\bar{A}}(x) = \begin{cases}
          0 & x \in \bar{A} \\
          \uparrow & \mbox{otherwise}
        \end{cases}
      \]
      that is computable. This means that $\exists e_0, e_1 \in \nat$ such that
      \[
        \varphi_{e_0} = sc_A \quad \varphi_{e_1} = \mathbf{1} -
        sc_{\bar{A}}
      \]
      therefore we can ``combine two machines'' and wait until one
      of the two terminates. Since either $x \in A$ or $x \in \bar{A}$,
      then the process will terminate for sure. We can build the characteristic function of $A$ as
      \[
          \chi_A(x) = (\mu \omega . |\chi_{S(e_0, x,
            (\omega)_1, (\omega)_2 \wedge S(e_1, x, (\omega)_1,
            (\omega)_2))}-1|)_1
        \]
        which is computable, therefore $A$ is recursive.
    \end{itemize}
  \end{proof}
\end{observation}

\begin{observation}
  the set $K = \{x \mid x \in W_x\}$ is r.e. In fact
  \[
    sc_K(x) = \begin{cases}
      1 & x \in K \\
      \uparrow & \mbox{otherwise}
    \end{cases}
    = \mathbf{1}(\varphi_x(x)) = \mathbf{1}(\univ(x,x))
  \]
  is computable by definition and by \ref{th:aiffanota}
  \[
    \bar{K} = \{x \mid x \notin W_x\}
  \]
  is \emph{not} r.e, otherwise $K,\bar{K}$ would have been both r.e.,
  and therefore $K$ would have been recursive, which is a
  contradiction.
\end{observation}

\begin{theorem}[Structure of semi-decidable predicates]\label{th:structure}
  Let $ P(\vec{x}) \subseteq \nat^k $ be a predicate.
  $ P(\vec{x})$ is decidable if and only if there is a decidable predicate $
  Q(t,\vec{x}) \subseteq \nat^{k+1} $ such that $ P(\vec{x}) = \exists
  t. Q(t,\vec{x}) $
  \begin{proof}
    \begin{itemize}
    \item[($\Rightarrow$)] Let $P(\vec{x})$ be
      semi-decidable. It has a computable semi characteristic function
      $sc_P$ so
      \[
        P(\vec{x}) \equiv \exists t . H(e,\vec{x}, t)
      \]
      therefore if we can rewrite $H$ as
      $Q(t, \vec{x}) = H(e,\vec{x}, t)$, in this way $Q$ is decidable as
      we wanted and \[P(\vec{x}) \equiv \exists t . Q(t, \vec{x})\]

    \item[($\Leftarrow$)] Let
      \(P(\vec{x}) \equiv \exists t . Q(t, \vec{x})\) with $Q(t, \vec{x})$
      decidable. Observe that
      \[
        sc_P(\vec{x}) = \mathbf{1}(\mu t . |\chi_Q(t,\vec{x}) - 1|)
      \]
      which is computable by definition, and therefore $P(\vec{x})$ is
      semi-decidable.
    \end{itemize}
  \end{proof}
\end{theorem}

\section {Projection theorem}

From the last theorem we had a hint about the fact that the class of
semi-decidable predicates is closed under \emph{existential
  quantification}. The projection theorem states this:
\begin{theorem}[Projection theorem]
  Let $ P(x,\vec{y}) $ be semi-decidable; then
  \[
    \exists x.P(x,\vec{y}) = P'(\vec{y})
  \]
  is semi-decidable.

  \begin{proof}
    Let $P(x,\vec{y})$ be semi-decidable. The by
    (\ref{th:structure}), there exists $Q(t,x,\vec{y})$ decidable such that
    \[
      P(x, \vec{y}) \equiv \exists t . Q(t,x,\vec{y})
    \]
    Thus 
    \begin{align*}
      P^\prime(\vec{y}) &\equiv \exists x . P(x, \vec{y}) \\
        &\equiv \exists x .\exists t .
        Q(t,x,\vec{y}) \\
        &\equiv \exists \omega . Q((\omega)_1, (\omega)_2, \vec{y})
    \end{align*}    
    since $Q((\omega)_1, (\omega)_2, \vec{y})$ is decidable,
    by (\ref{th:structure}) $P^\prime(\vec{y})$ is
    semi-decidable.
  \end{proof}
\end{theorem}

\begin{theorem}[Closure property]
  Let $ P_1(\vec{x}), P_2(\vec{x}) $ be semi-decidable predicates. Then
  \begin{itemize}
  \item $  P_1(\vec{x}) \lor P_2(\vec{x}) $;
  \item $ P_1(\vec{x}) \land P_2(\vec{x}) $
  \end{itemize}
  are semi-decidable.
  \begin{proof}
    Let $ P_1(\vec{x}), P_2(\vec{x}) $ be semi-decidable
    predicates. Then by (\ref{th:structure}) there are two
    decidable predicates \( Q_1(t, \vec{x}), Q_2(t, \vec{x})\) such that
    \begin{gather*}
      P_1(\vec{x}) \equiv \exists t . Q_1(t, \vec{x}) \\
      P_2(\vec{x}) \equiv \exists t . Q_2(t, \vec{x})
    \end{gather*}
    Hence
    \begin{enumerate}[label=(\arabic*)]
    \item
      \[
        \begin{split}
          P_1(\vec{x}) \lor P_2(\vec{x}) &\equiv \exists t .
          Q_1(t, \vec{x}) \lor \exists t . Q_2(t, \vec{x}) \\
          &\equiv \exists \omega . (Q_1((\omega)_1, \vec{x})
          \lor Q_2((\omega)_2, \vec{x}))
        \end{split}
      \]
      This means that by (\ref{th:structure}),
      $P_1(\vec{x}) \lor P_2(\vec{x})$ is semi-decidable.
    \item Analogously
      \[
        P_1(\vec{x}) \land P_2(\vec{x}) \equiv \exists t .
        (Q_1(t, \vec{x}) \land Q_2(t, \vec{x}))
      \]
    \end{enumerate}
  \end{proof}
\end{theorem}

\begin{observation}
  The set of semi-decidable predicates is closed under $\land, \lor$
  and $\exists$. It is not closed under $\forall$ and $\lnot$.
\end{observation}

\begin{exercise}
  Prove that if $ P(\vec{x}) $ is semi-decidable and is not decidable
  then $ \lnot P(\vec{x}) $ is not semi-decidable.
\end{exercise}

\begin{observation}
  \begin{enumerate}[label=(\arabic*)]
  \item $A \subseteq \nat$ is recursive if and only if $A, \bar{A}$ are r.e.
  \item if $A \subseteq \nat$ r.e. and $f : \nattonat$ computable
    $\Rightarrow f^{-1}(A)$ is r.e. (projection)
  \item $A,B \subseteq \nat$ r.e. $\Rightarrow A \cup B, A \cap B$
    are r.e.
  \end{enumerate}
\end{observation}

\subsection{r.e. sets and reducibility}
Properties similar to those already seen for recursive sets hold for r.e. sets:
\begin{observation}
  Given $ A,B \subseteq \nat, A\leq_m B $, then
  \begin{itemize}
  \item B is r.e. $ \Rightarrow $ A is r.e.
  \item A is not r.e. $ \Rightarrow $ B not r.e.
  \end{itemize}
  \begin{proof}
    \begin{enumerate}
      \item If $B$ r.e., then
      \begin{equation*}
        sc_B(x) = \begin{cases}
          1 & x \in B \\
          \uparrow & $ otherwise $
        \end{cases}
      \end{equation*}
      is computable.  Let $ f:\nat\rightarrow\nat $ be a total
      computable reduction function for $ A\red B $. Then
      $ sc_A(x) = sc_B(f(x)) $, therefore $ sc_A $ is computable by
      composition and $A$ is r.e.
      \item equivalent.
    \end{enumerate}
  \end{proof}
\end{observation}
