\chapter{Computability on other domains}
Since the URM is only able to manipulate natural numbers, our definition of computability concerns only functions and predicates on $\nat$.

The concept of computability can be extended to other domains referencing a notion of effective encoding.

Suppose that we are interested in computability on an object domain $D$. 
Does our concept of computability extend to this domain? 
One of the necessary conditions is the possibility of encoding the elements of $D$ as natural numbers. 
Suppose there exists $ \alpha: D \rightarrow \nat $, which is biunivocal and that $ \alpha, \alpha^{-1} $ are ``effective''. 
We can't have a formal notion of effectiveness; it means that everyone would agree on its calculability (if there's any justice in this world).

The domain $D$ must be countable. For example, take the strings of a certain alphabet $ \Sigma $, $ D = \Sigma^* $. 
The set of rational numbers $ \mathbb{Q} $ is also countable, and so is the set of integers $\mathbb{Z}$, while $D$ can't be  $ \mathbb{R} $ or $A^\omega$ (streams).

At this point we must ask ourselves when a function on a generic domain $D$ is URM-computable. 

\begin{definition}[Computable function on generic domain]
  Given $ f: D \rightarrow D $, we say it is \textbf{computable} if $ f^* = \alpha \circ f \circ \alpha^{-1}$
  \[\begin{tikzcd}
    D & D \\
    \nat & \nat
    \arrow["{\alpha^{-1}}", from=2-1, to=1-1]
    \arrow["f", from=1-1, to=1-2]
    \arrow["\alpha", from=1-2, to=2-2]
    \arrow["{f^*}"', from=2-1, to=2-2]
  \end{tikzcd}\]
  is URM-computable.
\end{definition}

We will see that if $\alpha$ is effective, its' inverse is also effective.

\begin{example}
  To work on integers we need a function $ \alpha: \mathbb{Z} \rightarrow \nat $; one way to define it is
  \begin{equation*}
    \alpha(z) = \begin{cases}
      2z    & z \geq 0 \\
      -2z-1 & z < 0
    \end{cases} 
  \end{equation*}
which is an effective function with inverse
\begin{equation*}
  \alpha^{-1}(n) = \begin{cases}
    \dfrac{n}{2}    & n $ is even $ \\
    -\dfrac{(n+1)}{2} & n $ is odd $
  \end{cases}
\end{equation*}

An example of a computable function is $f (z) =  |z| $. 
It is computable if $ f^*=\alpha\circ f\circ \alpha^{-1} $ is URM-computable. We have
\begin{align*}
  f^*(n) &= (\alpha\circ f\circ \alpha^{-1})(n)\\ 
         &=  
\begin{cases}
  (\alpha\circ f)\left(\dfrac{n}{2}\right) & n $ even $ \\
  (\alpha\circ f)\left(-\dfrac{n+1}{2}\right) & $ otherwise $
\end{cases}\\
&=
\begin{cases}
  \alpha\left(\dfrac{n}{2}\right) & n $ even $\\
  \alpha\left(\dfrac{n+1}{2}\right) & $ otherwise $
\end{cases}\\
&= \begin{cases}
  n   & n $ even $ \\
  n+1 & $ otherwise $
\end{cases} 
\end{align*}
that is URM-computable, so $f$ is computable.
\end{example}

