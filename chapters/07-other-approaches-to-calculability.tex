\chapter{Other approaches to computability}
We already observed that the URM machine is just one of the many possible computational models that allow us to formalize the notion of computable functions.

We could have used:
\begin{itemize}
\item Turing machine
\item Canonical deduction systems of Post
\item $\lambda$-calculus of Church
\item Partial recursive functions of Gödel-Kleene
\end{itemize}

All of these approaches define the \textbf{same class of computable functions}, leading to the

\textbf{Church-Turing thesis}: a function is computable through an effective procedure 
if and only if it is URM-computable

Now, we introduce another formalism for the definition of computable functions, the set $\mathcal{R}$ of \textbf{partial recursive functions} of Gödel-Kleene and prove that it is equivalent to the URM, meaning it defines the same class of functions: $\mathcal{R} = \mathcal{C}$.

\section{Partially recursive functions}

\begin{definition}[Partially recursive functions]
  The class $ \mathcal{R} $ of \textbf{partially recursive functions} is the least class of partial functions on the natural numbers which contains
  \begin{enumerate}[label=(\alph*)]
    \item zero function;
    \item successor;
    \item projections
    \end{enumerate}
    
    and \textbf{closed} under
    \begin{enumerate}
    \item composition;
    \item primitive recursion;
    \item minimalisation.
    \end{enumerate}
\end{definition}
it is a well given definition.

\begin{definition}[Rich class]
    A class of functions $\mathcal{A}$ is said to be \textbf{rich} if it includes (a),(b) and (c) and it's closed under (1), (2) and (3).
\end{definition}

$\mathcal{R}$ is rich and for all $\mathcal{A}$, we have $\mathcal{R}\subseteq\mathcal{A}$

\begin{remark}
  The property of being a rich class is \textbf{closed for intersection}:
  let $\{\mathcal{A}_i\}_{i\in I}$ a family of rich classes, then $\bigcap_{i\in I}\mathcal{A}_i$ rich.
\end{remark}

Finally we observe that

\begin{proposition}
  The set of the partially recursive functions can be characterised as
\begin{equation*}
  \mathcal{R} = \bigcap_{
  \mathcal{A} \text{ rich}} \mathcal{A}
\end{equation*}
\end{proposition}


\begin{theorem}\label{reqc} 
  $\mathcal{R} = \mathcal{C}$
  \begin{proof}
    ($ \mathcal{R} \subseteq \mathcal{C} $) \\
    $\mathcal{C} $ is a rich class, $\mathcal{R}$ is the
    smallest rich class, so this is trivial.

    ($ \mathcal{C} \subseteq \mathcal{R} $) \\
    Given $ f : \nat^k \to \nat \in \mathcal{C} $ be a function, we show
    that $ f \in \mathcal{R} $. We know that there exists a URM program $P$ such that $ f_P^{(k)} = f$.

    We consider the following functions
    \begin{itemize}
      \item $c_P^1:\nat^{k+1}\to\nat$ with $ c_P^1(\vec{x}, t) $ be the content of $R_1$ after $t$ steps of $ P(\vec{x}) $. 
            If $P(\vec{x})$ terminates in less than $t$ steps, $ c_P^1(\vec{x}, t) $ gives the content of $R_1$ in the final configuration, i.e. the output of the function $f$;
      \item $j_P:\nat^{k+1}\to\nat$ with $ j_P(\vec{x},t) $ be the instruction to be executed after $t$ steps of $P(\vec{x})$. If the program has already ended, then $ j_P(\vec{x},t) = 0$.
    \end{itemize}

    $c_p^1$ and $j_p$ are total functions.

    Given $\vec{x}\in\nat^k$
    \begin{itemize}
      \item if $ f(\vec{x})\downarrow $ then $ P(\vec{x})\downarrow $ in a number of steps $ t_0 = \mu t.j_P(\vec{x},t)$, so 
      \begin{equation*}
        f(\vec{x}) = c_P^1(\vec{x},t_0) = c_P^1(\vec{x}, \mu t.j_P(\vec{x},t))
      \end{equation*}
      \item otherwise, if $ f(\vec{x})\uparrow $ then $P(\vec{x})\uparrow$ and $ \mu t.j_P(\vec{x},t)\uparrow $, sorting
      \begin{equation*}
        f(\vec{x}) = c_P^1(\vec{x}, \mu t.j_P(\vec{x},t))\uparrow
      \end{equation*}
    \end{itemize}
    therefore
    \begin{equation*}
      f(\vec{x}) = c_P^1(\vec{x}, \mu t.j_P(\vec{x},t)) \quad \forall \vec{x}\in\nat^k
    \end{equation*}

    % =======================================================

    If we knew that $ c_P^1, j_P \in \mathcal{R} $ then we could argue that $ f \in \mathcal{R} $.

    The idea of the proof is the following
    \begin{itemize}
    \item work on sequences encodings that represent the registers and program counter configurations
    \item manipulate such sequences with functions such as ($ p_x, qt, \text{div}, \dots $) that were built by:
      \begin{itemize}
      \item composition
      \item primitive recursion
      \end{itemize}
    \end{itemize}
    in this way, we obtain $c_P^1, j_P$ through primitive recursion.
    
    A register configuration in which a finite number of registers
    contains a value other than 0 can be encoded in the following way:
    \begin{equation*}
      c = \prod\limits_{i \geq 1}p_i^{r_i} = \prod\limits_{i = 1}^{k}p_i^{r_i}
    \end{equation*}
    such that
    \begin{equation*}
      r_i = (x)_i
    \end{equation*}

    Thus, using this encoding, we can consider the function $c_P:\nat^{k+1} \rightarrow \nat$,
    that is the registers' configuration after t steps of $P(\vec{x})$. 

    We define $c_P$, $j_P$ by primitive recursion:
    \begin{itemize}
      \item base cases \begin{align*}
        &c_P(\vec{x}, 0) = \prod\limits_{i=1}^k p_i^{x_i}\\
        &j_P(\vec{x}, 0) = 1
      \end{align*}
      \item recursive cases   
      \[
        c_P(\vec{x},t+1) = \begin{cases}
          qt(p_n^{(c)_n}, c)                    & \text{if }1 \leq j \leq l(P) \; \And \; I_j = Z(n) \\
          p_n \cdot c                           & \text{if }1 \leq j \leq l(P) \; \And \; I_j = S(n) \\
          qt(p_n^{(c)_n}, c)  \cdot p_n^{(c)_m} & \text{if }1 \leq j \leq l(P) \; \And \; I_j = T(m,n) \\
          c                                     & \text{otherwise}
        \end{cases}
      \]
  
      \[
        j_P(\vec{x}, t+1) = \begin{cases}
          j + 1       & \text{if } 1 \leq j < l(P) \And I_j = Z(n), S(n), T(m,n) \\
                      & \text{ or } J(m,n,t) \text{ with } (c)_m \neq (c)_n \\
          u           & \text{if } 1 \leq j \leq l(P) \And I_{j} = J(m,n,u) \\
                      & \And (c)_m = (c)_n \And \text{if } 1 \leq u \leq l(P) \\
          0 & \text{otherwise}
        \end{cases}
      \]
    \end{itemize}
    thus $c_P$, $j_P$ are in $\mathcal{R}$.
    So it is $c_P^1$, since $c_P^1(\vec{x}, t) = (c_P(\vec{x}, t))_1$ for all $\vec{x}\in\nat^k$ and $t\in\nat$.
    Thus $f$, defined by composition and minimalisation of $c_P^1$ and $j_P$ is in $\mathcal{R}$ as wanted.

  \end{proof}
\end{theorem}
