\chapter{Cantor diagonalization method}

The diagonalization technique allows to build an
object that differs from an enumerable infinity of similar objects.
The idea behind is: given an enumerable set of objects
$\{x_1, x_2, x_3, \dots \}$
we can build another object $x$ with the
same nature of all the other objects, but different from all of them
by making it ``differ from $x_n$ on $n$''.

\begin{proposition}[Cantor diagonalization]
  For all $X$ set, we have
  \[ 
    |X| < |2^X|
  \]
\end{proposition}

This is the original method used by Cantor, one of the founders of the classic theory of sets, to prove that there are
  various ``levels of infinity''.

\begin{corollary}
  $|\nat| < |2^\nat|$
  \begin{proof}
    By contradiction $|\nat| \geq |2^\nat| $, i.e. $|2^\nat| $ countable. This means that there exists an
    enumeration of $2^\nat$: $x_0, x_1, x_2, \dots$

    Consider
    \[
      \begin{tabu}{c | c c c c}
        & x_0 & x_1 & x_2 & \dots \\ \hline
        0 &  ?  & NO & \dots & \\
        1 &  NO  & ? & YES & \\
        2 &  YES  & NO & ? & \\
        \vdots & & & &
      \end{tabu}
    \]
    We can define $D = \{i \mid i \notin X_i\} \subseteq \nat$ which differs from
    $X_i$ on its $i^{\mbox{th}}$ element. Obviously $D \in 2^\nat $
    which means that there exists $k$ such that $D = X_k$. But is $k$ in $D$?
    \begin{gather*}
      k \in D \quad \Rightarrow \quad k \notin X_k = D \\
      k \notin D \quad \Rightarrow \quad k \in X_k = D
    \end{gather*}
    which is absurd. Therefore $|\nat| < |2^\nat|$.
  \end{proof}
\end{corollary}

\newcommand{\nattonat}{\nat \to \nat}
\begin{corollary}\label{corollary:nattonat}
  Consider $\nattonat = \{f \mid f : \nattonat \}$, 
  we have \[|\nattonat| > |\nat|\]
  \begin{proof}
    There are two approaches to prove the corollary:
    \begin{enumerate}
      \item Define
      \[
        \mathcal{F} = \{f \mid f : \nattonat \mbox{ total }, \forall x \ f(x)\in \{0, 1\}
        \} \subseteq \nattonat
        \]
      there is a bijection between $ \mathcal{F} $ and $ \nat \to 2$
      and so 
      \[|\nattonat| \geq |\nat \to 2| > |\nat|\]
      \item Let $f_1, f_2, f_3, \dots$ be an enumeration of elements in
      $\nattonat$ and consider
      \[
        \begin{tabu}{c | c c c c}
          & f_0 & f_1 & f_2 & \dots \\ \hline
          0 &  f_0(0)  & \dots & \dots & \\
          1 &  \dots  & f_1(1) & \dots & \\
          2 &  \dots  & \dots & f_2(2) & \\
          \vdots & & & &
        \end{tabu}
      \]
  
      we can define a function $f$ that differs from every other
      function on the diagonal based on the input:
      \[f(i) = \begin{cases}
          0 & \quad \mbox{if } f_i(i)\uparrow \\
          \uparrow & \quad \mbox{if } f_i(i) \downarrow
        \end{cases}
      \] so that
      \[
        \forall i \quad f(i) \neq f_i(i) \quad (f \neq f_i)
        \]
      thus no enumeration of functions in $\mathcal{F}$ can contain all $\mathcal{F}$,
      so $\mathcal{F}$ is not countable.
    \end{enumerate}
    

    
  \end{proof}
\end{corollary}

\newcommand{\noc}{\bar{\mathcal{C}}}

\begin{corollary}
  The set
$\noc = \{f : \nattonat \; | \; f \mbox{ not computable}\}$ is not
enumerable.
\begin{proof}
  We know that $|\mathcal{C}| = |\nat|$. If $\noc$ were enumerable, then
  $\nattonat = \mathcal{C} \cup \noc$ would be enumerable, which is
  absurd for the previous corollary.
\end{proof}
\end{corollary}

\begin{observation}
  There exists a total non-computable function $f : \nattonat$ defined by
  \[
    f(n) = \begin{cases}

      \varphi_n(n) + 1 & \quad \mbox{if } \varphi_n(n) \downarrow  \\

       0 & \quad \mbox{if } \varphi_n(n)\uparrow 
    \end{cases}
  \]
  $f$ is not computable because
  \begin{itemize}
    \item if $\varphi_n(n) \downarrow$, then $f(n) = \varphi_n(n) + 1 \neq \varphi_n(n) $
    \item if $\varphi_n(n)\uparrow$, then $f(n) = 0 \neq \varphi_n(n)$
  \end{itemize}
  so \[\forall n \; f \neq \varphi_n\]
\end{observation}


\begin{observation}
  There are infinitely many total non-computable
functions of the followin shape
\[
  f(n)  = \begin{cases}
    \varphi_n(n) + k & n \in W_n \\
    k & n \notin W_n
  \end{cases}
\]

\end{observation}

\begin{exercise}
  Let $f: \nat \rightarrow \nat$, $m \in \nat$.
  Show that there exists a non-computable function $g : \nat \rightarrow \nat$ 
  such that \[g(x) = f(x) \quad \forall x < m\]

  Idea: use a ``translated diagonal'';
  \[
    g(x) = \begin{cases}
      f(x) & x < m \\
      \varphi_{x - m}(x) + 1 & x \geq m \mbox{ and } x \in W_{x-m} \\
      0 & x \geq m \mbox{ and } x \notin W_{x-m}
    \end{cases}
  \]
  $g$ is not computable since
  $g(x + m) \neq \varphi_x(x+m)$ for all $x$, so
  \[\forall x \ g\neq \varphi_x\]

  Another approach is to define $g$ in the following way
  \[
    g(x) = \begin{cases}
      f(x) & x < m \\
      \varphi_x(x) + 1 & x \geq m \mbox{ and } x \in W_{x} \\
      0 & x \geq m \mbox{ and } x \notin W_{x}
    \end{cases}
  \]
  because each function appears infinitely many times in the enumeration,
  and skipping the first $m-1$ steps doesn't create any
  problem. Formally
  \[
    \forall x \geq m \quad g \neq \varphi_x
    \]
    so for all $y$
  \[
    \forall y\ \exists x \geq m \ \varphi_y = \varphi_x
  \]
    thus \[
      \forall y \; \varphi_y \neq g
      \]
  then $g$ is not computable.
\end{exercise}

\begin{exercise}
  Given a family of functions
  $\{f_i\}_{i\in \nat}$ with $ f_i : \nattonat$, construct $g: \nattonat$
  such that $\dom{g} \neq \dom{f_i}$ for all $i \in N$

  Idea:
  \[
    g(n) = \begin{cases}
      0 & \mbox{if } n \notin \dom{f_n} \\
      \uparrow & \mbox{if } n \in \dom{f_n}
    \end{cases}
  \]
  In this way 
  \[
    \forall n \ n \in \dom{g} \Leftrightarrow 
  n \notin \dom{f_n}
    \]
\end{exercise}

\begin{exercise}
  Define a non-computable total function that returns $0$ when the input
  is even

  Idea:
  \[
    f(x) = \begin{cases}
      0 & x \mbox{ is even} \\
      \varphi_{\frac{x-1}{2}}(x) + 1 & x \mbox{ is odd, and } x \in
      W_{\frac{x-1}{2}} \\
      0 & x \mbox{ is odd, and } x \notin W_{\frac{x-1}{2}}
    \end{cases}
  \]
  it is total not computable. In fact
  \begin{itemize}
    \item if
    $2n+1 \in W_n \Rightarrow f(2n+1) = \varphi_{n}(2n+1) + 1 \neq
    \varphi_n(2n+1)$
    \item if
    $2n+1 \notin W_n \Rightarrow f(2n+1) = 0 \neq \varphi_n(2n+1)
    \uparrow $
  \end{itemize}
  so \[
  \forall n \ f(2n + 1) \neq \varphi_n(2n+1)\]

  

  

\end{exercise}
% Same as the previous chapter, these notes are very different form
% the italian version and not complete in the same way

% Given a set $X$ we know that $ |X| \geq |2^X| $ is never valid, that
% is, the set of its parts is always bigger. Suppose there is
% $ f:X\rightarrow2^X $ surjective. Hence $ R = \{y . y \in X $ and
% $ x \not \in f(y) \ \} \in 2^X $, since f is surjective then
% $ \exists y_R \in X . f(y_R) = R$. I consider the cases separately:
% \begin{itemize}
% \item $ y_R \in R $ then $f(y_R) \Rightarrow y_R \not \in R $
% \item $ y_R \not \in R $ then $ f(y_R) \Rightarrow y_R \in R$
% \end{itemize}
% Now we prove that the set of the parts of the natural numbers is not
% countable. We assume there is a surjective function
% $ \nat \rightarrow 2^\nat $. This means that I can enumerate subsets
% $ X_i $ s.t. $ i \in \nat $. At this point I create a matrix where
% each row $i$ of column $j = 1$ if $ x_i \in X_j $ and 0 otherwise. Now
% consider the inverted diagonal, that is if $ x_i \not \in X_i $, in
% this way it is different from all the columns, that is
% $ \forall i . R \not= X_i $ because
% $ n \in R \Leftrightarrow b \not \in X_n $ absurd since I assumed that
% $ \{X_0 \dots X_n \} = 2^\nat$.

% We conclude that $ |\nat| \not \geq |2^\nat| $

% But we want $ |\nat| < |2^\nat| $.

% But:
% $ |\nat| \leq |2^\nat| \land |\nat| \not\geq |2^\nat| \implies |\nat|
% < |2^\nat| $

% I take the set of the characteristic functions $g$ of $ 2^\nat $ and I
% call it $Y$, I take the set of all the functions $ \mathcal{F} $, of
% course it holds that $ Y \subseteq \mathcal{F} $ and therefore
% $ |Y| \leq |\mathcal{F}| $ but being that $ |\nat < |2^\nat| $ then I
% also have that $ |\nat| \leq |\mathcal{F}| $

% There is a total function that cannot be computed. We know how to
% enumerate computable functions because we can enumerate them in the
% form of numbers, repeating some of them. Now let's define $ f(n) $ as
% non-total computable.

% Matrix where the columns are the functions computed by the program
% $ i \in \nat $, that is $ \{\phi_i . i \in \nat \} $ and the rows are
% the arguments 0,1, \dots.

% \begin{equation*}
%   f(n) = \begin{cases}
%     \phi_n(n)+1 & $ se $\phi(n)\downarrow \\
%     0           & $ se $\phi(n)\uparrow
%   \end{cases}
% \end{equation*}

% we observe that $f$ is total by construction and
% $ f \not= \phi_n \forall n, n \in \nat $ Furthermore by construction
% it is different from all computable functions and therefore it is not
% computable.
