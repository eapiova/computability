\chapter{Second recursion theorem}
Let $f:\nat \to \nat$ be total, computable and extensional i.e.
\[
\forall e, e'\ \varphi_e=\varphi_{e'}\Rightarrow\varphi_{f(e)}=\varphi_{f(e')}
\] 
Then, for the
theorem (\ref{th:myhill-shepherdson2}) there exists a unique recursive
functional $\Phi$ such that
\[ \forall e \in \nat \quad \Phi(\varphi_e) = \varphi_{f(e)} \] Since $\Phi$ is
recursive, for the first recursion theorem (\ref{th:first-recursion})
it has a least fixed point $f_\Phi :\nat \to \nat$ computable, therefore exists
$e_0\in \nat$ such that
\[
  \varphi_{e_0} = f_\Phi = \Phi(f_\Phi) = \Phi(\varphi_{e_0}) = \varphi_{f(e_0)}
\]
This means that if $f$ is total computable and extensional, then there
exists $e_0$ such that \[\varphi_{e_0} = \varphi_{f(e_0)}\]

The second recursion
theorem states that this is valid also when $f$ is not extensional. The
consequence is that a function $f : \nattonat$ cannot be thought as a
functional on computable functions.
\begin{theorem}[Second recursion theorem (Kleene)]\label{th:second-recursion}
  Let $f : \nattonat$ a total computable function. Then there exists
  $e_0 \in \nat$ such that
  \[
    \varphi_{e_0} = \varphi_{f(e_0)}
  \]
  \begin{proof}
    Let $f : \nattonat$ a total computable.
    Take
    \begin{align*}
      g(x,y) &= \varphi_{f(\varphi_x(x))}(y)\\
             &= \univ(f(\varphi_x(x)), y)     \\
             &= \univ(f(\univ(x,x)), y) 
    \end{align*}
    it is computable. This means that for the
    $smn$ theorem exists $s : \nattonat$ total computable such that
    \[
      \varphi_{s(x)}(y) = g(x,y) = \varphi_{f(\varphi_x(x))}(y) \quad \forall x,y
    \]
    Since $s$ is computable there exists $m\in\nat$ such that 
    \[
    S = \varphi_m
    \]
    hence
    \[
      \varphi_{\varphi_m(x)}(y) = \varphi_{f(\varphi_x(x))}(y) \quad \forall x,y
    \]
    For $x=m$
    \[
      \varphi_{\varphi_m(m)}(y) = \varphi_{f(\varphi_m(m))}(y) \quad \forall y
    \]
    We set $e_0 = \varphi_m(m)\downarrow$ and we replace in the previous equation
    \[
       \varphi_{e_0}(y) = \varphi_{f(e_0)}(y) \quad \forall y
    \]
    i.e.
    \[
      \varphi_{e_0} = \varphi_{h(e_0)}
    \]
  \end{proof}
\end{theorem}

This theorem can therefore be interpreted in the following manner
\emph{given any effective procedure to transform programs, there exists a
program such that when properly modified does exactly what it did
before} or \emph{it is impossible to write a program that edits the core of
all programs}.

The proof of the theorem can appear mysterious, but after a closer
inspection, it clearly appears to be a simple diagonalization.
Nevertheless, the result of this theorem is extremely
deep; in this way, many theorems we've seen up until now are just
corollaries.
\begin{corollary}[Rice's theorem]
  Let $\emptyset \neq A \subset \nat$ saturated, then
  $A$ is not recursive.
  \begin{proof}
    Let $\emptyset \neq A \subset \nat$ saturated.
    Take $e_1\in A$ and $e_0\notin A$ and
    assume by contradiction that $A$ is recursive.
    Define $f : \nattonat$
    \begin{align*}
      f(x) &= \begin{cases}
        e_0 & x \in A \\
        e_1 & x \notin A
      \end{cases} \\
            &= e_0 \cdot \chi_A(x) + e_1 \cdot \chi_{\bar{A}}(x)
    \end{align*}
    Since $A$ is recursive then also $\bar{A}$ is recursive, thus $\chi_A$ and $\chi_{\bar{A}}$ are computable.
    Thus, $f$ is computable and total, then for the second recursion
    theorem (\ref{th:second-recursion}) there exists $e\in\nat$ such that
    $\varphi_e = \varphi_{f(e)}$; there are two possibilities
    \begin{itemize}
      \item if $e\in A$, then $f(e)=e_0 \notin A$ and since $A$ saturated, $\varphi_e \neq \varphi_{e_0} = \varphi_{f(e)}$
      \item if $e\notin A$, then $f(e)=e_1 \in A$ and since $A$ saturated, $\varphi_e \neq \varphi_{e_1} = \varphi_{f(e)}$
    \end{itemize}
    that is absurd, so $A$ can't be recursive.
  \end{proof}
\end{corollary}

\begin{corollary}
  The halting set $K = \{ x \mid \varphi_x(x)\downarrow\}$ is not recursive.
  \begin{proof}
    Let $k = \{ x \mid x \in W_x\}$ recursive for the sake of the
    argument. and let $e_0, e_1$ be indexes s.t.
    $\varphi_{e_0} = \emptyset$ and
    $\varphi_{e_1} = \lambda x \; . \; x$. 
    
    Define $f: \nattonat$
    \begin{align*}
      f(x) &= \begin{cases}
        e_0 & x \in K \\
        e_0 & x \notin K
      \end{cases}\\
      &= e_0 \cdot \chi_K(x) + e_1 \cdot \chi_{\bar{K}}(x)
    \end{align*}
    If $K$ were recursive, then $\chi_K$ and $\chi_{\bar{K}}$ would be computable, thus
    $f$ would be both computable and total, then by
    (\ref{th:second-recursion}), there would be
    $e\in\nat$ such that $\varphi_e = \varphi_{f(e)}$, but
    \begin{itemize}
      \item if $e\in K$, then $f(e)=e_0$, so $\varphi_e(e)\downarrow \neq \varphi_{f(e)}(e) = \varphi_{e_0}(e)\uparrow $
      \item if $e\in \bar{K}$, then $f(e)=e_1$, so $\varphi_e(e)\uparrow \neq \varphi_{f(e)}(e) = \varphi_{e_1}(e)=e\downarrow  $
    \end{itemize}
    which is absurd, so $K$ can't be recursive.
  \end{proof}
\end{corollary}

\begin{corollary}
  $K = \{ x \mid \varphi_x(x)\downarrow\}$ is not saturated.
  \begin{proof}
    Let $n_0$ s.t. $\varphi_{n_0} = \{(n_0, n_0)\}$. We know that
    there are infinitely many indices for the same function; so let
    $n \neq n_0$ s.t. $\varphi_n = \varphi_{n_0}$.
    \[
      \varphi_n(n) \uparrow \Rightarrow n \notin K
    \]
  \end{proof}
\end{corollary}
