\chapter {Rice theorem}

Rice's theorem gives a general undecidability result. It states that \emph{no property} of computable functions
(besides the obvious ones) is decidable.

Formally, we'll need the notion of \emph{saturated} set.
\section{Saturated sets}
\begin{definition}[Saturated set]
  A subset $A \subseteq \nat$ is \emph{saturated} (or \emph{extensional}) if
  \[
     x \in A \wedge \varphi_x = \varphi_y \Rightarrow y \in A 
  \]
  In other words, $A$ is saturated if it expresses a property of
functions, independently from indices
\[A = \{x \mid P(\varphi_x)\}\]
or, again, if there exists $\mathcal{A} \subseteq \mathcal{C}$ such that
\[A = \{ x \mid \varphi_x \in \mathcal{A}\}\]
\end{definition}

\begin{example}
  The following set is saturated
  \begin{align*}
    T &= \{ n \mid P_n \mbox{ always terminate} \} \\
    &= \{n \mid \phi_n \in \mathcal{T} \}
  \end{align*}
  where
  \[ \mathcal{T} = \{ f \mid f \mbox{ is total} \} \]
\end{example}

\begin{example}
  The following set is saturated
  \begin{align*}
    ONE &= \{ n \mid P_n \mbox{ computes } \mathbf{1}\} \\
    &= \{n \mid \phi_n = \mathbf{1} \} \\
    &= \{n \mid \phi_n \in \{ \mathbf{1} \} \}
  \end{align*}
\end{example}

\begin{example}
  Consider
  \begin{align*}
    T_2 &= \{ e \mid P_e(e)\downarrow \mbox{ in two steps } \} \\
    &=
    \{e \mid \phi_e \in \mathcal{T}_2 \}
  \end{align*}
  two programs can compute the same function, one terminates in
  less than 2 steps and the other in more than 2. Thus, the set is not
  saturated.
\end{example}

\begin{example}
  Consider
  \begin{align*}
    K &= \{e \mid e\in W_e \} \\ 
    &= \{e \mid \phi_e\in \mathcal{K} \}
  \end{align*}
  where we would like
  \[   \mathcal{K} = \{f \mid ? \} \]
  It is not saturated. It can be shown that there is a program $e$ such that
  \begin{equation*}
    \phi_e(x) = \begin{cases}
      0 & x = e \\
      \uparrow & $ otherwise $
    \end{cases}
  \end{equation*}
\end{example}

\section {Rice's theorem}

\begin{theorem}[Rice's theorem]
  Let $ A \in \nat, A \neq \emptyset, A \neq \nat$ be saturated.
  Then it is not recursive.
\end{theorem}

\begin{proof}
  We show that $K \red A$. 
  Let $ e_0$ such that $\phi_{e_0}(x)\uparrow\forall x$
  \begin{itemize}
  \item[($e_0 \notin A$)]
     Suppose
    $e_0\notin A$ and let $e_1\in A$.
    Now define
    \begin{align*}
      g(x,y) &= \begin{cases}
        \phi_{e_1}(y) & x \in K \\
        \phi_{e_0}(y) & x \notin K
      \end{cases} \\
      &=
      \begin{cases}
        \phi_{e_1}(y) & x \in K \\
        \uparrow & x \notin K
      \end{cases} \\
      &= \phi_{e_1}(y) \cdot \mathbf{1}(\univ(x,x))
    \end{align*}
    it is computable. By \emph{smn} theorem there is $s \nat \to \nat$ such that
    $ \phi_{s(x)}(y) = g(x,y)$.

    Now observe that $s$ is a reduction function for $K \red A$
    \begin{itemize}
      \item $x \in K
      \Rightarrow \forall y  \ \varphi_{s(x)}(y) = \varphi_{e_1}(y)
      \Rightarrow s(x) \in A$
      \item $x \notin K
      \Rightarrow \forall y  \ \varphi_{s(x)}(y) = \varphi_{e_0}(y)\uparrow 
      \Rightarrow s(x) \notin A$
    \end{itemize}
    Hence $K \leq_m A$, $K$ not recursive, thus $A$.
    
  \item[($e_0 \in A$)] If $e_0 \in A$ then $ e_0 \not \in \bar{A}
    $. Then
    $ \bar{A} \subseteq \nat, \bar{A} \neq \emptyset, \bar{A} \neq
    \nat $, so $\bar{A}$ is not recursive, and
    therefore $A$ is not recursive either.
  \end{itemize}
\end{proof}

\begin{example}[Output problem]
  \[ B_n = \{e \mid n \in E_e \} \quad \mbox{is not recursive} \]
  we showed that $K \red B_n$. We can conclude the same by observing
  \begin{itemize}
  \item $B_n$ is saturated;
  \item $B_n \neq \emptyset$;
  \item $B_n \neq \nat$.
  \end{itemize}
  By Rice's theorem $ B_n $ is not recursive.
\end{example}
