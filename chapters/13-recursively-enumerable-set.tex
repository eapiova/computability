\chapter{Recursive and recursively enumerable sets}
Up to now we saw many computable functions, decidable problems, but
only in few cases we gave examples of the large class of
non-computable functions and undecidable predicates. Fro this reason
we want to start a mathematical study of:
\begin{itemize}
\item classes of undecidable predicates
\item techniques to prove the undecidaility of some predicates
\end{itemize}
This way we can highlight the limits of computers abilities and give a
structure to problems classes (spoiler: the majority of interesting
predicates are undecidable).

We'll focus on \emph{numerical sets} $X \subseteq \nat$ trying to find
an answer to the problem ``how difficult it is to determine weather
$x \in X$''? This way we'll find
\begin{itemize}
\item recursive sets
\item recursively enuerable sets
\end{itemize}

\section{Recursive sets}
\begin{definition}
  A set $A \subseteq \nat$ is \emph{recursive} if its charateristic
  function
  \begin{gather*}
    \chi_A : \nattonat \\
    \chi_A(x) = \begin{cases}
      1 & x\in A \\
      0 & x \notin A
    \end{cases}
  \end{gather*}
  is computable.
\end{definition}

In other words, if the predicate ``$x \in A$'' is decidable.

\textbf{Note:}
\begin{itemize}
\item if $\chi_A \in \pr$ we'll say that $A$ is \emph{primitively}
  recursive.
\item the notion can be extended to subsets of $\nat^k$, but we'll
  stick to $\nat$ subsets, since every $\nat^k$ subset can be encoded
  into $\nat$
\end{itemize}

\paragraph{Examples}
Are recursive:
\begin{enumerate}[label=(\alph*)]
\item $\nat$, since $\chi_\nat (x) = 1$ is computable;
\item prime numbers
  \[
    Pr(x) = \begin{cases}
      1 & \mbox{if $x$ is prime} \\
      0 & \mbox{otherwise}
    \end{cases}
  \]
  is computable;
\item each and every finite set. In fact, given $A \subset \nat$ with
  $|A| < \infty \quad A = \{x_1, x_2, \dots, x_n\}$ we have that
  \[
    \chi_A(x) = \overline{sg}\left( \prod_{i=1}^n|x - x_i| \right)
  \]
  is computable
\end{enumerate}

On the other hand, are definetly not recursive:
\begin{enumerate}[label=(\alph*)]
\item $k = \left\{ x \; | \; x \in W_x \right\} $, since
  \[
    \chi){k(x)} = \begin{cases}
      1 & x \in W_x \\
      0 & x \notin W_x
    \end{cases}
  \]
  is not computable;
\item $\left\{ x \; | \; \varphi_x \mbox{ total } \right\} $
\end{enumerate}

\mbox{\\}

\paragraph{Closure of recursive sets}
If $A,B \subset \nat$ are recursive, then are alsp recursive
\begin{enumerate}[label=\arabic*)]
\item $\overline{A} = \nat - A$
\item $A \cap B$
\item $A \cup B$
\end{enumerate}

\subsection{Reduction process}
It is a simple but very important process for study decidability
problems. It tries to formalize the intuition of a problem
$\mathcal{A}$ being ``easyer'' then another one, $\mathcal{B}$

\newcommand{\red}{\ensuremath{\leq_m}}
\begin{definition}
  Let $A,B \subseteq \nat$. We say that the problem $x \in A$ is
  \emph{reducible} to the problem $x \in B$ (or also that directly $A$
  is reducible to $B$), and we write $A \red B$ if exists
  $f : \nattonat$ computable and total s.t.
  \[x \in A \quad  \Leftrightarrow \quad f(x) \in B\]
\end{definition}
In this case, $f$ is the \emph{reduction function}.

\textbf{Proposition:} Let $A,B \subseteq \nat$ s.t. $A \red B$ then
\begin{enumerate}[label=\arabic*)]
\item if $B$ is recursive, then $A$ is recursive
\item of $A$ is not recursive, then $B$ is not recursive
\end{enumerate}

\begin{proof}
We just need to observe that $\chi_A = \chi_B \circ f$
\end{proof}

\section{Recursively enumerable sets}
We say that $ A \subseteq \nat $ is \emph{recursive} if
the semi-characteristic function is computable.
\begin{equation*}
  sc_A(x) = \begin{cases} 1 & x \in A \\ \uparrow & $
    otherwise $
  \end{cases}
\end{equation*}

Predicate $ Q(\vec{x}) \subseteq \nat^k $ semi-decidable

\begin{equation*} sc_Q(\vec{x}) = \begin{cases} 1 & Q(\vec{x}) \\
    \uparrow & $ altrimenti $
  \end{cases}
\end{equation*}

So say $A$ is r.e. is like saying that the predicate $ Q(x)=``x \in A"
$ is semi-decidable

\textbf{Proprietà}: All recursive sets are also recursively
enumerable:

\textit{A} recursive if $ A, \bar{A} $ r.e.

If $A$ recursive, \begin{equation*} \mathcal{X}_A(x)=\begin{cases} 1 &
    x\in A \\ 0 & $ otherwise $
  \end{cases}
\end{equation*}

Then $ sc_A(x) = \mu z.\bar{sg}(\mathcal{X}(x)) + 1 $
computable. Computable, therefore \textit{A} is r.e. Same goes for $
\bar{A} $.

\section {Theorem of structure of semidecidable predicates}

Let $ Q(\vec{x}) \subseteq \nat^k $ be a predicate.

This is decidable $ \Leftrightarrow $ there is a predicate $
Q'(t,\vec{x}) \subseteq \nat^{k+1} $ s.t. $ Q(\vec{x}) = \exists
t. Q'(t,\vec{x}) $

So try 0,1,2, \dots if there is a point where it holds then yes,
otherwise I try endlessly.

\section {Projection theorem}

Let $ P(x,\vec{y}) $ be semi-decidable; then $ \exists x $ s.t. $
P(x,\vec{y}) = P'(\vec{y})$ is semi-decidable.

So if you use existential identifier you go out of the set of the
decidable predicates and enter the semi-decidable set, but if you use
it twice you don't go outside the semi-decidable set.

\textbf{observation:} $ P_1(\vec{x}), P_2(\vec{x}) $ semi-dec. $
\Rightarrow P_1(\vec{x}) \lor P_2(\vec{x}) $; $ P_1(\vec{x}) \land
P_2(\vec{x}) $ semi-dec.

Because you quantify existentially a single number whose components
are equivalent to quantifying the two numbers of the individual
predicates. In the \textit{OR} case and in the \textit{AND} case you
look for the same number directly.

\textbf{Exercize:} If $ P(\vec{x}) $ is semi-decidable and is not
decidable then $ \lnot P(\vec{x}) $ is not semi-decidable.

\textbf{Observation:} $ A,B \subseteq \nat, A\leq_m B $ then:
\begin{itemize}
\item B is r.e. $ \Rightarrow $ A is r.e .;
\item A is not r.e. $ \Rightarrow $ B not r.e.
\end{itemize} Demonstration: If B r.e. then
\begin{equation*} SC_B(x) = \begin{cases} 1 & x \in B \\ \uparrow & $
    otherwise $
  \end{cases}
\end{equation*} This is computable. Let $ f:\nat\rightarrow\nat $ be a
total computable reduction function\\ $ A\leq B $ Then $ SC_A(x) =
SC_B(f(x)) $, therefore $ SC_A $ is computable by composition.
