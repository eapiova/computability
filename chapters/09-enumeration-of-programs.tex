\chapter{Enumeration of programs}
\newcommand{\pr}{\mathcal{PR}}
The objective here is to define an \emph{effective enumeration} of URM
programs and URM-computable functions. These results will be
fundamental for our theory, and in particular to
\begin{itemize}
\item prove the existence of non computable functions
\item the \textit{smn} theorem
\item the universal function/machine.
\end{itemize}

\begin{remark}
  Let $ A, B $ be two sets,
\begin{itemize}
\item $ |A| = |B| $ if there exists $ f:A\to B$ bijective.

\item $|A| \leq |B|$ if there exists $f:A\to B$ injective or
  there exists $g:B\to A$ surjective.

\item if $|A| \leq |B|$ and $|B| \leq |A|$, then $|A|=|B|$.
  Assuming the axiom of choice, if we have $\{A_i\}_{i \in I}$ family
  of non-empty sets, then
  there exists a function $$f:I \to \bigcup_{i \in I}A_i \quad
  \text{ s.t. }  \forall i \in I\; f(i) \in A_i$$
\end{itemize}
\end{remark}

\begin{definition}[Countable set]
  $A$ is \textbf{countable} if $ |A| \leq |\nat| $, i.e. we have
$f: \nat \to A$ surjective. We say that
$f$ is an enumeration of $X$, because we can enumerate all
elements in $X$ with \[f(0), f(1), f(2), \dots \]
\end{definition}

\begin{definition}[Enumeration without repetitions]
  An enumeration is \textbf{without repetitions} if it is injective.
  It is also called \textbf{effective}.
\end{definition}

\begin{lemma}
  There are effective enumerations of
  \begin{enumerate}[label=(\arabic*)]
  \item $\nat^2$
  \item $\nat^3$
  \item $\bigcup_{k\geq 1} \nat^k $
  \end{enumerate}
  \begin{proof}
    \begin{enumerate}[label=(\arabic*)]
    \item we already saw that 
      \begin{align*}
        &\pi : \nat^2 \to
        \nat \\
        &\pi(x,y) = 2^x(2y+1)-1
      \end{align*}  
     is computable
       with inverse
       \begin{align*}
        &\pi^{-1} : \nat \to \nat^2\\
        &\pi^{-1}(x) = (\pi_1(x), \pi_2(x))
       \end{align*}
      where $\pi_1,\pi_2 : \nat \to \nat$
      \begin{align*}
        &\pi_1 (n) = (n+1)_1\\
        &\pi_2(n) = \left(\left(\dfrac{n+1}{2^{\pi_1(n)}}\right)-1\right)
       \end{align*}
       are computable.
    \item{
        consider 
        \begin{align*}
          &\nu : \nat^3 \to
          \nat\\
          &\nu(x,y,z)= \pi(\pi(x,y),z)
        \end{align*}
        with inverse built upon projections
        \begin{align*}
          &\nu^{-1} : \nat \to \nat^3\\
          &\nu^{-1}(x) = (\nu_1(x), \nu_2(x), \nu_3(x))
        \end{align*}
        with $\nu_1, \nu_2, \nu_3$ are computable.}
    \item{ The following tuple encoding
    \begin{align*}
      &\tau : \bigcup_{k \geq 1} \nat^k \to \nat\\
      &\tau(x_1, \dots, x_k) = \prod_{i=1}^{k}p_i^{x_i}-1
    \end{align*}
    does not work, since it is not injective.
    The idea is that we can \emph{increment} the last component, in this way
    \[
    \tau(x_1, \dots, x_k) = \left(\prod_{i=1}^{k-1}p_i^{x_i}\right) \cdot p_k^{x_k+1} - 2
    \]
    with inverse $\tau^{-1} : \nat \to \bigcup_{k \geq 1} \nat^k$ defined out of the following functions:
    \begin{itemize}
      \item $l :\nat \to \nat$
            \[
            l(n) =   \max\{k : p_k | (x+2)\} = x - \mu (h \leq x) \;
            . \; p_{x-h} | (x+2)
            \]
      \item $a :\nat^2 \to \nat$
            \[
              a(n,i)=\begin{cases}
                (n+2)_i & i = 1, \dots, \ell(x)-1 \\
                (n+2)_i - 1 & i = \ell(x)
              \end{cases}
              \]
    \end{itemize}
    An alternative encoding is the following
    \begin{itemize}
      \item $\tau(x_1, \dots, x_k) = \pi(\prod_{i=1}^k p_i^{a_i}, k)$
      \item $l(n) = \pi_2(n)$
      \item $a(n,i) = (\pi_1(n))_i$
    \end{itemize}
      
    
    
    
    
    
    
    
    
    
    
    
    
    
    
    
    
    \iffalse
    Observe that the encoding
        \[(a_1, \dots, a_k) \mapsto \prod^k_{i=1}p_i^{a_i+1}\] can't
        be used since it is injective, but not surjective (the
        co-domain does not contain $0,1$ and any number $x$
        s.t. \(\exists i,j \; i<j \quad p_i \times x \wedge p_j | x\))

        The idea is to leverage the uniqueness of the binary
        representation of natural numbers (positional).
        \[\tau : \cup_{k \geq 1} \nat^k \to \nat \]

        \[\tau (a_1, \dots, a_k ) = 2^{a_1}+2^{a_1+a_2+1}+ \dots +
          2^{a_1+\dots + a_k + (k-1)} - 1\]

        \[\tau (a_1, \dots, a_k ) = \sum^k_{i=1}2^{(\sum_{j=1}^i a_j)
            + (i-1)}-1\]

        \begin{itemize}
        \item $\tau$ is intuitively effective (it uses only
          exclusively functions), but the domain does not allow us to
          prove its' computability.
        \item $\tau$ is bijective: let $x \in \nat$, we need to find
          $(a_1, \dots, a_k)$ s.t. $\tau(a_1,\dots,a_k) = x$. Thanks
          to the uniqueness of binary representation of $x+1$
          \[x = (\alpha_m \alpha_{m-1} \alpha_1 \alpha_0)_2 = \alpha
            2^m + \dots + \alpha_1 2^1 + \alpha_0 - 1 \quad \text{with
            } \alpha \in \{0,1\} \] We can then easily define a
          function $\alpha(x,j)$ s.t. $\alpha(x,j) = \alpha_j$, in
          fact
          \[\alpha(x, j) = \text{rm}(2, \text{qt}(2^j, x+1)) \quad \in
            \pr \subseteq \mathcal{C}\]

          Now, based on the definition of $\tau$
          \[\tau(a_1, \dots, a_k) = (1 0 0 1 0 0 0 1 \dots 1 0 0 1 0
            0)\] where the first digit is $a_1+\dots+a_k+k-1 =
          b_k$. We can already express the length $k$ of the tuple
          as \[\ell (x) = \sum_{j \leq x}\alpha(x,j)\]. Now considering
          only the digits 1 and noting their position with $b_k$ we
          can define $b(x,i)$ s.t. $b(x,i) = b_i$ in fact

          \[b(x,1) = \mu(j \leq x) \; . \; |\alpha(x,j) - 1|\]

          \[b(x,i+1) = \mu(j \leq x) \; . \; |\alpha(x,j)-1| +
            (b(x,i)+1 \dotdiv j)\]

          This means that $b(x,i)\in \pr$.  Eventually we can write a
          function $a(x,i)$ s.t. $a(x,i) = a_i$.
          \[a(x,1) = b(x,1)\]
          \[a(x,i+1) = (b(x,i+1) \dotdiv b(x,i))-1\]

          We can finally express $\tau^{-1}$ for $x \in \nat$ as
          \[\tau^{-1} = (a(x,1), \dots, a(x, \ell(x)))\] and is
          effective.
        \end{itemize}
    \fi
    
    
       
      }
    \end{enumerate}
  \end{proof}
\end{lemma}

\begin{theorem}
  Let $\mathcal{P}$ the set of all URM programs. 
  Then there exists an effective bijective enumeration of $\mathcal{P}$.
  \[
    \gamma : \mathcal{P} \to \nat
    \]
  \begin{proof}
    Let $\mathcal{F}$ the set of all URM instructions.
    First, we'll prove that there exists
    \[\beta : \mathcal{F} \to \nat \]
    a bijective effective correspondence. The idea is to use the
    enumeration of pairs and triples, sending
    \begin{itemize}
    \item $Z(n)$ instructions to multiples of 4
    \item $S(n)$ instructions to numbers equal $ 1 \mod 4$
    \item $T(m,n)$ instructions to numbers equal $2 \mod 4$
    \item $J(m,n,t)$ instructions to numbers equal $3 \mod 4$
    \end{itemize}
    this means that
    \begin{equation*}
      \begin{cases}
        \beta(Z(n)) = 4*(n-1)\\
        \beta(S(n)) = 4*(n-1) + 1\\
        \beta(T(m,n)) = 4*\pi(m-1, n-1) + 2\\
        \beta(J(m,n,t)) = 4*\nu(m-1, n-1, t-1) + 3
      \end{cases}
    \end{equation*}
    with inverse $\beta^{-1} : \nat \to \mathcal{F}$ such
    that, let $r = rm(4,x)$ and
    $q = qt(4,x)$,
    \[
      \beta^{-1}(x) = \begin{cases}
        Z(q+1) & \mbox{if } r=0 \\
        S(q+1) & \mbox{if } r=1 \\
        T(\pi_1(q)+1, \pi_2(q)+1) & \mbox{if } r=2 \\
        J(\nu_1(q)+1, \nu_2(q)+1, \nu_3(q)+1) & \mbox{if } r=3
      \end{cases}
    \]
    so both $\beta$ and $\beta^{-1}$ are effective. 
    Now
    $\gamma : \mathcal{P} \to \nat$ can be defined as follows:
    if 
    $P = I_1 \dots I_s$, then
    \[\gamma(P) = \tau(\beta(I_1), \dots, \beta(I_s))\]
    with inverse $\gamma^{-1}(x) = P = I_1 \dots I_{l(x)}$, where
    $I_i = \beta ^ {-1} (a (n, i))$.
    Thus, $\gamma$ is bijective because is composition of bijective
    functions. Since $\gamma, \gamma^{-1}$ are effective,
    $\mathcal{P}$ is effectively enumerable.
  \end{proof}
\end{theorem}

\begin{definition}[Gödel number]
  Given $P \in \mathcal{P}$ the value $\gamma(P)$ is called code (or
  Gödel number) of $P$. Usually we'll write $P_n$ to represent
  $\gamma^{-1}(n)$ (In other words, the $n^{th}$ program of the
  enumeration).
\end{definition}

\begin{observation}
  From now on we will consider a fixed
enumeration $\gamma$ of programs, which
determines the meaning of $P_n$. This fixed enumeration can be
defined in another way, but it is absolutely necessary. In other words:
\begin{itemize}
\item given a program $P$ we can compute in an effective way its code
  $\gamma(P)$;
\item given a number it is possible to find the $n^{th}$ program
  $P_n = \gamma^{-1}(n)$.
  
\end{itemize}
\end{observation}

\begin{example}
  Let us consider the program $P$
  \begin{itemize}
  \item[] $T(1,2)$
  \item[] $S(2)$
  \item[] $T(2,1)$
  \end{itemize}
  encoded by
  \begin{itemize}
  \item[] $\beta(T(1,2)) = 4 * \pi(1-1,2-1) + 2 = 4 * \pi(0,1) + 2 = 10$
  \item[] $\beta(S(2)) = 4 * (2-1) + 1 = 5$
  \item[] $\beta(T(2,1)) = 4 * \pi(2-1,1-1) + 2 = 4 * \pi(1,0) + 2 = 6$
  \end{itemize}
  then
      \begin{align*}
        \gamma(P) & = \tau(10,5,6) \\
                  & = p_1^{10} \cdot p_2^5 \cdot p_3^{6+1} - 2 \\
                  & = 2^{10} \cdot 3^5 \cdot 5^7 -2 \\
                  & = 19439999998
      \end{align*}
  % or with an alternative encoding
  % \begin{align*}
  %   \gamma(P) & = \tau(10,5,6) \\
  %             & = 2^{10} + 2^{10+5+1} + 2^{10+5+6+2} - 1 \\
  %             & = 2^{10} + 2^{16} + 2^{23} - 1 \\
  %             & = 8455167
  % \end{align*}

  What does this program compute? $\lambda x . x+1$.

  The program $P^\prime : S(1)$ computes the same function. 
  In this case the encoding is
  \[\beta(S(1)) = 4 * (1-1) + 1 = 1\] and so
  \[\gamma(P^\prime) = \tau(1) = 2^{1+1} - 2 = 2\]
  % with the alternative encoding
  % \[\gamma(P^\prime) = \tau(1) = 2^1 - 1 = 1\]
\end{example}

\begin{example}
  Show what $P_{100} = \gamma^{-1}(100)$ is.

  We observe that % $100 = 1100101 - 1$. 100 is the encoding of a
  % quadruple (program with 4 instructions). So
  % $\tau^{-1}(100) = (a_1 a_2 a_3 a_4)$ and the components can be
  % expressed as before.

  % \[
  %   \begin{split}
  %     a_1 = b_1 = 0 & \to Z(1) \\
  %     a_2 = b_2 - b_1 - 1 = 1 & \to S(1) \\
  %     a_3 = b_3 - b_2 - 1 = 2 & \to T(1,1) \\
  %     a_4 = b_4 - b_3 - 1 = 0 & \to Z(1)
  %   \end{split}
  % \]

  % \[\gamma^{-1}(100) = (Z(1), S(1), T(1,1), Z(1))\]

  % With the alternative encoding:
  \[100 + 2 = 2^1 * 3^1 * 17^1 = p_1^1 \cdot p_2^1 \cdot p_3^1 \cdot
    p_4^0 \cdot p_5^0 \cdot p_6^0 \cdot p_7^1 \]

  hence the program contains $7$ instructions:
  \[
    \begin{split}
      \beta^{-1}(1) & \to \quad S(1) \\
      \beta^{-1}(1) & \to \quad S(1) \\
      \beta^{-1}(0) & \to \quad Z(1) \\
      \beta^{-1}(0) & \to \quad Z(1) \\
      \beta^{-1}(0) & \to \quad Z(1) \\
      \beta^{-1}(0) & \to \quad Z(1) \\
      \beta^{-1}(0) & \to \quad Z(1)
    \end{split}
  \]
\end{example}

Clearly, an enumeration of URM programs induces an enumeration of computable functions

\begin{definition}
  Fixed an effective enumeration
  $\gamma : \mathcal{P} \to \nat$ we define:
  \begin{enumerate}[label=\arabic*.]
  \item $\varphi_n^{(k)}$: the function of k arguments (k-ary
    function) computed by the program $P_n = \gamma^{-1}(n)$ (with
    the notation presently introduced: $\varphi_n^{(k)} = f_{P_n}^{(k)}$)
  \item $W_n^{(k)} = \mbox{dom}(\varphi_n^{(k)}) \subseteq \nat^k $
  \item $E_n^{(k)} = \mbox{cod}(\varphi_n^{(k)}) \subseteq \nat$
  \end{enumerate}

  usually if $k=1$, it is omitted. $\varphi_n = \varphi_n^{(1)}$
\end{definition}

\textbf{Observation:} The function \[
  \begin{split}
    \varphi^{(k)} : & \quad \nat \to \mathcal{C}^{(k)} \\
    & \quad n \mapsto \varphi_n^{(k)}
  \end{split}
\]

is obviously surjective (each computable function is computed by a
program!), and so $\mathcal{C}^{(k)}$ is enumerable:
\[|\mathcal{C}^{(k)}| = |\nat|\]

The existence of a surjective function $\nat \to \mathcal{C}$
follows that $|\mathcal{C}^{(k)}| \leq |\nat|$, but obviously there
exists infinitely many computable functions, for example constants
$\lambda x_1 \dots x_k . c$, and so $|\mathcal{C}^{(k)}| \geq |\nat|$
is also true.

Clearly $\varphi^{(k)} : \nat \to \mathcal{C}^{(k)}$ is not
injective. In fact, for each computable function there are infinitely
many programs that computes it
\[\forall f \in \mathcal{C} \quad \quad |(\varphi^{(k)})^{-1}(f)| =
  |\nat|\] which means
\(\varphi_0^{(k)}, \varphi_1^{(k)}, \varphi_2^{(k)}, \dots\) is an
enumeration of $\mathcal{C}$ with infinitely many repetitions. An
enumeration without repetitions can be defined as:
\begin{itemize}
\item[] $\chi(0) = 0$
\item[]
  $\chi(n+1) = \mu z \; . \; (\varphi_z \notin \{\varphi_{\chi(0)},
  \dots, \varphi_{\chi(n)}\})$
\end{itemize}
which rises the enumeration
$\varphi_{\chi(0)}, \varphi_{\chi(1)}, \varphi_{\chi(2)}, \dots$

But this enumeration is highly ineffective.

(Can be proved that there exists an enumeration
$h : \nat \to \nat$ which is total and computable s.t.
$\varphi_{h(0)}, \varphi_{h(1)}, \varphi_{h(2)}, \dots$ is an
enumeration without repetitions \cite{firedberg:1958}. However, an
enumeration with repetitions is sufficient for us).

\begin{theorem}[$|\mathcal{C}| = |\nat|$]
  The class $\mathcal{C}$ of computable functions is enumerable.
  \begin{proof}
    \[ \mathcal{C} = \bigcup_{k \geq 1}\mathcal{C}^{(k)} \] Since the
    union of enumerable sets is enumerable, $\mathcal{C}$ is
    enumerable.
  \end{proof}
\end{theorem}

\textbf{Note:} from now on we will implicitly use the enumeration of
programs $\gamma$. The meaning of
$\varphi_n^{(k)}, W_n^{(k)}, E_n^{(k)}$ is fixed and determined
starting from $\gamma$.
% \textbf{Observation:}m

% In particular the pairs in $\nat^2$ can be encoded as
% $\pi(x,y) = 2^x(2y+1)-1$ which is computable. The inverse is
% $\pi^{-1}(n) = (\pi_1(n), \pi_2(n))$

% The triple instead is a pair of a pair and an element.
% $ \upsilon (x,y,z) = \pi (x, \pi(y,z))$. The inverse is also obtained
% from the inverse of the first.

% For lists we need an encoding
% $\tau . \bigcup_{K \geq 1} \nat^k \to \nat$ we exploit the
% uniqueness of the prime numbers:
% $ \tau(a_1,\dots,a_k) = \pi_{i=1}^k p_i^{a_i}$ where $ p_i $ is the
% $i$-th prime number. \\
% This, however, leads us to lose any zeros, since
% the encodings for (1,1) and (1,1,0) would be the same because the
% exponential function in 0 = 1.

% So we use something that works:
% $ (\pi_{i=1}^{k-1} p_i^{a_i}) \times p_k^{a_k+1} - 2$. For decoding we
% can proceed as usual, but bounded minimalisation can be used.\\ Hint:
% $ max \{z \leq x . P(z)\} x - min\{\delta \leq x . P(x-\delta)\}$

% But I do not just want the length, I also need the list of items. I
% need a function:
% \begin{equation*}
%   a(x,i) = \begin{cases}
%     (x+2)_i   & 1 \leq i \leq k-1 \\
%     (x+2)_k-1 & i = k
%   \end{cases}
% \end{equation*}
% And this is the inverse function of $\tau$. And these functions are
% computable recursive primitives.

% To compute instructions: Let us take $\mathcal{F}$ set of URM
% instructions, $\mathcal{P}$ URM programs. Let us take function
% $\beta:\mathcal{F}\to\nat$
% \begin{itemize}
% \item $\beta(z(n)) = 4 \times (n-1)$;
% \item $\beta(s(n)) = 4 \times (n-1)+1$;
% \item $\beta(t(m,n)) = 4 \times \pi(m-1,n-1)+2$;
% \item $\beta(j(m,n,t)) = 4 \times \upsilon(m-1,n-1,t-1)+3$;
% \end{itemize}

% The decoding of this monstrosity is obtained from the previous inverse
% functions applied on the basis of the dimension and the rest of the
% number.

% \textbf{Example:} P = T (1,2); S (2); T (2,1) =
% $\tau(10,5,6) = 2^{10} 3^5 5^{6+1} -2$

% The two-way function between programs and numbers has been demonstrated.

% \begin{notation} given an effective enumeration (in our case the one
%   defined previously) we say that $\gamma(P)$ is the code of P (also
%   called G\"{o} of the number), if $\gamma(P) = n$ then $P$ is the
%   $n$-th program.
% \end{notation}

% \begin{notation} $\Phi_n^{(k)}: \nat^k\to\nat$ function of
%   $K$ arguments computed by program $n$, that is, by program
%   $\gamma^{-1}(n)$, if $k = 1$ is is omitted. The function domain is
%   $ W_n^{k} = dom(\Phi_k^{(k)}) = \{\vec{x} |
%   \Phi_k^{(k)})(\vec{x})\downarrow \} \subseteq \nat^k$

%   The function codomain:
%   $ E^{(k)}_n = \Phi_k^{(k)}) = \{\vec{x} | \vec{x} \in W_n^{(k)} \}
%   \subseteq \nat^k$
% \end{notation}

% So for example the program $\Phi_{19439999998} = x+1$,
% $W_{19439999998} = \nat$, $E_{19439999998} = \nat \setminus \{0\}$

% Now we have an enumeration of all the unary computable functions that is \{$\Phi_n$ . $n \in \nat$ \} each function with infinite repetitions.

% Remember we have indicated the computable functions of $k$ arguments as $\mathcal{C} ^ {(k)}$, where $| \mathcal{C} ^ {(k)} | \leq (=) | \nat |$ and therefore being $\mathcal{C} = \bigcup_{K \geq 1} \mathcal{C} ^ {(k)}$ a union of countable sets it is still countable.
