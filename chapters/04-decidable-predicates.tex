\chapter{Decidable Predicates}

In mathematics we often want to establish \textbf{properties}, for example \emph{$m$ is a divisor of $n$}. 
We can define it using a relation such as
\begin{align*}
    div & \subseteq \nat \times \nat\\
    div & = \{(m,k \cdot m) \mid m \in \nat, k \in \nat \}
\end{align*}

We can also characterise $div$ as 
\begin{align*}
    div & : \nat \times \nat \rightarrow \{true, false\}\\
    div & = \begin{cases}
                true & \mbox{if $m$ is a divisor of $n$}\\
                false & \mbox{otherwise}
            \end{cases}
\end{align*}

In the field of calculability theory we talk about \textbf{predicates}.

A \textbf{k-ary predicate} on $\nat $ indicated with $Q(x_1,\dots,x_k)$ is a property that can be true or false, formally we can see it as

\begin{itemize}
\item a function $Q: \nat^k\rightarrow \{true,false\}$
\item a set $Q \subseteq \nat^k$
\end{itemize}

we write $Q(x_1,\dots,x_k)$ to denote $(x_1,\dots,x_k) \in Q$ or $Q(x_1,\dots,x_k) = true$

When is $Q$ computable? When there exists a URM such that given a k-tuple $(x_1,\dots,x_k)$ in input, it returns $true$ if $Q(x_1,\dots,x_k)$ and $false$ otherwise. 

To represent $true$ and $false$ we conventionally use values $1$ and $0$.

\begin{definition}
    A predicate $Q \subseteq \nat^k$ is said to be \textbf{decidable} if its \textbf{characteristic function}
\begin{equation*}
\mathcal{X}_Q(x_1,\dots,x_k) = \begin{cases}
1 & $ if $ Q(x_1,\dots,x_k) \\
0 & $ otherwise $
\end{cases}
\end{equation*}

is (URM) computable.
\end{definition}


\begin{remark}
    $\mathcal{X}_Q$ is a \textbf{total} function (dealing with decidability of predicates, involves only total functions).
\end{remark}


\section {Examples of decidable predicates}
\begin{enumerate}
    \item Equality\\
    $ Q \subseteq \nat^2 $, $ Q(x,y) \equiv x = y $

    The characteristic function
    \begin{equation*}
    \mathcal{X}_Q(x,y) = \begin{cases}
    1 & $ if $ x = y  \\
    0 & $ otherwise $
    \end{cases}
    \end{equation*}
    is computed, for instance, by the program
    \begin{quote}
    \begin{tabular}{lll}
    $I_1$ & J(1,2,3)  \\
    $I_2$ & J(1,1,4)       \\
    $I_3$ & S(3) \\
    $I_4$ & T(3,1)
    \end{tabular}
    \end{quote}

    \item $ Q(x) \equiv x \text{ is even} $

    \begin{quote}
    \begin{tabular}{lll}
    $I_1$ & J(1,2,6)   \\
    $I_2$ & S(2)        \\
    $I_3$ & J(1,2,7)   \\
    $I_4$ & S(2)        \\
    $I_5$ & J(1,1,1) \\
    $I_6$ & S(3)        \\
    $I_7$ & T(3,1)
    \end{tabular}
    \end{quote}
    
    \begin{tabu}{|c|c|c|}
    \hline
    x & k & r \\
    \hline
    \end{tabu} in memory where k is a growing index and r is the result.
    
    \item $Q(x,y) \equiv x \leq y$\\
    We can either increment both $x$ and $y$ until $x+k=y$, so that $x\leq y$, or until $y+k=x$, so that $x>y$ (not equal for the order of comparisons).
    
    \begin{quote}
    \begin{tabular}{lll}
    & T(1,3)      &        \\
    & T(2,4)      &        \\
    LOOP: & J(2,3,SI)   & \comment{x+k=y?} \\
    & J(1,4,NO)   & \comment{y+k=x?} \\
    & S(3)        &        \\
    & S(4)        &        \\
    & J(1,1,LOOP) &        \\
    SI:   & S(5)        &        \\
    NO:   & T(5,1)      &
    \end{tabular}
    \end{quote}
    
    Memory: $\begin{tabu}{|c|c|c|c|c|}
    \hline
    x & y & x+k & y+k & r \\
    \hline
    \end{tabu}$ where $r$ is the result.
    
    Another approach is to increment a register starting from 0. If we reach $x$ first then $x \leq y$, otherwise $x > y$.
    
    \begin{quote}
    \begin{tabular}{lll}            
    LOOP: & J(1,3,SI)   & \\
    & J(2,3,NO)   & \\
    & S(3)        & \\
    & J(1,1,LOOP) & \\
    SI:   & S(4)        & \\
    NO:   & T(4,1)      &
    \end{tabular}
    \end{quote}
    
    $\begin{tabu}{|c|c|c|c|}
    \hline
    x+k & y & k & r \\
    \hline
    \end{tabu}$ where $r$ is the result.
    
    Example of $div(x,y)$, suppose $x \not= 0$:
    
    \begin{quote}
    \begin{tabular}{lll}            
    LOOP: & J(2,3,SI)   &                                   \\
    & Z(4)        & \comment{sum $x$ to $R_2$}         \\
    ADDX: & J(1,4,LOOP) &                                   \\
    & J(2,3,NO)   & \comment{if for $h<x$  $kx+h=y$ then no!} \\
    & S(3)        &                                   \\
    & S(4)        &                                   \\
    & J(1,1,ADDX) &                                   \\
    SI:   & S(5)        &                                   \\
    NO:   & T(5,1)      &
    \end{tabular}
    \end{quote}
    
    $\begin{tabu}{|c|c|c|c|c|}
    \hline
    x & y & kx+h & h & r \\
    \hline
    \end{tabu}$ where $r$ is the result.
\end{enumerate}


