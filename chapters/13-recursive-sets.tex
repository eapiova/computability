\chapter{Recursive sets}
In previous chapters, we saw many computable functions and decidable problems, but
only in few cases we gave examples of the large class of
non-computable functions and undecidable predicates. For this reason,
we want to start a mathematical study of
\begin{itemize}
  \item classes of undecidable predicates
  \item techniques to prove the undecidability of some predicates
\end{itemize}
In this way we can highlight the limits of computers abilities and give a
structure to problems classes (the majority of interesting
predicates are undecidable).

We'll focus on \emph{numerical sets} $X \subseteq \nat$ and try to find
an answer to the problem ``$x \in X$?''. We'll get
\begin{itemize}
\item recursive sets
\item recursively enumerable sets
\end{itemize}

\section{Recursive sets}
\begin{definition}
  A set $A \subseteq \nat$ is \emph{recursive} if its characteristic
  function
  \begin{gather*}
    \chi_A : \nattonat \\
    \chi_A(x) = \begin{cases}
      1 & x\in A \\
      0 & x \notin A
    \end{cases}
  \end{gather*}
  is computable.
\end{definition}
In other words, if the predicate ``$x \in A$'' is decidable.

\begin{observation}
  \begin{itemize}
    \item if $\chi_A \in \pr$ we'll say that $A$ is \emph{primitively}
      recursive.
    \item the notion can be extended to subsets of $\nat^k$, but we'll
      stick to $\nat$ subsets, since every $\nat^k$ subset can be encoded
      into $\nat$
    \end{itemize}
\end{observation}


\begin{example}
  These are recursive:
\begin{enumerate}[label=(\alph*)]
\item $\nat$, since $\chi_\nat = \mathbf{1}$ is computable;
\item $\varnothing$, because $\chi_\varnothing = \mathbf{0}$ is computable;
\item prime numbers $\mathbb{P}$, since
  \[
    Pr(x) = \begin{cases}
      1 & \mbox{if $x$ is prime} \\
      0 & \mbox{otherwise}
    \end{cases}
  \]
  is computable;
\item each and every finite set. In fact, given $A \subset \nat$ with
  $|A| < \infty$, $A = \{x_1, x_2, \dots, x_n\}$, we have that
  \[
    \chi_A(x) = \overline{sg}\left( \prod_{i=1}^n|x - x_i| \right)
  \]
  is computable.
\end{enumerate}
\end{example}


On the other hand, these are definitely not recursive:
\begin{enumerate}[label=(\alph*)]
\item $K = \left\{ x \; | \; x \in W_x \right\} $, since
  \[
    \chi_{K}(x) = \begin{cases}
      1 & x \in W_x \\
      0 & x \notin W_x
    \end{cases}
  \]
  is not computable;
\item $\left\{ x \; | \; \varphi_x \mbox{ total} \right\} $
\end{enumerate}

\begin{observation}
  If $A,B \subseteq \nat$ are recursive, then
\begin{enumerate}[label=\arabic*)]
\item $\overline{A} = \nat - A$
\item $A \cap B$
\item $A \cup B$
\end{enumerate}
are recursive.
\end{observation}


\subsection{Reduction}
Reduction is a process used to study decidability
problems. It formalizes the intuition of a problem
$\mathcal{A}$ being ``easier'' then another one, $\mathcal{B}$.

\newcommand{\red}{\ensuremath{\leq_m}}
\begin{definition}
  Let $A,B \subseteq \nat$. We say that the problem $x \in A$
  \emph{reduces} to the problem $x \in B$ ($A$ reduces to $B$), 
  written $A \red B$ if there exists
  $f : \nattonat$ computable and total such that, for every $x \in \nat$
  \[x \in A \quad  \Leftrightarrow \quad f(x) \in B\]
\end{definition}
In this case, $f$ is the \emph{reduction function}.

\begin{observation}
  Let $A,B \subseteq \nat$ such that $A \red B$ then
\begin{enumerate}[label=\arabic*]
\item if $B$ is recursive, then $A$ is recursive
\item of $A$ is not recursive, then $B$ is not recursive
\end{enumerate}

\begin{proof}
Just observe that $\chi_A = \chi_B \circ f$
\end{proof}
\end{observation}

We know that $K = \{ x \mid x \in W_x \}$ is not recursive. Let's see how
the recursiveness of other sets can be proven by reduction to this
set which we know for certain is not recursive.
\begin{example}
  $K \red T = \{x \mid \varphi_x $ total$ \}$
  \begin{proof}
    we need to prove that there exists $s : \nattonat$ computable and total
    such that $x \in k \Leftrightarrow s(x) \in T$. In other words
    \[ x\in W_x \Leftrightarrow  \varphi_{f(x)} \mbox{ is total} \]
    To do so, we can define
    \[
      g(x,y) = \begin{cases}
        1 & x\in W_x \\
        \uparrow & \mbox{otherwise}
      \end{cases}
    \]
    which is computable. This fact is easily proven by rewriting it as
    \[
      g(x,y) = \mathbf{1}(\varphi_x(x)) = \mathbf{1}(\univ(x,x))
    \]
    then, by \emph{smn} theorem we have that there exists $s: \nattonat$
    computable and total such that 
    \[\varphi_{s(x)}(y) = g(x,y)\] and 
    \[x \in K \Rightarrow x \in W_x \Rightarrow \forall y\
      \varphi_{s(x)}(y) = g(x,y) = 1 \Rightarrow \varphi_{s(x)} \mbox{
        total } \Rightarrow s(x) \in T\]
    \[x \notin K \Rightarrow x \notin W_x \Rightarrow \forall y\
      \varphi_{s(x)}(y) = g(x,y) \uparrow \Rightarrow \varphi_{s(x)}
      \mbox{ not total } \Rightarrow s(x) \notin T\]
  \end{proof}
\end{example}

\begin{example}[Input problem]
  For every $ n \in \nat$
  \[
    A_n = \{x  \mid \varphi_x (n) \downarrow\}
  \]
  is not recursive.

  \begin{proof}
    We will prove that $K \leq A_n$. We have to define a function $f$
    s.t.
    \begin{gather*}
      x \in K \Leftrightarrow f(x) \in A_n \\
      x \in W_x \Leftrightarrow \varphi_{f(x)}(n) \downarrow
    \end{gather*}
    Define
    \begin{align*}
      g(x,y) &= \begin{cases}
        1 & x \in W_x \\
        \uparrow & \mbox{otherwise}
      \end{cases} \\
      &= \mathbf{1}(\univ(x,x))
    \end{align*}
    which is computable, and therefore by the \emph{smn} theorem, there exists
    $f: \nattonat$ computable and total such that
    $g(x,y) = \varphi_{f(x)}(y)$, and
    \begin{gather*}
      x \in k \Rightarrow f(x) \in A_n \\
      x \notin  k \Rightarrow f(x) \notin A_n
    \end{gather*}
  \end{proof}
\end{example}

\begin{example}[The output problem]
  For every $ n \in \nat$, $B_n\{ x \mid n \in E_x\}$ is not recursive
  \begin{proof}
    We show that $K \leq_m B_n$
    \[
      \begin{split}
        g(x,y) &= \begin{cases}
          n & x \in W_x \\
          \uparrow & \mbox{otherwise}
        \end{cases} \\
        &= n \cdot \mathbf{1}(\univ(x,x))
      \end{split}
    \]
    computable, by \emph{smn} theorem, there exists a function
    $s : \nattonat$ such that
    \[
      \forall x,y \quad g(x,y) = \varphi_{s(x)}(y)
    \]
  moreover
  \begin{gather*}
    x \in k \Rightarrow s(x) \in B_n \\
    x \notin k \Rightarrow s(x) \notin B_n
  \end{gather*}
\end{proof}
\end{example}

\begin{observation}
Let $A,B \subseteq \nat$ with $A \leq_m B$
through an injective reduction function $f : \nattonat$ (total and computable). 
One could think that, since $f^{-1}$ is computable, then
also $B \leq_m A$. This does not happen, since $f^{-1}$ is
not total and so reduces $A$ to a ``subproblem'' of $B$ even though
it has no complexity relationships with $B$.
\end{observation}
