\chapter{URM computability}

\section {Which model?}

To give a formal notion of computability we must choose a concrete model of computation that induces a class of algorithms and therefore of computable functions. 
Despite the fact that we focus on an abstract ideal model, there are still a lot of possibilities. Many models have been considered in the literature:

\begin{enumerate}
\item Turing machine (Turing, 1936)
\item $\lambda$-calculus (Church, 1930)
\item Partial recursive functions (Godel-Kleene 1930)
\item Canonical deductive systems (Post, 1943)
\item Markov systems (Markov, 1951)
\item Unlimited register machine (URM) (Shepherdson - Sturgis, 1963)
\end{enumerate}

In principle, each computational model determines a class of computable functions.
We may be concerned thinking that the developed theory is valid only for a specific one
model. In fact, it can be verified that the class of computable functions for all
models cited (for all the models "sufficiently expressive" considered
in literature) is always the same. This leads to the so-called Church-Turing
thesis:

\textbf{Church-Turing thesis}: A function is computable by an
effective procedure (i.e., in a finitary computational model, obeying the conditions (a)-(e) from the chapter before) if and only if it is computable
by a Turing machine.


This means that the notion of ``computable function'' is robust (i.e. independent of the specific computational model), and we can choose our favorite one for developing our theory.

\begin{remark}
  The \emph{Church-Turing thesis} is called a thesis and not a theorem due
  of its informal nature. 
  It cannot be proved w.r.t effective procedures, but is supported only by evidence: 
  several computational models have been considered and all respect the thesis 
  (e.g. Yuri Gurevich, argues that it should be proved on the basis of a formal
  axiomatization of conditions (a) - (e)).
\end{remark}

Sometimes we resort to the Church-Turing thesis to shorten the proof that a certain entity is computable, however it can only be used when it is not strictly necessary, i.e. when it could be replaced by a formal proof
(however, this could hide the intuitive idea under a bunch of technical details).

\section{URM (Unlimited register machine)}

We will formalise the notion of \textbf{computable function} by using an \textbf{abstract machine} 
called \textbf{URM-machine} (Unlimited Register Machine), 
which is an abstraction of a computer based on the Von Neumann's model. It is characterized by

\begin{itemize}
\item \textbf{unbounded memory} that consists of a infinite sequence of \textbf{registers}, each of which can store a  natural number


  $\begin{tabu}{|c|c|c|c|c|}
    \hline
    R_1 & R_2 & \dots & R_n & \dots \\
    \hline
    r_1 & r_2 & \dots & r_n & \dots \\
    \hline
  \end{tabu}$

  the $n$-th register is indicated with $R_n$, its content with $r_n$

  the sequence $(r_1, r_2,\dots, r_n,\dots) \in \nat^\omega$ is called
  \textbf{configuration} of the URM;

\item a \textbf{computing agent} capable of executing an URM program;

\item an execution of a \textbf{URM program}, i.e. a finite sequence of instructions
  $I_1, I_2, \dots, I_s$ that can ``locally'' alter the configuration
  of the URM.
\end{itemize}


Program instructions can be the following

\begin{itemize}

\item \textbf{zero} $Z(n)$ sets the content of the register $R_n$ to zero: $r_n \leftarrow 0$;

\item \textbf{successor} $S(n)$ increments the content of the $R_n$ register by 1: $r_n \leftarrow r_n+1$;

\item \textbf{transfer} $T(m,n)$ transfers the content of the register $R_m$ in the register $R_n$, $R_m$ stays untouched: $r_n\leftarrow r_m$.
\end{itemize}
These are called arithmetic functions, where the instruction to be executed in the next step
 follows the current instruction in the program.
Then there is another instruction
\begin{itemize}
\item \textbf{conditional jump} $J(m,n,t)$ compares the content of the registers $R_m$ and $R_n$
  \begin{itemize}
  \item if $r_m = r_n$ it jumps to the $t$-th instruction;
  \item otherwise, it continues with the next instruction.
  \end{itemize}
\end{itemize}


\begin{example}
  An example of program is the following:
  \begin{quote}
    \begin{tabular}{llr}
      $I_1$: & J(2,3,5) &                       \\
      $I_2$: & S(1)     &                       \\
      $I_3$: & S(3)     &                       \\
      $I_4$: & J(1,1,1) &  \comment{unconditional jump}
    \end{tabular}
  \end{quote}

  Disregard what this program computes for the moment. The computation starting from the configuration below is:

  \begin{center}
    $\begin{tabu}{|c|c|c|c|}
      \hline
      R_1 & R_2 & R_3 & \dots \\
      \hline
      1   & 2   & 0   & \dots \\
      \hline
    \end{tabu}
    %
    \xrightarrow{I_1, I_2}
    %
    \begin{tabu}{|c|c|c|c|}
      \hline
      R_1 & R_2 & R_3 & \dots \\
      \hline
      2   & 2   & 0   & \dots \\
      \hline
    \end{tabu}
    %
    \xrightarrow{I_3}
    %
    \begin{tabu}{|c|c|c|c|}
      \hline
      R_1 & R_2 & R_3 & \dots \\
      \hline
      2   & 2   & 1   & \dots \\
      \hline
    \end{tabu}
    %
    \xrightarrow{I_4, I_1, I_2}
    %
    \begin{tabu}{|c|c|c|c|}
      \hline
      R_1 & R_2 & R_3 & \dots \\
      \hline
      3   & 2   & 1   & \dots \\
      \hline
    \end{tabu}
    %
    \xrightarrow{I_3}
    %
    \begin{tabu}{|c|c|c|c|}
      \hline
      R_1 & R_2 & R_3 & \dots \\
      \hline
      3   & 2   & 2   & \dots \\
      \hline
    \end{tabu}
    \xrightarrow{I_4, I_1, I_5}
    $
  \end{center}
\end{example}


The \textbf{state} of the URM machine in which it executes a program $P = I_1 \dots I_s$ 
is given by a pair $\langle c, t \rangle$ that consists of a

\begin{itemize}
\item \emph{register configuration} $c$\\
  a total function $c : \nat \to \nat$ such that $c(n)$ is the content
  of register $R_n$;

\item \emph{program counter} $t$, i.e., index of the current instruction.
\end{itemize}

An \emph{operational semantics} can easily be defined via a set of deduction rules 
axiomatising the state transitions  $\langle c, t \rangle \rightarrow \langle c', t' \rangle$. 
However we do not need this degree of formality, and we will rely on an informal description of the program execution.


\begin{remark}
  A computation might \textbf{not terminate}! Consider for instance the program

  \begin{quote}
    \begin{tabular}{ll}
      $I_1$: & S(1)     \\
      $I_2$: & J(1,1,1)
    \end{tabular}
  \end{quote}

  Then the computation will not terminate. For instance
  \begin{center}
    $\begin{tabu}{|c|c|c|c|}
      \hline
      R_1 & R_2 & R_3 & \dots \\
      \hline
      0  & 0   & 0   & \dots \\
      \hline
    \end{tabu}
    %
    \xrightarrow{I_1, I_2}
    %
    \begin{tabu}{|c|c|c|c|}
      \hline
      R_1 & R_2 & R_3 & \dots \\
      \hline
      1   & 0   & 0   & \dots \\
      \hline
    \end{tabu}
    %
    \xrightarrow{I_1, I_2}
    %
    \begin{tabu}{|c|c|c|c|}
      \hline
      R_1 & R_2 & R_3 & \dots \\
      \hline
      2   & 0  & 0   & \dots \\
      \hline
    \end{tabu}
    %
    \xrightarrow{\ldots}
    %
    $
  \end{center}
\end{remark}


\begin{notation}
  Let $P$ be an URM program, and $(a_1,a_2,a_3,\dots) \in \nat^\omega$ a sequence
  of natural numbers. We indicate the computation of $P$ starting from the
  initial configuration by $P(a_1,a_2,\dots)$:

  \begin{center}
    $\begin{tabu}{|c|c|c|c|}
      \hline
      R_1 & R_2 & R_3 & \dots \\
      \hline
      a_1 & a_2 & a_3 & \dots \\
      \hline
    \end{tabu}$
  \end{center}

  and

  \begin{itemize}
  \item $P(a_1,a_2,\dots) \downarrow$ if the computation \textbf{halts}.
  \item $P(a_1,a_2,\dots) \uparrow$ if the computation \textbf{never
      halts} (i.e. it \textbf{diverges}).
  \end{itemize}


  We will work on computations that start from an initial configuration
  where only a \textbf{finite number of registers contain a non-zero value} for
  the majority of the time (almost always for obvious reasons of input
  finiteness). Hence; given $a_1,a_2,\dots,a_k \in \nat$ we will write

  \begin{center}
    $P(a_1,\dots,a_k)$ for 
    $P(a_1,\dots,a_k,0,\dots,0)$
  \end{center}

  The notation extend to $P(a_1,\dots,a_k)\downarrow$ or
  $P(a_1,\dots,a_k)\uparrow$.
\end{notation}

\section{URM-computable functions}

Let $f : \nat^k \rightarrow \nat$ be a partial function. What does it mean for  $f$ to be computable by an URM machine?

Intuitively, it means that there exists a program $P$ such that for each $(a_1,\dots,a_k) \in \nat^k$, 
$P(a_1,\dots,a_k)$ computes the value of $f$, i.e. when $(a_1,\dots,a_k) \in \dom{f}$, 
$P$ terminates and outputs $f(a_1, \ldots, a_k)$. 
However, $P$ does not terminate if $(a_1,\dots,a_k) \not\in \dom{f}$.

A doubt could be about where the output is stored. 
We conventionally decide that the output will be in the first register $ R_1 $ (at the end of the computation, any register other than the first register contains irrelevant data). 
For this reason we introduce the following notation

\begin{notation}
  Let $P$ be a program and $(a_1,\dots,a_k) \in \nat^k$, we write
  $P(a_1,\dots,a_k)\downarrow a$ if $P(a_1,\dots,a_k) \downarrow$ and
  the final configuration contains $a$ in $R_1$
\end{notation}

\begin{definition}[URM-computable function]
  A function $f:\nat^k\rightarrow\nat$ is said to be
  \textbf{URM-computable} if there exists a URM program $P$ such that for all
  $(a_1,\dots,a_k) \in \nat^k$ and $ a\in\nat$,
  $P(a_1,\dots,a_k)\downarrow$ if and only if $(a_1,\dots,a_k)\in dom(f)$ and
  $f(a_1,\dots,a_k) = a$. 
  
  In this case we say that $P$ computes $f$.

  We denote by $\mathcal{C}$ the class of all URM-computable
  functions and by $\mathcal{C}^{(k)}$ the class of the k-ary
  URM-computable functions.
  Therefore we have
  $\mathcal{C} = \bigcup_{k\geq 1} \mathcal{C}^{(k)}$
\end{definition}

\section{Examples of URM-computable functions}

\begin{enumerate}
\item $f:\nat^2 \rightarrow \nat$\\
  $ f(x,y) = x+y$

  \begin{quote}
    \begin{tabular}{lll}
      $I_1$: & J(2,3,5) &                    \\
      $I_2$: & S(1)     &                    \\
      $I_3$: & S(3)     &                    \\
      $I_4$: & J(1,1,1) &  \comment{unconditional jump}
    \end{tabular}
  \end{quote}

  \begin{center}
    $\begin{tabu}{|c|c|c|c|}
      \hline
      R_1 & R_2 & R_3 & \dots \\
      \hline
      x   & y   & 0   & \dots \\
      \hline
    \end{tabu}$
  \end{center}

  \emph{Idea}: Increment $R_1$ and $R_3$ until $R_2$ and $R_3$ contain
  the same value. This results in adding to $R_1$ the content of
  $R_2$.

\item $f:\nat \rightarrow \nat$\\
  $f(x) = x\dot{-}1 = \begin{cases} 0 & x=0 \\ x-1 & x>0 \end{cases}$

  \begin{center}
    $\begin{tabu}{|c|c|c|c|}
      \hline
      R_1 & R_2 & R_3 & \dots \\
      \hline
      x   & 0   & 0   & \dots \\
      \hline
    \end{tabu}$
  \end{center}

  \emph{Idea}: if $x=0$ it trivially terminates; if $x>0$ keep a value $k-1$ in
  $R_2$ and $k$ in $R_5$, with $k>1$ ascending until $R_3=x$, at that
  point $R_2 = x-1$.

  Here's the program

  \begin{quote}
    \begin{tabular}{lll}
      $I_1$: & J(1,3,8) \\
      $I_2$: & S(3)     \\
      $I_3$: & J(1,3,7) \\
      $I_4$: & S(2)     \\
      $I_5$: & S(3)     \\
      $I_6$: & J(1,1,3) \\
      $I_7$: & T(2,1)   \\
    \end{tabular}
  \end{quote}


\item $f:\nat \rightarrow \nat$\\
  $f(x) = \begin{cases}
    \dfrac{x}{2} & \mbox{if $x$ even}\\
    \uparrow      & \mbox{otherwise}
  \end{cases}$

  \emph{Idea:} Store and increasing even number in $R_2$ and store its' half in
  $R_3$.
  \begin{center}
    $\begin{tabu}{|c|c|c|c|}
      \hline
      R_1 & R_2 & R_3 & \dots \\
      \hline
      x   &  2k  & k   & \dots \\
      \hline
    \end{tabu}$
  \end{center}

  \begin{quote}
    \begin{tabular}{lll}
      $I_1$: & J(1,2,6) \\
      $I_2$: & S(2)     \\
      $I_3$: & S(2)     \\
      $I_4$: & S(3)     \\
      $I_5$: & J(1,1,1) \\
      $I_6$: & T(3,1)   \\
    \end{tabular}
  \end{quote}

\end{enumerate}

\section {Function computed by a program}
Given a program $P$, for some fixed number $k \geq 1$ of parameters, there exists a unique \textbf{function computed by $P$} that we denote by $f_P^{(k)} : \nat^k \to \nat$ defined by:

\begin{equation*}
  f_P^{(k)}(a_1, \dots, a_k) = \begin{cases}
    a        & $ if $ P(a_1, \dots, a_k) \downarrow a  \quad \\
    \uparrow & $ if $ P(a_1, \dots, a_k) \uparrow
  \end{cases}
\end{equation*}

\begin{remark}
  The same function can be computed by different programs, for the following two reasons

  \begin{itemize}
  \item we can add useless instructions to a program (dead code, $T(n,n)$, ...)

  \item the same function can be computed via different algorithms
    (e.g., for sorting we have quicksort, mergesort, heapsort, etc.)
  \end{itemize}

  A function can be computed either by no program or by infinitely many programs.
\end{remark}

\section {Exercises}

\subsection{Reduced URM}

Let URM be reduced without transfer instruction $T(m, n)$. We indicate the class
of functions that can be computed with the reduced machine $ \mathcal{C}^- $ and
we compare it with $ \mathcal{C} $. Obviously $ \mathcal{C}^- \subseteq \mathcal{C}
$. Let's see if $ \mathcal{C} \subseteq \mathcal{C}^-$? \\
(The answer is yes because $T(m, n)$ can be replaced with other instructions.)

\begin{quote}
  \begin{tabular}{lll}
    $I_t$: & T(m,n) \\
  \end{tabular}
\end{quote}

can be replaced with a subroutine at the right place

\begin{quote}
  \begin{tabular}{lll}
    $I_{t'}$:   & J(m,n,t+1)  \\
    $I_{t'+1}$: & Z(n)        \\
    $I_{t'+2}$: & J(m,n,t+1)  \\
    $I_{t'+3}$: & S(n)        \\
    $I_{t'+4}$: & J(1,1,t'+2) \\
  \end{tabular}
\end{quote}

But let's prove it: ($ \mathcal{C} \subseteq \mathcal{C}' $), $ f \in \mathcal{C} $, $ f: \nat^K \rightarrow \nat $ There is an URM $P$ s.t. $ f_P^{(K)}  = f$ the program $P$ can be transformed into $P ^R $ of the reduced URM machine s.t. $ f_{P^R}^{(K)}  = f_{P}^{(K)}$.

There is a demonstration by induction. We show that $P$ can be transformed into $P' $ s.t. $ f_{P'}^{(K)}  = f_{P}^{(K)} $ by induction on $h$ = number of transfer instructions $T$ in $P$.

$h = 0$ trivial.

$h \rightarrow  h + 1$:

$P$ contains $h + 1 \quad T$ instructions.

Transform $P$ into $P''$ where all instructions from 1 to \textit{l} are as same as before, while instead of $T$ we put a jump $J(1,1, SUB)$ where the subroutine is written before. We assume that if $P$ ends, it does so at instruction $l + 1$. And at position $l + 1$ we put a $J (1,1, END)$. After these replacements, we have $h$ instructions $T$ and therefore we can say that we have the program of the reduced URM s.t. the computed function is the same by inductive assumption.

\subsection{Exercises}

\subsection{URM with swap instruction}
Let $URM^S $ be the model obtained by removing transfer function and inserting the swap function $ T_S(m,n) $, what relationship is there between the class of this model and the other?

Proof that $ \mathcal{C}^S \subseteq \mathcal{C} $ The exchange $T (m, n)$ is equivalent to:

\begin{lstlisting}
  T(n,i)
  T(m,n)
  T(i,m)
\end{lstlisting}

Formalization:

Let $ f \in \mathcal{C}^S f:\nat\rightarrow \nat $. There exists $P$  $URM^S $ s.t. $ f_P^{(K)} = f $. Let's proceed by induction on the number of transfer functions $h$.

If $h = 0$ the program is already URM. therefore $ P' = P $

Otherwise if there is at least one transfer instruction we show that the inductive case is valid $ h \rightarrow h+1 $, if $P$ ends it does so in $l + 1$, the program has $h + 1$ exchange instructions.

Let $i$ be a register not used by $P$ found by inspecting the program. We replace at step $t \quad J (1,1, SUB)$ and add a subroutine as the one above, thus creating $ P'' $, at the end of the subroutine we return to the starting point with $J (1,1, t + 1)$. By inductive hypothesis there is $ P' $ URM s.t. $ f_{P'}^{(K)} = f_{P''}^{(K)} = f_{P}^{(K)}$. \\
\textbf{But all of this is wrong!}

Why is it wrong? Because with the replacement that we made we have 1 transfer instruction, but $n-1$ exchange instructions.

Let's prove something stronger: Given $P$ program that uses both URM instructions and  $URM^S $ instructions, there is $ P'' $ program that uses $URM$ instructions s.t. $ f_{P}^{(K)} = f_{P'}^{(K)} $.

The proof procedure is the same but we are demonstrating something stronger, the inductive case is now correct. This proves that $ \mathcal{C}^S \subseteq \mathcal{C} $

To show $  \mathcal{C} \subseteq \mathcal{C}^S $ we know that we have shown that $ \mathcal{C} \subseteq \mathcal{C}^R $ and therefore given that $ \mathcal{C}^R \subseteq \mathcal{C}^S $ is demonstrated by transitivity.

So we can say that $ \mathcal{C}^S = \mathcal{C} $

\subsection{URM without jump instructions}

URM$ ^{nj} $ is a model without jumps, meaning without $J(m,n,t)$ instructions.

Demonstrate that $ \mathcal{C}^{nj} \subset \mathcal{C} $ where the first is the set of computable functions without a jump instruction. We know that $ f: \nat \rightarrow \nat, f(x)\uparrow \forall x $ is computable in URM, but it is not computable in URM $ ^{nj} $

Which functions $ f: \nat \rightarrow \nat $ can be computed without jumping? Keep in mind that we also removed the only conditional statement, so they always end. Then we have the following cases:

$f(x) = c  \forall x $ or $ f(x) = x + c \forall x $

We see $ r_1(h,x) $ the contents of register 1 after h steps with initial content x. Given $P$ program, the function computed by him $ f_p(x) = r_1(l(P), x) $ where $ l(P) $ is the length of $P$;

We show by induction on $h$ that after $h$ execution steps on $P \quad  r_1(h,x) $ is equal to $x + c$ or to $c$.

By induction $h = 0: r_1(0,x) = x $ OK

Case $ h \rightarrow h+1 $: We know $ r_1(h,x) = x+c $ or $ c $ by inductive hypothesis. The next instruction can be one of three cases:
\begin{itemize}
\item The instruction is $Z (n)$. Then if $n = 1  \quad  r_1(h+1,x) = 0 $, otherwise $ r_1(h+1,x) = r_1(h,x) $ in both cases it's OK. The first is constant, the second is fine by inductive hypothesis.
\item The instruction is $S (n)$. Then if $n = 1$ we find that $ r_1(h+1,x) = r_1(h,x)+1 $ which by induction hypothesis is fine. Even if $n$ is other than 1 (see the previous bullet point)
\item The instruction is $T (m, n)$. In cases where $ n>1 $ or $ n=m=1 $ then $ r_1(h+1,x) = r_1(h,x) $ and that's fine. Otherwise $ n = 1, m > 1 $ we do not know what $ r_1(h+1,x) $ is worth. So who knows?
\end{itemize}

Actually we should prove this not only for register $ r_1 $ but for a generic
register $ r_j $.
