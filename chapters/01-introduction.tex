\chapter{Introduction}\label{chap:intro}

In this chapter, we informally discuss the notion of \textbf{effective procedure} and \textbf{function computable} by means of an effective
procedure. This will lead us to single out the main features of an
algorithm/computational model.  Despite being informal, these
considerations will allow us to derive the existence of non-computable
functions for every effective computational model.
%
Subsequently these notions and
considerations will be formalized by setting a specific computational model,
a kind of idealized computer.

\section{Algorithm or effective procedure}

\textbf{Effective procedure}s and \textbf{algorithms}, even though we do not always call them in this way, are a part of our everyday life.

For example, at the primary school we are not only taught that given two numbers their sum exists, but we are also provided with a procedure to compute the sum of two numbers!

In general terms, an \emph{algorithm} can be defined as the
description of a sequence of \emph{elementary steps} (where an
``elementary step'' is a step which can be performed ``mechanically'', without
any intelligence) which allows one reach some objective.  Typically,
the aim is transforming some input into a corresponding output,
suitably related to the input.
%
This could be transforming ingredients into a cake, although normally
we are interested in computational problems.

\begin{example}
  Some examples are:
  \begin{enumerate}

  \item given $n \in \nat$, establish whether $n$ is prime;
  \item find the $n^{th}$ prime number;
  \item differentiate a polynomial;
  \item perform the square root $\sqrt{n}$;
  \item find least common multiple $\mathit{lcm}$ and greatest common divisor $\mathit{GCD}$.
  \end{enumerate}
\end{example}

Therefore we can think of an algorithm as a black box
\begin{center}
  in $\rightarrow$ \boxed{black box} $\rightarrow$ out
\end{center}
where the transformation is performed by executing a sequence of
elementary instructions.

If each step is \emph{deterministic} (i.e., for each state of the
system, the instruction to be executed next and the new state it produces are
uniquely determined), then each possible input will uniquely determine
the corresponding output (or the procedure might not terminate, in which case we will have no output).

In mathematical terms, the algorithm determines a (partial) function

\begin{center}
  $f : \mathit{input} \rightarrow \mathit{output}$.
\end{center}

We say that $f$ is the function \emph{computed} by the algorithm and
that $f$ is effectively computable. Thus, we can give the following
first definition of an algorithm (still informal since it refers to a
generic notion of algorithm).

\begin{definition}[Computable function]
  A function $f$ is \emph{computable} if there exists an algorithm
  that computes $f$.
\end{definition}

We stress that for $f$ to be computable, it is not important to know which is the algorithm that computes $f$, but we just need to know that some algorithm that computes $f$ exists.

\begin{example}
  According to the above definition, we expect the the following
  functions to be computable:

  \begin{itemize}

  \item GCD (greatest common divisor), e.g., exploiting Euclid's
    algorithm.

  \item the function $f : \nat \to \nat$ defined as

    \begin{equation*}
      f(n)=
      \begin{cases}
        1 & n \mbox{ prime} \\
        0 &   \mbox{otherwise}
      \end{cases}
    \end{equation*}

  \item
    $g(n) = p_n$\\
    where $p_n$ is the $n$-th prime number.

    This is computable by generating numbers
    and testing for primality until the $n$-th prime is found.


  \item
    $h(n) = \pi_n$
    where $\pi_n$ is the $n$-th digit of the decimal representation of $\pi$.

    Indeed there are
    \begin{itemize}
    \item series that converge to $\pi$
    \item techniques to estimate (by excess) the error caused by
      \begin{itemize}
      \item truncating a series
      \item rounding in the computation of the value of the truncated series
      \end{itemize}
    \end{itemize}
  \end{itemize}
\end{example}

What about the function below?

\begin{equation*}
  g(n) = \begin{cases}
    1 & $ there is a sequence of exactly  \textit{n} consecutive $5$'s in $ \pi \\
    0 & $ otherwise $
	\end{cases}
\end{equation*}

For example $g(3) = 1$ if and only if  $\pi = 3.14 \dots k 555 h \dots $, with $k, h \neq 5$.

A naive algorithm could consist in generating the digits of $\pi$ until a sequence of $5$'s of the desired length $n$ is found.
Clearly, if such a sequence exists, it will be eventually found and the answer $1$ will be returned. 
However, at no stage of the computation, we can exclude that the desired sequence of $n$ $5$'s will appear later; hence, apparently, we have no way of returning $0$.

\begin{remark}
  On the basis of the considerations above, the following
   \begin{itemize}
   \item generate all digits in the decimal representation of $\pi$;
   \item if they include a sequence of $n$ consecutive 5's then
    $g(n) = 1$;
  \item otherwise $g(n) = 0$.
\end{itemize}
is \emph{not} an effective procedure.
\end{remark}

Note that this does not mean that $g$ is not computable, i.e., that an
effective procedure for computing $g$ does not exist, but at the moment this procedure is not known (to us)!

We do not know if $g$ is computable, but there might be a property
of $\pi$ that allows us to conclude. In particular, there is a conjecture that all
finite sequences of digits appear in $\pi$, which would imply that $g$
is simply the constant $1$, whence computable.

\medskip

Consider now a slightly different function $h: \nat \to \nat$, defined by
\begin{equation*}
  h(n) = \begin{cases}
    1 & $ there is a sequence of at least  \textit{n} consecutive $5$'s in $ \pi \\
    0 & $ otherwise $
  \end{cases}
\end{equation*}

The function seems very similar to the one considered before. However, note that if $\pi = 3.14 \dots k 555 h \dots $, then we deduce, not only that $h(3)=5$, but also $h(2)=h(1)=h(0)=1$. 
More generally, whenever $h(n) =1$ then $h(m)=1$ for all $m < n$. This suggests that $h$ could have a quite simple shape.

More precisely, consider $K = sup\{ n \mid \pi\ \text{contains}\ n\ \text{consecutive digits}\ 5 \}$. Then we have 2 possibilities:
\begin{enumerate}
\item $K$ is finite, and thus
  \begin{center}
    $h(n) = \begin{cases}
      1 & \mbox{if $n\leq K$}\\
      0 & \mbox{otherwise}
    \end{cases}
    $
  \end{center}
\item $K$ is infinite, and thus
    \begin{center}
      $h(n) = 1$ for all $n \in \nat$
    \end{center}
  \end{enumerate}

This implies that $h$ is computable because it is either a step
function or a constant function, and these function can are computed by simple
programs. One could object that we do not know which shape the function
has and thus we do not know exactly which is the program that computes the
function. This is true, but irrelevant for computability.

Trying to replicate the same argument for function $g$ fails. In fact, one could think of defining $A = \{n \mid \mbox{$\pi$ contains exactly $n$ consecutive 5's}\}$. Then

\begin{equation*}
  g(n) =
    \begin{cases}
      0 & n \in A     \\
      1 & n \not\in A
    \end{cases}
\end{equation*}

This does not suggest that $g$ is computable. Set $A$ is possibly infinite and we do not see a way of providing a finite representation of $A$ which can be included in a program.

Bringing the argument to the extreme, one could consider the function
$G : \nat \to \nat$ defined by
\begin{center}
  $G(x) = \begin{cases}
    1 & $ if God exists $ \\
    0 & $ otherwise $
  \end{cases}
  $
\end{center}
Since the condition does not depend on the variable $x$, the function is either the constant $0$ or the constant $1$. 
Independently of which of the two cases applies, the function is computable.