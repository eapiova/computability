\chapter {Universal Function}
\newcommand{\Psiex}{\ensuremath{\Psi_{\mathcal{U}}^{(k)} (e, \vec{x})}}
\newcommand{\Psiuex}{\ensuremath{\Psi_e^{(k)} (\vec{x})}}
\newcommand{\univ}{\ensuremath{\Psi_{\mathcal{U}}}}
We'll now see how the theory which was described up until now allows us to
prove the existence of a
universal function is able to reproduce the behaviour of each
and every other computable function. 
Let's consider the function $\Psi : \nat^{2} \rightarrow \nat$
\[
  \Psi(x,y) = \varphi_e(y)
\]
We can observe that $\Psi$ is a \emph{universal function}, i.e. it
captures all unary computable functions $\varphi_1, \varphi_2, \dots$. In fact,
for all $e \in \nat$
\[
  g(y) = \Psi(e,y) = \varphi_e(y) \quad \rightsquigarrow \quad g = \varphi_e
\]
so $\Psi$ represents all the computable functions of the
form $\nattonat$.

\begin{definition}
  The universal function for $k$-ary functions (with $k \geq 1$) is
  defined as
  \begin{align*}
    &\univ^{(k)} : \nat^{k+1} \rightarrow \nat \\
    &\Psiex = \varphi_e^{(k)}(\vec{x})
  \end{align*}
\end{definition}
Is it computable? If yes, then a program $P_{\mathcal{U}}$ computing $\univ$ would
be able to compute all $k$-ary functions, so it includes in itself all
other computable programs, and we can call it a Universal Computer
\cite{davis:2011}.

It receives in
input
\begin{itemize}
\item $e$ (the index of the program, a \textit{description} of the
  program $P_e$ to run)
\item $\vec{x}$ the arguments
\end{itemize}
then it fits the description of an interpreter.

In fact, the following result applies
\begin{theorem}
  The universal function $\Psi_{\mathcal{U}}^{(k)}$
  is computable.
  \begin{proof}
    Fixed $k \geq 1$,
    $e \in \nat$ and $\vec{x} \in \nat^k$ we want
    $\Psiex = \varphi_e^{(k)}(\vec{x})$.
    Idea:
    \begin{itemize}
    \item get the program $P_e = \gamma^{-1}(e)$;
    \item execute $P_e$ on input $\vec{x}$;
    \item if $P_e(\vec{x})\downarrow$, the value $\Psiex$ is in
      $R_1$, otherwise there is no problem.
    \end{itemize}
    Thus, effectiveness follows and by Church-Turing thesis also computability.

    More formally, we will need some programs
    in order to accomplish this.

      Consider
      \[
        \begin{tabu}{|c|c|c|c|c}
          \hline
          r_1 & r_2 & r_3 & 0 & \dots \\ \hline
        \end{tabu}
      \]
      the \textit{configuration of registers} is given by
      \[ c = \prod_{i \geq 1} p_i^{r_i} \]
      The value of the registers is computed as $ r_i = (c)_i$

    We define the following functions:
    \begin{itemize}
      \item \[
        c_k : \nat^{k+2} \rightarrow \nat
      \]
      where 
      $c_k(e, \vec{x}, t)$
      is the configuration after $t$ steps of computation of $P_e(\vec{x})$,
      if $P_e$ does not stop on $\vec{x}$ in $t$ or fewer steps; instead, it is
      the final configuration if $P_e(\vec{x})$ stops in $t$ or fewer steps.
      \item  \[
        j_k : \nat^{k+2} \rightarrow \nat
      \]
      where
      $j_k(e, \vec{x}, t)$ is the number of instructions to be executed after $t$
      steps of $P_e(\vec{x})$; instead it is $0$ if $P_e(\vec{x})$ stops in $t$ or fewer steps.
    \end{itemize}
    assuming that $c_k, j_k$ are computable
    \begin{itemize}
      \item if $P_e(\vec{x}) \downarrow$, then it stops in $\mu t . j_k(e, \vec{x}, t)$ steps, so
            \[\varphi_e^{(k)}(\vec{x}) = (c_k(e,\vec{x}, \mu t . j_k(e, \vec{x}, t)))_1\]
      \item if $P_e(\vec{x}) \uparrow$, then $\mu t \ . \ j_k(e, \vec{x}, t) \uparrow$, hence
            \[\varphi_e^{(k)}(\vec{x}) \uparrow = (c_k(e,\vec{x}, \mu t . j_k(e, \vec{x}, t)))_1\]
    \end{itemize}
   
    Hence in all cases
    \[\Psiex = \varphi_e^{(k)}(\vec{x}) = (c_k(e,\vec{x}, \mu t . j_k(e, \vec{x}, t)))_1\]
    so if we prove that $c_k, j_k$ are computable, we can conclude that
    $\Psi_{\mathcal{U}}^{(k)}$ is also computable. 
    
    We proceed in the
    same way we did in the proof \S\ref{reqc}, by proving that
    $c_k, j_k \in \pr$ (in fact, this can be seen as a more formal
    proof of the same fact, the only difference is that we defined
    $c_P, j_P$ with \emph{a fixed $P$} in the latter demonstration.
    Nevertheless, $P$ is instead a parameter here).

    \newcommand{\uarg}{{\mbox{arg}}}
    \newcommand{\uargh}{{\mbox{arg}_h}}
    \begin{enumerate}[label=(\alph*)]
    \item given $i \in \nat$ instruction code $( i = \beta(\mbox{Instruction}))$
      \begin{align*}
        Z_\uarg (i) &= qt(4, i) + 1 \\
        S_\uarg (i) &= qt(4,i) + 1 \\
        T_\uargh(i) &= \pi_h(qt(4,i)) + 1 \quad h \in \{1,2\} \\
        J_\uargh(i) &= \nu_h(qt(4,i)) + 1 \quad h \in \{1,2,3\}
      \end{align*}
    \item effect of executing an algebraic instruction on a configuration
      \begin{align*}
        zero(c,n) &= qt(p_n^{(c)_n}, c) \\
        succ(c,n) &= p_n \cdot c \\
        transf(c,m,n) &= p_n^{(c)_m} \cdot zero(c,n)
      \end{align*}

    \item effect on the configuration of registers of the execution of
      the instruction with code $i$
      \[
        change(c,i) = \begin{cases}
          zero(c, Z_\uarg(i)) & rm(4,i) = 0 \\
          succ(c, S_\uarg(i)) & rm(4,i) = 1 \\
          transf(c, T_{\uarg_1}(i), T_{\uarg_2}(i)) & rm(4,i) = 2 \\
          c & rm(4,i) = 3
      \end{cases}
    \]

  \item configuration of the registers starting from $c$,
    after executing instruction $t$ of program $P_e$ 
    \[
      nextconf(e,c,t) = \begin{cases}
        change(c, a(e,t)) & 1 \leq t \leq \ell(e) \\
        c & \mbox{otherwise}
      \end{cases}
    \]

  \item number of next instruction if we execute $i=\beta(\mbox{Instruction})$ 
        and this is in position $t$ of the program
    \[
      ni(c, i, t) = \begin{cases}
        t+1 & rm(4,i) \neq 3 \vee (rm(4,i) = 3 \wedge (c)_{J_{\uarg_1}(i)} \neq (c)_{J_{\uarg_2}(i)}) \\
        J_{\uarg_3}(i) & \mbox{otherwise}
      \end{cases}
    \]

  \item next instruction, if we execute instruction $t$
    in program $P_e$ in configuration $c$
    \[
      nextinstr(e,c,t) = \begin{cases}
        ni(c, a(e,t), t) & 1 \leq t \leq \ell(e) \wedge 1 \leq ni(c, a(e,t), t) \leq \ell(e) \\
        0 & \mbox{otherwise}
      \end{cases}
    \]
  \end{enumerate}

 Now
 \begin{align*}
  c_k(e, \vec{x}, 0) &= \prod_{i=1}^kp_i^{x_i} \\
  j_k(e, \vec{x}, 0) &= 1 \\
  c_k(e, \vec{x}, t+1) &= nextconf (e, c_k(e, \vec{x}, t), j_k(e,\vec{x},t)) \\
  j_k(e, \vec{x}, t+1) &= nextinstr (e, c_k(e, \vec{x}, t), j_k(e,\vec{x},t))
 \end{align*}
 they are defined by primitive recursion of computable functions, therefore $c_k, j_j \in \pr$, so they are computable.
 Thus,
  \[
    \Psiex = (c_k(e, \vec{x}, \mu t \; . \; j_k(e,\vec{x},t)))_1
  \]
  is computable.
  \end{proof}
\end{theorem}

As a corollary, we obtain the decidability of two statements that will
be really useful in the next chapters.
\begin{corollary}
  The following predicates are decidable:
  \begin{enumerate}[label=(\alph*)]
  \item $H_k(e, \vec{x}, t) \equiv$ ``$P_e(\vec{x})\downarrow$ in $t$
    or less steps''
  \item $S_k(e, \vec{x}, y, t) \equiv$ ``$P_e(\vec{x})\downarrow y$ in
    $t$ or less steps''
  \end{enumerate}
  \begin{proof}
    \begin{enumerate}[label=(\alph*)]
    \item 
      \begin{align*}
        \chi_{H_k}(e, \vec{x}, t) &= 
        \begin{cases}
          1 & \mbox{if } H_k(e, \vec{x}, t)\\
          0 & \mbox{otherwise}
        \end{cases} \\
        &= \overline{sg}(j_k(e,\vec{x},t))
      \end{align*}
      it is computable by composition.
    \item \[
      \chi_{S_k}(e, \vec{x}, y, t) = \chi_{H_k}(e, \vec{x}, t) 
      \cdot \overline{sg}(|(c_k(e,\vec{x},t))_1 - y|) 
    \]
    it is computable by composition.
    \end{enumerate}
  \end{proof}
  If $k=1$ we usually omit it.
\end{corollary}

Also, from the theorem we deduce the possibility to express every
computable function in Kleene Normal Form (KNF).
\begin{corollary}[Kleene Normal Form]
  For every $e,k \in \nat$ and $x \in \nat^k$
  \[
    \Psiuex = (\mu z \;.\; |\chi_{S_k}(e, \vec{x}, (z)_1, (z)_2) - 1|)_1
  \]
\end{corollary}

\begin{observation}
  \begin{enumerate}[label=\roman*.]
    \item This corollary highlights how each computable function can be obtained by primitive recursion
      functions using minimimalisation at most one time (we can use just one
      \texttt{while})
    \item Minimimalisation allows us to ``search'' a single value that has a
      certain property. The one we used is a technique to search couples
      of values generalizable to tuples.
    \end{enumerate}
\end{observation}


\section{Applications}
We observed that if $f : \nattonat$ is a total
computable injective function, then
\[
  f^{-1}(y) = \begin{cases}
    x & \quad \mbox{if exists $y$ s.t. } f(x) = y \\
    \uparrow & \quad \mbox{otherwise}
\end{cases}
\]
is computable since $f^{-1} = \mu x \; . \; |f(x) = y|$. 
The hypothesis of \emph{totality} can be omitted.
\begin{exercise}
  Let $f: \nattonat$ computable and injective. Then
  $f^{-1}: \nattonat$ is computable.
  \begin{proof}
    Since $f$ is computable, there exists $e \in \nat$ such that $\varphi_e =
    f$. Now it is sufficient to observe that
    \[
f^{-1}(y) = (\mu w \; . \; |\chi_S(e, (w)_1, x, (w)_2) - 1|)_1
    \]
  \end{proof}
\end{exercise}

We can now find new non-computable functions and undecidable predicates:

\begin{exercise}
  The statement ``$\varphi_x$ is total'' is undecidable
  \begin{proof}
    Let $Tot(x)$ be the predicate
    \[ Tot(x) \equiv \mbox{ ``$\varphi_x$ is total''} \]
    and assume that it is decidable.
    Define
    \[
      f(x) = \begin{cases}
        \varphi_x (x) + 1 & \varphi_x \mbox{ total}\\
        0 & \mbox{otherwise}
      \end{cases}
    \]
    it is total.
    For every $x$ if $\varphi_x$ is total, then $\varphi_x \neq f$, since
    \[
    f(x) =   \varphi_x (x) + 1 \neq \varphi_x (x)
    \]
    so $f$ it is not computable.
    But we can write $f(x)$ as
    \begin{equation*}
      \begin{split}
        f(x) = (\mu w . (S(x,x,(w)_1,(w)_2)) \wedge Tot(x) \wedge (w)_3 = (w)_2 + 1) \\
        \vee ((w)_3 = 0 \wedge \lnot Tot(x))
      \end{split}
    \end{equation*}
    which is absurd.
  \end{proof}
\end{exercise}
\begin{observation}
The same applies to prove that the following
statements are undecidable (Halting problem):
\begin{itemize}
\item
  $P_1(x) \equiv \mbox{``}x \in W_x\mbox{''} \equiv
  \mbox{``}\varphi_x(x) \downarrow \mbox{''}$
\item
  $P_2(x,y) \equiv \mbox{``} y \in W_x \mbox{''} \equiv
  \mbox{``}\varphi_x(y) \downarrow\mbox{''}$
\end{itemize}
\end{observation}

\section{Effective operations on computable functions}
The existence of the universal function, together with the \emph{smn}
theorem allows us to prove the effectivity of various
operations.
\begin{proposition}[Effectiveness of product]
  There exists a function $s: \nat^2 \rightarrow \nat$ total and computable
such that for every $x, y \in \nat$
 \[\varphi_{s(x,y)} = \varphi_x \cdot \varphi_y\]
\begin{proof}
  We define a function $g : \nat^3 \to \nat$
  \begin{align*}
    g(x,y,z) &= \varphi_x(z) \cdot \varphi_y(z) \\
    &= \Psi_{\mathcal{U}}(x,z) \cdot \Psi_{\mathcal{U}}(y,z)
  \end{align*}
  it is computable since it arises as composition of computable functions. 
  By the \emph{smn} theorem there
  exists $s: \nat^2 \rightarrow \nat$ total computable such that for every $x,y,z$
  \[
    \varphi_{s(x,y)}(z) = g(x,y,z) = \varphi_x(z) \cdot \varphi_y(z)
  \]
  thus
  \[
    \varphi_{s(x,y)} = \varphi_x \cdot \varphi_y
  \]
\end{proof}
\end{proposition}


\begin{proposition}[Effectiveness of squaring]
There exists $k:\nattonat$ total and computable such that, for every $x \in \nat$,
\[\varphi_{k(x)} = \varphi_x^2\]
\begin{proof}
$k(x) = s(x,x)$
\end{proof}
\end{proposition}

\begin{proposition}[Effectiveness of primitive recursion]
Recall the notion of primitive recursion
\begin{align*}
  h(\vec{x}, 0) &= f(\vec{x}) \\
  h(\vec{x}, y+1) &= g(\vec{x}, y, f(\vec{x},y))
\end{align*}
if we know that
$f,g$ are computable then $h$ is computable too. In other words, if
$f = \varphi_{e_1}^{(k)}$ and $g = \varphi_{e_2}^{(k+2)}$, then there exists $r: \nat^2 \rightarrow \nat$ total
computable such that
\[h = \varphi_{r(e_1, e_2)}^{(k+1)}\]
\end{proposition}

\begin{proposition}[Effectiveness of the inverse function]
  There exists $k: \nattonat$ total and computable such that
\[\forall x \in \nat \quad \mbox{if } \varphi_x \mbox{ is injective }
  \Rightarrow \varphi_{k(x)} = (\varphi_x)^{-1}\]
\begin{proof}
  We define a function $g : \nat^2 \to \nat$
  \begin{align*}
    g(x,y) &= (\varphi_x)^{-1}(y) \\
           &= \begin{cases}
      z & \exists z \mbox{ s.t. } \varphi_x(z) = y \\
      \uparrow & \mbox{otherwise}
    \end{cases}\\
    & = (\mu \omega \; . \; |\chi_{S(x, (\omega)_1, y, (\omega)_2)} - 1|)_1
  \end{align*}
  it is computable by minimalisation.
  Hence, by \emph{smn} theorem, there is a $k: \nattonat$ total and
  computable such that for every $x,y$
  \[\varphi_{k(x)}(y) = g(x,y) = (\varphi_x)^{-1}(y)\]
\end{proof}
\end{proposition}


\begin{proposition}
  There is a total computable function $s : \nat^2 \rightarrow \nat$
such that, for every $x, y$
 \[W_{s(x,y)} = W_x \cup W_y\]
\begin{proof}
  We want $\varphi_{S(x,y)}(z)\downarrow$ iff
  $\varphi_x(z)\downarrow$ or $\varphi_y(z) \downarrow$. We define
  a function $g : \nat^3 \to \nat$
  \[
    g(x,y,z) = \begin{cases}
      1 & z \in W_x \vee z \in W_y \\
      \uparrow  & \mbox{otherwise}
    \end{cases}
  \]
  which is computable:
  \[
    g(x,y,z) = \mathbf{1}(\mu \omega \; . \; |\chi_{H(x,z,\omega)
      \wedge H(y,z,\omega)} - 1|)
  \]
  

  Hence by \emph{smn} theorem exists $s:\nat^2 \rightarrow \nat$
  computable and total such that \[\varphi_{S(x,y)}(z) = g(x,y,z)\]
\end{proof}
\end{proposition}

\begin{proposition}
  There exists a $s:\nat^2 \rightarrow \nat$ computable and total such that
\[\forall x,y \quad E_{s(x,y)} = E_x \cup E_y\]
\begin{proof}
  We want the value of $\varphi_{S(x,y)}$ to be the same of the
  functions $\varphi_x and \varphi_y$. In order to do this, we can simulate
  $\varphi_x$ on even numbers and $\varphi_y$ on odd numbers. We
  define a function $g : \nat^3 \to \nat$
  \[
    g(x,y,z) = \begin{cases}
      \varphi_x(\frac{z}{2}) & \mbox{if } z \mbox{ even} \\
      \varphi_y(\frac{z-1}{2}) & \mbox{if } z \mbox{ odd}
    \end{cases}
  \]
  computable since
  \begin{multline*}
    g(x,y,z) = (\mu \omega \; . \; |\max\{\chi_S(x,qt(2,z),(\omega)_1,(\omega)_2) \; \cdot \; \overline{sg}(rm(2,z)), \\ \chi_S(y,qt(2,z),(\omega)_1, (\omega)_2) \; \cdot \; sg(rm(2,z))\}- 1|)_1
  \end{multline*}
  By \emph{smn} theorem there exists $s: \nat^2 \rightarrow \nat$
  computable and total such that \[\varphi_{s(x,y)}(z) = g(x,y,z)\] for every $x,y,z$.
  So
  \[
    \begin{split}
      v \in E_{s(x,y)} & \Leftrightarrow \exists z \; . \; \varphi_{S(x,y)}(z) = g(x,y,z) = v \\
      & \Leftrightarrow \exists z \; . \; \begin{cases}
        z \mbox{ even and } & \varphi_x(\frac{z}{2}) = v \\
        z \mbox{ odd and } & \varphi_y(\frac{z-1}{2}) = v \\
      \end{cases} \\
      & \Leftrightarrow \exists z \; . \; \varphi_x(z) = v \wedge
      \varphi_y(z) = v \Leftrightarrow \omega \in E_x \cup E_y
    \end{split}
  \]
\end{proof}
\end{proposition}

\begin{proposition}
  There is $k : \nattonat$ computable and total such that $E_{k(x)} = W_x$
\begin{proof}
  Define
  \begin{align*}
    g(x,y) &= \begin{cases}
      y & y \in W_x \\
      \uparrow & \mbox{otherwise}
    \end{cases} \\
    &= \mathds{1}( \univ (x,y)) \cdot y
  \end{align*}
  it is computable by composition, so by \emph{smn} theorem there exists $k : \nattonat$
  computable and total such that, for every $x,y$
  \[\varphi_{k(x)}(y) = g(x,y)\]
  In other words
  \[y \in E_{k(x)} \Leftrightarrow \varphi_{k(x)}(y) = y
    \Leftrightarrow g(x,y) = y  \Leftrightarrow y \in W_x\]
\end{proof}
\end{proposition}

\begin{proposition}
  Given $f : \nattonat$ computable, there exists $k : \nattonat$
computable and total such that, for every $x$, $W_{k(x)} = f^{-1}(W_x)$
\begin{proof}
  Define
  \[g(x,y) = \varphi_x(f(y)) = \univ(x, f(y))\] computable by
  definition. By the \emph{smn} theorem, there exists $k: \nattonat$ computable
  and total such that \(\varphi_{k(x)}(y) = g(x,y)\). So
  \[
    \begin{split}
      y \in W_{k(x)} & \Leftrightarrow \varphi_{k(x)}(y) = g(x,y) = \varphi_x(f(y)) \downarrow \\
      & \Leftrightarrow f(y)\downarrow \mbox{ and } f(y) \in W_x \\
      & \Leftrightarrow y \in f^{-1}(W_x)
    \end{split}
  \]
\end{proof}
\end{proposition}

\begin{proposition}
There exists $k : \nattonat$ computable and total such that if
$\varphi_x = \chi_Q$ is the characteristic function of a decidable
predicate $Q$, then $\varphi_{k(x)} = \chi_{\neg Q}$
\begin{proof}
  Define \[g(x,y) = 1 \dotdiv \varphi_x(y) = 1 - \univ(x,y) \]
  which is computable by definition. By the \emph{smn} theorem, there exists
  $k$ computable and total such that \[g(x,y) = \varphi_{k(x)}\]
  In this way,
  if $\varphi_x = \chi_Q$
  \[
    g(x,y) = 1-\varphi_x(y) = \varphi_{k(x)}(y) = 1 \Leftrightarrow
    \varphi_x(y) = 0 \Leftrightarrow \chi_Q(y) = 0
  \]
  therefore
  \[
    \varphi_{k(x)} = \chi_{\neg Q}
  \]
\end{proof}
\end{proposition}


