\chapter{First recursion theorem}

In programming languages, we have functions that use other functions as arguments,
e.g. in ML, the function \texttt{succ} that
given a function $f$ returns $f+1$ can be defined as
\begin{equation*}
    fun\ succ\ f\ x = f\ x + 1
\end{equation*}
From the computability point of view it is still somewhat natural to ask
how effective/computable operations can be characterized on
functions. We'll later see that this idea leads to the concept of
\emph{recursive functional}.

\begin{definition}
  We'll call $\mathcal{F}(\nat^k)$ the set of all the functions
  (computable and not) of k arguments $\nat^k \to \nat^k$.
\end{definition}

a \emph{functional} is just a function
\[
  \Phi : \mathcal{F}(\nat^k) \to \mathcal{F}(\nat^h)
\]
(only total functions will be considered)

When can we say that a functional is effective (computable), in fact,
given $\Phi : \mathcal{F}(\nat^k) \to \mathcal{F}(\nat^h)$
\begin{itemize}
\item a function $f \in \mathcal{F}$ and its image
  $\Phi(f) \in \mathcal{F}(\nat^k)$ are both infinite objects in
  general.
\item We can not ask for $\Phi(f)$ to be computable in a finite time
  from $f$
\end{itemize}

What we need is a way to see functions as numbers

\subsection{Encoding of finite functions}
for each finite function its' encoding $\tilde{\theta} \in \nat$ is defined
as
\begin{itemize}
\item if $\theta = \emptyset$ then $\tilde{\theta} = 0$
\item if $\theta = \{(x_1, y_1), \dots, (x_2, y_2)\}$ then
  $\tilde{\theta} = \prod_{i=1}^n p_{x_1}^{y_i+1}$
\end{itemize}
which is both injective and effective. Given the encoding of a function:
$z= \tilde{\theta}$;
\[
  x \in \dom{\theta} \quad \mbox{iff} \quad (z)_1 \neq \emptyset
\]
\[
  \begin{aligned}
    app(z,x) = \theta(x) & = \begin{cases}
      (z)_x = 1 & x\in\dom{\theta} \\
      \uparrow & \mbox{otherwise}
    \end{cases} \\
    &= ((z)_x \dotdiv 1) \mathds{1}(\mu \omega \; . \; |1 - (z)_x|)
  \end{aligned}
\]

this way we can give the following definition of functional

\begin{definition}
  A functional
  $\Phi : \mathcal{F}(\nat^k) \to \mathcal{F}(\nat^h)$ is
  \emph{recursive} if exists $\varphi : \nat^{h+1} \to \nat$
  computable s.t.
  $\forall f \in \mathcal{F}(\nat^k) \quad \forall \vec{x} \in \nat^h
  \quad \forall y \in \nat$

  \[
    \Phi(f)(y) = y \quad \mbox{iff} \quad \exists \theta \subseteq f
    \mbox{ finite s.t. } \varphi(\tilde{\theta}, \vec{x}) = y
  \]
\end{definition}

\begin{example}
  The functional
  \newcommand{\fib}[1]{\ensuremath{\mbox{fib}(#1)}}
  \[
    \mbox{fib} : \mathcal{F}(\nat) \to \mathcal{F}(\nat)
  \]
  \[
    \fib{f}(x) = \begin{cases}
      1 & x=0 \lor x=1 \\
      f(x-2) + f(x-1) & x \geq 2
    \end{cases}
  \]
  is recursive, the function $\varphi: \nat^2 \to \nat$ can be
  \[
    \begin{aligned}
      \varphi(z, x) &= \begin{cases}
        1 & x=0 \lor x=1 \\
        \theta(x-2) + \theta(x-1) & x \geq 2 \land z = \tilde{\theta}
      \end{cases} \\
      &= \begin{cases}
        1 & x = 0 \lor x = 1 \\
        app(z, x-2) + app(z, x-1) & x > 2
      \end{cases}
    \end{aligned}
  \]
  which is computable
\end{example}

\begin{example}
Te functional associated to Ackermann function:
\[
  \Psi_{ack} : \mathcal{F}(\nat^2) \to \mathcal{F}(\nat^2)
\]
\[
  \begin{tabu}{r l l}
    \Psi_{ack}(f)(0,y)  & = & y+1 \\
    \Psi_{ack}(f)(x+1, 0) & = & f(x,1) \\
    \Psi_{ack}(f)(x+1, y+1)  & = & f(x, f(x+1, y)) \\
  \end{tabu}
\]
which is computable as before.
\end{example}
We can also verify:

\begin{theorem}\label{th:unknown}
  Let $\Phi : \mathcal{F}(\nat^k) \to \mathcal{F}(\nat^h)$
  be a recursive functional and let $f \in \mathcal{F}(\nat^k)$ be
  computable. Then $\Phi(f) \in \mathcal{F}(\nat^h)$ is computable
\end{theorem}

\section{Myhill-Sheperdson theorems}
Given a recursive functional $\Phi$, for (\ref{th:unknown})

\[
  f \mbox{ computable } \rightsquigarrow \Phi(f) \mbox{ computable }
\]
\[
  f = \varphi_e \rightsquigarrow \Phi(f) = \varphi_{e'}
\]

so we can see a recursive functional as a function that transforms
indices (programs) in indices (other programs), but with the property
that the transformation depends on the firstly indexed function and
\emph{not} on the index itself

\begin{definition}[Extensional function]
  Let $h : \nattonat$ a total function. It is \emph{extensional} if
  \[
    \forall e,e' \quad \varphi_e = \varphi_{e'} \to
    \varphi_{h(e)} = \varphi_{h(e')}
  \]
\end{definition}

\begin{theorem}[Myhill-Shepherdson (I)]
  If $\Phi : \mathcal{F}(\nat^k) \to \mathcal{F}(\nat^h)$ is
  recursive functional then it exists a total computable function
  $h_\Phi : \nattonat$ s.t.
  \[
    \forall e \quad \Phi(\varphi_e) = \varphi_{h_\Phi(e)}^{(k)}
  \]
\end{theorem}

Intuitively, the behaviour of the recursive functional on computable
functions is captured entirely by a total extensional function on the
indices

\begin{theorem}[Myhill-Shepherdson (II)]\label{th:myhill-shepherdson2}
  If $h : \nattonat$ is a total computable extensional function, then
  exists \emph{only one} recursive functional $\Phi_h$ s.t.
  \[
    \forall e \quad \Phi_h(\varphi_e) = \varphi_{h(e)}
  \]
\end{theorem}
That is to say that a total computable extensional function identifies
only one recursive functional.  Is interesting to observe that
$\Phi_n$ exists also for non computable functions.

\begin{theorem}[First recursion theorem (Kleene)]\label{th:first-recursion}
  Let $\Phi : \mathcal{F}(\nat^k) \to \mathcal{F}(\nat^h)$ be
  a \textbf{recursive functional}. Then $\Phi$ has a \textbf{least fixed point}
  $f_\Phi$ which is \textbf{computable}, i.e.
  \begin{enumerate}
  \item $\Phi(f_\Phi) = f_\Phi$
  \item $\forall g \in \mathcal{F}(\nat^k) \quad \Phi(g) = g \Rightarrow f_\Phi \subseteq g$
  \item $f_\Phi$ is computable
  \end{enumerate}
  and we can see that $f_\Phi = \bigcup\limits_n
  \Phi^{n}(\emptyset)$, and since a functional can be associated to a
  fixed point; the theorem proves the closure of the set of computable
  functions with respect to extremely general forms of recursion.
\end{theorem}
 The following are the exceptions:
\paragraph{\textbf{Primitive recursion}}
given $f : \nat^h \to \nat$ and
$g : \nat^{h+2} \to \nat$, the function defined by primitive
recursion is the minimum fixed point of
$\Phi_r \in \mathcal{F}(\nat^{h+1})$, defined by
\[
  \begin{tabu}{r l l}
    \Phi_r(h)(\vec{x}, 0) & = & f(\vec{x}) \\
    \Phi_r(h)(\vec{x}, y+1) & = & g(\vec{x}, y h(\vec{x},y))
  \end{tabu}
\]
and if $f,g$ are computable, then $\Phi_r$ is a recursive
functional. This way the theorem assures us that:
\begin{itemize}
\item exists a minimal fixed point;
\item it is computable.
\end{itemize}

\mbox{}\\*
\paragraph{\textbf{Minimization}}
given a function $f: \nat^{k+1} \to \nat$ we can see the
minimization $\mu y \; . \; f(\vec{x}, y)$ as a fixed point. Let us
consider, for a fixed f
\[
  \Phi_\mu \in \mathcal{F}(\nat^{k+1})
\]
\[
  \Phi_\mu(h)(\vec{x}, y) = \begin{cases}
    y & f(\vec{x},y) = 0 \\
    h(\vec{x}, y+1) & f(\vec{x}, y)\downarrow \land f(\vec{x}, y) \neq 0 \\
    \uparrow & \mbox{otherwise}
  \end{cases}
\]
which is recursive and has a minimum fixed point:
\[
  f_{\Phi_\mu}(\vec{x}, y) = \mu (z \geq y) \; . \; f(\vec{x}, y)
\]

\mbox{}\\*
\paragraph{\textbf{Ackermann function}}
We saw that $\Phi_{ack}$ is recursive, therefore it has a computable
minimum fixed point (the Ackermann function itself $\Psi$). The fact
that $\Psi$ is total implies that such fixed point is the
\emph{only} fixed point.


\begin{observation}
  Generally speaking, the fixed point is not unique. Counter-example:
  \[
    \Phi(f)(x) = \begin{cases}
      0 & x=0 \\
      f(x+1) & \mbox{otherwise}
    \end{cases}
  \]
  is recursive, and therefore has a minimum fixed point
  \[
    f_\Phi(x) = \begin{cases}
      0 & x=0 \\
      \uparrow & \mbox{otherwise}
    \end{cases}
  \]
  but has other fixed points, for example:
  \[
    f(x) = \begin{cases}
      0 & x=0 \\
      k & x>0
    \end{cases}
  \]
\end{observation}
